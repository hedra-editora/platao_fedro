\chapterspecial{Final}{}{}
 

[265e] S: Poder novamente (\emph{pálin}) discernir
(\emph{diatémnein}) em espécies (\emph{eídê})\emph{,}~segundo as
articulações naturais, procurando não causar roturas em nenhuma parte,
ao modo do cozinheiro inexperiente. Mas que sirvam de exemplo os dois
discursos anteriores, que reuniram a insanidade do pensamento
(\emph{áphron tês dianoías}) a uma ideia comum
(\emph{koinêi}~\emph{eidos}). Tal como de um só corpo nascem membros
duplos e homônimos (\emph{diplâ kaì homṓnuma}), chamados sinistros e
destros, [266a] assim também o discurso nos apresentou uma ideia do
desvio do intelecto (\emph{paranóias}). Um deles, discernindo
(\emph{temnómenos}) a sua parte esquerda, não cessou de novamente
(\emph{pálin}) discerní"-la (\emph{témnôn}) enquanto não descobriu
(\emph{epheurṑn}) aí uma espécie de amor denominado sinistro, a quem com
toda a razão encheu de censuras (\emph{eloidórêsen}), e o outro nos
levou para a parte destra da loucura (\emph{manías})\emph{,}~homônima
àquela (\emph{homônumon mèn ekeínôi}), mas divina (\emph{theîon}) foi
essa parte descoberta (\emph{epheurṑn}) do amor, apresentando"-a diante
dos nossos olhos e cantando"-lhe elogios (\emph{epḗinesen}), como sendo a
causa dos nossos maiores bens (\emph{agathôn}).

 

[266b] F: É verdade o que dizes.

 

S: Eu mesmo sou um amante (\emph{erastḗs}), ó Fedro, dessas divisões e
sínteses (\emph{diairéseôn kaì sunagôgôn}), meio pelo qual é possível
falar e pensar (\emph{légein te kaì phroneîn}). Se considero qualquer
outra pessoa capaz de observar a natureza do uno e do múltiplo, este eu
persigo, ``seguindo seus passos como os de um deus''. Os que são capazes
disso, quer tenha eu os designado bem ou não, deus o sabe, até agora os
referi como dialéticos (\emph{dialektikoús}). [266c] Mas aos que
aprendem (\emph{mathóntas}) junto a ti e a Lísias, como é necessário que
os designemos? Ou essa não é aquela arte discursiva (\emph{lógôn
téchnê}), segundo a qual Trasímaco e outros sábios manejavam o falar,
proporcionando que outros assim também o fizessem, aqueles que queriam
presenteá"-los como se fossem reis (\emph{basileûsin})?

 

F: Há de fato uma realeza (\emph{basilikoì}) nesses homens, embora não
conheçam isso que perguntas. Parece"-me correta essa forma de dizer, que
esses são chamados de dialéticos (\emph{dialektikòn}), mas parece"-me,
todavia, que a retórica (\emph{rhêtorikòn}) ainda nos escapa
(\emph{diapheúgein}).

 

[266d] S: Como dizes? Onde poderia existir algo belo que, mesmo
afastado dessas mesmas características, fosse adquirido como uma arte
(\emph{téchnêi})? Em todo caso é preciso que não a desprezemos, tu e eu,
mas que falemos o que ficou de lado sobre a retórica
(\emph{rhêtorikês}).

 

F: E é bastante vasto, ó Sócrates, aquilo que foi escrito nos livros
acerca da arte discursiva (\emph{en toîs biblíois toîs perì lógôn
téchnês gegramménois}).

 

S: Bem me recordaste (\emph{hupémnêsas}) disto. Segundo creio,
primeiramente é necessário proferir no início dos discursos o
``proêmio''. É~a isso que te referes ou não? A esses refinamentos da
arte?

 

[266e] F: Sim.

 

S: Em segundo lugar vem a ``narração'' (\emph{diḗgesín}) e alguns
``testemunhos'' (\emph{marturías})\emph{~}que lhes dizem respeito, em
terceiro lugar a ``prova'' (\emph{tekmḗria}) e em quarto as
``verossimilhanças'' (\emph{eikóta})\emph{.~}Também existe, segundo
creio, a ``confirmação'' (\emph{pístôsin}) e a ``confirmação
suplementar'' (\emph{epipístôsin}), nos dizeres do excelente burilador
de discursos (\emph{logodaídalon}), o homem de Bizâncio.

 

F: Mencionas o auspicioso Teodoro?

 

[267a] S: Quem mais senão ele, o qual disse haver nas composições
uma ``refutação'' (\emph{elegchón}) e uma ``refutação suplementar''
(\emph{epexélegchon}), tanto na acusação como na defesa
(\emph{katêgoríai te kaì apologíai}). E~o belíssimo Eveno de Paros, não
o traremos para o debate? Ele foi quem primeiro inventou (\emph{hêuren})
as ``insinuações'' (\emph{hypodḗlosin})\emph{~}e os ``para"-elogios''
(\emph{parepaínous})\emph{.~}Dizem que ele compôs ``para"-vitupérios''
(\emph{parapsógous}) em versos, para auxiliar a memória (\emph{mnḗmes
chárin}). Foi, portanto, um homem sábio (\emph{sophòs gàr anḗr}).~E
Tísias~e Górgias, vamos deixá"-los dormindo, eles que souberam honrar
mais a verossimilhança do que a verdade, que pela força discursiva
fizeram o grande parecer pequeno e o pequeno parecer grande, [267b]
o novo parecer arcaico, bem como o seu contrário, o arcaico parecer
novo, e que acerca de todos os assuntos encontraram (\emph{anêuron}) a
concisão discursiva e seu prolongamento
indefinido?\textsuperscript{~}Ouvindo isso de mim, outrora, Pródico
sorriu e disse que somente ele havia descoberto (\emph{heurêkénai}) o
que é preciso na arte discursiva, discursos que não sejam nem longos nem
curtos, mas na medida (\emph{metríôn}).

 

F: Ó Pródico, sapientíssimo!

 

S: Não falamos ainda de Hípias?~Creio eu que o estrangeiro de Élis
votaria também com Pródico.

 

F: E por que não?

 

S: E o que diremos do~\emph{Museu Discursivo}~de Polo? [267c] Com
sua duplicação discursiva (\emph{diplasiologían})\emph{,~}coleção de
máximas (\emph{gnômologían}) e estilo imagético (\emph{eikonologian}).~E
do~\emph{Vocabulário}~que Licímnio~havia presenteado Polo, em vista da
composição do seu belo falar?

 

F: E de Protágoras, ó Sócrates, não há nada desse tipo?

 

S: A~\emph{Dicção Correta~}(\emph{Orthoépeia}) é uma delas, ó jovem,
entre muitas outras e belas composições. E~dos discursos piedosos
escritos sobre a velhice e a pobreza, o que me parece dominar pela arte
é o do grande Calcedônio, homem terrível que a muitos enfurecia e, em
seguida, novamente, conduzia pelos encantos da palavra (\emph{epáidôn
kêleîn}), a dissipar essa fúria, como ele mesmo dizia. [267d]
Fortíssimo ele era em gerar e destruir qualquer tipo de calúnia. Quanto
à conclusão dos discursos parecem estar todos em comum acordo, embora
alguns a chamem de peroração (\emph{epánodon}) e outros estabeleçam
outro nome.

 

F: Aludes à recordação (\emph{hypomnêsai}) de cada um dos pontos
capitais, no final do que foi dito aos ouvintes?

 

S: Falo disso mesmo, e se tens algo mais a dizer acerca da arte
discursiva…

 

F: Insignificâncias indignas de menção.

 

[268a] S: Deixemos de lado as insignificâncias e, à luz do sol,
vejamos melhor que potencialidade (\emph{dúnamin}) tem quem detém essa
arte.

 

F: Muita é sua força, ó Sócrates, sobretudo nas reuniões populares
(\emph{plḗthous}~\emph{sunódois}).

 

S: Tem mesmo, mas, ó divino, vê também se essa urdidura não te parece
frouxa, como a mim me parece.

 

F: Mostra"-me.

 

S: Diz"-me, se alguém chegasse a teu amigo Erixímaco ou a seu pai Acúmeno
afirmando o seguinte: ``Eu conheço aplicações para aquecer o corpo ou,
se desejar, resfriá"-lo, e se me parecer adequado fazê"-lo vomitar ou, ao
contrário, evacuar, além de muitos outros tantos efeitos semelhantes.
[268b] Tendo conhecimento disso, considero"-me um médico capaz de
fazer com que outros assim procedam, transmitindo tais saberes''. O~que
pensas que os seus ouvintes, nesse caso, diriam?

 

F: Que outra coisa, senão perguntar se ele sabe em quem e quando é
preciso aplicar cada um deles, e também em que quantidade?

 

S: Se então dissesse: ``De modo algum, mas considero que aquele que
junto a mim aprender essas coisas, poderá fazer o que perguntas''.

 

[268c] F: Poderiam dizer, creio eu, que esse homem estivesse louco,
pois só por ter colhido de algum livro~ou por calhar de conhecer alguns
fármacos, considera"-se um médico, sem nenhum conhecimento da arte.

 

S: E o que diria, se alguém chegasse junto a Sófocles e Eurípides,
dizendo saber compor grandes falas sobre temas insignificantes e sobre
temas grandiosos falas curtas, e quando quisesse falas piedosas ou, ao
contrário, terríveis e ameaçadoras, e tantas outras desse tipo. E~ainda
que, com tais ensinamentos, considerava"-se capaz de transmitir a própria
arte da composição de tragédias?

 

[268d] F: E esses também, ó Sócrates, creio que ririam, se alguém
considerasse ser a tragédia outra coisa senão a adequação
(\emph{prépousan}) dos elementos entre si e com o todo (\emph{tôi
holôi}) da composição.

 

S: Penso eu que tais censuras não seriam grosseiras, tal qual o músico
que encontra uma pessoa que se considera um harmonista, só porque lhe
ocorreu aprender a fazer soar uma corda aguda e grave. O~músico não
seria grosseiro dizendo: ``Ó desgraçado, melancólico'', mas por ser
músico, gentilmente diria: ``Ó meu querido, para quem deseja vir a ser
um harmonista, esses conhecimentos são necessários, mas ninguém adentra
nem aprende o mínimo da harmonia, só por possuir essa tua disposição.
Conheces então os saberes prévios necessários à harmonia
(\emph{prò}~\emph{harmonías}), mas não a harmonia propriamente dita
(\emph{tà harmoniká})''.

 

F: Corretíssimo.

 

[269a] S: Então Sófocles e Eurípides mostrariam que aquilo era
apenas um rudimento para a composição da tragédia (\emph{prò
tragoidías),~}mas não a arte trágica propriamente dita (\emph{tà
tragiká),~}bem como Acúmeno diria que apresentavam os saberes prévios
necessários à medicina (\emph{prò}~\emph{iatrikês}), mas não detinham a
medicina mesma (\emph{tà iatriká}).

 

F: Sem dúvida.

 

S: E o que pensaremos de ``Adrasto voz de mel'' ou de Péricles, se
ouvissem o que agora mesmo percorremos sobre todas as artes, acerca dos
discursos breves (\emph{brachylogiôn})\emph{,~}dos estilos imagéticos
(\emph{eikonologiôn})\emph{~}e tantos outros a que chegamos, os quais já
mencionamos e foram verificados à luz do dia? [269b] Qual deles
seria o mais cruel, assim como eu e tu, ao falar mal, pela própria
rusticidade (\emph{hup' agroikías}), desses escritores e professores da
arte retórica? Ou, sendo mais sábios que nós, eles nos reprovariam
dizendo: ``Ó Fedro e Sócrates, não é preciso ser odioso para com eles,
mas desculpá"-los, se alguns que não conheceram a dialética se tornaram
incapazes de definir o que é a retórica, e por essa mesma condição
consideraram ter descoberto a arte retórica, quando apenas eram
detentores de conhecimentos prévios necessários à arte. [269c] Por
ensinarem tais coisas a outros, eles se consideram perfeitos professores
de retórica, embora dizer em cada ocasião o persuasivo e arranjar o todo
no discurso não lhes pareça um trabalho especial, de modo que seus
aprendizes precisam por si próprios adquirir essas habilidades nos
discursos''.

 

F: Ó Sócrates, isso parece ser aquela arte retórica que os homens
ensinaram e sobre a qual escreveram, e me parece ser proclamada
verdadeira. Mas a arte da retórica propriamente dita e sua
credibilidade, [269d] como e a partir de onde seria possível
alcançá"-la?

 

S: Para ser capaz, ó Fedro, de se tornar um competidor
(\emph{agonistḕn}) perfeito é verossímil (\emph{eikós}) -- talvez
necessário \mbox{---,} que detenhas outras coisas. Se está na tua natureza
(\emph{phýsei}) ser um retórico, serás um rétor consumado, acrescendo a
isso conhecimento (\emph{epistḗmen})\emph{~}e exercício
(\emph{melétên}). Se deixares de lado qualquer um desses elementos,
serás imperfeito.~Tal é a arte, a qual não me parece evidente atingi"-la
através do método de Lísias e de Trasímaco.

 

F: E por que modo então?

 

S: É possível, ó querido, que Péricles tenha se tornado de modo
verossímil (\emph{eikótôs}) o maior perito na arte retórica.

 

F: E por quê?

 

S: Todas as grandes artes requerem a tagarelice e a meteorologia
(\emph{adoleschías}~\emph{kai}~\emph{meteôrologías phýseos péri}) acerca
da natureza, pois parece que é justamente delas que se pode adquirir
sublime pensamento (\emph{hypselónoun}) e perfeição. Péricles foi capaz
de adquiri"-las, além da sua inclinação natural (\emph{euphyḕs}), e me
parece que envolto com Anaxágoras, pleno daquela meteorologia, chegou à
natureza do intelecto e da sua ausência (\emph{noû te kaì anoías}),~tal
qual Anaxágoras apresentava em muitos dos seus discursos, e a partir
disso Péricles forjou utilidades para sua arte discursiva
(\emph{enteûthen~eílkysen~epì tèn tôn lógôn téchnên tò prósphoron
autêi}).

 

F: Como dizes?

 

S: [270b] É que o recurso técnico (\emph{trópos téchnês}) da
medicina é similar ao da retórica.

 

F: Como assim?

 

S: Em ambas é preciso dividir a natureza (\emph{phýsin}), numa delas a
do corpo, noutra a da alma. Se pretendes, por um lado, fornecer
fármacos\emph{~}e alimento para a saúde e para a força, e, por outro,
discursos e estudos prescritos como úteis à persuasão desejada e à
virtude transmitida, isso não só aconteceria por treino e experiência,
mas por arte (\emph{mḕ~tribêi mónon kai empeiríai allà téchnêi}).

 

F: É verossímil (\emph{eikós}) que seja assim, ó Sócrates.

 

[270c] S: E tu consideras ser possível compreender o valor da
natureza da alma sem, contudo, compreender a natureza do todo (\emph{tês
toû hólou phýseôs})?

 

F: Se devemos acreditar em Hipócrates, que pertence aos Asclepíades,
quando diz que sem esse método nem mesmo o corpo seria possível
conhecer.

 

S: Belo dizer, ó companheiro, mas é necessário, todavia, examinarmos
isso em vista do discurso (\emph{tòn lógon}) de Hipócrates e
verificarmos se estamos de acordo.

 

F: É o que digo.

 

S: Verifica então agora o que diz Hipócrates e a verdadeira razão
(\emph{alêthḕs lógos}) acerca da natureza (\emph{perì phýseos}). Não é
preciso, para compreender a natureza de qualquer coisa, primeiramente
verificar se ela é simples ou de múltiplas formas, sobretudo se
desejaremos nós mesmos sermos os artífices (\emph{technikoì}) e capazes
de transmitir isso a outros, em seguida, se forem simples, verificar sua
potencialidade, saber em que tipo de relação sua natureza produz uma
determinada ação ou pelo que é afetada por algo externo. Se são
múltiplas as suas formas, estas devem ser também enumeradas e, tal qual
a unidade, devem ser observadas, cada uma delas, em que sua natureza
pode produzir ou ser afetada por algo externo?

 

F: É possível, ó Sócrates.

 

[270e] S: Sem esse método pareceríamos fazer uma travessia de cego
(\emph{typhloû poreíai}), mas não devemos comparar um cego ou um surdo
àquele que persegue com arte alguma coisa, como é evidente a quem
ofereça a arte discursiva e mostre, de forma acurada, a essência da
natureza para quem os discursos serão oferecidos. E~essa será, sem
dúvida, a alma.

 

F: Seguramente

 

[271a] S: Então, o combate se estende por toda alma, pois a
persuasão (\emph{peithṑ}) é produzida nela, ou não?

 

F: Sim.

 

S: É evidente que Trasímaco e tantos outros dos que zelosamente nos
legaram tratados da arte retórica, inicialmente e com muito cuidado
inscreverão e produzirão uma visão\emph{~}da alma, em seguida observavão
se sua natureza é una e semelhante, ou se, como o corpo, tem múltiplas
formas. Dizemos que mostrar a natureza de algo é isso.

 

F: É assim mesmo.

 

S: Em segundo lugar, é preciso entender o que sua natureza pode produzir
e pelo que é afetada.

 

 

F: Seguramente.

 

[271b] S: Em terceiro lugar, é preciso realizar uma disposição entre
gêneros de discursos e almas, bem como todas as causas que as afetam,
harmonizar cada qual ao seu correspondente e ensinar por quais causas,
necessariamente, alguns são persuadidos por determinados discursos e
outros não são persuadidos.

 

F: Parece que essa é a mais bela maneira de agir.

 

S: Não haverá então, ó querido, outra forma de demonstrar ou proferir
com arte, pela qual se dirá ou se escreverá algo, seja acerca desse ou
de outro tema. [271c] Os que hoje escrevem, como tu ouviste, são
hábeis em todas essas artes discursivas e as dissimulam, tendo muito
conhecimento acerca da alma. Antes de falarem e escreverem com tais
recursos, não nos deixemos persuadir de que eles escrevem com arte.

 

F: Quais são esses recursos?

 

S: Dizer com todas as letras não é fácil. Mas o modo pelo qual é preciso
escrever, se queres mesmo dominar a arte, tanto quanto se possa admitir,
acerca disso eu quero falar.

 

F: Diz então.

 

S: Uma vez que a potência do discurso está na condução das almas
(\emph{psychagôgía}),\emph{~}[271d] aquele que pretende ser rétor
deve necessariamente conhecer as formas (\emph{eíde}) que a alma tem. Há
tantas e tantos tipos de almas, motivo pelo qual existem umas de um modo
e outras de outro. Tendo assim realizado tal distinção, existem
discursos correspondentes para cada uma delas e cada uma tem o seu
específico. Por causa disso, umas [almas] são levadas à plena
persuasão (\emph{eupeitheîs}) pelo efeito do discurso, e outras, por
outro lado, pelo mesmo discurso, são levadas à desconfiança
(\emph{duspeitheîs}). É~preciso ter pensado (\emph{noḗsanta})
suficientemente nessas coisas, ter contemplado os próprios seres em ação
e ter praticado [271e], bem como ser capaz de acompanhar com agudeza
pela sensação tudo isso, ou não haverá plenitude dos saberes que outrora
ouviu nos discursos que trazia consigo. Quando for capaz de dizer como e
pelo que há persuasão, quando for capaz de, junto a alguém, perceber e
mostrar a si mesmo qual é a natureza [272a] acerca da qual versavam
os discursos de outrora, depois disso, junto a eles tendo trabalhado, é
preciso administrar os discursos pelos quais serão persuadidos. É~necessário ser detentor disso tudo, escolhendo também o momento oportuno
(\emph{kairós}) de falar e o de calar, os discursos breves
(\emph{brachylogias}),\emph{~}os discursos piedosos
(\emph{eleinologias}) e cada uma das formas dos discursos veementes
(\emph{deinṓseos})\emph{~}aprendidos, reconhecendo neles o momento
oportuno (\emph{eukairían}) e a falta de oportunidade (\emph{akairían}).
Bela e acabada estará, nesse ponto, a arte adquirida, antes disso não.
[272b] Mas se alguém deixar de lado qualquer um desses elementos ao
falar, ensinar ou escrever, dizendo que o faz com arte, não terá força
persuasiva. Daí talvez dissesse o escritor: ``O quê? Ó Fedro e Sócrates,
parece"-vos mesmo assim? Não há então outro modo de conceber a arte dos
discursos?''

 

F: Impossível, ó Sócrates, que seja de outro modo, e não me parece uma
tarefa de pouca monta.

 

S: Dizes a verdade. E, em consequência disso, é necessário percorrer de
cima abaixo todos os discursos para verificar se há, em alguma parte, um
caminho mais fácil e curto, [272c] a fim de que este não seja vão e
muito árduo, mas, se possível, curto e suave. Se em alguma ocasião
tiveste o auxílio da audição de Lísias ou de algum outro, procura
lembrar"-te e diz.

 

F: Eu poderia até tentar, mas assim, agora, não posso.

 

S: Queres que eu te exponha um discurso que ouvi de alguns, acerca desse
tema?

 

F: Como não?

 

S: Dizem, ó Fedro, que é justo mencionar também a razão do lobo.

 

[272d] F: Então faça isso.

 

S: Dizem alguns que não é preciso honrar tanto essas coisas, nem se
elevar tanto a altos rodeios. Foi perfeito o que dissemos no início
dessa discussão, que aquele que pretende ser um rétor pleno, não precisa
participar da verdade (\emph{alêtheías}), nem da justiça
(\emph{dikaíôn}), nem da bondade (\emph{agathôn}) acerca daquilo que
trata, quer tenham os homens tais naturezas, quer as tenham adquirido
pela educação. Ocorre que nos tribunais ninguém se preocupa com a
verdade (\emph{alêtheías}), mas com o persuasivo (\emph{pithanoû}), isto
é, com o verossímil (\emph{eikós}), questão na qual é preciso aplicar"-se
quem pretenda falar com arte. [272e] Algumas vezes, nem mesmo os
fatos ocorridos devem ser mencionados, caso não sejam verossímeis
(\emph{eikótôs}), mas somente as coisas verossímeis (\emph{eikóta}),
seja na acusação seja na defesa, bem como tudo aquilo que se diz de
forma verossímil (\emph{eikòs}) é necessário perseguir, motivo pelo qual
muitas vezes é preciso renunciar à verdade (\emph{alêtheî}). [273a]
O verossímil surge ao longo dos discursos e proporciona toda a arte
(\emph{téchnên}).

 

F: Detalhaste com propriedade, ó Sócrates, os dizeres daqueles que
professam serem detentores dessa arte dos discursos. Recordo"-me que, no
começo dessa discussão, tocamos rapidamente nesse tema, o qual parece
ser muito importante para aqueles que se ocupam disso.

 

S: Mas tu certamente tens degustado bem do seu Tísias. [273b] Pois
então ele que nos diga, também, se o verossímil (\emph{tò eikòs}) é
outra coisa senão a opinião da maioria (\emph{tôi plḗthei dokoûn}).

 

F: E o que mais seria?

 

S: Como parece, esse sábio artista aqui encontrado (\emph{heurṑn})
escreveu que, se por acaso, um homem fraco e corajoso assaltasse um
homem forte e covarde, levando sua toga e outros pertences, ao serem
ambos levados ao tribunal, seria necessário que nenhum deles dissesse a
verdade, uma vez que o covarde diria que não havia sido assaltado
unicamente pelo corajoso, o qual por sua vez diria que estava sozinho,
[273c] usando aquele argumento: ``Como eu, sendo assim fraco,
poderia ter executado o assalto contra ele que é forte?'' E o outro não
expressaria sua própria maldade, mas usaria alguma outra mentira para,
rapidamente, oferecer refutação ao oponente. E~acerca de outras tantas
coisas dessas, também ligadas à arte, não dizemos que é assim, ó Fedro?

 

F: Como não?

 

S: Ah! Parece"-me terrivelmente escondida essa arte encontrada
(\emph{aneureîn}) por Tísias ou por outro qualquer, bem como o nome pelo
qual ela pode ser designada. Mas, ó companheiro, qual deles, dizemos ou
não --.

 

[273d] F: O qual?

 

S: Ó Tísias, antes de ti, os mais antigos já haviam dito que o
verossímil surge para a multidão pela semelhança que ele tem com o
verdadeiro (\emph{to eikòs toîs polloîs dia homoiótêta toû alethoûs
tygchánei eggignómenon}). E~as semelhanças (\emph{homoiótêta}), como
dissemos há pouco, são encontradas (\emph{heurískein}) em toda parte e
da melhor maneira possível por aquele que conhece a verdade. Então, se
tens algo mais acerca da arte dos discursos, diz, pois te escutaremos,
caso contrário chegaremos ao mesmo ponto onde estávamos persuadidos
[273e]. Se alguém não enumera a natureza dos seus ouvintes, discerne
os seres segundo sua forma, não é capaz de levá"-los a uma só ideia,
abarcando cada uma delas, e não será um artista do discurso, tanto
quanto isso é possível ao homem. Essas coisas não se podem adquirir sem
muito empenho (\emph{pragmateías}). E~não é por causa do falar e do agir
com outros que o homem moderado precisa cultivar sua prudência
(\emph{sṓphrona}), mas para poder agradar aos deuses, ao falar, ao agir
e em tudo que seja possível. Não é assim, ó Tísias? Dizem"-nos os mais
sábios que é necessário seguir e agradar o que possui intelecto
(\emph{tòn noûn échonta}), [274a] porque esse não é simples
acessório, mas senhor bondoso em tudo aquilo que é bom. Se essa estrada
é longa, não te espantes, pois é pelas grandes coisas que se faz esse
percurso, ao contrário do que tu supões. Como o discurso já afirmou, se
for algo que desejas, o que surge desse trajeto será belíssimo.

 

F: Perece"-me que afirmas algo magnífico, ó Sócrates, se for assim mesmo.

 

S: É belo ocupar"-se das coisas belas, bem como suportar aquilo que lhes
advém.

 

F: É certo.

 

[274b] S: Então, acerca da arte e da sua ausência (\emph{téchnês te
kai atechnías}) nos discursos, dissemos o suficiente.

 

F: E o que mais haveria?

 

S: E acerca da conveniência ou inconveniência da escrita, como ela pode
ser bela ou inconveniente, omitimos isso?

 

F: Sim.

 

S: Conheces, pois, o meio pelo qual em matéria de discursos, devemos
agradar aos deuses, agindo ou falando?

 

F: Não conheço, e tu?

 

[274c] S: Escuta o que posso te contar dos antigos, pois eles
conheciam a verdade. Se nós a descobríssemos (\emph{heuroimen}), não nos
preocuparíamos com a opinião dos homens, não é mesmo?

 

F: Engraçada tua pergunta, mas conta o que ouviste.

 

S: Escutei que perto de Náucratis, no Egito, existia um dos deuses
antigos, cujo pássaro sagrado era chamado Íbis, o nome
desse~\emph{daímon}~era Theuth. Ele foi o primeiro a inventar
(\emph{heureîn}) os números, o cálculo, a geometria, a astronomia, o
jogo de tabuleiro, o jogo de dados e especialmente a escrita
(\emph{grámmata}).~[274d] O rei de todo Egito nessa época era
Thamous, que vivia na grande cidade alta, a qual os gregos chamavam de
Tebas egípcia e onde o mesmo Thamous Amon era o deus.~Theuth veio junto
ao rei para mostrar"-lhe suas artes, que segundo ele deveriam ser
presenteadas a todos os egípcios. Indagado, então, acerca das utilidades
(\emph{ôphelían}) de cada uma delas, ele as expunha, de modo que o rei
dizia o que parecia, aos seus olhos, ser belo ou feio em cada uma, ora
elogiando ora vituperando.~[274e] Muitas foram as artes para as
quais Thamous apresentou seu comentário a Theuth, e todo o seu discurso
seria muito longo para referi"-lo aqui, mas acerca da escrita, foi assim:
``Ó rei, disse Theuth, esse conhecimento tornará os egípcios mais sábios
e com maior disposição para a memória. Foi inventado (\emph{heuréthê})
então o fármaco da memória e da sabedoria``. Ao que o rei replicou: ''Ó
artificiosíssimo Theuth, enquanto uns são capazes de criar uma arte,
outros são capazes de julgá"-la, especialmente em que aspectos elas serão
nocivas ou úteis para quem poderá usá"-las. [275a] Agora, aqui, tu,
como pai da escrita que és, por tua benevolência para com ela, dizes o
contrário do que ela é capaz. Ela produz esquecimento nas almas daqueles
que aprendem pela falta de cuidado com a memória, sendo então por
escritos externos e alheios que adquirem a crença
(\emph{pístis})\emph{,~}não adquirindo mais a reminiscência internamente
e por si mesmos. Portanto, não inventaste (\emph{hêures}) o fármaco da
memória (\emph{mnḗmês})\emph{,~}mas o da recordação
(\emph{hypomnḗseos}). Ela oferece uma aparente sabedoria aos discípulos,
que não alcançam a verdade propriamente dita. [275b] Muitos dos teus
ouvintes, sem aprendizado, parecem conhecedores de muitas coisas, quando
na verdade são geralmente ignorantes e difíceis no trato, tornando"-se
aparentemente sábios (\emph{doxósophoi}) em vez de sábios
(\emph{sophôn})''.

 

F: Ó Sócrates, que facilidade tens para apresentar histórias egípcias e
de qualquer lugar que queiras.

 

S: Ó amigo, dizem que os antigos discursos divinatórios provinham de um
carvalho situado no templo de Zeus em Dodona. Naquele tempo os homens
não eram tão sábios quanto vós, os jovens, motivo pelo qual lhes
bastava, devido à sua simplicidade, ouvir um carvalho ou uma pedra,
desde que estes lhes dissessem somente a verdade. [275c] Tu talvez
possas discernir qual é o discurso e de onde ele provém. E~não observes
somente se é assim ou não.

F: Correta é a tua repreensão e me parece que, acerca dos escritos,
ocorre o que tebano já havia afirmado.

 

S: Tanto aquele que supõe deixar alguma arte por meio da escrita
(\emph{en grámmasi}), quanto aquele que espera recebê"-la por esse meio,
ambos consideram que a escrita (\emph{grammátôn}) porta algo de claro e
seguro, o que é muita ingenuidade e prova de desconhecimento
(\emph{agnooî}) do oráculo de Amon, [275d] segundo o qual os
discursos escritos (\emph{lógos gegramménous}) nada mais são do que um
meio de recordar (\emph{hypomnêsai}) aquele que já conhece
(\emph{eidóta}) os assuntos tratados nos escritos (\emph{gegramména}).

 

F: Corretíssimo.

 

S: É terrível mesmo, ó Fedro, essa escrita (\emph{graphḕ}) e como tem
verdadeira semelhança (\emph{hómoion}) com a pintura
(\emph{zôgraphíai}). Os frutos desta são estabelecidos como vivos, mas
se lhe perguntas algo, ela permanece sempre num silêncio sagrado
(\emph{semnôs pánu sigâi}), e assim também acontece com os discursos
(\emph{oi lógoi}). Eles parecem dizer algo de sensato, mas, se alguém
que deseja aprender lhes pergunta algo sobre o que foi dito, eles só
significam a mesma coisa sempre. [275e] E a grafia (\emph{graphêi})
roda por todo lado conservando o mesmo discurso, seja para os que a
elogiam, seja para os que nela não têm nenhum interesse, pois ela não
conhece o momento de falar ou de calar. E~se ela for atacada num
tribunal, sempre haverá a necessidade que o seu pai a socorra
(\emph{boêthoû}) das injúrias, pois ela não é capaz de defender ou
socorrer (\emph{amúnasthai oute boêthêsai}) a si mesma.

 

F: Também isso que dizes é corretíssimo.

 

[276a] S: O quê? Dizemos que há outro discurso, irmão legítimo
deste, mas surgido por outro modo, melhor quanto à natureza e mais
poderoso?

 

F: Sobre qual discurso te referes e como ele surge?

 

S: Sobre aquele que é inscrito na alma (\emph{gráphetai en têi psychêi})
daquele que aprende, segundo o conhecimento (\emph{met'epistḗme}), ele é
capaz de defender (\emph{amûnai}) a si mesmo, conhecedor da ocasião
frente a qual é preciso falar (\emph{légein}) ou calar (\emph{sigân}).

 

F: O discurso de quem efetivamente sabe, ao qual te referes, é vivo e
animado (\emph{zônta kaì empsuchon}), de modo que o discurso escrito
(\emph{gegramménos}), poderíamos dizer com justiça, é um ídolo
(\emph{eídôlon}) seu.

 

[276b] S: É assim mesmo, agora me diz, quanto ao agricultor
(\emph{geôrgós}) que tem inteligência (\emph{noûn}) e deseja cuidar das
suas sementes para que frutifiquem, o que ele faria? Haveria de
lançá"-las, durante o verão, no jardim de Adônis (\emph{Adónidos
kêpous}), para homenagear a sua festa, para que floresçam em oito dias?
Ou isso ele poderia fazer só por brincadeira (\emph{paidiâs}) e
exclusivamente de bom grado para o festival, quando muito. Ou, quanto às
sementes que ele realmente despende atenção, valendo"-se da arte da
agricultura (\emph{georgikêi}), ele semearia em local adequado,
felicitando"-se em oito meses, quando as sementes atingem sua maturidade?

 

[276c] F: Ó Sócrates, num caso ele faria com atenção, no outro não,
como tu dizes.

 

S: O que dizemos daquele que tem conhecimento do justo, do belo e do
bom? Que ele tem menor inteligência (\emph{noûn}) que a do agricultor,
com relação às suas sementes?

 

F: De modo algum.

 

S: Então não vai cuidadosamente escrevê"-las na água escura com uma pena,
compondo discursos incapazes de socorrerem"-se (\emph{boêtheîn}) a si
mesmos, insuficientes para ensinar a verdade (\emph{adunátôn dè ikanôs
talêthḕs didáxai}).

 

F: Não é mesmo verossímil (\emph{eikós}).

 

[276d] S: Não? Mas nos jardins da escritura (\emph{en grámmasi
kḗpous}), como parece (\emph{eoike}), semeiam e escrevem pelo deleite da
brincadeira (\emph{paidiâs chárin spereî te kaì grápsei}), e quando
escrevem entesouram recordações (\emph{hypomnḗmata}) de si mesmos, para
o ``oblívio da velhice'' (\emph{lḗthes}~\emph{gêras}), se ela
``chegar''. E~todos que buscam seguir seus passos~serão agraciados pela
contemplação dessas delicadas plantas. Por outro lado, quando outros se
valem de outras diversões, bebendo nos simpósios, entregues a prazeres
similares a este, e, como parece, divertir"-se-ão exatamente com as
coisas referidas.

 

[276e] F: Boa diversão frente àquela frívola, ó Sócrates, essa de
poder brincar (\emph{paídzein}) com os discursos, sejam eles judiciais
ou outros que dizes nos quais possamos também narrar
(\emph{mythologoûnta})

 

S: É assim, ó querido Fedro, considero muito mais belo o empenho daquele
que pela arte da dialética toma uma alma para plantar e nela semear
discursos com conhecimento (\emph{met'epistḗmês}), [277a] aqueles
que são capazes de socorrer (\emph{boêtheîn}) quem os plantou. Então, os
discursos não são infrutíferos, mas têm sementes, pelas quais outros em
outros lugares se habituarão a crescer, tornando"-as sempre imortais o
bastante, tornando felizes os homens, tanto quanto possível.

 

F: Muito mais belo é o que dizes agora.

 

S: Agora que chegamos a esse acordo, ó Fedro, somos então capazes de
julgar (\emph{krínein}).

 

F: Julgar o quê?

 

S: Aquilo que queríamos saber e que nos trouxe até aqui, justamente para
que pudéssemos examinar a censura endereçada a Lísias pelos seus
discursos escritos (\emph{tês tôn lógôn graphês péri}), [277b] e
para examinarmos os próprios discursos, se haviam sido escritos com arte
(\emph{téchnêi}) ou sem arte (\emph{aneu~techné}). Os que estão de
acordo com a arte (\emph{éntechnon}) parecem"-me terem sido expostos de
modo bem medido (\emph{metríôs})\emph{.}

 

F: Parece mesmo. Mas recorda"-me (\emph{hypómnêson}) novamente
(\emph{palin}) como.

 

S: Antes, devemos saber a verdade acerca de cada coisa sobre o que se
fala e escreve, tudo deve poder ser definido por si mesmo, e uma vez
definido, devemos conhecer como dividi"-lo (\emph{témnein}) novamente
(\emph{pálin}) até a forma indivisível (\emph{toû atmétou}). E~a
respeito da natureza da alma, que se distinga tudo da mesma forma,
[277c] descobrindo (\emph{aneurískôn}) a forma natural que se
harmoniza com cada uma delas, para então estabelecer e ordenar o
discurso. Um discurso variegado é oferecido para uma alma variegada, um
simples para uma alma simples, antes disso não é possível haver um
gênero discursivo que faça uso natural da arte, nem para ensinar nem
para persuadir, como nos foi revelado pelo discurso anterior.

 

F: É tudo mesmo dessa forma, tal qual nos pareceu.

 

S: E a respeito do falar e do escrever discursos ser algo belo ou
vergonhoso, e de quando é possível dizer, com justiça, o que é
vergonhoso ou não. O~que há pouco foi dito não ficou bem claro?

 

F: O quê?

 

S: Que Lísias ou qualquer outro que tenha escrito ou venha a escrever
leis particulares ou públicas, quando consideram o tratado escrito sobre
política algo grandioso, estável e claro, é nesse momento que eles podem
se envergonhar dos discursos, quer isso seja mencionado ou não. O~fato
de alguém ignorar, sob o efeito do sono, [277e] o justo e o injusto,
o mau e o bom, não pode livrá"-lo da verdade de ser censurado, ainda que
toda a turba o elogie.

 

F: Não mesmo.

 

S: É necessário que haja muito divertimento (\emph{paidián}) em cada um
desses discursos escritos, e que nenhum deles, em metro ou sem, mereça
grande esforço para ser escrito, ou mesmo lido como fazem os rapsodos,
sem preparo ou didática naquilo que é dito para persuadir. [278a] Os
melhores entre eles são os que, pela recordação (\emph{hypómnêsin}),
levam ao saber. Por outro lado, os que são feitos para ensinar,
discursos que agradam ao aluno e inscrevem na alma (\emph{graphoménois
en psychêi}) algo acerca do justo, do belo e do bom, somente estes são
visíveis, acabados e dignos de esforço. É~preciso que tais discursos
sejam enunciados como filhos legítimos, [278b] primeiro por eles
mesmos, se eles os descobrirem (\emph{heuretheìs}) em si, e, em seguida,
se alguns desses seus descendentes e irmãos plantam concomitantemente em
outras almas, em outros lugares, de acordo com a dignidade. Quanto a
outros discursos, é melhor afastar"-nos deles, ó Fedro, pois essa é a
atitude do homem que ambos, eu e tu, gostaríamos de ser.

 

F: Quanto a mim, desejo e faço votos para que seja assim, tudo da
maneira que dizes.

 

S: Então nós já nos divertimos (\emph{pepaísthô}) o bastante
(\emph{metríôs}) acerca dos discursos, e tu vai até Lísias e diz a ele
que nós dois descemos até a fonte das ninfas e ao santuário das Musas e
que escutamos um discurso [278c] para ser enviado a Lísias e para
qualquer outro que componha discursos, a Homero e a qualquer outro que
tenha composto poesia com ou sem acompanhamento musical (\emph{ôidêi}),
e em terceiro lugar a Sólon e aos que escreveram discursos políticos,
tratados que foram chamados de leis escritas: ``Se conheces a verdade
daquilo que está composto nesse escrito e és capaz de socorrê"-lo
(\emph{boêtheîn}), nas refutações que lhes são endereçadas, e ainda és
capaz de mostrar o que é ineficiente no teu próprio escrito, então, na
verdade, pelo qual epônimo deverá ser designado, por esta atividade de
escrever ou por aquela atividade a qual se dedicou?''

 

F: Qual dos epônimos tu atribuis a ele?

 

S: O de sábio, ó Fedro, acredito parecer demasiado, conveniente somente
a um deus. O~de filósofo ou outro desse tipo poderia ser mais ajustado e
adequado.

 

F: E de nenhum modo inapropriado.

 

S: Aquele que não tem, por outro lado, nada de mais honrado
(\emph{timiṓtera}) do que aquilo que outrora escreveu e passa o tempo a
percorrer (\emph{stréphôn}) seus escritos de cima abaixo, separando
trechos e trocando"-os de lugar, [278e] é com justiça que o
designarás por poeta, compositor de discursos ou escritor de leis?

 

F: É certo.

 

S: E é isso mesmo que deves dizer ao teu companheiro.

 

F: E tu? Como farás? Não deves pôr de lado o teu companheiro.

 

S: Qual deles?

 

F: O belo Isócrates.~O que dirás a ele, ó Sócrates, e nós diremos o quê?

 

S: Isócrates é jovem ainda, ó Fedro, [279a] entretanto adivinho algo
sobre ele e quero dizer.

 

F: O quê?

 

S: Parece"-me que ele é superior a Lísias quanto à natureza de seus
discursos, e ainda temperado por um caráter mais nobre (\emph{ḗthei
gennikôtéroi}), de modo que não seria espantoso se, com a idade, ele
superasse nessa prática os que hoje em dia se ocupam disso, tornando
infantis os que sempre se ocuparam de discursos. Se isso ainda não for
suficiente, ele será guiado por um impulso maior e mais divino. Pois há,
ó querido, certa filosofia no intelecto desse homem. [279b] É isso,
então, que eu vou, junto aos deuses, anunciar a Isócrates, o meu
favorito, e tu, por sua parte, faça o mesmo ao teu Lísias.

 

F: Assim será, partamos agora que o calor se tornou ameno.

 

S: Não é adequado fazermos uma prece antes de partir?

 

F: Sim, é.

 

S: Ó querido Pã e outros deuses, concedam"-me uma beleza interior. Que
tudo que há fora de mim possa ser amigo do que está no meu interior
[279c]~e que eu considere rico o sábio. Quanto à quantidade de ouro,
que eu possua tanto quanto o homem prudente seja capaz de levar e
trazer. Precisamos de algo mais, ó Fedro? Pois me parece bem medida
(\emph{metríôs}) a prece.

 

F: E eu partilho dessa súplica, pois tudo é comum entre amigos.

 

S: Partamos.

 
