\chapterspecial{Introdução}{Qual é o verdadeiro tema do Fedro?~\footnote{Tomo aqui como
  parâmetro um dos títulos empregados por Hermias: ``Qual é o verdadeiro
  escopo do diálogo (\emph{Tís ho alêthès toû dialόgou skopόs})'' (Herm.
  \emph{In Phdr} 63.26 Couvreur = 10, 24 Lucarini e Moreschini).}}{}
%Qual é o verdadeiro tema do Fedro?

\epigraph{[Apolo a seu irmão Hermes] Dar-te-ei o esplêndido caduceu da felicidade e da riqueza, dourado, de
três pétalas, ele te protegerá e tornará imune, acompanhando as
palavras e ações dos homens bons, as quais afirmo conhecer da voz de
Zeus.}{(\emph{Hino Homérico a Hermes,} v. 529-532)} 


\section{Eros e Psique}
 

O \emph{Fedro} pode ser entendido como uma interpretação filosófica de
diversos mitos, mas entre eles o mito de Eros e Psique parece estar à
frente, não apenas do ponto de vista teológico, mas também do ponto de
vista lógico, retórico e dialético. Tudo isso se entrelaça na natureza
do deus em questão. O diálogo propõe uma aplicação filosófica
do \emph{lógos} a partir da tensão natural entre Eros e Psique, tensão
que é responsável por conduzir (\emph{movere}) a alma, os discursos e as almas
pelos discursos. Não é irrelevante que a psicagogia, ou
literalmente a ``condução das almas'', tenha sido abordada amplamente em
diversos estudos como um dos temas mais importantes do diálogo (Moss, 2012), seja através de cantos, seja através de discursos proferidos, de
discursos escritos, das imagens (\emph{eikôna}), pois tudo isso tem
poder psicagógico nesse retrato complexo.

Apesar de a fonte de transmissão mais antiga acerca desse mito ser
Apuleio, seria plausível pensar que a narrativa (\emph{mythos}) acerca de Eros e
Psique estivesse presente na mentalidade e na oralidade já há tempos.
Grimal (1997, p.~148-149) a localiza nas fábulas milésias e
mostra também que sua iconografia está impregnada em toda arte romana da
época de Apuleio (séc. \versal{II} d.C.). O mito remonta também a padrões
mitológicos comuns, como o fato de Eros ser retratado como um deus psicopompo,
literalmente, condutor das almas. Nele há também um casamento entre
mortal e divindade, cujos exemplos são abundantes -- Teseu e Tétis,
Dioniso e Ariadne, Afrodite e Adônis --, sendo este último talvez o mais
próximo ao ambiente mitológico do \emph{Fedro}, no qual as festas de
Adônis são citadas, bem como seus jardins comemorativos
(\emph{Phdr}. 276b). Esses jardins de Adônis são emblemáticos por serem
semeados em uma semana, para decorar a festa, numa imagem da
efemeridade, pois Adônis acompanha sempre Perséfone em seu ciclo natural de
inverno e verão, morte e vida.

A semeadura na alma, a inscrição educativa em Platão, seria o longo
caminho do aprendizado, não o atalho, mas uma imagem da imortalidade e
da permanência da alma e da sua memória. Não é possível detectar uma
versão acabada desse mito antes de Apuleio, entretanto, em Eurípides, esse
mesmo Eros condutor (\emph{eiságôn}) das almas aparece mobilizado pela
visão:

 

\begin{quote}
\begin{verse}
\{\gr{Χο}.\} \gr{Ἔρως} \gr{Ἔρως}, \gr{ὁ} \gr{κατ}' \gr{ὀμμάτων}\\
\gr{στάζων} \gr{πόθον}, \gr{εἰσάγων} \gr{γλυκεῖαν}\\
\gr{ψυχᾶι} \gr{χάριν} \gr{οὓς} \gr{ἐπιστρατεύσηι},\\
\gr{μή} \gr{μοί} \gr{ποτε} \gr{σὺν} \gr{κακῶι} \gr{φανείης}\\
\gr{μηδ}' \gr{ἄρρυθμος} \gr{ἔλθοις}.\\


\bigskip


\{Co.\} Eros, Eros, o que pela visão\\
mobiliza o desejo, doce condutor\\
gracioso das almas para o combate,\\
que a mim não me apareças com um mau,\\
nem tampouco provoques arritmia.\\

\hfill (Eurípides, \emph{Hipólito} 525-529)
\end{verse}
\end{quote}

 

É notável a proximidade entre Eurípides e Platão. Os versos do coro em
Eurípides funcionam como um exemplo das afecções provocadas pela visão
em Platão, contexto em que a reciprocidade entre amante e amado estimula
o \emph{hímeros}, o fluxo do desejo, e irriga plenamente as asas da
alma, sendo esse o melhor e o mais lícito dos desejos. Ainda em
Eurípides, temos um exemplo adicional da ligação Eros/\allowbreak{}Psique:

 

\begin{quote}
\begin{verse}
\gr{Ἔρωτα}, \gr{πάντων} \gr{δυσμαχώτατον} \gr{θεόν}.\\
\gr{Ἔρως} \gr{γὰρ} \gr{ἄνδρας} \gr{οὐ} \gr{μόνους} \gr{ἐπέρχεται}\\
\gr{οὐδ}' \gr{αὖ} \gr{γυναῖκας}, \gr{ἀλλὰ} \gr{καὶ} \gr{θεῶν} \gr{ἄνω}\\
\gr{ψυχὰς} \gr{χαράσσει} \gr{κἀπὶ} \gr{πόντον} \gr{ἔρχεται}.\\


\bigskip


Eros, o deus mais invencível de todos.\\
Pois Eros não chega só para os homens,\\
nem só para as mulheres, mas atinge também\\*
as almas dos elevados deuses, bem como as conduz ao mar.\\

 

\hfill (Eurípides, \emph{Frg. }431)
\end{verse}
\end{quote}

 

Em Anacreonte, também a imagem de um jovem, bem como seu olhar puro,
mobilizam um amante a ponto de ele considerar o amado um auriga, ou
seja, o condutor de seu impulso amoroso. Há uma relação entre o olhar
puro do amado, sua figura e a afecção do amante, que descreve justamente
esse conduzir da alma:

 

\begin{quote}
\begin{verse}
\gr{ὦ} \gr{παῖ} \gr{παρθένιον} \gr{βλέπων}\\
\gr{δίζημαί} \gr{σε}, \gr{σὺ} \gr{δ}' \gr{οὐ} \gr{κλύεις},\\
\gr{οὐκ} \gr{εἰδὼς} \gr{ὅτι} \gr{τῆς} \gr{ἐμῆς}\\
\gr{ψυχῆς} \gr{ἡνιοχεύεις}.\\


\bigskip 

Ó jóvem de olhos puros,\\
busco-te, mas tu não ouves,\\
não sabes que de minha\\
alma és o auriga.\\

 

\hfill (Anacreonte, \emph{Frg}. 4 [15])
\end{verse}
\end{quote}

 

Aqui a relação se torna especial, pois no \emph{Fedro} a imagem
escolhida para ilustrar a alma divina e humana é justamente a da
carruagem, meio pela qual a alma é apresentada em sua tripartição, sendo
o auriga o intelecto (parte lógica). Nesses versos de Anacreonte, a
afecção amorosa faz com que o amado seja agora um auriga externo à alma
do amante. Em Platão esse mesmo tipo de condução é descrita, mas, ao
contrário de Anacreonte, a psicagogia é explicada em dupla perspectiva,
da ação e da paixão. Platão pretende que a alma, mais ou menos capaz de
administrar os impulsos que nela surgem, conheça o modo pelo qual os
estímulos a afetam, sejam eles visuais ou discursivos. Reconhecer essas
afecções é estar cada vez mais preparada (iniciada) para tantos e tão
violentos ataques. Menandro, que apresenta diversas modalidades do
gênero epidíctico, cita Platão e sua abordagem acerca de Eros. Segundo
ele, essa exibição (\emph{epideixis}) é do tipo fisiológica, porque Eros
é uma afecção da alma (\emph{páthos estì tês psuchês ho Erôs})
(Men. \emph{Rh}. 337, 6-7).

A imagem do alimento para um animal conduzido é uma forte metáfora desse
poder psicagógico do discurso no \emph{Fedro}:

 

\begin{quote}
Sócrates: \redondo{[…]} Tal como os que agitam um ramo para uma
criatura faminta, ou algum fruto que os conduza, tu, do mesmo modo,
estendendo discursos provenientes de livros, parece que me conduzirás
por toda a Ática ou para qualquer outro lugar que queiras.
(Pl. \emph{Phdr}. 230d6-e1).
\end{quote}

 

Não vou me alongar nas evidências da relação entre Eros e Psique
paralelas ao \emph{Fedro}, e passarei prontamente à dicotomia que Platão
apresenta, não só no \emph{Fedro}, mas também
no \emph{Banquete }(180c-181d), entre dois tipos de amores. Quanto aos
dois Eros, um deles é vulgar e o outro é nobre, assim como há uma maneira
vulgar de usar a linguagem, a retórica e a logografia, e uma nobre, a
filosofia. Tal qual no discurso de Pausânias no \emph{Banquete},
no \emph{Fedro} é evidente também que concorrem esses mesmos dois Eros,
ou, na linguagem de Pausânias, um Eros Pandêmio e um Urânio, filhos de
duas Afrodites distintas.

Esses dois amores estão atrelados a dois tipos de conduta e compreensão
do mundo, uma conduta ímpia, do Amor pragmático que visa estender ao
máximo os prazeres e evitar seus efeitos colaterais, evidenciados como
puramente nocivos, e do Amor lícito, ligado naturalmente à loucura
erótica, que é mais saudável e nobre ao presidir as relações
intelectuais, pedagógicas e psicagógicas. Esse amor é o mais saudável
porque considera a reciprocidade entre amante e amado dentro de
parâmetros da educação grega, sem desconsiderar ou rechaçar um deus tão
importante como Eros, ou seja, sem cometer qualquer falta teológica e
ética. Nesse sentido, é necessário entender que a imagem de Eros que
Platão reabilita tanto no \emph{Banquete} quanto no \emph{Fedro} é a do
deus intermediário entre deuses e homens, um \emph{daímon} responsável
por todo tipo de mescla. Esse Eros platônico não está distante do Eros
hesiódico, também responsável pela mistura cósmica
(Hes. \emph{Th}. 116-120)\footnote{Cf. Torrano, JAA, especialmente sua
  introdução à \emph{Teogonia} de Hesíodo, São Paulo: Iluminuras, 1995,
  p. 39-48.}. Então, desde esse prisma, talvez o mais importante tema
do \emph{Fedro} seja mesmo Eros em suas
múltiplas acepções, especialmente em sua relação com a Alma, e essa
tensão que se revela no \emph{lógos}.

Na tensão e mescla de Eros, Platão descreve a natureza tripartida da
alma, seu trajeto, suas possibilidades e especialmente as maneiras pelas
quais ela se move, bem como é capaz de, manejando o \emph{lógos}, mover
outras almas lícita ou ilicitamente. Atraída por diversos impulsos de
diversos matizes a alma deseja, seja em modalidades nobres ou
vulgares, por isso o \emph{lógos}, não só em Górgias, mas também em
Platão, é fonte de inúmeros deleites e encantos. Tudo isso por Eros ser
duplo ou ter essa natureza dúbia, e propiciar o fluxo do \emph{lógos}.
Eros ``doce e amargo'' (\emph{glukúpikron}), eternizado por Safo e
reafirmado por Luciano quanto à duplicidade: ``duplo deus é Eros
(\emph{diploûs theòs ho Érôs})''\footnote{Luciano Samósata,
  \emph{Amores} 37,7.}, de modo que Eros unifica o diverso por
ser duplo. Há uma proximidade entre Eros, Phanes e Protogonos, pelo que
fica claro que o Platão assimilou muitos elementos da cosmogonia órfica
na sua filosofia, e a interpretação dessa assimilação se acentua no
neoplatonismo\footnote{Cf. Kern, O. \emph{Orphicorum Fragmenta},
  Zurich: Weidmann,1972; Sorel, R. \emph{Les cosmogonies grecques},
  Paris: PUF, 1994; Barnabé, A. \emph{Platón y el orfismo}, Madrid:
  Agapea, 1990.}.

Eros e Psique são parâmetros hermenêuticos do diálogo. Eros manifesta
dois impulsos distintos ligados a essa força cósmica que permanece dupla
sempre, e a alma pode estar preparada para esse ataque ou não, do ponto
de vista dialético. A questão da arte discursiva se apresenta em três
níveis: (a) primeiramente, a logografia é exposta, nas
três \emph{epideixis} (recitações); (b) depois, há uma reflexão
bastante controlada acerca da história da retórica, da arte da palavra,
de seus principais inventores e inventos; (c) uma apresentação
bastante diligente da superioridade da dialética frente à retórica
vulgar. Nesse trajeto, torna-se possível enunciar uma retórica
filosófica, sem nenhum equívoco, superior à arte da palavra praticada
pelos logógrafos e sofistas, pois o foco de Platão passa a ser a arte de
pensar e falar, em seus fundamentos lógicos, a arte de reunir e separar
no pensamento e no discurso, de acordo com categorias, e bem aplicá-las
nas almas através do discurso (\emph{lógos}). A dialética torna possível
observar quando somos enganados pelos artífices de uma certa persuasão,
homens que, com seu discurso, frequentam assiduamente e querem nos fazer
frequentar o lado obscuro do não ser, bem como torna possível falar,
escrever e agir de acordo com parâmetros éticos e filosóficos. Platão
propõe fazer soar melhor esse \emph{lógos}, em sua face mais brilhante.

Platão, nesse percurso longo, mostra o fundamento do verossímil,
atrelado à reputação do ditado Lacônico, segundo o qual toda mentira
está, necessariamente, alicerçada em alguma verdade. O verossímil se
constrói pela semelhança (\emph{homoiosis}) com o verdadeiro, de modo
que não há mentira que não esteja, ainda que parcialmente, alicerçada em
algo verdadeiro. O dialético observa como se deriva da parte um todo, do
ser um não ser, e, a partir dessa indução, como atribuem-lhe um nome
qualquer, bonito ou feio.

Na imagem da mudança desse \emph{lógos}, desse \emph{nous}, está o
aspecto ético do \emph{Fedro}. Nesse efeito protréptico, de conversão,
está também a imagem do cessar do fluxo do desejo, do fazer parar, do
interromper um determinado fluxo. Essa interrupção, essa pausa, esse
autodomínio, que vem do estancar fluxos do corpo na medicina, aqui se
aplica à contensão e subsequente conversão dos impulsos da alma na
direção considerada mais elevada.

As cigarras interpelam Sócrates e Fedro acerca do impulso intelectual
rumo à dialética, pois o fato de não estarem encantados com as cigarras interrompe
algo, faz com que se encantem por outra coisa, impulsiona suas ações e
discursos para lugares menos previsíveis. A logografia e a retórica são
expostas em seus fundamentos, de modo que podemos perceber, com nitidez,
o que seria para Platão de fato discursar e/\allowbreak{}ou escrever bem ou mal,
dentro de técnicas tradicionais, com fins precisos e preparo dialético.

 

\section{Recitações e partes da alma}
	

Cada uma das recitações do diálogo revela um nível distinto de
compreensão da realidade e graus diferentes de participação no ser e no
divino. Nessa perspectiva, as recitações apresentam uma história
retrospectiva da relação dos homens com a linguagem (\emph{lógos}),
especialmente tendo como referência a descoberta/\allowbreak{}invenção da escrita.

Do ponto de vista histórico, a Atenas na qual viveu Platão, agora
retratada no final da penosa guerra do Peloponeso, foi o resultado do
predomínio da democracia e da sofística, representados por Lísias,
Isócrates e pelos demais sofistas, estejam eles mencionados ou não. Ao
mesmo tempo, as etapas seguintes, que apontam para uma saída dessa
condição, são ambas repletas de elementos tradicionais e arcaicos,
sobretudo poéticos. Na primeira recitação de Sócrates se observa uma
mimese da poesia rapsódica, algumas vezes mimetizando o ditirâmbico
(invectivo), enquanto a palinódia, ou segundo discurso de Sócrates, está
repleta de elementos da lírica arcaica, elogios (encômios), bem como
elementos ligados aos mistérios, às revelações e iniciações. Entre essas
duas poéticas, apenas a segunda se aproxima da dialética e guarda com
ela assimilações. Apesar de Sócrates dizer que aquele lugar supraceleste
nunca foi cantado dignamente por nenhum poeta, denotando a limitação da
própria linguagem poética, por outro lado aposta na alegoria, na imagem
(\emph{eikôna}; \emph{eidôlon}), como único recurso possível, ou o mais
próximo, para descrever o lugar supraceleste de onde a alma veio e para
onde deve voltar naturalmente, se bem direcionada. Sócrates aproxima-se
do registro poético lícito, o estesicórico, colhendo-lhe elementos
catárticos, e procede de modo encomiástico, absorvendo a ambiência
coral arcaica, por ser um registro mais apropriado para a empreitada de
descrever e louvar filosoficamente elementos cósmicos e teológicos. 
A tópica que Sócrates retoma de Estesícoro remete à imagem de Helena e aos
poderes de Afrodite, mãe de Eros.

Hermias enxerga a correspondência entre discursos e partes da alma,
usando os termos ``vidas'' (\emph{zôás}) e ``visões'' (\emph{horâtai,
enoratai}), que correspondem aos três impulsos principais da alma:

 

\begin{quote}
em três partes pode-se dividir todo o pensamento do discurso e em três
vidas,
%visões?
 a primeira é a indisciplinada, visão de Lísias, a segunda é a
prudente e se manifesta pelo discurso e pela visão de Sócrates, e a
terceira é pelo último discurso de Sócrates, a palinódia, no qual se vê
o entusiasmo. (Herm. \emph{In Phdr.} 64, 5-10 Couvreur = 12, 6-10
Lucarini e Moreschini).
\end{quote}

 

Há diferentes impulsos ou predominâncias das almas. Em algumas almas
predomina o apetitivo (\emph{epithumétikos}), nas almas dos
indisciplinados, noutros predomina o irascível (\emph{thymoeides}), os
quais podem ser tomados erroneamente por prudentes, e noutros ainda
predomina o racional (\emph{logoi}), os que alcançam a verdadeira razão
e o aprendizado, juntamente com o entusiasmo. Esses três níveis ou
predominâncias da alma são apresentados pelas recitações, e nossa leitura
segue Hermias nessa correspondência entre os três discursos iniciais e
as três partes da alma. Vejamos alguns detalhes de cada uma das
recitações:


\subsection{Lísias (230e6-234c5)} 


A logografia de Lísias é apresentada
como a cadeia em que estão presos nesse período histórico da narrativa,
momento em que os tribunais estão repletos de escritos de outros, um
mercado de discursos estabelecido por necessidades jurídicas. 
A logografia de Lísias é uma peroração presenteada a Fedro e que ele
procura memorizar, fora dos muros, demonstrando sua própria ingenuidade,
acreditando que algo escrito possa, de fato, trazer-lhe algum
conhecimento verdadeiro, não tendo sido assimilado anteriormente.

O discurso de Lísias, conhecido também como discurso erótico, apresenta
um Amor nocivo. Nele, o hedonismo é evidente e o Amor é entendido como
uma doença que deve ser evitada. Para tanto, é importante deixá-lo de
lado nas relações, especialmente nas pedagógicas, que devem, segundo
essa perspectiva, primar pelo pragmatismo. Só uma amizade sem Eros
seria adequada para Lísias, especialmente porque o costume de agradar
(\emph{charídzesthai}) um amante faz parte da instituição pedagógica
ateniense, meio pelo qual todos haviam sido educados até então. Nesse
sistema antigo não havia espaço para educadores estrangeiros, grande
praga política na visão de Platão. E nesse sentido é que Lísias
defenderá sua tese, grande mote do diálogo, de que é melhor não agradar
um amante tomado por Eros, mas um qualquer que não esteja tomado por
essa força, especialmente porque quer dizer ser melhor não seguir a
educação tradicional ateniense, mas sim pagar um sofista estrangeiro,
alguém que não faça parte dessas tradições. O discurso de Lísias
corresponde à parte apetitiva da alma ou região do \emph{epithymetikós},
descrita na \emph{República}, meio pelo qual Lísias seria exemplo do
desejo por ganhos e vantagens (\emph{philochrématon kaì
philokerdès} \emph{R.} 581a6-7)


\subsection{Sócrates 1 (237a7-241d)} 

Em seguida, a poesia rapsódica é
mimetizada por Sócrates. Podemos lê-la também como uma alusão aos
festivais teatrais, época em que Atenas concentrou muitos recursos como
consequência dos seus acordos militares, os quais, diga-se de passagem,
levaram à guerra do Peloponeso. Sócrates será levado a defender a mesma
tese ímpia que Lísias havia sustentado em seu discurso, mas, como ele
prevê a falta teológica que está prestes a cometer, realiza o discurso
com a cabeça coberta. Sócrates defenderá ali um segundo grau de uso da
linguagem, imitando a inspiração poética e rapsódica, defendendo um Eros
prudente, não louco, em um discurso com técnica apurada, mas que, apesar
disso, mantém como fundamento a mesma tese comprometedora de Lísias,
pois falar mal de Eros não é aconselhável. Sócrates, ao atravessar o
Ilisso, será alertado pelo seu \emph{daímon} acerca da impiedade que, de
algum modo, já tinha ciência de ter cometido, uma vez que cobriu sua
cabeça antes de falar.

Sócrates não fala a partir de um texto escrito, mas sim de sua própria
``arte'', que ele a todo tempo pretende manifestar como algo que recebe
das entidades locais, como as ninfas, as Musas ou de Dioniso, mas que,
na realidade, não passa de uma técnica bastante programada dentro das
tópicas poéticas que estão à sua disposição. Aliás, em termos de
invenção e disposição, o discurso de Sócrates é realmente superior ao de
Lísias, que foi exemplo de discurso mal-acabado.

Sócrates é superior a Lísias por ser capaz de extrair, de inventar, um
discurso epidíctico (\emph{epideixis}) organizado a partir das tópicas
poéticas. A disputa é um pouco desigual porque o gênero epidíctico ou
demonstrativo nunca foi exatamente a especialidade de Lísias, que
escreveu sobretudo discursos jurídicos. Como meteco, Lísias só pôde
pisar no tribunal ateniense por um curtíssimo período em que foi aceito
como cidadão, rapidamente teve esse direito revogado.

Sócrates, por não ser capaz de recuar frente ao desafio proposto,
exemplifica esse impulso do \emph{thymos}, mesmo que tenha ciência da
falta religiosa que comete. Esse seria o nível irascível
(\emph{thymoeides}) da alma em sua sede por discursos, por emparelhar-se
em competições, nos quais predominam o amor pela vitória (\emph{niké}) e
pela honra (\emph{timé}). Descrito na \emph{República}, esse impulso é
o da amizade pela vitória (\emph{philónikon}) e pelas honras
(\emph{philótimon}) (\emph{R.} 580d-581c). Sócrates mostrará em seguida
que é possível e necessário passar dessa condição irascível para uma
condição mais elevada.

\subsection{Sócrates 2 (243e-257b6)}

Aqui a idade de ouro da civilização
grega é apresentada, juntamente com um retrato da época arcaica
propriamente dita, louvada em todo o diálogo como superior, época de
Estesícoro, do círculo pitagórico, das invenções e valores da sociedade
oral que Platão, de algum modo, pretende restituir na sua filosofia
escrita, valores ligados à imortalidade da alma, à catarse
musicoterapêutica, à harmonia da alma e à revelação dos mistérios.
Nesse lugar, a parte intelectual, do auriga, predomina, pois é nele que
Sócrates expõe essa natureza das almas dos deuses e homens.

Será apenas nesse segundo discurso de Sócrates (palinódia) que
veremos a defesa de um Eros lícito, atrelado à loucura, unicamente capaz
de demover e corrigir a alma na direção correta, de nutrir suas asas a
ponto de ela ser capaz de mimetizar os deuses e alçar novamente
(\emph{pálin}) a planície da verdade, lugar de onde são naturais as
almas.

A imagem filosófica é usada por Sócrates para descrever o lugar
supraceleste, a natureza da alma divina e humana, bem como a semelhança
que há entre elas, sem deixar de referir diferenças. Sócrates recorre a
uma imagem para descrever algo que está para além da imagem sensível. 
A linguagem está em seu limite, pelo uso analógico e alusivo, embora não
afirme ser capaz de realizar algo para além da sua própria
potencialidade. A imagem da biga alada é utilizada nesse grande discurso
em que a tripartição da alma é exposta.

A palinódia socrática, de alguma forma, pretende substituir as
impiedades anteriormente proferidas e, ao evidenciar a conversão
(\emph{protrepse}) de Sócrates, mudar a visão do leitor e de Fedro, operando
uma mudança na alma através de um discurso escrito, o que, na verdade, só
pode funcionar se aquele que lê ou ouve for capaz de recuperar os
padrões aprendidos anteriormente (anamnese) e reativá-los.
A palinódia é o reverso das posições anteriores acerca de Eros,
momento em que a tópica da \emph{Palinódia} de Estesícoro é retomada,
justamente por desejar criar com seu discurso em prosa um efeito
catártico similar ao poético, algo curativo e agora dialético. Esse
seria um uso da escrita como \emph{hypomnema}, como recordação, somente
útil para quem já aprendeu.

Esse discurso de Sócrates corresponde à parte da alma que aprende
(\emph{R.} 581b). Sócrates transforma sua conduta, inverte a tese ímpia
contra Eros, e segue uma nova vida atrelada a Eros e à loucura amorosa.
Nessa mudança, o homem se torna capaz de levantar a cabeça para a
contemplação dos deuses no supraceleste: ``levantando a cabeça
(\emph{anakúpsasa}) para o verdadeiro ser (\emph{tò ón óntos})
(\emph{Phdr}. 249c2-3)''. Sócrates defende naturalmente as instituições
pedagógicas atenienses e traz a imagem do trajeto das almas divinas e
humanas, valendo-se de um modo discursivo próximo ao dos mistérios
(Schefer, 2003).

 

\asterisc{}

 

Resumidamente, o discurso de Lísias (\emph{Phdr}. 230e6-234c5) sustenta
a visão de quem é conduzido especialmente pelo apetitivo
(\emph{epithymétikos}), o amor vulgar, fortemente ligado aos impulsos
mais violentos, enquanto o primeiro discurso de Sócrates
(\emph{Phdr}. 237a7-241d) sustenta uma visão de quem é conduzido
predominantemente pelo irascível (\emph{thymoeides}), havendo nele uma
argumentação ``aparentemente'' mais elevada e prudente
(\emph{sophosyne}), enquanto a parte lógica é apresentada apenas pela
palinódia de Sócrates (\emph{Phdr}. 243e-257b6) como uma visão de quem é
conduzido pelo entusiástico (\emph{enthoun}), no qual encontramos, além
do lógico (\emph{logoi}), o místico, a revelação, admitindo o papel da
loucura no processo de reconhecimento/\allowbreak{}aprendizado da verdade. Sócrates,
tomado pela loucura de Eros, é capaz de narrar a alegoria cósmica que
descreve a natureza da alma, divina e humana, em seu trajeto e suas
potencialidades.

 



​

%\part{Fedro}

 
