\chapterspecial{Título do capítulo}{}{}
 

 Existem outros males ainda, mas algum~\emph{daímôn}~os misturou
(\emph{émeixe}) aos maiores prazeres (\emph{hêdonén}) momentâneos,
[240b] como no caso do adulador, terrível fera de enorme prejuízo
(\emph{kólaki, deinôi thêríôi kaì blábêi megálêi}), para o qual, ao
mesmo tempo, a natureza mesclou (\emph{epémeixen hê phýsis}) algum tipo
de prazer requintado (\emph{hêdonén tina ouk ámouson}).~Como os
[prazeres] de uma cortesã (\emph{hetaíran}), que poderia ser
censurada como danosa (\emph{blaberòn pséxeien}), assim como outras
similares criaturas e ocupações, prazerosas (\emph{hedístoisin}) no
início, pelo menos durante um dia. O~amado, então, com relação ao seu
amante, torna"-se danoso, bem como o convívio prolongado torna"-se o maior
desprazer possível (\emph{aêdéstaton}). [240c] Como diz o antigo
ditado, ``cada idade agrada aos da mesma idade'', pois considero que
idades similares conduzem a uma similaridade de prazeres
(\emph{hedonàs}) e proporciona uma semelhante amizade (\emph{philían}),
da mesma maneira como a convivência contínua entre eles causa a
saciedade. E~dizem que o constrangimento é pesado em todos os casos e
para todos, especialmente aos amantes (\emph{erastḕs}) que tem essa
diferença etária com relação ao amado (\emph{pròs paidikà}). O~velho que
convive com um jovem, nem de dia nem de noite abandona voluntariamente
seu amado, [240d] mas é conduzido pela necessidade e pelo aguilhão
daquele que sempre lhe oferece prazer (\emph{hêdonàs}), vendo,
escutando, tocando, com todos os sentidos percebendo o amado, como se
ele servisse justamente aos seus prazeres (\emph{hêdonês}). Que tipo de
exortação ou de prazeres (\emph{hêdonàs}) o amante oferece ao amado
durante o tempo de convívio, para que não cheguem a extremos do
desprazer (\emph{aêdías})? Uma vez que o jovem vê aquele olhar que já
não está na flor da idade, acompanhado de outras coisas desse tipo, que
nem são agradáveis de ouvir falar, [240e] para não ser obrigado
sempre a estar disposto a essa ocupação, sendo vigiado com suspeita
constante do seu guardião em meio a todos, ouvindo elogios
(\emph{epaínous}) hiperbólicos e inoportunos (\emph{akaírous}), bem como
censuras (\emph{psógous}) inadmissíveis a um sóbrio (\emph{nḗphontos}).
E~quando ele está entregue à bebida, todas essas coisas, além de
intoleráveis, passam a ser vergonhosas, especialmente pela tagarelice
excessiva e pelo atrevimento empregado.

Esse amante, danoso e desagradável (\emph{erôn mèn blaberós te kaì
aêdḗs}), quando deixa de amar, logo se torna indigno de confiança
(\emph{ápistos}), pois todos os juramentos (\emph{hórkôn}) e todas as
súplicas professadas mantinham a companhia com dificuldade, [241a]
uma vez que a relação já era penosa de suportar, mesmo quando havia a
esperança de lhe trazer coisas boas (\emph{agathôn}). Quando é
necessário mudar (\emph{metabalôn}) sua própria conduta, o amante passa
a dominar a si mesmo e a estar preparado, com a inteligência e a
prudência em vez do amor e da loucura (\emph{noûn kaì sôphrosýnên ant'
érôtos kaì manías}), e dessa forma ele esquece o seu predileto
(\emph{lélêthen ta paidiká}). Enquanto o amado demanda as graças
[prometidas], relembrando (\emph{hypomimnḗiskon}) os feitos e ditos,
como se pudesse dialogar ainda com ele, o amante, por outro lado, por
vergonha não diz a ninguém o que ocorreu, e de nenhum modo confirma os
juramentos impensados do início (\emph{anoḗtou archês horkômósiá}) e as
promessas realizadas, [241b] pois agora está em sua plena
inteligência (\emph{noûn})~e salvo pela prudência
(\emph{sesôphronêkós}), que o impede de agir de maneira semelhante ou
fazer aquelas coisas novamente (\emph{pálin}). Ele foge de tudo isso,
tendo cometido uma falta pela força do amor anterior (\emph{prìn
erastḗs}), e sendo alterada (\emph{metapesóntos}) a concha de lado, ele
se retira na direção alternada (\emph{híetai phugêi~metabalṓn}).

Aquele que, por outro lado, é levado agora a perseguir a irritação e a
imprecação contra os deuses, desconhece tudo desde o início, de que não
devia agradar (\emph{charídzesthai}) ao amoroso (\emph{erônti}), forçado
pela falta de intelecto (\emph{anoétôi}) [241c], mas que seria muito
melhor não estar sob o efeito do amor (\emph{m}ḕ~\emph{erônti}) e manter
o intelecto (\emph{noûn}).~Caso contrário, seria obrigado a entregar"-se
a alguém sem crença (\emph{apístôi}), mal"-humorado, invejoso,
desagradável (\emph{aêdeî}), danoso (\emph{blaberôi}) para a essência
(\emph{pròs ousían}), danoso (\emph{blaberôi}) para a disposição do
corpo e, sobretudo, muitíssimo danoso (\emph{blaberôtátôi}) para a
educação da alma (\emph{psychês paídeusin}), a qual em verdade é a mais
honrada (\emph{timiṓteron}), e não haverá nada no futuro tão honrado
entre homens ou deuses. É~preciso conhecer tais coisas, ó criança, e
saber que a amizade do amante (\emph{t}ḕ\emph{n erastoû philían}) não
surge entre favores (\emph{eunoías}), mas como alimento, para agradar
(\emph{chárin}) a saciedade, [241d] pois os amantes amam
(\emph{philoûsin}) seus prediletos como os lobos amam (\emph{agapôsin})
os cordeiros''.

 

Isso é tudo, ó Fedro, e nada mais ouvirás de meu discurso, pois esse é o
seu fim.

 

F: E eu considerei que estavas no meio e que dirias semelhantes coisas
também acerca do que não está sob o efeito do amor
(\emph{m}ḕ\emph{~erôntos}), de como é melhor agradá"-lo
(\emph{charídzesthai}), mencionando o quanto isso tem de bom, mas agora,
ó Sócrates, porque interrompeste (\emph{apopaúêi})?

 

[241e] S: Não percebeste, ó bem"-aventurado, que eu proferia há pouco
um épico, não um ditirambo, o qual é mais conveniente ao vitupério
(\emph{pségôn})? E se eu começasse a elogiar (\emph{epaineîn}) o outro,
o que pensas que eu faria? Consideras que sob a influência das Ninfas,
as quais tu me colocaste premeditadamente, eu estaria obviamente
inspirado (\emph{enthousiásô})? Digo, então, que por meio de um só
discurso recusamos o outro, o qual nos oferece os benefícios contrários.
E~para que, então, um discurso tão extenso (\emph{makroû lógoû})? Acerca
de ambos é suficiente o que foi dito, que a narrativa (\emph{mythos})
que te ofereci sofra (\emph{páschein}) o que for.~[242a] Quanto a
mim, vou partir e atravessar esse rio, antes que seja obrigado por ti a
algo mais grave.

 

F: Não ainda, ó Sócrates, pelo menos antes que esse calor se vá, ou não
vês que é quase meio"-dia, aquilo que chamamos de sol a pino? Vamos
permanecer e dialogar acerca do que foi dito, partiremos assim que o
tempo esteja mais fresco.

 

S: És divino em matéria discursiva, ó Fedro, e espantoso por ser isento
de arte (\emph{atechnôs}).~[242b] Por conta dos discursos que
ocorreram por tua força, creio que ninguém é melhor que tu ao
pronunciá"-los ou forçando outros a proferi"-los -- exceto o discurso de
Símias de Tebas, que é o mais forte (\emph{krateîs}) entre todos. E~agora tu me parece ser a causa (\emph{aitiós}) mesma de outro discurso
(\emph{lógôi}) que vou proferir (\emph{rhêthênai}).

 

F: Então é uma guerra o que anuncias! Mas diz como foi e a qual deles te
referes?

 

S: Quando decidi, ó meu caro, atravessar o rio, um~\emph{daimon}~que me
é familiar surgiu e me gerou um sinal (\emph{sêmeîón}) -- ele sempre me
impede quando estou prestes a fazer algo \mbox{---,} e parecia que eu ouvia a
sua voz, ela não me deixava partir antes de me purificar
(\emph{aphosiṓsōmai}), como se tivesse cometido alguma falta contra o
divino (\emph{hēmartēk\gr{ό}ta eis to theîon}).~[242c] É que eu sou um
adivinho (\emph{mántis}), mas não muito aplicado (\emph{ou pánu dè
spoudaîos}), talvez como aqueles que são ruins na escrita (\emph{ta
grámmata phaûloi}), no entanto para mim isso já é suficiente.~Aí então
compreendi com clareza a minha falta (\emph{hamártēma}). É~certo agora,
ó companheiro, que a alma (\emph{hē psykhḗ}) tem alguma adivinhação
(\emph{mantikón gé ti}), pois algo também me inquietou, ao proferir o
discurso passado, e temi, do mesmo modo que Íbico, com relação à sua
falta contra os deuses:

 

que ela não altere minha honra (\emph{timàn}) junto aos homens

 

[242d] Só agora percebi a minha falta (\emph{ḗisthēmai tò
hamártēma}).

 

F: Diz então, qual é [a falta]?

 

S: Terrível, ó Fedro, terrível foi o discurso (\emph{lógon}) que
trouxeste e o que me forçaste a dizer (\emph{eipeîn}).

 

F: Como é?

 

S: Foi uma tolice e uma espécie de impiedade (\emph{ti asebê}).~Pode
haver algo mais terrível (\emph{deinóteros})?

 

F: Não, se é que dizes a verdade.

 

S: O quê? Não consideras que o Eros é um deus, filho de Afrodite?

 

F: Assim dizem.

 

S: Mas não segundo Lísias, nem pelo teu discurso, aquele que saiu da
minha boca envenenada (\emph{katapharmakeuthéntos}) pelo que tu disseste
(\emph{eléchtê}). [242e] Se é assim, tal qual sabemos, Eros é um
deus ou alguma divindade, e de nenhuma forma poderia ser mau, mas ambos
os discursos proferidos falaram dele como se assim ele fosse. Dessa
maneira, cometeram uma falta (\emph{hēmartanétēn}) contra Eros, além
disso, pretenderam"-se bondosos e muito civilizados (\emph{asteía}), mas
não foram discursos saudáveis (\emph{hygiès}) tampouco verdadeiros
(\emph{alêthès}), embora tenham disso se gabado, [243a] uma vez que,
ao enganar alguns homens, tornaram"-se bem reputados entre eles. Ó
querido, eu preciso me purificar (\emph{kathérasthai}). Há uma
purificação arcaica (\emph{katharmòs archaîos}) para os que cometem
faltas em mitologia. Homero não a conheceu, mas Estesícoro~sim. Privado
da visão pela linguagem abusiva contra Helena, não ignorou a causa como
Homero, mas, conhecendo a causa (\emph{égnô tèn aitían}), o músico de
Himera compôs:

 

\begin{quote}

\begin{verse}[\versewidth]
Esse não é um discurso verdadeiro,\\
 nem embarcaste em naves bem
assentadas,\\
 nem foste à cidade de Troia.\\!
\end{verse} 
\end{quote}

 

[243b] E ao compor toda a obra, chamada de~\emph{Palinódia},
imediatamente recuperou a visão (\emph{anéblepsen}).~Eu, então, agora me
torno mais sábio (\emph{sôph}ṓ\emph{teros}) que eles, pelo menos nesse
ponto, pois, antes de sofrer (\emph{patheîn}) algo pela linguagem
abusiva contra Eros, trato de ofertar"-lhe uma~\emph{palinódia}~com a
cabeça descoberta (\emph{gymnêi têi kephalêi}), e não como havia
ocorrido, por vergonha, com a cabeça velada (\emph{egkekalumménos}).

 

F: Nada poderia ser mais prazeroso (\emph{hedíô}), ó Sócrates, do que
isso que tu afirmas.

 

[243c] S: Pois então, ó bom Fedro, põe na tua mente (\emph{ennoeîs})
a falta de pudor (\emph{anaidôs}) com que os discursos foram proferidos,
tanto esse meu [discurso] como o teu, a partir do livro lido. Se,
por acaso (\emph{týchoi}), o caráter (\emph{êthos}) dos que nos ouvem
for nobre e gentil, eles pensariam que estão sendo levados a escutar
marinheiros e que de nenhum modo veriam um amor entre homens livres?
Quer seja o amado ou o amante, se os mencionarmos simplesmente como
enamorados, não seria através das pequenas coisas que ambos poderiam ser
tomados por grandes ódios (\emph{échthras}), ciúmes (\emph{phthonerôs})
e danos (\emph{blaberôs}) aos seus prediletos (\emph{ta paidikà}).
[243d] Eles, nesse caso, não concordariam plenamente conosco ao
vituperarmos o Amor (\emph{hêmîn homologeîn há pségomen tòn Érota})?

 

F: É possível, ó Sócrates, por Zeus!

 

S: Desse mesmo homem eu me envergonho e temo pelo próprio Amor, a ponto
de desejar, com um discurso potável, lavar"-me dessa audição salgada
(\emph{epithymô potímôi lógôi oîon halmuràn akoḕn apoklúsasthai}).~Diga
a Lísias para escrever (\emph{grápsai}) o mais depressa possível que
partindo de condições semelhantes, é melhor agradar
(\emph{charídzesthai}) ao amante (\emph{erastêi}) em vez do não afetado
pelo amor (\emph{mè erônti}).

 

F: E veja bem que será assim mesmo, pois tu, ao fazer o elogio do amante
(\emph{tòn toû erastoû épainon}), gerarás em Lísias a necessidade de
escrever (\emph{grápsai}), impelido por mim, [243e] sobre esse mesmo
discurso (\emph{lógon}).

 

S: Eu acredito.

 

F: Diz agora com bravura.

 

S: Onde está o jovem com o qual eu falava? Quero que ele ouça também
isso, e que não se antecipe, por não ter ainda escutado, agradando
(\emph{charisámenos}) ao não afetado pelo amor (\emph{mḕ~erônti}).

 

F: Ele está junto a ti, muito perto e sempre a acompanhar"-te, quando tu
quiseres (\emph{boúlêi}).

 

S: ``Deste modo, ó bela criança, compreende (\emph{ennóêson}) que o
primeiro discurso foi o de Fedro, filho de Pítocles, homem de
Mirrinunte, [244a] e o discurso seguinte será o de Estesícoro, filho
de Eufemo, natural de Himera.~Que seja dito que\emph{~não é um discurso
verdadeiro}~(\emph{ouk ést'étymos lógos}),\textsuperscript{~}aquele que
diz, perto de um amante (\emph{erastoû}), ser melhor agradar
(\emph{charídzesthai}) a quem não está afetado pelo amor
(\emph{mḕ~erônti}), porque um está louco (\emph{maínetai}) e o outro
sóbrio (\emph{sôphoneî}). Se a loucura (\emph{manían}) fosse
simplesmente (\emph{aploûn}) má, este seria um belo discurso, mas os
maiores bens (\emph{agathôn}) nos surgem por intermédio da loucura
(\emph{dià manías}), a qual seguramente é um presente divino
(\emph{theíai}). Tanto a profetisa do oráculo em Delfos, quanto as
sacerdotisas em Dodona, executaram para a Hélade muitas e belas coisas,
sejam particulares ou públicas, tomadas pela loucura (\emph{maneîsai}),
[244b] ao passo que sóbrias (\emph{sôphronoûsai}) elas pouco ou nada
fizeram.~E se dissermos que a Sibila e tantos outros, valendo"-se da
adivinhação entusiástica (\emph{mantikêi ch}ṓ\emph{menoi enthéoi}),
muitas vezes e para tantos, predisseram um futuro correto, estaríamos
nos alongando sobre o que é evidente para todos.

Eis um testemunho digno (\emph{áxion epimartúrasthai}), que os antigos
instituidores dos nomes não consideravam a loucura (\emph{manían}) nem
ruim (\emph{aischròn}) nem vergonhosa (\emph{óneidos}), pois não a
teriam misturado à arte mais bela, a que interpreta o futuro, [244c]
designando"-a pelo nome de~\emph{maniké}. Julgaram"-na bela porque a
loucura surgia por parte da divindade (\emph{theíai}). Nossos
contemporâneos, inexperientes em beleza, enfiando o ``\emph{tau''}~no
meio, chamam"-na de~\emph{mantiké}. E~os sóbrios (\emph{tôn
emphrónôn})~que buscam o futuro pelos pássaros e por outros sinais
(\emph{diá orníthôn poiouménôn kai tôn állôn semeíôn}), os mesmos que
partem da reflexão e abrem o caminho da suposição humana, do pensamento
e da observação (\emph{ek dianoías poridzoménôn anthrôpínei oiḗsei noûn
te kaí historían}), esses chamaram"-na de~\emph{oionoïstikḗn}. Hoje em
dia os jovens imponentemente dizem~\emph{oiônistiké}, com um ``ô''
longo. [244d] Quanto mais perfeita e honrada é a~\emph{adivinhação
oionística~}(\emph{mantiké oiônistiké}), e o nome da primeira atividade
com relação ao nome da segunda, mais bela é a loucura (\emph{manían}) em
vista da sobriedade (\emph{sôphrosýne}), testemunham os antigos
(\emph{marturoûsin oi palaioì}), pois uma surge por intermédio do deus
(\emph{ek theoû}) e a outra junto aos homens (\emph{par'anthópôn}).

Com efeito, a loucura (\emph{manían}) surgiu para algumas famílias que
necessitavam, profetizando as maiores enfermidades (\emph{nósôn}) e
dores (\emph{pónôn}) vindas de antigos ressentimentos, e elas
encontraram (\emph{heúreto}) refúgio em preces e cultos aos deuses.
[244e] Daí então surgiram purificações e iniciações (\emph{katharmôn
te kai teletôn}) praticadas para suas próprias isenções, tanto para o
tempo presente quanto para os tempos vindouros, sendo assim encontrado
(\emph{heuroméne}) o correto (\emph{orthôs}) afastamento dos males
coetâneos na loucura e na possessão (\emph{manéti te kai
kataschoménoi}).

[245a] A terceira possessão e loucura (\emph{katokochḗ~te kai
manía}) vem das Musas, as quais se apoderam da alma delicada e
inviolada, despertando e tornando"-a báquica por meio de odes e outras
poesias, as quais ordenam (\emph{kosmoûsa})\textsuperscript{~}inúmeras
obras dos antigos e educam (\emph{paideúei}) os pósteros. Aquele que
chegar às portas da poética sem a loucura das Musas (\emph{aneu manías
Mousôn}), acreditando (\emph{peistheìs}) que somente por força da arte
seria poeta perfeito (\emph{ék téchnês ikanòs poietês}), está incompleto
(\emph{atelḕs}), sem contar que a poesia (\emph{poíesis}) dos
enlouquecidos (\emph{mainoménôn}) ofusca (\emph{êphanísthê}) a dos
sóbrios (\emph{sôphronoûntos}).

[245b] Tenho dito a ti acerca da grandeza das belas obras repletas
da loucura que vem dos deuses, de tal maneira que dela não fujamos
(\emph{phobómetha}), nem nos perturbe algum discurso que amedronte o
amante na direção da necessidade da escolha (\emph{proaireîsthai}) do
amigo sóbrio (\emph{tón sṓphrona phílon}). Aquele [amante] leva a
honra da vitória antes deste [sóbrio], mostrando que não é pela
utilidade (\emph{ôpheléiai}) que o amor (\emph{érôs}) é enviado, pelo
deus, ao amante e ao amado. Então, é preciso que demostremos o contrário
(\emph{apodeiktéon aû tounantíon}): como pela maior das sortes essa
loucura é dádiva dos deuses. [245c] Essa demonstração
(\emph{apódeixis}) não será persuasiva (\emph{ápistos}) aos terríveis
(\emph{deinoîs}), mas será persuasiva (\emph{pistḗ}) aos sábios
(\emph{sophoîs}).~Será necessário primeiramente tratarmos da natureza da
alma, divina e humana, vislumbrando suas paixões e ações (\emph{páthê te
kaì érga}), entendendo a verdade (\emph{talêthès noêsai}). O~princípio
(\emph{archḕ}) da demonstração (\emph{apodeíxeôs}) é o seguinte.

Toda alma é imortal. Tudo aquilo que está sempre em movimento é imortal,
ao passo que o que move outro, ou por outro é movido, ao cessar do
movimento (\emph{kinéseos}) cessa também a vida (\emph{dzōẽs}). Somente
o auto"-movido não se desliga de si mesmo, visto que nunca cessa seu
movimento, e para todas as coisas que são movidas essa é a fonte e o
princípio do movimento (\emph{pegḗ~kai archḗ~kinḗseos}). Princípio é sem
geração (\emph{archḕ~dè agénêton}). [245d] É necessário que todo o
gerado advenha de um princípio (\emph{ex archês}), e ele mesmo não
[advenha] de nenhum, pois se ele surgisse de um princípio
(\emph{archè}) não mais poderia ser considerado um princípio. De modo
que é necessário que [o mesmo princípio] seja sem geração e sem
corrupção, pois nem se corrompe, nem é gerado, se é que todas as coisas
necessariamente surgem (\emph{gígnesthai}) de um princípio. Neste caso o
movimento tem um princípio que lhe é idêntico, dele não podendo sofrer
corrupção ou geração, ou todas as coisas e toda a gênese do céu estariam
conjuntamente perecendo e nunca teriam recebido movimento a partir de
algo. [245e] Esclarecida a imortalidade daquilo que é movido por si
mesmo, a essência da alma (\emph{psychês ousían}) e sua explicação não
foram enunciadas (\emph{legôn}) de modo vergonhoso. De todos os corpos,
os que recebem movimento externo (\emph{éxôthen}) são inanimados, ao
passo que os que de dentro de si e por si [recebem o movimento] são
animados, tal é a essência natural da alma (\emph{hos taútês oúsês
phýseôs psychês}). Sendo assim, a alma não é outra coisa senão aquilo
que move a si mesmo, necessariamente a alma não é gerada, o que a torna
imortal.

[246a] Acerca da imortalidade é o suficiente. Falemos [agora]
acerca dessa ideia (\emph{idéas}).~Quanto ao que ela é teríamos uma
grandiosa e divina (\emph{theías}) exposição (\emph{diegḗseos}), e, ao
que parece, uma exposição inferior, conveniente aos homens, falemos por
meio desta última. Convencionemos que ela tenha uma potência
(\emph{dynámei}) e uma natureza similar (\emph{sumphýtôi}) a uma biga
alada e seu auriga. Os cavalos e cocheiros dos deuses são essencialmente
todos bons e vindos do que é bom (\emph{agathoì kaì ex agathôn}), ao
passo que os dos outros são misturados (\emph{mémiktai}). [246b]
Primeiramente dirige a biga aquele que para nós é o comandante, em
consequência disso um dos cavalos é bom e belo, enquanto o outro é o seu
contrário, sendo ele mesmo um contrário. Entre nós, portanto, o ofício
de auriga (\emph{henióchêsis}) é necessariamente penoso e adverso.

Experimentemos dizer o motivo pelo qual a vida foi enunciada como mortal
e imortal (\emph{thnetón te kaì athánaton}).~Toda alma ocupa"-se
inteiramente do que é inanimado, circula por todo o céu tomando, algumas
vezes, outros aspectos (\emph{eídesi}). [246c] Estando em sua
perfeição [a alma] é alada, atravessa as alturas
(\emph{meteôroporeî}) e habita todo o cosmo (\emph{pánta tòn kósmon
dioikeî}), mas quando é levada à perda das asas, então, de algum sólido
se apodera e ali se instala (\emph{ou katoikistheîsa}), tomando corpo
terrestre, o qual parece mover a si mesmo devido àquela potência. Como
viventes que são enunciados conjuntamente, alma e corpo, fixados, ganham
o epíteto de mortal\emph{.~}O imortal não é deduzido por um raciocínio,
mas modelamos (\emph{pláttomen}) o deus, não somente pela visão
(\emph{idóntes}), nem somente pelo pensamento (\emph{noḗsantes}), como
um vivente imortal que, tendo alma e corpo, mantém"-se assimilado para
sempre no tempo. [246d] Que essas coisas assim sejam e que tenham
sido expostas ao agrado do deus.

Tomemos agora a causa da queda das asas, motivo pelo qual a alma se
perde. É~da natureza da potência alada levar o que é pesado para o alto,
alcançando a casa do gênero divino, por onde ela se põe em comum
(\emph{kekoinṓnêke}), no mais alto grau corpóreo, com a alma do deus. O~divino é belo, sábio, bom e tudo o que é dessa mesma classe, [246e]
e justamente por essas coisas que são mais bem acrescidas e alimentadas
as asas da alma. As coisas contrárias a estas são corruptíveis e perecem
pela maldade e pelo vício.

Zeus é o grande condutor no céu com seu carro alado (\emph{ptênòn
hárma}), adianta"-se em primeiro lugar, zelando por todas as coisas
através do cosmo (\emph{diakosmôn}). Ele é seguido por um exército de
deuses e~\emph{daimônes}~(\emph{theôn te kaì daimónôn}) ordenados
(\emph{kekosmêménê}) em onze partes, permanecendo Héstia sozinha na casa
dos deuses (\emph{ménei gar Hestía en theôn oíkôi mónê}). [247a]
Dentre os outros tantos deuses, em sua formação de doze partes, são
conduzidos pelo chefe, seguindo a composição que lhes foi
atribuída.~Então, muitas divindades bem"-aventuradas seguem trajetos no
interior do céu (\emph{entós ouranoû}) e circulam (\emph{epistréphetai})
no gênero feliz dos deuses, cada uma delas fazendo o que lhes é próprio
(\emph{práttôn hékastos autôn tò autoû}). Seguem sempre que querem e
podem, uma vez que a inveja (\emph{phthónos}) permanece fora do coro dos
deuses (\emph{éxô theíou choroû}).~Quando vão ao cume para um festim ou
banquete, atravessam para o ápice das escarpas que sustentam o céu
(\emph{ákran epí ten hypouránion apsîda poreúontai}), de modo que as
carruagens dos deuses, estando num dócil equilíbrio, [247b]
ultrapassam facilmente, já as outras, [ultrapassam] com dificuldade.
O~cavalo que partilha do mal é pesado, inclina"-se para a terra e impede
o trajeto do auriga que não foi bem"-educado.

Ali mesmo fica o último (\emph{éskatos}) grau de sofrimento e disputa a
que a alma se dispõe; as almas dos imortais, quando chegam ao extremo,
atravessando exteriormente (\emph{éxô poreutheîsai}), estabelecem"-se sob
o dorso do céu (\emph{epì tôi toû ouranoû nôtôi}), sendo levadas e
trazidas ao seu redor; [247c] as outras [almas dos mortais]
contemplam ali as coisas fora do céu (\emph{theoroûsi ta éxô toû
ouranoû}).

Esse lugar supraceleste (\emph{hyperouránion tópon}) ainda não foi
cantado por nenhum dos poetas e nunca será cantado de forma digna. É~necessário ousar dizer a verdade (\emph{alêthès eipeîn}), sobretudo ao
falarmos da verdade (\emph{alêtheías
légonta}).~\emph{eidôlon}\textsuperscript{~}(\emph{ousía óntôs oûsa})
não tem cor, é sem figura, intangível e somente contemplada pelo
pensamento do piloto da alma (\emph{psykês kybernḗtei monôi
theatḕ~nôi}), região na qual tem lugar o gênero verdadeiro do
conhecimento (\emph{tò tês alêthoûs epistḗmês génos}). [247d] Então,
bem como pensamento do deus (\emph{theoû diánoia}), [a alma] é
nutrida pela pureza do intelecto (\emph{nôi}) e do conhecimento
(\emph{epistḗmêi}), como todas as almas que possam vir a mostrar tal
preocupação, tendo visto o ser através do tempo (\emph{idoûsa dià
chrónou tò ón}). Ela é nutrida por ter contemplado a verdade
(\emph{theôroûsa talêthê}), sentindo"-se completa, até que possa chegar,
pelo ciclo, ao ponto inicial do trajeto pelo qual foi levada. Nesse
período veriam a própria justiça (\emph{dikaiosúnên}), a prudência
(\emph{sôphrosúnên}), o conhecimento (\emph{epistémên}), [247e] não
aquilo que pertence à gênese, nem o que está em outras coisas, em outros
que agora chamamos seres (\emph{óntôn}), mas conheceriam a própria
essência que está no ser (\emph{ho estin òn óntôs epistémên oûsan}). E,
do mesmo modo, tendo contemplado (\emph{theasaméne}) a essência dos
seres em seu posto (\emph{estiatheîsa}), mergulham de volta (\emph{dûsa
pálin}) para o interior do céu e chegam a casa.\textsuperscript{~}O
auriga chega ao estábulo, coloca ali os cavalos, oferece"-lhes ambrosia e
lhes dá néctar para beber.

[248a] Esta é a vida dos deuses. Quanto às outras almas, a que
melhor acompanha e se assimila (\emph{epoménê kaì}~\emph{eikasménê}) ao
deus, eleva a cabeça do auriga até o lugar exterior (\emph{tòn éxô
tópon}) e acompanha a volta circular, perturbadas pelos cavalos e com
muita fadiga veem do alto os seres (\emph{kathorôsa ta ónta}). Já a
[alma] que ora se eleva, ora mergulha, tendo forçado os cavalos,
algumas coisas vê, outras não. Outras [almas] ainda, apegando"-se a
tudo o que é do alto (\emph{toû ano épontai}) são incapazes
(\emph{adynatoûsai}) de ter êxito, mas seguem submersas na volta
circular, em pisoteio e confronto mútuos, tentando se adiantarem frente
as outras. [248b] Então ocorre tumulto, luta e suor extremos, é
quando muitas almas claudicam pela maldade do auriga (\emph{kakíai
henióchôn}) e destroçam suas asas. Todas estas, tendo muita fadiga, sem
chegarem à contemplação do ser (\emph{toû óntos théas}), afastam"-se e
servem"-se do alimento da opinião (\emph{trophêi doxastêi chrôntai}).~Eis
o grande empenho que há para ser capaz de ver a planície da verdade onde
ela está (\emph{tò alêtheías ideîn pedíon oû estin}), pois o pasto que
convém ao melhor da alma provém desse prado e a natureza do alado
(\emph{pteroû phýsis}), que eleva a alma, [248c] ali é alimentada.

Eis a lei de Adrasteia: A alma que tenha se tornado acompanhante do deus
(\emph{theôi sunopadòs}) e que tenha visto algo das verdades
(\emph{katídêi ti tôn alêthôn}) fica ilesa (\emph{apḗmona}) até o outro
percurso, e se puder fazer isso sempre, fica sempre intacta
(\emph{ablabê}). Quando não lhe é possível gerir"-se, não se vale da
visão nem do sucesso, e, ao aplicar muito peso, perde as asas,
despencando por terra em função do fardo do esquecimento (\emph{léthês})
e da maldade. Então é lei, na primeira geração, não nascer em nenhuma
natureza de fera [248d]. Os que viram o máximo do gênero humano
tornar"-se-ão filósofos, amigos do belo, músicos ou algum dentre os
eróticos. Em segundo lugar estão o rei na lei, o guerreiro ou o
comandante; no terceiro lugar um político, economista ou administrador;
na quarta posição um amigo das fadigas, da ginástica ou alguém para
curar o corpo; na quinta um adivinho (\emph{mantikón}) ou alguém que
pode cuidar das iniciações (\emph{telestikòn}); [248e] na sexta um
poeta, alguém que se ocupa da mimese ou outras [atividades]
concordes; na sétima um demiurgo ou homem do campo; na oitava um sofista
ou aquele que fere o povo (\emph{demokopikós}); e na nona um tirano.

Em todas elas, os que se conduzem com justiça tomam o melhor destino
(\emph{moíras}), os que o fazem injustamente, o pior.~Cada uma das almas
não chega ao mesmo ponto de onde saiu antes de dez mil anos, pois não
criam asas antes desse tempo, exceto aquela que foi, de maneira honrada,
amante do saber (\emph{philosophḗsantos}) ou [249a] amante dos
jovens de acordo com a filosofia (\emph{paiderastḗsantos metà
philosophías}).\emph{~}Estas, na terceira volta de mil anos, se
conduziram este tipo de vida por três vezes seguidas, no terceiro
milênio se afastam de modo alado. Quanto as outras, há as que ao término
da primeira vida ocorre uma separação (\emph{kríseôs étychon}) e há
julgamento no tribunal subterrâneo, no qual elas prestam contas, ao
passo que há também as que chegam a algum lugar do céu, elevam"-se pela
justiça e são levadas à dignidade da vida humana que tiveram. [249b]
Tanto umas como outras, no milésimo ano, sorteiam (\emph{klḗrôsín}) e
escolhem (\emph{aíresin}) a próxima vida, sendo que cada uma escolhe
(\emph{airoûntai}) a que quiser.~Ali mesmo, os homens que foram feras
serão novamente (\emph{pálin}) homens, e se a alma não atingir tal
figura (\emph{schêma}) é por não ter visto a verdade
(\emph{mḗpote}~\emph{idoûsa tḕn alḗtheian}).

É necessário ao homem atingir (\emph{suniénai}) a ideia (\emph{eîdos})
que vai do múltiplo sensível ao uno, tomado conjuntamente pelo
raciocínio (\emph{logismôi}). [249c] Isso é a reminiscência
(\emph{anámnêsis}) daquilo que nossa alma viu, atravessando com o deus,
vendo além do que agora nos é dito e levantando a cabeça
(\emph{anakúpsasa})~para o verdadeiro ser (\emph{tò ón óntôs}). É~por
isso que, justamente, só cria asas o pensamento (\emph{diánoia}) do
filósofo, para o qual sempre há, na medida do possível, memória
(\emph{mnḗmêi}), e para o qual os deuses são divinos. Homens de tal
valor servem"-se corretamente da recordação (\emph{hypomnḗmasin}), sempre
se iniciam corretamente em mistérios e tornam"-se os únicos perfeitos.
Mudam a dignidade dos homens ao tornarem"-se próximos aos deuses
(\emph{pròs tôi theôi gignómenos}), e são advertidos por muitos que ao
seu lado se moviam. Pelo seu entusiasmo [249d]
(\emph{enthousiádzôn}) eles são esquecidos (\emph{lélêthen}) pela
maioria.

Até aqui temos o discurso todo a respeito da quarta loucura, quando [a
alma] vê alguma dessas belezas, rememorando o verdadeiro
(\emph{alêthoûs anamimneiskómenos}) e tem as asas crescidas, momento em
que a alma está alada e cheia de disposição, entretanto, quando não pode
voar, ela parece um pássaro que vê o que há acima dele, mas descuida do
que está embaixo, por isso é vista como uma alma louca (\emph{manikôs}).
[249e] Essa é a melhor das coisas entusiásticas
(\emph{enthousiáseôn}) e provém das melhores, quem dela vier a ser
possuidor ou dela participar é chamado de amante do belo (\emph{kalôn
erastès}), porque aquele que ama é partícipe (\emph{metéchôn}) da
loucura (\emph{manías}). De acordo com o que foi dito, é da natureza de
toda alma humana ter contemplado os seres (\emph{tetheáthai tà ónta}),
ou não chegariam a essa vida.

[250a] Relembrar (\emph{anamimnḗiskesthai}) aquilo, a partir destas
coisas, não é fácil para todas as almas, nem para aquelas que tiveram
uma breve visão, nem para as que caíram, infortunadas dirigidas pela
injustiça da multidão, almas que esqueceram a visão sagrada
(\emph{léthên ôn tóte eîdon hierôn}) que outrora tiveram. Poucas
(\emph{olígai}) são as [almas] deixadas com suficiente memória
(\emph{mnḗmês}~\emph{ikanôs}). Estas mesmas, quando têm visão de algo
semelhante (\emph{homoíoma ídôsin}), ficam fora de si
(\emph{ekplḗttontai}) e de nenhum modo voltam a si. Outras ignoram a
afecção (\emph{tò páthos}) por não a ter percebido com força suficiente
(\emph{mè ikanôs}~\emph{diaisthánesthai}).

[250b] Justiça, prudência (\emph{dikaiosýnês mèn oûn kaì
sôphrosýnês}) e tantas outras preciosidades da alma não resplandecem
(\emph{phéggous})~em nenhuma das semelhanças (\emph{homoiómasin}) daqui,
mas poucos, através de órgãos obscuros, com fadiga, contemplam o gênero
da similitude (\emph{theôntai to toû eikasthéntos génos}) partindo dos
ícones (\emph{epí tas eikónas ióntes}).

Era de se ver a luminosa beleza quando, outrora, juntamente com o feliz
cortejo, [as almas] visualizavam e contemplavam a bem"-aventurança
(\emph{makarían}~\emph{ópsin te kaì théan}). Nós somos seguidores de
Zeus, outros seguem outros deuses. Vimos e nos completávamos nas
iniciações que, com justiça, são ditas as mais felizes que celebramos,
íntegros e indiferentes aos males que nos surgiriam em tempos
posteriores. Nas completas, simples, [250c] calmas e felizes
aparições estávamos iniciados e havíamos chegado à essa revelação~pela
mais pura luz (\emph{epopteúontes en augêi katharâi}), estando puros
(\emph{katharoì óntes}), pois não havia a marca que nos é trazida pelo
que agora chamamos corpo, motivo pelo qual permanecemos atados [a
ele] como uma ostra [a sua concha]. Que estas coisas sejam
agraciadas (\emph{kecharísthô}) pela memória (\emph{mnḗmêi}),~a qual
pela ausência de outrora foi agora longamente enunciada.

Sobre a beleza, como dissemos, sendo em cada um de nós luminosa,
chegamos aqui [250d] tomando"-a com a máxima clareza de nossos
sentidos, com o mais radiante brilho. A~visão (\emph{ópsis}) é a mais
aguda das sensações que nos chega pelo corpo, mas por ela a prudência
(\emph{phrónêsis}) não é vista.~Cairíamos em terríveis amores, se algum
ídolo (\emph{eídôlon}) de tal classe, por sua própria evidência, fosse
enviado e desejado pela visão (\emph{ópsin}), assim como tantas outras
coisas amáveis. Só a beleza (\emph{kállos}~\emph{mónon}) teve este
destino (\emph{moîran}), ser a mais evidente (\emph{ekphanéstaton}) e a
mais amada (\emph{erasmiṓtaton}).

[250e] Um recém"-iniciado ou alguém que veio a corromper"-se não é
rapidamente trazido daqui para lá, para a beleza mesma (\emph{pròs autò
tò kállos}), contemplando o mesmo que aqui leva seu nome. Não venera ao
olhar, mas entregue ao prazer (\emph{hêdonêi}), põe"-se a andar na lei de
um quadrúpede, produz filhos e, familiarizado com a desmesura
(\emph{hýbrei}), não teme nem se envergonha, perseguindo um prazer
contrário à natureza (\emph{parà phýsin hêdonḕn}).

[251a] O recém"-iniciado (\emph{artitelḗs}) que contemplou muitas
coisas (\emph{polytheámon}), quando vê um rosto de forma divina
(\emph{theoeidés}), bem imitando o belo (\emph{kállos eû memimêménon})
ou alguma forma corpórea (\emph{sṓmatos idéan}), primeiro estremece
(\emph{éphrixe}), enquanto algo dos medos de outrora chega até ele,
depois de ter visto, venera"-o como a um deus e, se não temesse a fama de
uma excessiva loucura (\emph{sphódra manías dóxan}), sacrificaria
(\emph{thúoi}) ao predileto (\emph{paidikoîs}) como a uma imagem e como
a um deus (\emph{agálmati kai theôi}). Depois dessa visão, surge nele
uma mudança do tremor (\emph{tês phríkês}~\emph{metabolḗ}), pois um suor
e um calor atípico o toma e o aquece, tendo recebido [251b] o fluxo
(\emph{aporroḕn}) da beleza pelos olhos (\emph{ommátôn}), motivo pelo
qual a natureza alada é irrigada (\emph{pteroû phýsis ádretai}).
Aquecida a asa, dissolve"-se uma natureza que há muito tempo não
germinava, por endurecimento, aprisionamento e clausura conjunta.
Túrgido de alimento, o caule da asa incha e começa a brotar da raiz em
todas as formas de almas, pois toda [alma] era anteriormente alada
(\emph{pterôtḗ}). Nesse momento, ela ferve toda e irrita"-se
(\emph{anakêkíei}), [251c] como quando sofremos pelo surgimento dos
dentes, que recém"-saídos raspam e irritam por toda a passagem, e o mesmo
sofre a alma no começo do crescer das asas (\emph{pterophyeîn}), ferve e
irrita"-se com as cócegas provocadas pelo brotar das asas (\emph{phýousa
tà pterá}). Quando, ao olhar para a beleza do amado, e dele recebendo
parte do fluxo que sobrevém -- o qual precisamente é chamado de desejo
(\emph{hímeros})~\mbox{---,} é irrigada e aquecida, recompondo"-se da dor ela
fica alegre. [251d] Quando ficam separadas e áridas, as vias que
deságuam onde crescem (\emph{hormâi}) as asas, ficam secas, fechando e
obstruindo o germinar das asas, as quais, em seu interior, após terem
sido fechadas ao fluxo do desejo (\emph{himérou}), ficam agitadas,
arranhando cada uma das vias de saída, justamente porque a alma enfurece
todas as feridas ao redor, causando dor.

Por outro lado, [a alma] alegra"-se tendo a memória (\emph{mnḗmen})
da beleza. Nessa mistura de ambos ela se atormenta pela estranheza da
afecção (\emph{atopíai toû páthous}), não conseguindo saída pela fúria,
e enlouquecida (\emph{emmanès}), nem a noite pode dormir, nem durante o
dia permanece num só lugar. [251e] Corre (\emph{theî}) ansiosa para
onde considera ver (\emph{ópsesthai}) o possuidor da beleza
(\emph{kállos}). Tendo visto e canalizado o desejo (\emph{hímeron}),
libera o que até então estava conjuntamente obstruído, tomando fôlego,
tendo apaziguado as picadas e dores. Este é o dulcíssimo prazer
(\emph{hedonḕn}) de que, no momento, ela desfruta. Por isso não é
voluntariamente afastada e a ninguém mais atende senão ao belo,
[252a] esquece de todos: mãe, irmãos, companheiros. Sendo arruinada
pela negligência, não realiza nada e, quanto aos hábitos e conveniências
com as quais antes se embelezava, a todos passa a desprezar, pronta a
escravizar"-se e deleitar"-se onde lhe permitam, o mais próximo possível
de seu desejo (\emph{póthou}). Além de adorar aquele que porta a beleza,
nele encontra único médico para os seus maiores sofrimentos
(\emph{iatrón hêurêke mónon tôn megístôn pónôn}).

[252b] Essa afecção (\emph{tò páthos}), ó bela criança a quem se
dirige meu discurso (\emph{lógos}), os homens denominam Amor
(\emph{Erôta}), mas se ouvires como os deuses o designam tu
possivelmente rirás, por conta de sua juventude. Alguns homéridas,
segundo penso, em dois de seus versos secretos (\emph{apothéton epôn})
falam o seguinte sobre o Amor (\emph{Érôta}) -- o segundo verso é
excessivo (\emph{hybristikòn}) e não precisamente na métrica \mbox{---,} eles
cantam assim:

\begin{quote}
 


\begin{verse}[\versewidth]
os mortais o designam por Eros (\emph{Érôta}) alado (\emph{potênón}),\\!

e os imortais por\emph{~Ptérôta}~(Alado), pela força do brotar das asas
(\emph{pterophýtor}).\\!

 \\!
\end{verse} 
\end{quote}

 

É possível ser persuadido por estes homens, assim como é possível não
ser. Apesar disso, a causa (\emph{aitía}) e a afecção (\emph{to páthos})
para aqueles que amam (\emph{tôn eróntôn}) são estas mesmas. [252c]
Dentre os acompanhantes, o que foi tomado com mais força pode carregar o
fardo de Zeus, aquele que é denominado como alado (\emph{pterônúmou}).
Quanto aos que foram servidores de Ares e com este circularam, quando
estão tomados por Eros e consideram que foram injustiçados pelo amado,
prontificam"-se ao homicídio, querem sacrificar a si mesmos e aos seus
prediletos (\emph{tà paidiká}). E~assim, cada qual sendo coreuta para
cada deus, honra"-o e imita"-o (\emph{mimoúmenos}) na vida o quanto podem.
[252d] Durante algum tempo, por não se corromperem, vivem aqui nesta
primeira geração, sendo levados a reunirem"-se aos amantes e a outros por
esse modo. Então, cada um elege (\emph{eklégetai}) a sua maneira no que
concerne ao Amor dos belos, e, sendo aquele amado como um deus
(\emph{theòn}), [os amantes] a si mesmos fabricam
(\emph{tektaínetaí}) e adornam (\emph{katakosmeî}) uma imagem
(\emph{ágalma}), para lhes honrar e celebrar (\emph{timḕsôn te kaì
orgiásôn}). [252e] Os que acompanham alguém tal qual o divino Zeus,
buscam que a alma do seu amado seja similar a dele, daí observam se sua
natureza (\emph{phýsin}) é de filósofo (\emph{philósophós}) ou de
comandante (\emph{hêgemonikòs}), tornam"-se amantes dele quando o
encontram (\emph{heuróntes}) e tudo fazem para que permaneça assim.~Caso
anteriormente eles não tenham se empenhado nestas ocupações, logo
atentamente aprendem, a partir de onde for possível, se acercam dos seus
amados e como rastreadores eles mesmos descobrem (\emph{aneurískein}) a
natureza do deus que lhes é próprio, prosperam assim através do severo
esforço em olhar para o deus (\emph{pròs tòn theòn blépein}).~[253a]
Sendo eles apoderados pela memória (\emph{têi mnémei}), tomam, em
entusiasmo (\emph{enthousiôntes}), os hábitos (\emph{tà éthê}) e
ocupações a partir daquele [deus], tanto quanto é possível um homem
partilhar (\emph{matascheîn}) da divindade.

E essa é a causa de tanto amarem (\emph{agapôsin}) seus amantes, tanto
os que pertencem a Zeus, como as bacantes que atingem a alma do amado e
adotam a máxima semelhança (\emph{homoiótaton}) com relação ao seu deus
[Dioniso], [253b] como os que são seguidores de Hera e buscam
[um amado de] natureza real (\emph{basilikòn}), tendo"-o encontrado
(\emph{heuróntes}) fazem com ele tudo do mesmo modo. [253b] Os que
são de Apolo, bem como de cada um dos deuses, avançando com o deus,
buscam que seu amado seja de tal natureza, e, depois de o enredarem,
persuadem (\emph{peíthontes}) e disciplinam (\emph{rhythmídzontes}) o
predileto (\emph{tà paidiká}) a imitá"-lo (\emph{mimoúmenoi}),
conduzem"-no assim à ocupação e ao aspecto (\emph{idéian}) daquele
[deus] tanto quanto é possível a cada um, não por inveja ou
mesquinha hostilidade para com o predileto (\emph{tà paidiká}), mas
tentando, em tudo, levá"-lo a maior semelhança (\emph{homoióteta})
possível consigo mesmo e, portanto, com o deus honrado, assim eles
fazem. [253c] A boa vontade (\emph{prothymía}) e a iniciação
(\emph{teletḗ}) dos verdadeiros amorosos (\emph{alêthôs erṓntôn}), caso
realizem essa benevolência que digo, é bela e feliz na loucura amorosa
do amante para com o amado, se ele foi mesmo arrebatado pelo amor. O~eleito é dessa maneira tomado.

Desde o começo dessa narrativa (\emph{mýthou}) dividimos a alma em três
partes, duas delas na forma de cavalos e a terceira na forma do auriga.
Então, agora, vamos manter isso. [253d] Entre os cavalos, dissemos
que um é bom e o outro não, mas não explicamos ainda qual é a virtude do
bom e o vício do mau. Façamos isso agora. Um deles tem uma bela postura,
uma forma correta e articulada, altivo, nariz adunco, branco, olhos
negros, amante da honra (\emph{timês erastḕs}) de acordo com a
temperança e o pudor (\emph{sôphosýnês te kaì aidoûs}).~Ele é
companheiro da opinião verdadeira (\emph{alêthinês dóxês hetaîros}), ele
não insulta, obedece a um só e é conduzido pela palavra (\emph{lógoi}).
[253e] Já o outro é oblíquo, vulgar, levado ao acaso, tem pescoço
forte e curto, nariz achatado, é negro, tem olhos acinzentados, é
sanguíneo, companheiro da desmesura (\emph{hýbreôs}) e da jactância
(\emph{aladzoneías}), orelhas peludas, surdo (\emph{kôphós}), e só
obedece com dificuldade ao açoite e ao aguilhão. Então, quando o auriga
vê o olhar do amante (\emph{tò erôtikòn ómma}), sente toda alma
aquecer"-se, enchendo"-se de prurido e dos aguilhões do desejo
(\emph{póthou}). [254a] Então, o cavalo que é bem persuadido pelo
auriga e sempre constrangido pela força do pudor (\emph{aidoî}),
permanece sob seu próprio domínio e não é levado para a direção do
amado. Já o outro, nem pelo aguilhão do auriga nem pelo açoite recua,
saltando e sendo conduzido pela força. Esse oferece todo tipo de apuros
ao companheiro de jugo e ao auriga, forçando"-os na direção do predileto,
fazendo com que ele rememore (\emph{mneían}) gracejos afrodisíacos
(\emph{aphrodisíôn cháritos}).

Ambos, desde o começo, opõem"-se de modo irritadiço, [254b] uma vez
que são forçados a coisas terríveis e violentas. Mas ao final, quando
nem mesmo conseguem evitar a maldade, atravessam e seguem, agindo como
se concordassem (\emph{homologḗsante}) em fazer o que lhes foi ordenado.
Ao chegarem diante do amado e observarem a face (\emph{ópsin}) luminosa
do predileto, a memória (\emph{hê mnḗmê}) do auriga, pela visão, é
levada à natureza do belo, momento em que novamente vê (\emph{pálin
eîden}) aquela beleza, de acordo com a prudência (\emph{sôphrosýnês}),
estabelecida num sagrado pedestal.

Ao vê"-lo, ele o teme e o sente vergonha, a ponto de cair de costas, e ao
mesmo tempo o constrange ao puxar as rédeas com tanta força que ambos os
cavalos se assentam sobre os próprios quadris [254c], um por vontade
própria, sem oferecer oposição, mas o outro, rebelde (\emph{hubrístḕn}),
faz isso muito a contragosto. Chegando a um lugar mais afastado, um por
estar com vergonha e estupefato, banha toda a alma com suor, o outro,
estando apaziguado da dor, causada pelo freio e pela queda, toma fôlego
e, com ímpeto, vitupera os muitos abusos do auriga e do companheiro de
jugo, como se por timidez ou covardia eles houvessem abandonado a ordem
(\emph{táxin}) e o acordo (\emph{homologían}).~[254d] E novamente
(\emph{pálin}), não desejando ser conduzido à força, às duras penas ele,
[o cavalo negro], aceita o que lhe foi demandado, adiando sua nova
investida (\emph{hyperbalésthai}). Chegado o tempo determinado, como se
estivesse esquecido (\emph{amnêmoneîn}), ele, [o cavalo negro], é
levado à rememoração (\emph{anamimnḗiskôn}), e usando toda sua energia,
relinchando, puxa fortemente para o lado contrário, arrastando
(\emph{hélkôn}) para onde está o favorito e oferecendo"-lhe os mesmos
discursos (\emph{lógous}). Logo em seguida, quando ele se aproxima,
agacha e estica a calda, morde o freio e arrasta (\emph{hélkei}) sem
nenhum pudor (\emph{anaideías}).

[254e] O auriga sofre a maior dessas afecções (\emph{páthos
pathón}), como se estivesse impedido por uma corda, uma vez que há a
desmesura (\emph{hybristoû}) do cavalo que é arrastado com força pelo
freio dos dentes, tendo a língua maledicente e a mandíbula
ensanguentadas, aí então suas patas e sua anca são lançadas à terra e
expostas ao sofrimento. Quando esse malvado sofre todas essas coisas,
cessa sua desmesura (\emph{hýbreôs}),~submetendo"-se à condução e à
intenção (\emph{pronoíai}) do auriga, e quando vê o belo [novamente]
ele é aniquilado pelo medo (\emph{phóbôi}). A~partir daí, ocorre que a
alma do amante (\emph{erastoû}) passa a seguir o predileto com pudor e
respeito (\emph{aidouménên te kaì dediuîan}). [255a] Por conta de
todos os cuidados dispensados pelo amante, que eram similares aos
dispensados a um deus, não por que o amante fingia, mas por sentir"-se
assim de verdade, fizeram com que o próprio amado, naturalmente, se
tornasse amigo do seu servidor, mesmo que anteriormente ele tivesse sido
reprovado pelos companheiros ou por quaisquer outros que diziam ser
vergonhoso associar"-se a um amante, e que por isso tivesse repelido o
amante, mas com o passar do tempo, [255b] a idade e a necessidade
fazem com que aquele seja aceito em sua companhia. Não quis o destino
nem que o malvado fosse amigo do malvado, nem que o bondoso não fosse
amigo do bondoso. Tendo oferecido o discurso e o recebido em sua
companhia, a proximidade e a benevolência do amante provoca no amado a
sensação de perturbação, uma vez que nem outros amigos, nem familiares,
ninguém frente ao amigo entusiasmado (\emph{éntheon})~oferece parcela
alguma de amizade. E~quando por muito tempo age assim, aproximando"-se
dele para tocá"-lo nos ginásios e em outras ocasiões, [255c] aí então
[surge] a fonte (\emph{pêgḗ}) daquele fluxo (\emph{rheúmatos}), o
qual Zeus, amante de Ganimedes, denominou de desejo (\emph{hímeron}),
que chega em abundância no amante, preenchendo"-lhe e, uma vez
preenchido, transborda para o exterior. Tal qual um sopro
(\emph{pneûma}) ou algum eco (\emph{êchô}) que numa superfície lisa ou
sólida é levado novamente (\emph{pálin}) ao ponto de partida, assim o
fluxo da beleza é novamente (\emph{pálin}) direcionado ao belo, através
dos olhos (\emph{ommátôn}), por onde a alma é acessada e tem as asas
acrescidas (\emph{anapterôsan}). [255d] Então, as vias das asas são
irrigadas (\emph{pterôn árdei}), iniciando o seu brotar
(\emph{pterophyeîn}), enquanto o amor preenche (\emph{érôtos enéplêsen})
a alma do amado (\emph{erôménou}). Ele ama, mas não sabe o quê. Não sabe
o que sofre e não tem como expressar isso. Tal qual uma oftalmia
(\emph{ophthalmías})~adquirida de outrem, ele não tem como expressar a
causa, uma vez que lhe escapa (\emph{lélêthen}) que vê a si mesmo no seu
amante, como se fosse em um espelho (\emph{katóptrôi}). E~quando está
junto dele, cessa o seu sofrimento, tal qual no amante, mas quando está
separado, ele deseja e é também desejado (\emph{potheî kaì potheîtai}),
pois adquire um ídolo do amor, um Ânteros (\emph{eídôlon érôtos antérôta
échôn}).~[255e] A este nomeia e considera não como amor
(\emph{érota}), mas como amizade (\emph{philían}). O~seu desejo
(\emph{epithymeî}) é quase o mesmo daquele, só que menos intenso, o de
ver, tocar, beijar, deitar"-se ao seu lado, ações que, como é verossímil
(\emph{eikós}), não tardará a realizar. Então, quando partilha o mesmo
leito do amante, o cavalo indisciplinado (\emph{akólastos}) tem algo a
dizer ao auriga, esperando depois de todos os sofrimentos tirar"-lhe um
pequeno benefício. [256a] O predileto não tem nada a dizer e, pleno
de desejo, perplexo, acerca"-se do amante e o adora, amigavelmente
saudando a quem bem lhe quer, como quem não pode recusar os gracejos
(\emph{charísasthai}) do amante quando está a seu lado, se por acaso
(\emph{tycheîn}) ele demandar. O~companheiro de jugo se opõe a isso,
juntamente com o auriga, seguindo seu pudor e sua razão (\emph{aidoûs
kaì lógou}). E~se, por acaso, predominarem as melhores partes do
pensamento (\emph{dianoías}), as que conduzem a um regime de vida
ordenado e amante da sabedoria (\emph{philosophían}), [256b] são
felizes e conduzem uma vida de concórdia, estando eles senhores de si
(\emph{egkrateîs}) e disciplinados (\emph{kósmioi}), subjugam aquilo que
faz nascer a maldade na alma e libertam aquilo que nela gera a virtude
(\emph{areté}). Então, no fim da vida, ganham asas e leveza
(\emph{hypópteroi kaì elaphroì}), pois venceram um dos três combates
verdadeiramente olímpicos, o qual é o maior bem (\emph{agathòn}), não
podendo ser alcançado pelo homem, nem pela prudência
(\emph{sôphrosýnê}), nem pela loucura divina (\emph{theía manía}). Mas
se, pelo contrário, levarem um regime de vida mais vulgar e sem amor
pela sabedoria (\emph{aphilosóphôi}), [256c] valendo"-se do amor pela
honra (\emph{philotímôi}), então, rapidamente, nas bebedeiras
(\emph{méthais}) ou em outras ocasiões de despreocupação
(\emph{ameleíai}), os dois [cavalos] libertinos (\emph{akolástô}),
sob o mesmo jugo, tomam as almas desprevenidas (\emph{aphroúrous}),
unindo"-se ambos para o mesmo fim, escolhendo (\emph{airesin}) o que a
maioria (\emph{tôn pollôn}) toma por excelente e assim praticam.~Tendo
realizado isso, valem"-se desse comportamento em ocasiões futuras, embora
raramente isso ocorra, visto que a praticam sem a aprovação plena da
reflexão (\emph{dianoíai}). Estes são obviamente amigos, mas em menor
grau que os anteriores, [256d] e, o amor lhes é recíproco e até
mesmo depois dele acabar, (\emph{písteis}) acreditam terem oferecido e
recebido mutuamente as melhores coisas, o que torna ilícito que fiquem
apartados a ponto de serem hostis entre si. No final da vida, sem asas
(\emph{ápteroi}), mas desejosos de as terem adquirido, eles saem do
corpo, e não é pequena a sua recompensa (\emph{âthlon}) advinda dessa
loucura amorosa (\emph{erôtikês manías}). Não há uma lei que designe que
aqueles que iniciam seu trajeto sob o céu devam passar pela travessia
escura e pelo subterrâneo, mas sim que atravessem celebrando entre si
uma vida luminosa (\emph{phanòn bíon}) e feliz (\emph{eudaimoneîn}),
[256e] e que sejam agraciados pelo amor com asas semelhantes
(\emph{homópteros érôtos chárin}), quando chegar o momento de seu
surgimento. São essas então, ó jovem, as coisas divinas (\emph{theîa})
que lhes são entregues pela amizade para com o amante (\emph{par'
erastoû philía}).

A familiaridade com o não afetado pelo amor (\emph{mè erôntos}),
mesclada com a prudência (\emph{sôphrosýnêi}) mortal na administração de
bens mortais e miseráveis (\emph{thnêtá te kaì pheidôlà}), é uma
servilidade (\emph{aneleutherían}) elogiada por muitos (\emph{plḗthous
epainouménen}) como uma virtude (\emph{aretḕn}) gerada pela amizade na
alma, [257a] mas que faz com que ela gire por nove mil anos, ao
redor e debaixo da terra, num percurso sem intelecto (\emph{ánoun}).

Esta é, ó querido Eros, dentro das nossas possibilidades, a mais bela e
melhor palinódia que eu poderia oferecer"-te como pagamento, entre tantas
outras razões, mas especialmente no vocabulário~poético a que fui
forçado por Fedro. Desculpe"-me (\emph{suggnṓmên}) pelos primeiros
[discursos] e que este último o tenha agradado (\emph{chárin}), seja
para mim benévolo e propício na arte de amar (\emph{tḕn erôtikḗn
téchnên}) que me destes, que eu não seja dela subtraído nem incapacitado
pelo impulso (\emph{orgén}), e que me seja concedido ser ainda mais
honrado (\emph{tímion}) junto aos belos. [257b] E, se com os dois
primeiros discursos eu e Fedro fomos dissonantes a ti, o causador foi
Lísias, o pai do discurso (\emph{tòn toû lógoû patéra}). Então,
interrompe (\emph{paûe}) nele tais discursos (\emph{lógôn}) e o conduz
(\emph{trépson}) para a filosofia, como foi conduzido (\emph{tétraptai})
o seu próprio irmão Polemarco, para que este seu amante (\emph{erastès})
aqui não fique mais entre dois caminhos (\emph{epamphoterídzêi}),
justamente como agora, mas tenha a vida devotada somente para o Amor, de
acordo com discursos filosóficos''.

 

F: Junto minhas preces às tuas, ó Sócrates, e, se isso for o melhor para
nós, que assim seja. [257c] O teu discurso há tempo que admiro
(\emph{thaumásas}), tanto mais belo que o anterior o fizeste. Assim,
receio que Lísias me apareça inferior mesmo, especialmente se queres
(\emph{ethelêsêi}) contra ele competir (\emph{antiparateînai}).~Pois é
algo assim, ó admirável (\emph{thaumásie}), agora mesmo um dos políticos
insultava e censurava Lísias, e entre os insultos proferidos o designava
por logógrafo (\emph{logográphon}). Talvez, então, tenha sido o amor
pela honra (\emph{philotimías}) o motivo pelo qual ele se absteve de nos
escrever (\emph{gráphein}).

 

S: Engraçado, ó jovem, o parecer (\emph{dógma}) que proferes, pois sobre
o teu companheiro (\emph{hetaírou}) estás completamente enganado
(\emph{diamartáneis}), se o consideras como alguém tímido. [257d]
Talvez aquele que o insultava considerasse censurável dizer o que disse
(\emph{légein hà élegen}).

 

F: É o que parece, ó Sócrates. Tu sabes como os poderosos e
reverenciados nas cidades envergonham"-se de escrever discursos
(\emph{lógous te gráphein}) e de deixar composições suas
(\emph{kataleípein suggrámmata heautôn}), temerosos da reputação
(\emph{dóxan}) que, com o tempo, pode atingi"-los, sendo designados por
sofistas (\emph{mè sophistaì kalôntai}).

 

S: Doce rodeio, ó Fedro, mas esqueces ainda do grande rodeio mencionado
pelos que descem o Nilo. [257e] E além desse rodeio, esqueceste que
os maiores amantes da logografia (\emph{erôsi logographías}), bem como
do legado de composições escritas (\emph{kataleípseôs suggrammátôn}),
são os grandes e os mais notáveis políticos (\emph{oi mégiston
phronoûntes tôn politikôn}).~Em seguida, em discurso escrito
(\emph{gráphôsi lógon}), eles agradam aos seus panegiristas
(\emph{epainétas}), uma vez que estes são os primeiros a elogiá"-los
(\emph{epainôsin}), previamente e em qualquer situação.

 

F: Como dizes isso? Não compreendo.

 

[258a] S: Não compreendes porque os políticos, no início das suas
composições escritas, inscrevem (\emph{gégraptai}) primeiramente o nome
dos seus panegiristas (\emph{epainétês}).

 

F: Como?

 

S: ``Foi resolvido'', como ele diz, ``pelo conselho'' ou ``pelo povo'',
ou por ambos, e ao dizer ``aquele que'', refere"-se ao seu próprio
discurso, no que há de mais sagrado e elogiável (\emph{egkômiádzôn}) no
escritor (\emph{suggrapheús}). Depois de tudo isso, mostra aos seus
panegiristas (\emph{epainétais}) a sua própria sabedoria
(\emph{sophía}), por vezes redigindo composições escritas
(\emph{sýggramma}) bastante longas. Que outra coisa te parece isso,
senão uma composição de discurso escrito (\emph{lógos syggegramménos})?

 

F: Não me parece outra coisa.

 

S: Então, se for bem recebido, o poeta deixa o teatro com júbilo, se for
rejeitado, é privado da logografia e da dignidade (\emph{áxios}) de
escrever (\emph{suggráphein}), lamentando"-se ele e os seus companheiros.

 

F: E muito.

 

S: Parece que não desprezam essa ocupação, mas a admiram.

 

F: Perfeitamente.

 

S: O quê? Quando alguém vem a ser um rétor ou um rei, tal qual Licurgo,
Sólon ou Dario, não é possível que ele venha a se tornar um logógrafo
imortal da cidade? [258c] Enquanto está vivo, ele é visto como um
deus~e, depois, os subsequentes cidadãos não o considerariam da mesma
maneira, ao contemplarem suas composições escritas (\emph{suggrámmata})?

 

F: E como.

 

S: Consideras que um desses, qualquer um, com qualquer tipo de desavença
contra Lísias, poderia censurá"-lo (\emph{oneidízein}) porque escreveu
(\emph{suggráphei})?

 

F: Não é verossímil (\emph{eikós}) pelo que dizes. Pois seria, como
parece, uma censura (\emph{oneidízoi}) contra o próprio desejo
(\emph{epithymíai}).

 

[258d] S: E isso é claro para todos, que não é vergonhoso por si só
escrever discursos (\emph{grapheîn lógous}).

 

F: Como?

 

S: Considero vergonhoso falar (\emph{légein}) e escrever
(\emph{gráphein}) sem nenhuma beleza, além de uma vergonha é algo
malvado.

 

F: É claro.

 

S: Qual é então a maneira (\emph{trópos}) de escrever com beleza ou sem?
Precisamos, ó Fedro, examinar (\emph{exetásai}) esse assunto junto a
Lísias ou a qualquer outro que tenha escrito (\emph{gégraphen}) ou que
ainda vá escrever (\emph{grápsei}), seja sobre um escrito político
(\emph{politikòn súggramma}) ou um assunto particular, seja na métrica
como poeta ou sem, como um prosador?

 

[258e] F: Perguntas (\emph{erôtais}) se precisamos? Que motivo teria
alguém para viver senão em vista, por assim dizer, desses mesmos
prazeres (\emph{hêdonôn})?~Pois não são daqueles que necessitam de
sofrimento prévio (\emph{prolupêthênai}), sem o que nem mesmo o prazer
(\emph{hêsthênai}) haveria, mas estão entre os poucos (\emph{olígou})
que fornecem todos os prazeres corpóreos (\emph{sôma hêdonaì}), motivo
pelo qual, justamente, são designados por servis (\emph{andrapodṓdeis}).

 

S: Temos tempo livre (\emph{scholḕ}), como parece. Enquanto isso as
cigarras cantoras~conversam entre si nesse calor e nos observam
(\emph{kathorân}) lá de cima. [259a] Se elas nos vissem, como a
maioria, ao meio"-dia e sem dialogarmos (\emph{mḕ~dialegoménous}), quase
dormindo, encantados pela preguiça da reflexão (\emph{dianoías}), elas
justamente nos desprezariam, considerando"-nos como criaturas cativas que
chegaram a um recanto, como ovelhas, ao meio"-dia, a dormir junto à fonte
(\emph{tḕn krḗnen eúdein}). Mas se elas nos vissem a dialogar
(\emph{dialegoménous}) e a evitá"-las, como [quem evita] as Sirenas,
sem nos deixarmos encantar (\emph{akêlḗtous}), então rapidamente nos
admirariam e conceder"-nos"-iam as dádivas divinas atribuídas aos homens.

 

[259b] F: Quais são essas dádivas? Não ouvi, como parece, acerca de
nenhuma delas?

 

S: Não é adequado (\emph{prépei}) a um homem amigo das Musas
(\emph{philómouson}) não ter ouvido falar nisso. Dizem que, antes do
tempo das Musas, as cigarras eram homens e que, quando estas [Musas]
surgiram e lhes mostraram os cantos (\emph{phaineísês}~\emph{oidês}),
alguns deles foram tomados por esse prazer (\emph{hêdonês}). Envolvidos
com o canto (\emph{áidontes}), eles, sem perceber, acabaram descuidando
da comida e da bebida, sendo levados à morte. Deles é que a família das
cigarras descende, pois, junto às Musas, tendo recebido essa dádiva
(\emph{géras}), elas não têm necessidade de alimentos, mas vivem a
cantar (\emph{aidein}) ininterruptamente, sem comer e sem beber, até a
morte e, depois disso, para as Musas relatam (\emph{apaggéllein}) quais
foram aqueles que as honraram (\emph{timâi}) aqui. Terpsicore
(Alegra"-coro) é venerada (\emph{tetimêkótas}) nas danças
(\emph{choroîs}), relato que proporciona maior benevolência aos seus
realizadores.~[259d] Érato (Amorosa) com a [poesia] erótica
(\emph{erôtikoîs}) é venerada, assim também em outras ocasiões, segundo
cada forma de honra (\emph{timês}). As mais velhas delas são Calíope
(Belavoz) e em seguida Urânia (Celeste), para aqueles que se dedicam à
filosofia e que estimam (\emph{timôntas}) a música, pois especialmente
as Musas enviam bela"-voz acerca do céu, dos discursos dos deuses e dos
homens.~Muitas são as razões para que falemos ao meio"-dia e não
cochilemos.

 

F: Falemos então.

 

[259e] S: Vamos agora estabelecer uma verificação
(\emph{sképsasthai}) sobre o discurso, verifiquemos (\emph{skeptéon}) em
que medida é possível falar e escrever (\emph{légein te kaì gráphein})
de modo belo (\emph{kalôs}) ou não.

 

F: Claro.

 

S: Não é necessário àqueles que desejam falar bem e de modo belo, que o
pensamento (\emph{diánoian}) de quem fala conheça a verdade
(\emph{eiduîan tò alêthès}) acerca do que será tratado?

 

[260a] F: Acerca disso ouvi o seguinte, ó querido Sócrates: aquele
que deseja tornar"-se rétor não necessita compreender (\emph{manthánein})
o que é verdadeiramente justo (\emph{tôi ónti díkaia}), mas o que parece
ser para aqueles muitos (\emph{tà dóxant' an plḗthei}), nem o verdadeiro
bom e belo, mas o que lhes parecer assim (\emph{tà óntos agathà ê kalà
all' hósa dóxei}). Disso deriva a persuasão (\emph{peíthei}), e não da
verdade (\emph{alêtheías}).

 

S: ``Palavras nada desprezíveis'', ó Fedro, essas que os sábios
(\emph{sophoí}) proferem, mas vamos examinar (\emph{skopeîn}) se elas
nos dizem algo. Certamente o que foi dito não deve ser abandonado.

 

F: Dizes bem.

 

S: Examinemos.

 

F: Como?

 

[260b] S: Se eu quisesse convencer"-te e ajudá"-lo na aquisição
(\emph{ktêsámenon}) de um cavalo de combate, ambos desconhecendo
(\emph{agnooîmen}) o que é um cavalo, mas, se alguma coisa, entretanto,
eu soubesse sobre você, que Fedro considera que ele é o animal doméstico
que tem a maior orelha.

 

F: Seria engraçado, ó Sócrates.

 

S: Nem tanto. Mas na ocasião de ocupar"-me da tua persuasão
(\emph{peíthoimi}), colocando o discurso elogioso (\emph{épainon}) no
asno, designando"-o por cavalo, falando acerca de todas as qualidades da
criatura no uso doméstico, na aquisição, na guerra, defendendo sua
utilidade (\emph{ôphélimon}) com bagagens e outras tantas tarefas.

 

[260c] F: Isso seria realmente engraçado.

 

S: Mas então não seria melhor o engraçado (\emph{geloîon}) e o amistoso
(\emph{phílon}) do que o terrível (\emph{deinón}), ou o hostil
(\emph{echthròn})?

 

F: Parece.

 

S: Mas, quando o rétor desconhece (\emph{agnoôn}) o bom (\emph{agathòn})
e o mau (\emph{kakón}), tomando uma cidade pela a persuasão
(\emph{peíthêi}), não faria um elogio (\emph{épainon}) da sombra
(\emph{skiâs}) de um asno como se fosse de um cavalo, mas elogiaria o
mau como sendo o bom, e, exercitado na opinião da maioria (\emph{dóxas
dè plḗthous memeletêkòs}), ele poderia persuadi"-los (\emph{peísêi}) a
fazer o mau e não o bom. Considerando isso tudo, que tipo de fruto
(\emph{karpòn}) a retórica poderia colher dessa semeadura?

 

F: Um fruto não muito agradável.

 

S: Então, ó bondoso, fomos mais grosseiros que o necessário ao
detratarmos a arte dos discursos (\emph{lógôn téchnên})? E ela talvez
nos dissesse: ``Ó admiráveis, porque dizeis tais bobagens? Eu não obrigo
ninguém que desconheça a verdade a aprender a falar (\emph{agnooûnta
talêthès anagkádzô manthánein légein}), mas, se em algo vale o meu
conselho (\emph{sumboulḗ}), que adquiram (\emph{ktêsámenon}) aquela
[verdade] antes de me tomar (\emph{lambánein}). Eis então o que digo
veementemente: que, sem mim, aquele que conhece a verdade nunca
alcançará a arte de persuadir (\emph{peíthein téchnei}).''

 

[260e] F: E ela não diria coisas justas ao proferir isso?

 

S: É o que digo, se os discursos apresentados testemunham que ela é uma
arte. É~como se eu ouvisse a aproximação de alguns contestadores da arte
do discurso a dizer que ela é falsa (\emph{pseúdetai}), que ela não é
uma arte, mas uma ocupação isenta de arte (\emph{ouk esti téchne alla
atechnos tribé}). Os Lacônios afirmam que ``não existe uma arte
verdadeira (\emph{étymos téchnê}) sem estar atada à verdade
(\emph{alêtheías}), nem mesmo poderá existir no futuro.''

 

 

[261a] Precisamos desses discursos, ó Sócrates, traze"-os agora para
junto de nós para examinarmos (\emph{exétaze}) o quê e como eles falam.

 

S: Vinde, nobres criaturas, persuadi (\emph{peíthete}) Fedro de belos
filhos de que, quem não filosofar suficientemente (\emph{ikanôs
philosophḗsêi}), não será nunca capaz de falar sobre coisa alguma.
Responde agora Fedro.

 

F: Pergunta.

 

S: Então, o todo da retórica não seria a arte da condução das almas
(\emph{téchnê}~\emph{psychagôgía}) por meio das palavras (\emph{dià
lógôn}), não só nos tribunais (\emph{dikastêríois}) e em outras
assembleias públicas (\emph{dêmósioi súllogoi}), mas também nas questões
particulares (\emph{idíois}), naquelas insignificantes e nas grandiosas,
e que não há nada de mais honrado (\emph{entimóteron}) que o seu
emprego, quando correto, seja nos assuntos sérios ou nos banais?
[261b] Ou como ouviste falar disso tudo?

 

F: Não, por Zeus, não foi assim absolutamente, mas que especialmente nos
tribunais (\emph{dikás}) fala"-se e escreve"-se com arte, bem como nas
assembleias públicas (\emph{dêmêgorías}). Não ouvi mais do que isso.

 

S: Mas então apenas ouviste sobre as artes discursivas de Nestor e de
Odisseu, as quais foram escritas (\emph{sunegrapsátên}) por eles em
Troia, nas horas vagas (\emph{scholázontes}), e nem mesmo chegaste a
ouvir aquela composta por Palamedes?

 

[261c] F: Por Zeus, nem mesmo ouvi a de Nestor, a não ser que
consideres Górgias uma espécie de Nestor, ou Trasímaco e Teodoro
distintos tais quais Odisseu.

 

S: Talvez, mas deixemos estes aí por hora. E~tu dize"-me o que fazem
(\emph{tí drôsin}) os que disputam nos tribunais (\emph{dikastêríois oi
antídikoi}), eles não entram em litígio (\emph{antilégousin})? Ou o que
diremos?

 

F: Isso mesmo.

 

S: Acerca do justo e do injusto (\emph{dikaíou te kaì adíkou})?

 

F: Sim.

 

S: Então, quem lançar mão dessa atividade com arte fará as mesmas coisas
parecerem justas às mesmas pessoas, e, por outro lado, quando quiser
(\emph{boúlêtai}), parecerem injustas?

 

[261d] F: O que tem isso?

 

S: E, também, nas assembleias públicas (\emph{dêmêgoríai}) da cidade,
fará parecer as mesmas coisas ora boas ora o seu contrário
(\emph{tanantía})?

 

F: É assim.

 

S: Então não conhecemos os dizeres com arte do eleático Palamedes, por
meio do qual mostrava aos ouvintes as mesmas coisas como semelhantes e
dessemelhantes (\emph{hómoia kaì anómoia}), unas e múltiplas (\emph{hén
kaì pollá}), em repouso e em movimento (\emph{ménontá te aû kaì
pherómena})?

 

F: E como!

 

[261e] S: Então, não só no tribunal e nas assembleias públicas
existe a antilogia (\emph{antilogikḕ}), mas, como parece, em todas as
coisas que são ditas há uma só arte (\emph{mía tis téchnê}), se é que
existe, aquela que é capaz de assemelhar tudo a todas as coisas
possíveis (\emph{pân pantì homoíoûn tôn dynatôn}), na medida do
possível, e também de trazer à luz (\emph{eis phôs agein}) o que outros,
operando essas semelhanças (\emph{homoioûntos}), tentam dissimular
(\emph{apokruptôménou}).

 

F: Como é que dizes isso?

 

S: Procuro mostrar"-te isso que buscamos. O~engano nasce
predominantemente naquilo que difere muito ou pouco?

 

[262a] F: No que difere pouco.

 

S: Mas então seria melhor para transportar às ocultas (\emph{metabaínôn
mâllon lḗseis}), conduzindo os discursos ao seu contrário, guiar"-se pelo
que difere pouco (\emph{smikròn}) ou, ao contrário, pelo que difere
muito.

 

F: Como não seria assim?

 

S: É necessário àquele que se prontifica a enganar (\emph{apatḗsein})
outros, sem enganar a si mesmo (\emph{autòn dè mḕ~apatḗsesthai}), que
conheça exatamente as semelhanças e as dessemelhanças entre os seres
(\emph{tèn homoiótêta tôn óntôn kaì anomoiótêta akribôs dieidénai}).

 

F: É necessário.

 

S: E será possível a este mesmo homem, desconhecendo a verdade,
reconhecer as semelhanças (\emph{homoiótêta}) menores e maiores em
outros seres?

 

[262b] F: Impossível.

 

S: Dessa forma, para os que opinam contra [a existência] dos seres e
são enganados, é evidente como sua afecção (\emph{páthos}) foi arrastada
por aquelas semelhanças (\emph{di'homoiotḗtôn}).

 

F: É assim mesmo que acontece.

 

S: Como o artífice (\emph{technikòs}) irá transladar
(\emph{metabibázein}), seguindo as menores semelhanças
(\emph{homoiotḗtôn}) entre os seres, levando cada um deles ao seu
contrário (\emph{tounantíon})? E como ele poderia esquivar"-se
(\emph{diapheúgein}) desse mesmo efeito sem reconhecer o que são cada um
dos seres (\emph{hékaston}~\emph{tôn óntôn})?

 

F: Não poderia.

 

[262c] S: Então a arte do discurso (\emph{lógôn téchnên}), ó
companheiro, sem o conhecimento do que é verdadeiro (\emph{tèn alétheian
mè eidôs}), é como uma caça das opiniões (\emph{dóxas dè tethêreukós}),
ocupação risível e, como bem parece, desprovida de arte (\emph{átechnon
paréxetai}).

 

F: É bem possível.

 

S: Sobre esse discurso de Lísias que trazes consigo ou esses que
pronunciamos em seguida, pretendes observar o que neles há desprovido de
arte (\emph{atéchnôn}), bem como o que está de acordo com a arte
(\emph{entéchnôn})?

 

F: É o melhor a fazer, especialmente porque até agora só falamos no
vazio, sem paradigmas suficientes.

 

S: E foi por sorte (\emph{tychên}), como bem parece, que nós temos dois
discursos como paradigmas, o que mostra que aquele que conhece a verdade
(\emph{eidôs tò alêthès}), brincando com as palavras (\emph{prospaídzôn
en lógois}), pode demover os ouvintes. [262d] E eu, ó Fedro, atribuo
isso aos deuses desse lugar (\emph{entopíous theoús}). Talvez tenham
sido as cigarras, intérpretes das Musas, que, sobre nossas cabeças,
cantam e inspiram"-nos essa honra, pois eu não partilho (\emph{métochos})
de nenhuma arte no meu discurso (\emph{téchnês toû légein}).

 

F: Que assim seja, mas apenas mostra o que dizes.

 

S: Lá vai então. Lê"-me o início do discurso de Lísias.

 

[262e] F: ``Já estás ciente acerca dos meus assuntos e creio que
ouviste acerca do que pode acontecer conosco. Espero que não me advenha
nenhum infortúnio (\emph{atychêsai}) só porque me ocorreu
(\emph{tygcháno}) de não estar te amando (\emph{ouk}~\emph{erastès}),
como aqueles que \redondo{[…]} arrependem"-se (\emph{metamélei})
\redondo{[…]}''.

 

S: Para. Tratemos agora do que ele errou (\emph{hamartánei}) e no que
procedeu sem arte (\emph{átechnon}), não é mesmo?

 

[263a] F: Sim.

 

S: Mas isso não é evidente para todos, que acerca de algumas coisas nós
concordamos (\emph{homonoêtikôs}) e de outras discordamos
(\emph{stasiôtikôs})?

 

F: Perece que entendo o que dizes, mas explica ainda de modo mais claro.

 

S: Quando dizes um nome como ferro (\emph{sidḗrou}) ou prata
(\emph{argúrou}), não entendemos (\emph{dienoḗthemen}) todos nós a mesma
coisa?

 

F: E como.

 

S: E quanto ao justo ou ao bom? Não ocorre que nos dirijamos uns para um
lado e outros para outro, fazendo como que entremos em controvérsia
mútua e até conosco mesmo (\emph{amphisbêtoûmen allélois te kaì hêmîn
autoîs})?

 

F: É assim mesmo.

 

[263b] S: Então, existem coisas a respeito das quais nós chegamos a
um acordo (\emph{sumphônoûmen}) e outras não?

 

F: De fato.

 

S: Em quais delas é mais fácil nos enganarmos e em qual delas a retórica
(\emph{rhetorikḕ}) tem maior poder (\emph{meîdzon dýnatai})?

 

F: É evidente que naquelas em que nós somos errantes
(\emph{planṓmetha}).

 

S: Então, para aquele que deseja seguir a arte retórica (\emph{téchnên
rhetorikḕn}), primeiramente, é preciso que diferencie esses dois
caminhos e que detecte os caracteres de cada um deles, onde
necessariamente a multidão erra
(\emph{to}~\emph{plêthos}~\emph{planâsthai}) e onde não.

 

[261c] F: Belo seria, ó Sócrates, deter essa forma de apreensão
(\emph{katanenoêkṑs}).

 

S: Em seguida, creio que não devemos deixar de observar cada um dos
assuntos que surgem, mas percebê"-los com agudeza (\emph{oxéôs
aisthánesthai}), bem como o gênero daquilo que falaremos.

 

F: Sem dúvida.

 

S: E o que diremos do amor? É algo controverso (\emph{amphisbêtêsímôn})
ou não?

 

F: Presumo que seja algo controverso (\emph{amphisbêtêsímôn}), ou
consideras ser possível admitir o que há pouco disseste acerca dele, que
é danoso (\emph{blábê}) ao amante e ao amado (\emph{tôi erôménôi kaì
erônti}) e, logo depois, que é o maior dos bens ocorridos (\emph{agathôn
tugchánei})?

 

[263d] S: Dizes muito bem, mas também diz"-me isso -- pois, pelo meu
entusiasmo (\emph{enthousiastikòn}), não me lembro bem (\emph{ou pánu
mémnêmai}) \mbox{---,} se defini o amor no início do discurso.

 

F: Sim, por Zeus, e com extraordinária precisão.

 

S: Ah! Proclamas superiores na arte (\emph{technikôtéras}) as Ninfas,
filhas de Aqueloo, e Pã, filho de Hermes, comparados a Lísias, filho de
Céfalo, em seu discurso. Talvez eu esteja errado, mas não é verdade que
Lísias, no início do seu discurso erótico (\emph{toû erôtikoû}),
forçou"-nos a entender o amor tal qual ele desejou (\emph{eboulḗthê}), e
a partir disso compôs (\emph{suntaxámenos}) tudo o que veio depois,
levando o discurso a seu termo? [263e] Queres novamente que leiamos
o seu começo (\emph{boúlei pálin anagnômen tèn archèn autoû})?

 

F: Se te parece conveniente. Mas o que procuras (\emph{dzêteîs}) não
está aí.

 

S: Diz, para que eu possa ouvi"-lo dele mesmo.

 

F: ``Já estás ciente acerca dos meus assuntos e creio que ouviste acerca
do que pode acontecer conosco. Espero que não me advenha nenhum
infortúnio (\emph{atychêsai}) só porque me ocorreu (\emph{tygcháno}) de
não estar te amando (\emph{ouk}~\emph{erastḕs}), [264a]como aqueles
que tão logo tenha cessado o seu desejo (\emph{epithymías paúsôntai}),
arrependem"-se (\emph{metamélei}) do que bem fizeram.''

 

S: Falta muito ainda, ao que parece, para que ele realize isso que
procuramos. Ele nem começa pelo começo, mas pelo final, empreendendo seu
discurso como alguém que nada de costas e para trás, iniciando pelas
coisas que o amante diria ao seu predileto somente no final. Ou não é
como digo, Fedro, querida cabeça.

 

[264b] F: É assim mesmo, ó Sócrates, uma peroração (\emph{teleutḗ})
em torno da qual é realizado o discurso.

 

S: E quanto ao resto? Não perece que foi lançado indiscriminadamente no
discurso? Ou o que veio depois do discurso deveria ser de fato colocado
depois por alguma necessidade, ou alguma outra coisa entre as que foram
ditas? Pois a mim me parece, como não sei de nada, que não é vil o que
foi proferido pelo escritor (\emph{tôi gráphonti}). E~tu conheces alguma
necessidade logográfica pela qual ele dispôs assim o discurso de modo
sucessivo, uma coisa ao lado da outra?

 

F: És muito gentil, uma vez que me considera suficientemente capaz de
assim discerni"-lo com precisão (\emph{akribôs diideîn}).

 

[264c] S: Mas te considero capaz de mostrar isso, que é necessário
que todo discurso esteja combinado como um ser vivo (\emph{hṓsper
dzôion}), tendo corpo próprio, não sendo acéfalo nem ápodo, e que tenha
tronco e membros convenientes entre si e com relação ao todo do escrito
(\emph{tôi hólôi gegramména}).

 

F: Como não?

 

S: Verifica (\emph{sképsai}) esse discurso do teu companheiro, seja ele
assim ou de outra maneira. Não encontras\emph{~}(\emph{heurḗseis}) no
escrito nada de diferente do epigrama (\emph{epigrámmatos}) da tumba de
Midas da Frígia, segundo alguns relatos escritos
(\emph{epigegráphthai}).

 

F: Como é e do que ele trata?

 

S: É assim:

 

\begin{quote}
Eu sou a virgem de bronze que jaz sobre a tumba de Midas,

enquanto a água fluir e grandes árvores florescerem,

eu permaneço sobre este túmulo tão chorado,

e anuncio aos que passam que Midas está aqui sepulto.
\end{quote}

 

Suponho que percebestes como não há diferença entre o que vem dito antes
ou depois.

 

[264e] F: Tu zombas do nosso discurso, ó Sócrates!

 

S: Deixemo"-lo então para não te irritar. Ainda que ele me pareça um
exemplo (\emph{paradeígmata}) àqueles que podem observá"-lo com algum
proveito, sem, contudo, imitá"-lo na performance, mas vamos para outros
discursos, pois neles há algo, como me parece, que diz respeito aos que
querem conhecer (\emph{ideîn}) e examinar discursos (\emph{perì lógôn
skopeîn}).

 

[265a] F: A que tipo de coisas te referes?

 

S: Meus dois discursos eram como que opostos (\emph{enantíô}), pois um
dizia que é necessário agraciar (\emph{charídzesthai}) ao amoroso
(\emph{tôi erônti}) e o outro ao que não é amoroso.

 

F: E com que virilidade.

 

S: Considerei que tu dirias ``com que loucura'' (\emph{manikôs}), que
seria o termo verdadeiro (\emph{talêthès}). Era ele que de fato eu
procurava. Pois dizemos ser o amor (\emph{erôta}) uma loucura
(\emph{manían}) ou não?

 

F: Sim.

 

S: Mas há duas (\emph{eîde dúo}) espécies de loucura (\emph{manías}): a
que afeta os homens como uma enfermidade (\emph{nosêmátôn}) e a que os
transporta (\emph{exallagês}) das normas habituais sob a influência da
divindade (\emph{hypò theías}).

 

[265b] F: Exato.

 

S: No que diz respeito aos deuses, são quatro as divindades e quatro as
partes pelas quais foram divididas. A~Apolo atribui"-se a inspiração da
adivinhação (\emph{mantikḕn}), a Dioniso as iniciações
(\emph{telestikḗn}), às Musas a poética (\emph{poêtikḗn}), e a loucura
amorosa (\emph{erôtikḕn manían}), a quarta, que dizemos ser a melhor
(\emph{arístên}), atribui"-se à Afrodite e ao Amor. E~não sei como, mas
ao apresentar (\emph{apeikádzontes}) a afecção amorosa (\emph{erôtikòn
páthos}), talvez por atingirmos algo verdadeiro (\emph{alethoûs tinos}),
talvez por chegarmos a outros lugares, forjamos um discurso não
totalmente isento de força persuasiva (\emph{apíthanon}) e celebramos
com uma espécie de hino, em algo mítico, [265c] bem medido e
respeitoso (\emph{metríôs te kaì euphḗmôs}) em nome desse meu e teu
senhor, o Amor, ó Fedro, o guardião dos belos jovens.

 

F: E a audição não me desagradou.

 

S: Captemos a partir disso como o discurso muda (\emph{metabênai}) do
vitupério (\emph{pségein}) ao elogio (\emph{epaineîn}).

 

F: Como dizes?

 

S: Parece"-me tudo isso ser uma brincadeira de criança, mas nessas
afirmações proferidas ao acaso (\emph{ek týchês}) há dois aspectos, e
não será desagradável se deles pudermos captar algo ligado à arte
(\emph{téchnêi}).

 

[265d] F: Quais deles?

 

S: Levar a uma só ideia, a uma visão de conjunto (\emph{sunorônta}), as
muitas coisas que estão dispersas (\emph{diesparména}), para que se
possa tornar evidente (\emph{dêlon}), pela definição
(\emph{horizómenos}), cada tema, sempre que pretendemos ensinar
(\emph{didáskein}), como agora mesmo foi feito com o Amor -- que foi
definido \mbox{---,} quer tenha sido bem ou mal definido, e que proporcionou, ao
mencionarmos o discurso, certa clareza (\emph{saphès}) e concordância
(\emph{homologoúmenon}) consigo mesmo.

 

F: E o outro aspecto (\emph{eîdos}) de que falas, ó Sócrates?

 
