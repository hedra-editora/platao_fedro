
\textbf{Platão} (Atenas, 428/427~a.C.--347~a.C.) foi filósofo e matemático grego, considerado um dos principais pensadores do período clássico da Grécia Antiga. Nascido em família abastada e envolvida na política ateniense, Platão teve uma educação conforme cabia aos jovens aristocratas de sua época: estudou retórica, poesia, leitura, escrita, matemática, pintura, música e ginástica. Tencionava ingressar na carreira política mas ainda em tenra idade tornou"-se discípulo de Sócrates, que o iniciou na filosofia e com quem conviveu durante oito anos. Depois da condenação à morte de seu mestre, acusado de corromper a juventude e de não acreditar nos deuses, Platão desiludiu-se da carreira política, opôs-se à democracia ateniense e abandonou sua terra natal, dedicando-se inteiramente aos estudos filosóficos, redigindo os diálogos e discursos de Sócrates, que morrera sem registrar nada por escrito. Viajou para Megara, Egito, Cyrene, Crotona, reunindo conhecimentos da metafísica, da matemática e da astronomia, estabelecendo também contato com os discípulos de Pitágoras, angariando saberes para a formação de sua própria escola teórica. Retornou a Atenas por volta de 387~a.C., quando funda sua Academia, onde recebia jovens a quem lecionava filosofia, ciências, matemática e geometria. Em pouco tempo, sua escola tornou"-se um dos maiores centros culturais da Gŕecia, recebendo alunos ilustres como Aristóteles, Ésquines, Demóstenes, Eudoxo de Cnido, entre outros grandes nomes da Antiguidade ocidental. Platão escreveu por mais de cinquenta anos de sua vida -- sua forma literária, por excelência, é o diálogo. O conjunto de sua obra compreende em torno de trinta diálogos, principalmente centrados em Sócrates, por meio dos quais transmite seus saberes filosóficos, versando sobre diversos temas como a retórica, a virtude, a amizade, o amor, a mentira, a natureza humana, a sabedoria, as ciências, a política, as leis etc. 

\textbf{Fedro} é um diálogo entre Sócrates e o próprio Fedro, por meio do qual Platão expõe três discursos acerca do amor: a oratória de Lísias, grande mestre da retórica, e outros dois discursos de Sócrates. [continua]


\textbf{Rogério de Campos} 

