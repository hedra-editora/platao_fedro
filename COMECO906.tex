\chapterspecial{Começo}{}{}
 

[227a] S: Ó querido Fedro, de onde vens e para onde vais?

 

F: Venho de junto de Lísias, o filho de Céfalo, ó Sócrates. Atravessei o
passeio por fora dos muros (\emph{éxô}~\emph{teíchous}) porque permaneci
sentado durante muito tempo, desde cedo. Persuadido pelo teu e meu amigo
Acúmeno, percorri o passeio pelas estradas (\emph{hodoùs}), pois,
segundo ele, esse caminho é menos cansativo que o realizado pelas vias
(\emph{drómois}) do pórtico.

 

[227b] S: É belo o que dizes companheiro. E~parece que Lísias estava
na cidade.

 

F: Sim, com Epícrates, na casa de Morico, aquela próxima ao templo de
Zeus Olímpico.

 

S: E qual era o debate (\emph{diatribḗ})? É certo que Lísias vos servia
um banquete (\emph{eistía}) de discursos!

 

F: Saberás, se estás livre para prosseguir comigo e escutar.

 

S: O quê? Não sabes que para mim, como diz Píndaro, ``não há nada mais
elevado''~que escutar o debate (\emph{diatribḕn}) que tiveste com
Lísias!

 

F: Avança, então.

 

[227c] S: Ao discurso.

 

F: Sócrates, de fato a audição interessa"-te, o discurso acerca do qual
debatíamos (\emph{dietríbomen}) versava, não sei com que precisão,
acerca do amor (\emph{erôtik\gr{ό}s}). Lísias escreveu (\emph{gégraphe}) a
respeito da sedução (\emph{peirṓmen\gr{ό}n}) dos belos, mas não dos que estão
sob a tutela de um amante (\emph{ouch hyp'erastoû dé}), e aí mesmo
reside a sua habilidade, pois Lísias diz ser melhor agradar
(\emph{charistéon}) o não afetado pelo amor (\emph{mḕ~erônti}) do que um
amante (\emph{erônti}).

 

S: Ó nobre! Espero que ele possa escrever ser melhor a pobreza frente à
riqueza, a velhice frente à mocidade, sem contar outras coisas que
comigo e a muitos de nós acontecem.~Nesse caso, os discursos seriam
agradáveis (\emph{asteîoi})~e úteis ao povo
(\emph{dêmôpheleĩs}).~[227d] Então desejo escutá"-lo, mesmo que tu
percorras o passeio até Mégara, segundo os preceitos de Heródico,
chegando até os muros e dali novamente (\emph{pálin}) retornando pelo
mesmo trajeto, nem mesmo assim eu te abandonaria.

 

F: O que dizes, ó querido Sócrates? O que Lísias compôs durante muito
tempo e com empenho (\emph{en pollôi chr\gr{ό}nôi katà scholḕn}), [228a]
ele que é hoje um dos mais hábeis (\emph{deinótatos}) a escrever
(\emph{grapheîn}), considera"-me capaz de lembrar
(\emph{apomnêmoneúsein}) dignamente disso, eu que sou um simplório?
Falta"-me muito ainda. Na verdade, almejaria mais isso do que ter muito
ouro.

 

S: Ó Fedro, se não conheço Fedro, estaria esquecido de mim mesmo, mas
não é nada disso. Bem sei que, tendo ouvido o discurso de Lísias
(\emph{Lysíou lógon akoúôn}), não só uma, mas muitas vezes, tu
remontavas seus dizeres (\emph{légein}), persuadido de boa vontade
(\emph{epeítheto prothúmōs}). Mas isso não [228b] era ainda o
suficiente. Aposto que tomavas o livro, passando a investigar o que mais
te interessava (\emph{epethýmei}), e, tendo feito isso desde cedo,
tratou de repeti"-lo ao redor do passeio (\emph{perípaton}), como bem
sei, e, pelo cão, decoraste o tal discurso, isso se não era muito
longo.~Depois de ter atravessado por fora dos muros (\emph{ektòs
teíchous}) para exercitar"-te, encontraste aquele que é doente para ouvir
discursos (\emph{nosoûnti peri lógôn akoḗn}) e que, ao vê"-lo, alegra"-se
porque terá um companheiro coribântico que o ordenará prosseguir. Sendo
obrigado a falar pelo [228c] amante dos discursos (\emph{lógon}),
ficas enternecido~(\emph{ethrúpteto}) como se não desejasses falar
(\emph{hos de ouk epithymôn légein}), mas, no final das contas,
falarias, mesmo que à força, ainda que não houvesse alguém para ouvir"-te
voluntariamente. Então, ó Fedro, obriga"-te a fazer imediatamente o que
certamente farias, em seguida, de qualquer modo.

 

F: Na verdade, é muito melhor que eu possa falar (\emph{légein}) assim,
mas parece também que tu, seja como for, não me largarias antes que eu
falasse.

 

S: Verdadeira é a tua impressão.

 

[228d] F: Assim farei. Quanto ao próprio [discurso], Sócrates,
eu não conheço completamente suas palavras (\emph{ta}~\emph{rhémata}).
Pretendo expor o pensamento (\emph{diánoian}) quase completo, que difere
o amante do não afetado pelo amor, em cada um dos seus pontos capitais
(\emph{kephalaíois}), começando pelo primeiro.

 

S: Então, primeiramente mostra, ó querido, o que trazes na mão esquerda,
debaixo do manto. Desconfio de que tenhas o próprio discurso (\emph{tòn
lógon autón}). [228e] Se é assim, entendas uma coisa a meu respeito,
que eu não me prestaria ao teu exercício estando junto de Lísias
(\emph{paróntos dè kaì Lysíou}), logo eu que o estimo
(\emph{philô})~tanto. Vai logo, mostra.

 

F: Para! Retiraste a minha esperança, ó Sócrates, de exercitar"-me
contigo.~Onde desejas tomar assento para lermos?

 

[229a] S: Desviemo"-nos daqui, seguindo a direção do Ilisso, para nos
sentarmos em algum lugar que te pareça tranquilo.

 

F: Oportunamente (\emph{eis kairón}), ao que parece, tive a sorte
(\emph{étychon}) de estar descalço hoje. Quanto a ti, Sócrates, sempre
estás assim.~Então, será facílimo, e nada desagradável, molharmos nossos
pés por esse fio d'água, especialmente nesta época do ano e nesta hora
do dia.

 

S: Prossegue e busca o lugar no qual descansaremos.

 

F: Vês aquele elevadíssimo plátano?\\\\

S: O que há nele?

 

[229b] F: Uma sombra, uma brisa moderada (\emph{pneûma métrion}),
uma relva para nos sentarmos e reclinarmos se quisermos
(\emph{boulṓmetha}).

 

S: Prossegue então.

 

F: Diga"-me, ó Sócrates, não é desse lugar do Ilisso que contam ter
Bóreas raptado (\emph{harpásai}) Orítia?

 

S: Dizem.

 

F: Então é aqui! A água parece agradável, pura e diáfana, própria às
donzelas que brincam (\emph{paídzein}) nessas margens.

 

[229c] S: Não é aqui, mas dois ou três
estádios\textsuperscript{~}abaixo, onde atravessamos na direção de Agra,
onde há um altar para Bóreas.

 

F: Não lembro, mas diz, por Zeus, ó Sócrates, estás persuadido
(\emph{peíthêi})~de que esse mitologema seja verdadeiro?

 

S: Se eu fosse um incrédulo (\emph{apistoíēn}), como fazem os sábios
(\emph{sophoí}), não seria nada extravagante (\emph{átopos}). Em
seguida, diria, com ar sofisticado (\emph{sophizómenos}), que ela foi
arrebatada das proximidades das pedras, lá debaixo, pelo sopro
(\emph{pneûma}) de Bóreas, quando brincava com Farmaceia. Alguns dizem
que ela teria morrido pelo rapto (\emph{anárpaston}) de Bóreas -- ou que
isso ocorrera no Areópago \mbox{---,} [229d] pois contam também essa outra
versão, de que ela foi (\emph{hērpásthē}) raptada de lá e não daqui.~Eu
considero graciosas essas coisas, mas uma ocupação terrível, laboriosa e
própria a homens não muito felizes. E~não por outra razão, senão por nos
obrigar a restaurar, necessariamente, a forma dos Hipocentauros, da
Quimera, de uma turba de Górgonas, Pégasos e muitos [229e] outros
seres formidáveis, por conta da extravagância (\emph{atopiai}) dessas
naturezas monstruosas. E~se alguém, entre os incrédulos
(\emph{apistôn}), conduzisse cada um deles à verossimilhança
(\emph{eikòs}), valendo"-se de uma sabedoria (\emph{sophíai}) grosseira,
precisaria de um bom tempo livre (\emph{scholés}). Eu não tenho nenhum
tempo livre (\emph{scholé}) para essas coisas e a causa disso, querido,
é que não fui ainda capaz de conhecer a mim mesmo
(\emph{gnônai}~\emph{emautón}), de acordo com a inscrição
délfica.~Pareceria risível [230a], ainda ignorante de mim mesmo, que
eu examinasse alguma outra coisa. Esse é o motivo pelo qual me agrada
renunciar a tudo, como disse agora mesmo, convencido
(\emph{peithómenos}) do que foi considerado sobre o tema. Nesse caso,
não examino (\emph{skopô}) essas coisas em outro lugar senão em mim
mesmo (\emph{emautón}), quer seja uma fera mais complexa e orgulhosa
(\emph{epitethymménon}) que Tifon (\emph{Tuphônos}), quer seja o animal
mais doméstico e simples, partícipe de uma natureza em algo divina e sem
nenhum orgulho (\emph{atúphou}). A~propósito, companheiro, não é esta a
árvore para a qual nos trazias?

 

[230b] F: A própria.

 

S: Por Hera, que bela pousada!~Um plátano corpulento e magnífico, com
uma folhagem excelente e digna desse ambiente sagrado. O~vigor da
floração oferece ao lugar o melhor dos aromas e a fonte agradabilíssima
sob o plátano nos traz água bem fria, como comprovamos com os pés.
Parece ser um templo para alguma Ninfa e para Aqueloo, a julgar pelas
estátuas votivas de argila e imagens de mármore (\emph{korôn te kaì
agalmátōn}).~[230c] Se desejas algo mais, há ainda uma doce e muito
agradável brisa (\emph{eúpnoun}) do lugar, estival e melodiosa, que ecoa
o coro das cigarras (\emph{tettígōn chorôi}). A~relva é certamente o
maior dos requintes (\emph{kompsótaton}), porque a escarpa suave
naturalmente solícita que reclinemos nossas cabeças de modo maravilhoso.
Assim, ó querido Fedro, és o melhor dos guias para estrangeiros
(\emph{exenágētai}).

 

F: Parece"-me muito extravagante (\emph{atopṓtat\gr{ό}s}), ó admirável. Pelo
que dizes, sem nenhum artifício (\emph{atéchnôs}), dá"-me a impressão de
seres guiado como um estrangeiro e não como um autóctone. [230d]
Parece que jamais te ausentaste da cidade rumo à terra estrangeira, nem
mesmo saíste para além dos muros (\emph{éxô teíchous}).

 

S: Perdoa"-me, ó excelente, é que eu sou um amante do aprendizado
(\emph{philomathḕs}), nem os campos nem as árvores querem me ensinar,
somente os homens da cidade.~E tu, realmente, pareces ter encontrado
(\emph{hēurēkénai}) o fármaco do meu êxodo.~Tal como os que agitam um
ramo para uma criatura faminta, ou algum fruto que os conduza, tu, do
mesmo modo, estendendo discursos provenientes de livros, parece que me
conduzirás por toda a Ática ou para qualquer outro lugar que queiras.
[230e] Agora, tendo chegado aqui, vou reclinar"-me e encontrarás a
posição que te seja mais cômoda à leitura. Depois disso, lê.

 

F: Escuta:

 

``Já estás ciente acerca dos meus assuntos e creio que ouviste acerca do
que pode acontecer conosco.~[231a] Espero que não me advenha nenhum
infortúnio (\emph{atychêsai}) só porque me ocorreu (\emph{tygchánō}) de
não estar te amando (\emph{ouk}~\emph{erastḕs}), como aqueles que, tão
logo tenha cessado o seu desejo (\emph{epithymías paúsōntai}),
arrependem"-se (\emph{metamélei}) do que bem fizeram. Outros, por outra
parte, não tem tempo hábil para mudar o pensamento (\emph{metagnônai}),
e não é por coação, mas espontaneamente, que bem fazem o que podem ao
amado, desejando"-lhes (\emph{bouleúsainto}) o melhor nos assuntos
pessoais. Os amantes (\emph{oi}~\emph{men}~\emph{erôntes}) observam o
que fizeram de bom e de mau pelo amor (\emph{érôta}), e, pelos
sofrimentos (\emph{pónon}) causados, consideram [231b] antiquadas as
gratificações (\emph{chárin}) endereçadas aos seus amados
(\emph{tois}~\emph{erôménois}) de outrora.

Os que não amam (\emph{mè}~\emph{erôsin}), por seu turno, não podem dar
tal pretexto para o abandono de assuntos pessoais, nem consideram os
sofrimentos (\emph{pónous}) passados, nem os desentendimentos causados
com os parentes. Ao despojarem"-se de todos esses males, não lhes resta
nada, senão fazer voluntariamente (\emph{prothýmos}) aquilo que
consideram poder agradar (\emph{chareîsthai}) o companheiro. [231c]
E se por esse valor aos amorosos tanto se faz, o motivo é que estes
sobretudo gostam de dizer que estão amando, e ficam a ponto de
hostilizarem quem quer que seja com palavras e ações, só para agradarem
(\emph{charídzesthai}) seus amados. Aliás, é fácil saber se dizem a
verdade, pelo tanto de amor (\emph{erasthôsin}) que lhes passam a
dedicar, pois fariam de tudo para ele, e é evidentemente que a outros
hostilizariam, se isso lhes fosse requisitado [pelo amado].
[231d] De algum modo, é verossímil (\emph{eikós esti})~aceitar
semelhante dificuldade àquele que passou por esse infortúnio, afinal
quem se livraria (\emph{apotrépein})~dele, mesmo sendo experiente
(\emph{émpeiros})? E eles mesmos concordam que estão mais doentes
(\emph{noseîn}) do que prudentes (\emph{sōphroneîn}), e sabem que pensam
(\emph{phronoûsin}) mal, mas não podem dominar"-se (\emph{krateîn}).~Como
é que em estado de bom pensamento (\emph{eû}~\emph{phronésantes}),
poderiam considerar belas as decisões (\emph{bouleúontai}) tomadas
naquele estado anterior? E se tu procuras escolher o melhor entre os
amantes (\emph{tôn eróntôn}), a eleição pode ser feita entre poucos
(\emph{ex olígôn}), mas seria mais proveitosa para ti se abarcasse
outros (\emph{ek tôn állôn}) entre muitos (\emph{ek pollôn}). [231e]
Desse modo, é muito maior a esperança de encontrar, na multidão
(\emph{en tois polloîs}), alguém que seja digno da tua amizade.

Se receares a lei (\emph{nómon}) estabelecida, não te afetes pelas
observações vergonhosas feitas pelos homens, pois é verossímil
(\emph{eikós esti}) [232a] que os amantes, considerando"-se honrados
pelo amado, tal qual eles mesmos o veneram, exaltem"-se em discursos e
honradamente (\emph{philotimouménous}) mostrem a todos que não têm
sofrido (\emph{pepónêtai}) em vão. Por outro lado, os que não estão sob
o efeito do amor (\emph{mḕ}~\emph{erôntas})~são superiores
(\emph{kreíttous}) ao escolher o melhor (\emph{béltiston}) em vez da
opinião (\emph{anti tês dóxês}) dos homens.~Ademais, necessariamente,
todos observam os amantes (\emph{erôntas}) com seus amados
(\emph{erôménois}), bem como as ações que praticam, e quando os vêem a
conversar [232b] entre si, pensam que já consumaram, ou que estão
para consumar, o seu desejo (\emph{epithymías}).

Por outro lado, aos que não estão sob o efeito do amor
(\emph{mḕ~erôntas}), não se culpa de tentar algo só por causa da
companhia, sabendo que é necessário dialogar (\emph{dialégesthai}) para
sedimentar a amizade (\emph{philían}) e para qualquer outro deleite
(\emph{hēdonḗn}). Se considerares difícil a permanência da amizade, uma
vez que qualquer tipo de diferença pode trazer desagrado a ambos,
[232c] quando tudo aquilo que fazes de grandioso torna"-se
prejudicial (\emph{blábên}), desse modo, é bem verossímil
(\emph{eikótôs}) que tenhas temor (\emph{phoboîo}) dos amantes
(\emph{erôntas}). Muitas são as aflições (\emph{lupoûnta}) desses
amantes, além de considerarem que tudo lhes traz prejuízo
(\emph{blábêi}). E é por isso que tentam afastar (\emph{apotrépousin})
os amados (\emph{erôménôn}) da companhia de outros, temendo
(\emph{phoboúmenoi}) que detentores de bens os superem no dinheiro ou
que lhes sejam superiores na educação. Protegem o amado de qualquer
outro que possa deter esses bens. [232d] Quando persuadem"-no a
odiá"-los, colocam"-no apartado dos amigos, e no caso de considerarem a si
mesmos melhores que aqueles, acabam provocando desavenças.

Os que, pela sorte, não estão sob o efeito do amor (\emph{mḕ~erôntes
étychon}), mas pela virtude (\emph{aretḕn})~praticam o desejado, não
sentem ciúmes (\emph{phthonoîen}) dos acompanhantes do amado, e,
certamente, não querem odiá"-los. Por considerarem"-se desprezados, eles
querem ser úteis aos ditos amigos do amado. A~partir dessa prática, é
muito maior a esperança de ter amizade (\emph{philían}) com eles em vez
de aversão (\emph{échthran}) [232e].

Na verdade, muitos amantes (\emph{erṓntōn}) desejam (\emph{epethúmēsan})
o corpo antes de lhes conhecerem seus hábitos e de experimentarem outras
familiaridades, de modo que não é evidente que decidam tornar"-se amigos
(\emph{boulḗsontai phíloi eînai}), tão logo tenha cessado o seu desejo
(\emph{epithymías paúsôntai}).~[233a] Para os que não estão amando
(\emph{mḕ~erôsin}) e que praticam primeiramente entre si a amizade, não
é verossímil (\emph{eikòs}), partindo daquilo que foi bem realizado, que
a amizade (\emph{philían}) diminua. Ao contrário, esse modo produz uma
memória (\emph{mnēmeĩa}) prévia que alimenta o que ainda está por vir.
Na verdade, é melhor que tu sejas persuadido (\emph{peithoménôi}) por
mim do que por um amante (\emph{erastêi}).~Os amantes elogiam as ações e
as palavras mesmo que estas não sejam as melhores, seja pelo receio de
serem odiados (\emph{mḕ~apéchthôntai}), seja por terem se tornado débeis
em seu discernimento pelo desejo (\emph{epithymían}). [233b] São
essas as coisas que o amor (\emph{érôs}) manifesta: desafortunados
(\emph{distychoûntas}) que consideram molesto o que não proporciona
sofrimento (\emph{mḕ}~\emph{lupén}) aos demais e afortunados
(\emph{eutychoûntas}) para os quais ocorre (\emph{tygchanein}) ter de
elogiar (\emph{epaínou}) forçosamente o valor daquilo que não lhes é
prazeroso (\emph{mḕ~hedonês}).~Assim, é muito mais conveniente
apiedar"-se dos amantes (\emph{eleeîn toîs erôménois}) do que invejá"-los
(\emph{zeloûn}). E~se por mim fores persuadido (\emph{peíthêi}), em
primeiro lugar, não serei o teu guardião do prazer
(\emph{hêdonḕn}~\emph{therapeúôn}), mas das utilidades (\emph{ôphelían})
futuras, [233c] não sendo vencido pelo amor (\emph{hyp'érôtos
hēttṓmenos}), mas permanecendo senhor de mim mesmo
(\emph{emautoû}~\emph{kratôn}). Não me arrastarei a um ódio extremo por
motivos fúteis e terei pouca ira em função de motivos maiores,
desculpando as faltas involuntárias e tentando evitar
(\emph{apotrépein}) as voluntárias. Esse é o testemunho
(\emph{tekmḗria}) de uma amizade (\emph{philían}) que durará muito
tempo. Se tu pensas que a mais forte das amizades não ocorreria sem a
presença do amor (\emph{erôn tygchánêi}), [233d] é necessário ainda
considerar (\emph{enthymeîsthai}) que nem aquela amizade aos muitos
filhos haveria, nem aos pais ou às mães, nem a fiel amizade dos amigos
teríamos adquirido (\emph{pistoùs phílous ekektḗmetha}), por isso a
amizade não provém daqueles desejos (\emph{epithymías}) antes
mencionados, mas de outras práticas.

Em seguida, se é necessário agradar (\emph{charídzesthai}) sobretudo a
quem precisa, convém que os beneficiados não sejam os melhores
(\emph{beltístous}), mas os muito isentos em recurso
(\emph{aporôtátous}), pois, livrados dos maiores males, esses terão
ampla gratidão (\emph{chárin}). Certamente então nos banquetes
particulares não vale a pena convidar os amigos (\emph{phílous}), mas os
que clamam e os que necessitam saciar"-se, pois estes se tornarão
carinhosos (\emph{agapésousin}), companheiros (\emph{akolouthḗsousin}),
e virão a nossa porta sabendo comprazer"-se e, não com pouca gratidão
(\emph{chárin}), desejar"-nos-ão boas coisas.~Da mesma maneira, convém
não agradar (\emph{charídzesthai}) aos muito necessitados, mas aos que
especialmente podem oferecer gratidão (\emph{chárin}), não só aos que
clamam [234a], mas aos dignos dessas práticas, não tanto aos que
desfrutam da juventude, mas aos que na velhice repartirão contigo os
benefícios, não aos que, por terem realizado seu intento, passam logo a
dedicar"-se aos outros, mas aos que se envergonham e calam diante de
todos, não aos que se dedicam por um curto tempo, mas aos que serão
amigos por toda a vida, não aos que, cessado o desejo (\emph{oi
pauómenoi tês~epithymías}), buscarão pretexto para o ódio, enquanto os
outros tendo [234b] passado a juventude mostrar"-lhes-ão a virtude
(\emph{hoì pausaménou tês hṓras t\gr{ό}te tḕn hautõn aretḕn epideíxontai}).

Então, recorda"-te do que foi dito e põe no teu ânimo (\emph{enthymoû})
que os amigos advertem aos seus amantes (\emph{tous mèn erôntas oi
phíloi nouthetoûsin}) por seu mau comportamento, ao passo que aos que
não estão sob o efeito do amor (\emph{mḕ}~\emph{erôsin}), nem mesmo seus
familiares lhes fazem censuras de qualquer tipo, pois eles decidem
(\emph{bouleuoménois}) os seus próprios males.

Talvez, então, tu me perguntes se eu te aconselho a agradar
(\emph{charídzesthai}) todos aqueles que não estão sob o efeito de Eros
(\emph{mḕ}~\emph{erôsi}). Eu não considero que o amante
(\emph{ton}~\emph{erônta}), em todo o caso, te incentivasse a essa
maneira de pensar (\emph{diánoian}) com relação aos amorosos
(\emph{tous}~\emph{erôntas}). [234c] Nem aquele que pelo discurso
recebe semelhante honra graciosa (\emph{cháritos}), nem tu, se quisesses
manter"-te escondido dos outros, poderias agir de modo semelhante. É~necessário que disso não surja nenhum dano (\emph{bláben}), mas que
ocorra o proveito (\emph{ôphéleían}) a ambos.~Eu considero o que foi
dito suficiente. Se desejares saber algo que foi negligenciado, pergunta
(\emph{erṓta})''.

 

Como te parece o discurso (\emph{ho}~\emph{lógos}), ó Sócrates? Não é
maravilhoso (\emph{huperphýôs}), entre tantas outras razões,
especialmente no vocabulário (\emph{onómasin}) empregado?

 

[234d] S: Divino mesmo, ó companheiro, a ponto de eu estar atordoado
(\emph{ekplagênai}). E~essa minha afecção (\emph{épathon}) foi gerada
por ti, ó Fedro, pois a visão (\emph{apoblépôn}) que tive de ti foi
radiante durante a leitura do discurso (\emph{toû}~\emph{lógou}).
Considero"-te melhor do que eu para apanhar esses discursos e sigo"-te,
cabeça divina (\emph{theías kephalês}), como em um cortejo báquico
(\emph{sunebákcheusa}).

 

F: Já estás brincando (\emph{paídzein}), não é?

 

S: Pareço por acaso brincar (\emph{paídzein}) e não me esforçar
(\emph{espoudakénai})?

 

[234e] F: De modo algum, ó Sócrates, mas como dizes a verdade,
diante de Zeus protetor da Amizade (\emph{alêthôs eipe pros Diòs
philíou}), considera"-te na iminência de pronunciar entre os helenos, a
respeito do mesmo assunto, outro, melhor e mais extenso [discurso]
que este?

 

S: O quê? É preciso que eu e tu elogiemos (\emph{epainethênai}) o
discurso (\emph{tòn lógon}) desse criador que disse o que devia? Ele não
é somente claro (\emph{saphê}) e perfeito (\emph{stroggúla}), mas também
exato (\emph{akribôs}) em cada uma das palavras entalhadas
(\emph{onomátôn apotetórneuetai})? Se for preciso agradeceremos
(\emph{chárin}) o autor, apesar da minha ignorância me obscurecer
(\emph{elathen}) [235a]. Tomando exclusivamente o pensamento
retórico dele (\emph{rhêtorikôi autoû mónôi tón noûn proseîchon}),
acredito que nem mesmo o próprio Lísias consideraria suficiente. E~me
pareceu, ó Fedro, se não queres dizer outra coisa, que ele afirmou o
mesmo duas ou três vezes, como se não tivesse muitos recursos adicionais
para fazê"-lo, acerca do mesmo assunto, sem nenhum interesse. E~pareceu"-me ainda uma demonstração juvenil (\emph{neanieúesthai
epideiknúmenos}) de quem quer falar tanto de uma perspectiva quanto de
outra e, em ambos os casos, da melhor maneira possível.

 

[235b] F: Não é como dizes, ó Sócrates, uma vez que isso é o que
justamente o discurso (\emph{ho lógos}) tem de melhor, o fato de não ter
negligenciado nenhum dos assuntos convenientes em sua performance. Desse
modo, junto a este discurso, ninguém seria capaz de proferir
(\emph{eipeîn}) outro, mais extenso e mais digno (\emph{pleíô kaì
pleíonos áxia}).

 

S: Nesse ponto, eu jamais poderia ser persuadido (\emph{pithésthai}) por
ti. Os antigos sábios~(\emph{sophoì}), homens e mulheres que proferiram
discursos e escreveram (\emph{eirêkotes kaì gegraphótes}),
refutar"-me"-iam (\emph{exelégksousí}) se, para agradar"-te
(\emph{charidzómenos}), eu concordasse contigo.

 

[235c] F: Quais são eles e onde ouviste algo superior?

 

S: De imediato, assim, não posso dizer. Mas é evidente que ouvi ou da
bela Safo ou do sábio (\emph{sophoû}) Anacreonte, ou de algum outro
escritor (\emph{suggraphéôn}). De onde tiro os testemunhos de que falo?
De certa plenitude (\emph{plêrés}), ó divino, que sinto no peito e pela
qual poderia dizer outras tantas coisas nada inferiores. Bem sei que não
é por mim que tenho em mente (\emph{ennenóêka}) essas coisas, pois
conheço minha própria ignorância (\emph{emautôi amathían}). Deixemos
isso de lado, creio que são outras as fontes que, pela audição, me
encheram (\emph{peplêrôsthai}) como a uma vasilha.~[235d] E foi por
estupidez que me esqueci (\emph{epilélêsmai}) dessas coisas, bem como de
quem as ouvi.

 

F: Mas, ó excelentíssimo, disseste do melhor modo possível. De quem e
como ouviste, eu não te ordeno que digas, desde que cumpras o seguinte.
Melhor que este livro, promete (\emph{hypéschêsai}) dizer outro
[discurso] em nada inferior, ficando dele afastado. E~eu prometo
(\emph{hypischnoûmai}), tal qual os nove arcontes, oferecer"-te um ícone
(\emph{eikóna})~dourado em tamanho natural (\emph{isométrêton}) no
templo de Delfos, não só o meu, mas também o teu.

 

[235e] S: És amicíssimo e de ouro verdadeiro, ó Fedro, se julgas que
eu poderia dizer todas as coisas em que Lísias se enganou [no
discurso] e que seria preciso proferir outro junto ao dele. Creio que
isso não abateria nem o mais inábil dos escritores (\emph{suggraphéa}).
Começando pelo discurso (\emph{ho lógos}), quem considera dizer que é
melhor e necessário agradar (\emph{charídzesthai}) o não afetado pelo
amor em vez do amante (\emph{mḕ~erônti mâllon~ḕ~erônti}) [236a] quer
evitar o encômio do prudente (\emph{to phónimon egkômiádzein}) e o
vitupério do insensato (\emph{to áphon pségein}), e, sendo assim,
necessariamente, terá ainda ele algo a dizer?~Mas penso que é necessário
livrar e desculpar o orador, e elogiar (\emph{epainetéon}) a sua
disposição (\emph{diáthesin}), não a invenção (\emph{heúresin}), e
quando a sua disposição (\emph{diathései}) não é suficiente ou difícil
de encontrar (\emph{heureîn}), elogiamos, para além dela, a invenção
(\emph{heúresin}).

 

F: Concordo com o que dizes e bem medido (\emph{metríôs}) me parece.
Farei então o seguinte, estabelecerei como base para ti que o amante
está mais doente que o não afetado pelo amor (\emph{tòn erônta toû
mḕ~erôntos mâllon noseîn}), [236b] deixando o resto de lado, ao
dizer outro [discurso] mais extenso e mais digno (\emph{pleíô kaì
pleíonos áxia})~que o de Lísias, estarás em ouro maciço junto a oferenda
dos Cipsélidas~em Olímpia.

 

S: Tomaste a sério, ó Fedro, porque eu brinquei acerca do teu favorito.
Consideras mesmo que eu verdadeiramente vá falar, contra aquela
sabedoria (\emph{sophían}), outro [discurso] em algo mais variado.

 

F: Nesse caso, ó querido, chegas a colher o mesmo. [236c] É
necessário que digas, tanto quanto lhe seja possível, um [discurso]
melhor em tudo, para que não nos obriguemos a realizar uma grosseria de
comediantes, trocando mutuamente de papéis, e que não queiras ainda
forçar"-me a dizer"-te aquele ``ó Sócrates, se não conheço Sócrates,
estaria esquecido de mim mesmo'', ou aquele ``desejas dizer, mas, no
entanto, fica enternecido''.~Põe na tua cabeça que não sairemos daqui
antes que digas o que tens a dizer. Estamos sozinhos num lugar ermo,
além do que sou mais forte e mais jovem, de modo que, de toda maneira,
``vê se entende o que te digo'', não queiras falar à força (\emph{pròs
bían boulêthêis}), mas de bom grado.

 

[236d] S: Ó bem"-aventurado Fedro, eu seria risível se, de improviso,
um inábil como eu procurasse equiparar"-me a um bom compositor
(\emph{agathòn poiêthḕn}) nesses assuntos.

 

F: Perceba que é melhor parar (\emph{paûsai}) de vangloriar"-te para cima
de mim, pois eu tenho algo a dizer para justamente forçar"-te a falar.

 

S: Não pode ser como dizes.

 

F: Não? Pois então eu digo, será um juramento (\emph{hórkos})\emph{~}a
minha fala (\emph{lógos}), prometo a ti -- e a alguém mais talvez, a
algum deus daqui? -- [236e] e até mesmo a esse plátano.~Se não
disseres outro discurso no lugar deste (\emph{enantíon autês taútes}),
nunca mais te comunico exposição nenhuma, de quem quer que seja.

 

S: Que infâmia, ó abominável, bem encontraste (\emph{eũ anēũres})~um
meio para forçar um homem que ama discursos (\emph{philológôi})~a fazer
o que dizes.

 

F: Então porque te esquivas (\emph{stréphēi})?

 

S: Em absoluto, uma vez que tu prometeste isso, como seria possível
apartar"-me desse banquete (\emph{thoínês})?

 

[237a] F: Diz então.

 

S: Sabes como farei?

 

F: A respeito de quê?

 

S: Falarei encoberto (\emph{enkalypsámenos}), para que rapidamente
percorra o discurso (\emph{tòn lógon}) e para que não te veja, temendo
vacilar de vergonha.

 

F: Simplesmente fala, de resto faz como queiras (\emph{boúlei}).

 

S: Vinde, ó Musas, tanto na forma de odes melodiosas (\emph{lígeias}),
como na dos músicos de Ligure (\emph{Ligúôn}), ambas são epônimas,
``toma da minha'' palavra (\emph{toû mýthou}) e me obriga a falar
(\emph{legeîn}) da melhor forma possível (\emph{béltistos}), para que o
teu companheiro, parecendo"-lhe ser sábio (\emph{sophòs}) num primeiro
momento, tenha agora melhor reputação (\emph{dóxêi}) ainda. [237b]
Era uma vez um menino, na verdade um jovem, que era muito belo
(\emph{kalós}), e que por isso tinha muitos amantes (\emph{erastaì}). Um
entre os seus aduladores (\emph{aimúlos}), não menos amoroso
(\emph{erôn}) que outros, foi capaz de convencê"-lo (\emph{epepeíken}) de
que não o amava (\emph{ouk}~\emph{erôiê}), e em seguida de que era
preciso antes agradar a quem não ama em vez dos amantes (\emph{hôs
mḕ~erônti prò toû erôntos déoi charídzesthai}), ele dizia o seguinte:

 

``Acerca de tudo isso, ó menino, só há um princípio (\emph{mía archḕ})
aos que desejam bem deliberar (\emph{bouleúsesthai}), [237c] saber
necessariamente acerca do que se trata em cada deliberação
(\emph{boulé}) ou então será forçoso que haja engano (\emph{hamartánei})
em todo o resto. Muitos se esquecem (\emph{lélêthen}) de que não
conhecem cada uma das essências (\emph{ousían}). E~como não sabem
acordarem"-se no princípio de uma verificação (\emph{diomologoûntai en
archêi tês}~\emph{sképseôs}), eles pagam o preço verossímil
(\emph{eikós}) de não concordarem (\emph{homologoûsin}) nem consigo
mesmos, nem com outros. Não soframos, eu e tu, daquilo que censuramos
neles, mas é necessário conhecermos o discurso (\emph{ho lógos}) que
propicia a melhor amizade (\emph{philían}), se é o direcionado ao amante
ou ao que não o é, para sabermos, em seguida, as potencialidades do amor
e de sua natureza, estabelecendo uma definição acordada (\emph{homología
thémenoi hóron}) [237d] para uma verificação (\emph{sképsin}), de
modo a vê"-lo e referi"-lo no que oferece de útil (\emph{ôpheleían}) e de
dano (\emph{blabên}).

É claro para todos que o amor é um desejo (\emph{epithymia}) e que
também os que não estão sob o efeito do amor (\emph{mḕ~erôntes}) desejam
(\emph{epithymoûsin}) os belos, isso nós sabemos. Mas como discerniremos
o amante do não afetado pelo amor (\emph{tòn erônta te kai
mḕ~krinoûmen})? É preciso entender (\emph{noêsai}) que há, em cada um de
nós, duas formas (\emph{idéa}) que nos presidem e nos conduzem: uma
delas seguimos onde quer que nos leve, é a do desejo dos prazeres inatos
(\emph{emphytos oûsa epithymía hedonôn}), a outra é a da opinião
(\emph{dóxa}) adquirida, que tende para o melhor (\emph{toû arístou}).
[237e] Algumas vezes, essas duas tendências em nós estão de acordo
(\emph{homonoeîton}), outras vezes em conflito (\emph{stasiádzeton}), de
modo que algumas vezes predomina uma delas, outras vezes a outra.

A opinião (\emph{dóxes}) da melhor razão domina e conduz pelo poder do
que é chamado de prudência (\emph{sophrosýne}), ao passo que o desejo
irracional arrasta para os prazeres (\emph{epithymías dè
alógôs~helkoûsês~epi hêdonàs}), deflagrando aquilo que no início
denominamos desmesura (\emph{hýbris}).~[238a] A desmesura
(\emph{hýbris}) tem muitos nomes, muitos membros e partes, e se por
acaso (\emph{týchêi}) ela tomar alguma dessas formas, oferece o epônimo,
nem belo e nem digno, a todos que a trazem consigo. Quando a ingestão de
alimentos domina a razão (\emph{lógou}) e o melhor (\emph{arístou}),
entre outros desejos (\emph{epithymiôn}), o desejo (\emph{epithymía}) é
o de um glutão (\emph{gastrimargía}), que conferirá esse título a quem o
possua. [238b] Sobre a tirania (\emph{tyranneúsasa}) das bebidas,
conduzindo aquele que a tem, produz aquela óbvia denominação. Da mesma
maneira, as coisas a estas aparentadas, bem como o surgimento de desejos
correlatos, sempre são soberanos (\emph{dynasteuoúses}) e tem cada qual
o seu nome.

De acordo com tudo o que foi dito antes, é quase evidente, podemos
dizer, ou não dizer, da forma mais clara possível (\emph{saphésteron}):
sem a razão (\emph{aneu lógou}), o desejo (\emph{epithymía}) domina a
opinião (\emph{doxês}) e conduz diretamente ao prazer da beleza
(\emph{hedonèn achtheîsa kállous}), [238c] então os desejos
(\emph{epithymiôn}) congêneres com força se lançam à beleza dos belos
corpos, vencendo todos, eles adquirem a força do seu nome, sendo
chamados de amor.''

 

E então, ó querido Fedro, pareço"-te, como a mim mesmo, afetado por algo
divino (\emph{theion páthos peponthénai})?

 

F: Completamente, ó Sócrates, foste tomado por uma não usual fluência.

 

S: Cala"-te agora e escuta"-me: realmente esse lugar parece divino
(\emph{theîos éoiken ho tópos eînai}) e não te espantes se eu, muitas
vezes, no discurso, for tomado pelas Ninfas. [238d] Agora mesmo, ao
falar, não estive longe de entoar um ditirambo.

 

F: Dizes a mais pura verdade.

 

S: E isso é por tua causa. Mas escuta o restante, pois talvez eu possa
me apartar (\emph{apotrápoito}) dessa ameaça. Deixemos isso ao cuidado
do deus (\emph{theôi}) e ocupemo"-nos novamente do discurso (\emph{pálin
tôi logôi}) dirigido ao jovem.

 

``Que seja assim, ó meu bravo amigo. O~assunto da deliberação
(\emph{bouletéon}) já foi mencionado e definido. Vejamos, agora, o que
nos falta ainda dizer sobre ele, qual é a utilidade (\emph{ôphelía}) ou
o dano (\emph{blábê}) que pode, de modo verossímil (\emph{ex eikótos}),
advir a quem agrada (\emph{charidzoménoi}) ao amante e a quem
[agrada] ao não afetado pelo amor.~[238e] Aos que começam a ser
escravizados pelo desejo (\emph{epithymías}) é necessário que, de alguma
maneira, pelo prazer (\emph{hedonêi}), busquem no amado o que lhes dá
esse prazer (\emph{hḗdiston}). Ao doente (\emph{nosoûnti}) é necessário
que tudo lhe seja agradável (\emph{hedù}) e que nada o contrarie
(\emph{antiteînon}), pois causa aborrecimento tudo o que é mais forte ou
de força similar ao seu desejo. O~amante não admite, da mesma forma, nem
superioridade nem igualdade ao seu predileto (\emph{paidikà}),
[239a] e sempre procura rebaixá"-lo, tornando"-o inferior. O~ignorante
(\emph{amathḕs}) é inferior ao sábio (\emph{sophoû}), assim como o
covarde é inferior ao corajoso, o inábil na fala inferior ao homem da
retórica (\emph{rhētorikoũ}) e o lento inferior ao sagaz.

Esses males e ainda outros maiores surgem no pensamento
(\emph{diánoian}) do amante com relação ao amado (\emph{erastḕn
erôménôi}), e, naturalmente, são [males] ligados ao prazer
(\emph{hḗdesthai}), podendo até mesmo provocá"-los
(\emph{paraskeuázdein}), quando [o amante] fica momentaneamente
apartado do seu prazer (\emph{hêdéos}). Necessariamente o amante é
ciumento e afasta [o amado] de todas as outras companhias que lhes
sejam proveitosas (\emph{ôphelímôn}), [239b] sobretudo das que
enriqueçam o homem. O~dano que isso causa é enorme, mas o maior deles é
o de evitar que [o amado] torne"-se prudentíssimo
(\emph{phronimṓtatos}).

Quando o amante mantém o afastamento necessário com relação ao seu
predileto, temendo por este ser depreciado, isso ocorre
(\emph{tugchánei}) devido à divina filosofia (\emph{theía
philosophía}).~Outras coisas são maquinadas para que [o amado] possa
ficar ignorante em tudo (\emph{pant' agnoôn}) e só tenha olhos
(\emph{apoblépôn}) para o seu amante (\emph{erastén}), como se fosse o
mais agradável (\emph{hédistos}), entretanto isso pode ser o mais danoso
(\emph{blablerṓtatos}) a si mesmo. [239c] Então, segundo esse
pensamento (\emph{diánoian}), quer para o tutor, quer para o
companheiro, em nenhuma parte é proveitoso ao homem sentir amor
(\emph{érôta}).

Depois dessas coisas é preciso conhecer a disposição e os cuidados
(\emph{héxin te kaì therapeían}) relativos ao corpo, de como ele deve
ser cuidado (\emph{therapeúsei}) pelo seu comandante e em que medida é
forçado a perseguir (\emph{diókein}) o prazer antes do que é o melhor
(\emph{hedù prò agathoû}). Veremos um amante a perseguir um jovem
delicado e não muito valente, alguém que não foi bem nutrido na pureza
do sol, mas na companhia da sombra, inexperiente nas fadigas e suores
dos labores masculinos, mas experiente no estilo de vida delicado dos
afeminados, [239d] embelezando"-se com cores e adornos incomuns, vida
acompanhada as outras práticas que dessas derivam, práticas evidentes e
indignas até mesmo para avançarmos no comentário, mas delimitemos a
questão primordial (\emph{kephálaion}) acerca disso para passarmos para
o outro assunto. Esse corpo, seja na guerra ou em outros afazeres,
proporciona, por um lado, coragem aos inimigos e, por outro, medo aos
amigos e mesmo aos amantes.

Deixemos de lado o que é evidente, pois é necessário que delimitemos
agora [239e] qual dessas atitudes nos é útil (\emph{ôphelían}) ou
danosa (\emph{blábên}) ao recebermos a companhia e a tutela de um
amante. É~claro para todos, e especialmente para o amante (\emph{tôi
erástêi}), que ele se regozije de que o amado possa ser privado de tudo
aquilo que é mais querido, mais amistoso e diviníssimo. Prefere que ele
fique afastado de pai, mãe, parentes e amigos, [240a]
considerando"-os todos como empecilhos e censores à sua prazerosa
(\emph{hêdístes}) convivência. Mas se [o amado] possui ouro ou
qualquer outro tipo de posse, não será da mesma maneira fácil ser
capturado e mantido sob controle. Por conta disso, é forçoso que o
amante (\emph{erastḕn})\emph{~}sinta ciúme de jovens que têm recursos,
e gostem (\emph{chaírein}) que eles sejam arruinados. E~ele ainda
preferiria que seu amado viesse a ficar sem se casar (\emph{ágamon}),
sem filhos (\emph{ápaida}) e sem casa (\emph{áoikon}), tanto tempo
quanto fosse possível, para que pudesse colher o seu doce desejo
(\emph{epithymôn}) no máximo tempo possível.

 
