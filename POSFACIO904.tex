



 \section{Do amor físico e
teológico}
 

``\emph{Fedro}~ou acerca do amor ético'' (\emph{Phaîdros ḕ perì
érôtos,~ḗthikós}~D.L.§58) é a maneira pela qual Diógenes Laércio
apresenta, no seu catálogo das obras de Platão, a definição do tema e do
gênero do diálogo. Um pouco antes, Diógenes diz que, segundo autores mais
antigos, o~\emph{Fedro}~guardava um estilo juvenil (\emph{meirakiôdés}),
pois estaria entre os primeiros escritos de Platão, e que Dicearco,
especificamente, teria considerado seu estilo vulgar (\emph{phortikón})
(D.L.§38). Essas, entre outras informações, fazem parte da interessante
biografia de Platão feita por Diógenes, meio pelo qual sabemos, por
exemplo, como Trasilo organizou os diálogos, tal qual já haviam sido
organizadas as tragédias (D.L.§56--60), segundo tetralogias.

Hermias de Alexandria, quase dois séculos à frente de Diógenes, mostra o
desdobramento dos temas do~\emph{Fedro}, ao comentá"-lo de modo
sistemático. Hermias diz que o~\emph{Fedro}~é:

 

\begin{quote}
\redondo{[…]} ético e catártico, refutativo e protréptico na direção da
filosofia (\emph{êthikós kaì kathartikós, elegktikós, protreptikòs eis
philosophían}), por isso há nele um discurso acerca do amor físico e
teológico, além de um [discurso] lógico acerca da retórica
(Herm.~\emph{In Phdr.}~63.23--25 Couvreur = 10,21--23 Lucarini \&
Moreschini).
\end{quote}

 

Hermias enumera os discursos acerca da alma e do amor prudente como
temas fundamentais do diálogo, ressaltando o sentido teológico pelo qual
Eros deve ser tratado. Mostra também que esses temas todos: (1) ética,
(2) catarse, (3) refutação, (4) mudança em direção à filosofia
(protréptico), (5) amor físico, (6) amor teológico e (7) retórica são
desdobramentos de um núcleo bastante importante, que está especialmente
na relação entre Alma e Amor (Eros e Psique). Desses elementos todos,
talvez o caráter~\emph{protréptico}~do diálogo seja o mais apto a
arrematar uma explicação geral, que perpasse todos esses pontos, pois é
importante nele o desejo de modificar uma determinada
conduta~\emph{ética}, o que pressupõe uma~\emph{refutação}~de conduta e
discursos anteriores, bem como uma~\emph{catarse}~intelectual, que
propicie um novo lugar dialético em que se possa pensar e agir com novas
ferramentas intelectuais e discursivas. Há nesse momento uma compreensão
de que determinadas ações e discursos estavam em desacordo com
concepções~\emph{éticas}~e~\emph{teológicas}, especialmente no que
concerne a Eros.

Hermias atentou para a proliferação de temas que o diálogo reúne sem
perder de vista sua unidade temática. Hermias manifesta a predominância
da alma e do amor como eixo principal, e o da retórica e da teologia,
como um eixo complementar, pois jamais se omite esse fundo retórico e
teológico. Nesse mesmo sentido, aspectos ligados à beleza e ao belo
ganham, em Hermias, fortes contornos neoplatônicos, ecoando por um lado
Plotino (\emph{Acerca do belo}) e por outro o vocabulário de Proclo
acerca dos ``deuses encósmios'' (\emph{In Plat. Tim.}~32c)\footnote{Cf.
  Brisson, L. \emph{Proclus et l'orphisme} In \emph{Proclus, lecteur et
  interprète des anciens}, Paris: Les Belles Letres, 1987, p.84-86;
  Berg, R. M. van den. \emph{Proclus' hymns}: essays, translations and
  commentary, Köln: Brill, 200; Saffrey, H. D. \emph{Accorder les
  traditions théologiques: une caractéristique du néoplatonisme
  athénien}, In \emph{Le Néoplatonisme après Plotin}, Paris: Vrin, 2000,
  p.143-158.}.

 

\section{Heath, Rowe e Kastely}

 

Há um debate acerca dessa profusão temática já referida por Hermias.
Alguns pesquisadores colocam em questão a suposta unidade entre as
recitações iniciais e a segunda parte, ou parte posterior às recitações,
enquanto outros aceitaram melhor a unidade. Malcon Heath, por exemplo,
tem claro que são os estudiosos modernos que procuram pela unidade
temática do~\emph{Fedro}, enquanto os estudiosos antigos, como Hermias,
reconhecem a pluralidade temática como marca. Como diz Heath, os antigos
têm mais a dizer acerca da variedade (\emph{poikilía}) do que sobre a 
unidade do~\emph{Fedro~}(Heath, 1989, p.\,163). Ao aproximar a unidade
dramática do diálogo à estrutura da tragédia, Heath mune"-se de
ferramentas que o permitem observar a alteração do tema inicial como
sutil, tal qual no teatro, quando se preservam temas de modo implícito.
Heath nota que a pluralidade temática não afeta ou enfraquece a unidade
do texto, para tanto ele colhe elementos interpretativos em Hermógenes,
especialmente uma descrição acerca do ``gênero'' diálogo. Heath menciona
brevemente essa definição que aqui trago na íntegra:

 

\begin{quote}
Nos diálogos há entrelaço de discursos ético e investigativo
(\emph{dzêtêtikoí}). Isso ocorre quando há mistura do diálogo com uma
investigação, os discursos éticos aplicados impedem a alma e, depois que
ela foi impedida, a investigação é conduzida, como um instrumento
[musical] que se torna tensionado ou distendido (Hermógenes
[sp.],~\emph{Do método terrível}, 36.30--33)\footnote{Διαλόγου πλοκὴ
  ἠθικοὶ λόγοι καὶ ζητητικοί. ὅταν ἀναμίξῃς προσδιαλεγόμενος καὶ ζητῶν,
  οἱ ἠθικοὶ παρεμβληθέντες λόγοι ἀναπαύουσι τὴν ψυχήν, ὅταν δ' αὖ
  ἀναπαύσηται, ἐπάγεται ἡ ζήτησις, ὥσπερ ἐν ὀργάνῳ ἡ τάσις καὶ ἄνεσις
  γίνεται.}.
\end{quote}

 

A descrição do gênero diálogo se adequa à disposição (\emph{táxis})
do~\emph{Fedro}, depois da alma ser refreada em seus impulsos baixos com
relação ao amor, desencorajada por um lado, passa a ser exortada em um
novo sentido, aí está a mudança de direção (\emph{protrepse}), depois da
qual investiga"-se os meios pelos quais isso ocorre. A~metáfora da música
também se adequa perfeitamente à ambiência poética do diálogo.

Rowe acredita que o início da palinódia divide o diálogo ao meio, sem,
no entanto, integrá"-lo. Segundo Rowe, o~\emph{Fedro}~não guarda uma
unidade dramática ou temática (Rowe, 1989, p.\,183) e a palinódia não tem
estatuto de discurso filosófico (p.\,179). Kastely, concordando com a
falta de unidade apontada por Rowe, mas discordando com relação ao lugar
em que isso ocorre, divide o~\emph{Fedro}~também em duas partes
desconexas. Para ele, a primeira metade do diálogo apresenta três
exemplos de prática retórica, o que chamamos aqui de recitações,
enquanto a segunda parte desenvolve uma justificativa teórica da
retórica enquanto arte. Assim, depois do final da palinódia começaria a
outra parte do~\emph{Fedro}~e Kastely percebe, entre essas partes, uma
potencial negociação, mas observa também, por outro lado, que essa
negociação não ocorre (Kastely, 2002, p.\,138).

Kastely e Rowe circunscrevem a segunda parte do diálogo -- seja ela depois
do primeiro discurso de Sócrates, como quer Rowe, seja ela depois da
palinódia, como quer Kastely -- como um novo trecho que não retoma, de
forma alguma, elementos dos discursos anteriores. O~texto platônico
divide os comentadores, pois a dita segunda parte não comenta realmente
trechos dos discursos anteriores, ainda que algumas vezes os citem, sem
jamais serem capazes de reavivar algo do que foi dito neles. Tudo fica
aparentemente num mesmo plano, pois as recitações, igualadas, jamais
podem ser reavivadas. Esse parece ser, justamente, o efeito mudo que
Platão pretende provocar -- e aí está a prova de que consegue \mbox{--,} ao
encerrar na letra morta do seu próprio discurso escrito as três
recitações inicias como práticas isoladas, sejam elas ``originalmente''
escritas ou não, pois elas não serão objetos de análise ou comentário.
Tal qual a pintura, para usar uma metáfora do próprio diálogo, esses
escritos mantêm"-se estáticos, sempre precisando de quem os socorra,
embora neles permaneçam o amor e a alma, atrelados à discussão acerca da
linguagem, escrita e falada. Esse silêncio posterior das recitações é
fundamental para o novo espaço discursivo, pois algo comum permanece na
constância tácita das recitações, que se confunde com o próprio limite
discursivo. Ainda que defendam posições diversas, todas as recitações
permanecem encerradas em si mesmas, distantes do discurso dialético e
vivo. Ainda que também mimetizado, o discurso dialético e vivo
pós"-recitações seria unicamente capaz de se aproximar dos seres e da
verdade. Só na trama da dialética pode haver uma explicação e uma
aplicação daquilo tudo, daquele antigo transbordar discursivo, numa nova
mimética didática e filosófica.

A segunda parte apresentará, no formato dialógico e dialético, ou como
diz Hermógenes, mais distendido, outro tipo de discurso, com novos ritmo
e dinâmica. As recitações haviam sido soberanas e
seus~\emph{intermezzos}~funcionaram como articulações que nos levavam,
em três momentos de monólogos (ou quase monólogos), de um lado a outro
do Ilisso, ou seja, do vitupério ao elogio de Eros. Agora, os blocos
textuais são muito menores, o diálogo mais intenso e as imagens ou mitos
ganham papéis diferentes na nova trama. Essa mudança rítmica e dinâmica
talvez seja um dos fatores que geraram aquele estranhamento da recepção
no que tange aos temas e à unidade do diálogo.

Embora haja um descolamento entre as recitações iniciais e o ``comentário''
posterior na economia do diálogo, entendemos a distância desse segundo
movimento como um efeito discursivo, como uma maneira prática de
apresentar o limite da comunicação escrita. Talvez seja essa a maior
crítica à escrita que Platão tenha feito no~\emph{Fedro}.

Curiosamente, no mesmo~\emph{Fedro}~há uma prescritiva para quem deseja
compor um discurso, que deve ser sempre como um corpo vivo, ``nem
acéfalo nem ápodo'', e tenha ``tronco e membros convenientes entre
si''(264c)\footnote{Cf. Trivigno (2009) que estuda esse aparente
  paradoxo, de Platão defender uma prescritiva retórica que ele mesmo
  não teria seguido.}.

Passarei em revista os temas do~\emph{Fedro}, ainda que seja um tema
polêmico, lembrando que essa apresentação tem apenas intenção didática,
pois outras tantas divisões do~\emph{Fedro}~já foram feitas e ainda
serão. Enumerarei de 1 a 7 os temas que a mim me parecem mais evidentes.
O~primeiro deles é: (1) o que se deve fazer com relação ao amante? É
melhor agradá"-lo (\emph{charísdzesthai}) ou não? Essa é a questão
principal das três recitações (\emph{epideixis}) e, portanto, o primeiro
grande tema do~\emph{Fedro}. A~essa pergunta Platão apresenta uma
resposta tripartida, vista de três diferentes modos, e desses discursos
emanam todos os outros temas. Essa pergunta acerca do que devemos fazer,
agradar ou não a um amante apaixonado, permanece entrelaçada à questão
ética de como devemos agir e falar. Depois das recitações, esse primeiro
tema não se perde. Platão apresentará os parâmetros lícitos, éticos,
desde a palinódia, mas também depois dela. Eros é o fundamento místico-teológico
 do~\emph{Fedro}, assim como é também, de um modo um pouco
diferente, no~\emph{Banquete}.

Em seguida, manifesta"-se o problema da conexão da palinódia com a dita
segunda parte ou segundo movimento. Fedro fica estarrecido pelo final da
palinódia, especialmente pelo efeito farmacológico desse ``peã''
apolíneo ou espécie de hino pronunciado por Sócrates. Desse momento em
diante, pressupõe"-se uma conversão (\emph{protrepse}), uma mudança
ética, uma mudança na ação e no discurso, uma cura da cegueira.

Fedro diz que naquele mesmo dia haviam insultado Lísias designando"-o
pelo nome de logógrafo, motivo pelo qual Sócrates porá em discussão (2)
a natureza desse saber, a escrita, em sua ligação com a reputação dos
nobres políticos da cidade. Sócrates mostra, por um lado, que eles não
se envergonhavam por escrever discursos e que, por outro, não queriam
ser lembrados como sofistas por terem deixado escritos. Depois, (3) o
desejo de escrever é apresentado como inevitável, de modo que a pergunta
passa a ser: (4) o que seria falar e escrever de modo belo ou feio? Em
seguida, essa questão ética e estética é transposta gradativamente para
(5) uma discussão acerca da~\emph{téchne}, da arte discursiva
propriamente dita. Depois, (6) a dialética como arte~de
discernimento do verdadeiro, do bom e do justo será explicada. 
A~dialética reabilita a ética, pois esse discernimento é um novo norteador
de ações e palavras. A~partir desses temas Platão oferecerá (7) um
retrato da diferença entre retórica/\allowbreak{}logografia e filosofia. É~claro que
em meio a esses temas explicados há outros temas, pois, como dissemos,
há inúmeras maneiras, mais ou menos interessantes, de dividir
o~\emph{Fedro}.

Nas recitações o diálogo passa da performance da leitura, quando Fedro
lê o discurso de Lísias, para um segundo nível, o da encenação
rapsódica, no qual uma suposta inspiração se confunde com a mera técnica
discursiva de Sócrates. O~discurso chega, em seguida, a um terceiro
patamar, que é mais próximo da lírica e dos corais em seus movimentos
combinados e contrapostos, dos monólogos e cantos solo combinados nas
coreografias. Nesse novo registro, as imagens e o~\emph{lógos}~estão
próximos do sacro (\emph{hierou}), pois a linguagem encontra seus
limites no indizível das revelações (\emph{epoptia}) místicas e catarses
diversas da alma.

Depois da palinódia haverá uma mudança rítmica e inaugura"-se o segundo
movimento, no qual predomina a dinâmica do diálogo socrático mesmo, mais
próxima portanto do teatro, na medida em que Fedro e Sócrates se
aproximam da palavra viva, sem abandonarem os mitos, diálogos
imaginários, analogias etc. O~diálogo de Platão, assim como o teatro,
funciona como cópia imperfeita do discurso vivo, e hierarquiza a
tradição literária grega na logografia de Lísias, na poesia
pseudo"-inspirada de Sócrates e no discurso inspirado da palinódia.

 

\section{A tríade de Estesícoro no~\emph{Fedro}}

 

Ainda que compareça somente com três versos no diálogo (243a5-b1), há um
elo formal e temático entre Estesícoro e a apropriação realizada por
Platão. Essa interpretação, fruto de um trabalho de doutorado acerca
do~\emph{Fedro}, acompanhada de tradução, aqui será apresentada de modo
sinótico (Campos, 2012). Nossa leitura da filosofia escrita de Platão
procurou utilizar elementos exteriores ao corpo dialógico platônico,
especialmente provenientes da doxografia ligada à poesia de Estesícoro
de Himera (Vürtheim, 1919; Campbell, 1991; Davies, 1991), os quais se
tornaram relevantes na interpretação do~\emph{Fedro}. Os três discursos
iniciais do~\emph{Fedro}~mimetizam também a poética tripartida da poesia
de Estesícoro, em sua forma, como três blocos textuais articulados que
correspondem aos três momentos da nova estrutura triádica estesicórica.

A aproximação entre a doxografia e o diálogo revela a importância de
algumas tópicas estesicóricas resgatadas por Platão, como (1) a
substituição de Helena por um ídolo (\emph{eidôlon}) e sua aplicação no
diálogo dentro do campo filosófico da diferenciação entre ser e imagem; 
(2) a oftalmia provocada pela falta teológico"-discursiva contra Helena,
reabilitada no temor de Sócrates pela falta teológico"-discursiva contra
Eros; (3) o canto curativo como catarse, restabelecendo a ordem e saúde
da visão, uma vez que em Platão essa catarse se dá pelo discurso de
Sócrates na palinódia. Essas seriam as tópicas estesicóricas mais
evidentes no~\emph{Fedro}, mas além delas é preciso observar, como
dissemos, que a forma de Estesícoro é mimetizada na escrita de Platão,
especialmente (4) em sua estrutura triádica transposta nos três grandes
discursos do~\emph{Fedro}.

No testemunho \versal{B}22c recolhido por Davies (1991, p.\,145) temos a expressão
``a tríade de Estesícoro~(\emph{tà tría tôn Stesichórou})'',
referindo à divisão ou disposição triádica dos movimentos ou momentos do
coro (tria~\emph{mére}~carminum choricorum), invenção atribuída a
Estesícoro e que ajuda a explicar a escolha de Platão por três recitações,
obedecendo o ``desenho estesicórico''. O~\emph{pattern}~\versal{AAB} (estrofe,
antístrofe e epodo), segundo Francesca D'Alfonso, provinha da mimese do
movimento da natureza, sendo a estrofe uma mimese das estrelas fixas, a
antístrofe mimese dos planetas, por seus movimentos contrários
(retrogradação), e o epodo mimese da fixidez da terra:

 

\begin{quote}
Os dançarinos, que na estrofe se movem para a direita (ou esquerda),
simbolizam o movimento do cosmo, na antístrofe, executando o movimento
contrário, são expressões do movimento do Sol e dos planetas, e enquanto
cantam em seu posto (epodo) figuram a posição da terra (\emph{stásis tês
gês}). (D'Alfonso, 1994, p.~19)\footnote{I danzatori che nella strophe si
  muovono verso destra (o sinistra) simbolizzano il movimento del cosmo,
  nell'antistrophe, esequendo il movimento contrario, sono espressione
  del movimento del sole o dei pianeti, mentre com il canto sul posto
  (epodo) raffigurano la \emph{stásis tês gês}.}.
\end{quote}

 

D'Alfonso, ao mostrar a origem astronômica desse vocabulário da
tripartição, apresenta uma leitura do caráter performático do poeta de
Himera, observando a importância do aspecto coral combinado à
performance citaródica. A~estrutura triádica permite uma execução coral
mais elaborada e Estesícoro teria sido o primeiro a estabelecer um coro
na citaródia tradicional (Suda, \versal{IV}, 433). Nesse modelo há também o
movimento mudo de parte do coro, cujos cantos poderiam soar combinados
agora com uma dança silenciosa, imitando sempre a complexidade dos
fenômenos celestes. Em seu exímio trabalho em torno de Estesícoro,
D'Alfonso aponta para a analogia, muito valiosa, entre o movimento
celeste e o do coro, reforçando elementos de religião~astral em
Estesícoro e em Platão.

Para ilustrar a relação mimética entre os movimentos da poética (do
coro) com relação aos astros, destacaremos um trecho antigo de Siriano
que descreve como o movimento triádico imitava os fenômenos celestes no
seu~\emph{Comentário ao de Hermógenes.~}Segundo Siriano, a tríade era um
``sistema métrico (\emph{systémata métron})'':

 

\begin{quote}
[326, 32] estrofe, antístrofe e epodo são um sistema métrico para
poemas cômicos, trágicos e líricos. Da estrofe surgem os primeiros
períodos, a maioria compostos por versos (\emph{kólon}) semelhantes ou
diferentes combinados, como em Alcman: \emph{hieròn hýmnon} e graciosa dança
dispostos\textgreater{}. A~estrofe parte de três [formas]: versos
(\emph{kólon}), dactílicos e isométricos. Há combinações entre
dessemelhantes, quando se diz: antístrofe é aquilo que vem depois da
estrofe, estabelecendo um período similar, semelhante à estrofe no
número e na extensão dos versos (\emph{kólon}). Esse nome é dado pelos
movimentos (\emph{strephoménos}) e contramovimentos
(\emph{antichoreúontas}) do coro na sua dança, que se alternam ao redor
dos altares, templos e nas orquestras, quando cantam aquela melodia que
imita o ritmo do movimento contrário do céu frente ao movimento dos
planetas. Epodo é um dos períodos da tríade, diferente da estrofe e da
antístrofe, tanto na quantidade de versos, como na extensão e
combinação. Eles cantam parados no coro, imitando a fixidez do trono
(\emph{hedraîon}) terrestre.\footnote{Siriano.
  \emph{Commentarium in Hermogenis librum} περὶ ἰδεῶν, \emph{Syriani in
  Hermogenem commentaria,} vol. 1, Ed. Rabe, H. Leipzig: Teubner, 1892,
  p.62. {[}326, 32{]}.}
\end{quote}

 

Siriano evidencia a assimilação entre o movimento do coro e dos
fenômenos celestes, bem como a mimese da fixidez terrestre,
classificando duas vezes o movimento do coro como imitação do céu. Nesse
trecho é possível reconhecer a tríade como uma matriz poética específica
e, a partir disso, observar como esse desenho triádico sugere uma
interpretação esquemática do~\emph{Fedro,~}de acordo com essa mimese da
natureza celeste.

As três recitações iniciais do~\emph{Fedro}~seguem então
a~\emph{dispositio}~poética da tríade: (a) movimento inicial (estrofe --
Lísias), (b) contramovimento (antístrofe -- Sócrates 1) e (c) recitação
solo, fixa sob os próprios pés (epodo -- Sócrates 2). Há uma relação
entre o~\emph{pattern}~\versal{AAB}, que é especialmente formal, com relação à
métrica, no sentido de serem similares (\versal{AA}) no número de versos e no
ritmo, enquanto o epodo (B) tem um desenho métrico à parte. No caso do
conteúdo também ocorre algo similar, pois \versal{AA} (que correspondem aos
discursos de Lísias e de Sócrates 1) são duas invectivas contra Eros,
enquanto B (o discurso de Sócrates 2) é, ao contrário, um encômio a
Eros. Boa parte do que está relacionado a Estesícoro é recolhida em
função de uma interpretação do~\emph{Fedro~}que destaca como Platão
reelabora temas e formas da poética arcaica em sua filosofia. É~possível
reconhecer, portanto, a mimese da estrutura triádica estesicórica
no~\emph{Fedro}, bem como os supracitados~\emph{tópoi}~estesicóricos.

Embora não tenha sido o único a ressaltar a questão do ídolo
(\emph{eidôlon}) como cópia, Platão o faz em função de uma construção
que funcionará em sua filosofia como diferença fundamental entre
aparência e essência. O ídolo resgatado de Estesícoro é em Platão um dos
motores na condução das almas (psicagogia), de acordo com a habilidade
do artífice em produzir a imagem ou um discurso persuasivo para a alma,
levando em consideração se ela é vulnerável ou não àquele encantamento.
A~imagem é um ídolo do ser, assim como a escrita é um ídolo da palavra
viva.

A mimese, um procedimento vital da cultura mnemônica grega, propicia que
Estesícoro mimetize Homero, construindo histórias diferentes a partir
desse corpo textual, como é o caso do ídolo (\emph{eidôlon}) de Helena,
lugar"-comum que o leva a novos lugares discursivos. A~poesia trágica
igualmente imita e transforma inúmeros episódios homéricos e Platão
talvez tenha escolhido Estesícoro justamente por representar bem, dentro
da tradição poético"-literária, o poder encantatório do ídolo
(\emph{eidôlon}) e do~\emph{lógos}~sobre as almas.

A poética de Estesícoro atualiza a relação entre ser e imagem (ídolo) na
filosofia platônica, especialmente ao circunscrever o poder das imagens
ao atingir as almas. Ela revela a importância do corpo, do ídolo, como
cópias imperfeitas da palavra viva, nas afecções da visão e nas imagens
discursivas. A~palavra dialogada, poética e logográfica, obedece, cada
uma delas, a um determinado fim, de modo que o diálogo percorre as
diversas modalidades pelas quais a alma pode ser demovida.
No~\emph{Fedro},~o discurso escrito e a performance discursiva oral são
habilidades distintas, lembrando que o discurso vivo seria um meio de
acesso à memória daquilo que está inscrito nas almas, com a ajuda da
dialética, bem como, às vezes, a palavra escrita, como vimos, pode ser
também um meio de acesso, caso o leitor tenha já acessado aqueles
saberes.

 

\section{Lísias e Isócrates -- dois mudos no tribunal de Platão}

 

Há uma correspondência entre os prediletos de Fedro e de Sócrates,
notadamente Lísias e Isócrates, no final do diálogo. Depois de separadas
definitivamente as características da retórica e da filosofia, ou da
retórica vulgar e da retórica filosófica (amparada pela dialética),
Lísias será aconselhado por Fedro a abandonar as suas impiedades
discursivas. Isócrates, a seu lado, embora pareça habilitado por
Sócrates como representante de uma retórica potencialmente filosófica,
não é efetivamente elogiado.

Lísias e Isócrates, apesar da fama de excelentes escritores, tendo cada
qual suas habilidades, permanecem rebaixados na filosofia de Platão.
Ambos, paravituperados (267a4), apresentam tudo aquilo com o que Platão
se indispõe, Lísias por ser meteco e escritor de discursos para o
tribunal, Isócrates por ser ateniense e escrever discursos políticos de
uma perspectiva com que Platão definitivamente não simpatizava. A~formação
(\emph{paideia}) isocrática, como diz Jaeger, é um meio termo entre a
sofística e um discurso político voltado para a ação humana, e não
pretende ser uma filosofia naqueles termos platônicos, mas uma
``filosofia retórica''. Isócrates representa uma classe de pensadores
avessos à especulação filosófica pura, mas que faz uso largo de alguns
desses recursos numa espécie de filosofia política.

Lísias era filho de um produtor de escudos siracusano, o meteco Céfalo,
enquanto Isócrates era filho de Teodoro, um ateniense que produzia
flautas. Enquanto Lísias foi a Túrios estudar com Tísias, Isócrates
estudou com Górgias, também fora de Atenas. Só na época do governo dos
Trinta (404-403 a.C.), quando Isócrates volta para Atenas, parece ter
aprofundado sua atividade de logógrafo, tempo em que Lísias já era
conhecido, e já havia fugido para Mégara. A~prescrição de Heródico de
Mégara, da caminhada de ida e volta de Atenas a Mégara, é uma alusão ao
breve exílio de Lísias. Esses pequenos detalhes, entre outros, elucidam
o motivo pelo qual Isócrates é chamado por Sócrates de muito novo na atividade
logográfica (279a), pois começou tarde, se comparado a Lísias.
Além disso, mostra o sentido do encontro de Fedro e Sócrates no
extramuros (\emph{exô teíchous}), fora da cidade, uma vez que a
discussão que travam acerca da retórica e da logografia diz respeito
especialmente a uma atividade ligada aos estrangeiros, metecos ou não,
bem como aos filhos dos ricos que foram estudar fora de Atenas a arte da
palavra, como Isócrates. Sem ter participado ativamente da~\emph{pólis},
e isso é obviamente condenável para Platão, Isócrates volta a Atenas
numa época em que sua própria família já não tinha tantos recursos.

Sem nenhuma habilidade oratória performática, Isócrates carrega essa
marca, que Platão não omite em seu decalque, de não ter uma boa dicção,
como se essa marca revelasse um aspecto nada nobre da sua ligação com
Atenas, lugar onde especialmente se falava bem, jogo primordial em que
Isócrates é completamente inábil. É~possível que essa limitação tenha
alimentado, por outro lado, uma excelência ao escrever discursos, embora
Platão enfatize negativamente sua~\emph{táxis}~(\emph{dispositio}), um
isocratismo foi acolhido e muito repetido pelas escolas retóricas
subsequentes. Menandro caracteriza Isócrates como teorizador da beleza e
dignidade na escolha das palavras, bem como na harmonia e na disposição
de figuras (Men.~\emph{Rhet}. 339,14-24).

A dissociação entre o logógrafo e o orador, na prática jurídica
ateniense, permitiu a adaptação da redação à personalidade dos
``clientes'', preservando o verossímil (\emph{eikós}), para evitar que
um homem de educação mediana proferisse um discurso elegante e culto,
típico de um~\emph{rhétor}~experiente. A~atividade do logógrafo
permanece diferente da atividade do orador ou~\emph{rhétor}, no sentido
performático, daquele que age/\allowbreak{}fala nas assembleias, nos confrontos
políticos, nas deliberações, embora alguns reunissem as duas
habilidades, sendo logógrafos e oradores. Nesse sentido, a logografia
pôde fazer com que um meteco como Lísias adquirisse uma grande
consideração social como logógrafo e que, ao mesmo tempo, um inábil ao
falar, como Isócrates, tivesse também fama de excelente escritor e
fundasse uma escola\footnote{Cf. Córdova, P. V., Aquino, S., Juárez M.
  G. y Vidal, G. R.; \emph{Oratória griega y oradores áticos del primer
  period}, México: Unam, 2004, p.27-34; 93-120.}.

Platão ressalta a condição de meteco de Lísias, em sua incapacidade nas
instituições, e Isócrates aparece como a versão ateniense dessa
debilidade, sua carência performática aparece como quase um traço
intrínseco, ligado à persona de Isócrates que, sob a perspectiva de
Platão, ganha esse estigma indelével.

 

\section{Edições e traduções}

 

Muitas edições e traduções foram usadas ao longo desse trabalho, e é
natural que algumas dessas traduções tenham apoiado resoluções aqui
apresentadas, assim como na redação das notas. De modo geral, a edição
mais usada foi a de Burnet (1901). Destaco também o comentário realizado
por De Vries (1969), pois expõe no detalhe as diferenças entre as cópias
disponíveis e suas principais discussões. Esse comentário serve à
resolução de alguns problemas específicos de tradução/\allowbreak{}interpretação.

As traduções com que tive contato primeiramente do~\emph{Fedro}~foram a de
José Ribeiro Ferreira (1973), amparada por boas notas e soluções
elegantes, a de Carlos A\,Nunes (1975), tradução bastante fluida, mas
sem aparato crítico, e a tradução de José Cavalcante de Souza (1997,
p.357--368) para a palinódia (\emph{Phdr}. 243e-257b), tradução rigorosa
com resultado filosófico excelente.

Entre as traduções mais recentes em espanhol que foram materiais de
consulta destaco a de Santa"-Cruz y Crespo (2007), com tradução e aparato
crítico impecáveis, edição que poupou muito da nossa pesquisa, pois
realiza uma síntese bastante refinada com notas completas. Outra
tradução notável é a de Porratti (2010), bilíngue e com escolhas
rigorosas, muitas das quais segui. Porratti diante da profusão
do~\emph{Fedro}~fornece introdução, notas, notas ao texto grego e
comentário por trechos em edição preciosa.

Em italiano, destaco a tradução de Diano (1934), que serviu muitas vezes
como material de consulta. Em francês, consultei especialmente Vicaire
(1991), Chambry (1992) e Brisson (2006), todas com bastante proveito.
Entre as edições em inglês, certamente as mais conhecidas e citadas,
ressalto a de Hackforth (1952), especialmente por combinar comentários e
notas entre os blocos, os quais se encontram separados por subtítulos e
sinopses. O~formato de Hackforth prejudica a fluência, mas se justifica
pelo rico resultado. Rowe (1986), por outro lado, em sua edição bilíngue,
faz com que sua tradução privilegie a fluidez, pois não recorta o
diálogo em subseções e não carrega de notas, reservando seus comentários
ao texto grego, bastante importantes aliás, à seção final.

A edição de Yunis (2011), texto grego sem tradução, mas com um aparato
crítico monumental, foi muito proveitosa na revisão dessa tradução, pois
Yunis é extremamente generoso nas discussões gramaticais, comparações
literárias, elementos históricos, interpretações diversas, de um modo
bastante organizado. A~edição de Yunis é altamente recomendável para
quem conhece grego e busca aparato rigoroso.

 

\section{Bibliografia}

\subsection{Edições, traduções e comentários específicos dedicados
ao~\emph{Fedro}:}

 

\begin{bibliohedra}
\tit{Brisson}, L.~\emph{Phèdre,~}Paris: Flammarion, 1989.
\tit{Burnet}, J.~\emph{Platonis Opera}~tomus \versal{II}, Oxford: Oxford Classical
  Texts, 1901.
\tit{Chambry}, E.~\emph{Phèdre}, Paris: Flammarion, 1992.
\tit{De} Vries, G\,J.~\emph{A~Commentary on the Phaedrus of Plato},
  Amsterdam: Adolf M\,Hakkert"-Publisher, 1969.
\tit{Diano}, C.~\emph{Fedro}, Dialoghi vol. \versal{III}, Bari: Laterza, 1934.
\tit{Fernandez}, L\,G.~\emph{Fedro}, Madrid: Alianza, 2004.
\tit{Ferreira}, J\,R.~\emph{Fedro}, Lisboa: Verbo, 1973.
\tit{Fowler}, H\,N.~\emph{Phaedrus}, translated by Fowler in Plato I, Loeb
  Classical Library, Cambridge, Mass., 1914.
\tit{Gil}, L\,F.~\emph{Fedro}, traducción, notas y studio preliminar,
  Madrid: Instituto de Estudios Políticos, 1957.
\tit{Hackforth}, R.~\emph{Plato's Phaedrus}, translated with introduction
  and commentary, Cambridge: Cambridge University press, 1952.
\tit{Hermias} Alexandrinus.~\emph{In Platonis Phaedrum Scholia}, ad fidem
  codicis parisini 1810 denuo collati edidit et aparatu critic ornavit,
  Ed. Couvreur, Paris: Émile Bouillon, 1901.
\tit{Hermias} Alexandrinus.~\emph{In Platonis Phaedrum Scholia}, Ed.~Carlo
  M\,Lucarini \& Claudio Moreschini, Teubneriana, Berlin: De Gruyer,
  2012.
\tit{Lledó}, E.~\emph{Fedro}, Madrid: Gredos, 2008.
\tit{Nunes}, C\,A.~\emph{Fedro},~\emph{Cartas},~\emph{Primeiro Alcebíades},
  Belém: \versal{UFPA}, 1975.
\tit{Poratti}, A.~\emph{Fedro}, introdução, tradução e notas, España, Akal,
  2010.
\tit{Pucci}, P.~\emph{Fedro}, Bari: Laterza, 1998.
\tit{Robin}, L.~\emph{Platon, Oeuvres completes}, tome \versal{IV},~\emph{Phèdre},
  Paris: Les Belles Letres, 1933.
\tit{Rowe}. C\,J.~\emph{Plato: Phaedrus}, with translation and commentary by
  C.J\,Rowe, Warminster, 1986.
\tit{Ryan}, P.~\emph{Plato's Phaedrus, a commentary for greek readers},
  University of Oklahoma Press, 2012.
\tit{Santa}"-Cruz M\,I\,y Crespo, M\,I.~\emph{Fedro}, introdução, tradução e
  notas, España: Losada, 2007.
\tit{Souza}, J\,C\,de.~\emph{O~amor alado}, Platão,~\emph{Fedro}, 243e-257b,
  In Letras Clássicas, ano 2, No 2, São Paulo: Humanitas, 1998.
\tit{Thompson}, W\,H.~\emph{The Phaedrus of Plato}, edited with commentary
  by W.H\,Thompson, London: Wittaker and Co.,1868; repr. 1973.
\tit{Vicaire}, P\,\& Moreschini. C.~\emph{Phèdre}, Paris: Les Belles Letres,
  1998.
\tit{Vicaire}, P.~\emph{Phèdre}, Paris: Les Belles Letres, 1991.
\tit{Yunis}, H.~\emph{Phaedrus}, edited with commentary by Yunis, Cambridge:
  Cambridge Press, 2011.
\end{bibliohedra}

 

\subsubsection{Autores antigos:}

 

\begin{bibliohedra}
\tit{Anacreon}. Trad. Gentili, B., Roma: Athenaei, 1958.
\tit{Antiphon}. In~\emph{Minor Attic Orators}, Volume I: Antiphon.
  Andocides, trad. K\,J\,Maidment, Harvard University Press, 1941.
\tit{Apulée}.~\emph{Les Métamorphoses}, trad. Vallete, P., Paris: Les Belles
  Lettres, 1972.
\tit{Apuleio}, Lúcio.~\emph{O~asno de Ouro}. Trad. Francisco Antônio de
  Campos. Portugal: Europa"-América, 1990.
\tit{Aristophanes}.~\emph{Clouds. Wasps. Peace}, trad. Henderson J., Harvard
  University Press, 1998.
\tit{Aristotele}.~\emph{Etica Nicomachea}.~Trad. Natali. C., Bari: Laterza,
  1999.
\tit{Aristotele}.~\emph{Politica}, trad. Viano, C\,A\,Milano: Bur, 2008.
\tit{Aristóteles}.~\emph{Física}~I"-\versal{II}, trad. Angioni, L., Campinas: Unicamp,
  2002
\tit{Aristotelis}.~\emph{Physica}, Ed.~Ross, W.D\,Oxford: Clarendon Press,
  1950, Repr. 1966.
\tit{Aristotle}.~\emph{On Rhetoric}, trad. G\,A\,Kennedy, Oxford: Oxford
  University Press, 1991.
\tit{Aristotle}\emph{. Posterior Analytics. Topica}, trad. Hugh Tredennick
  \& E\,S\,Forster, Harvard University Press, 1960.
\tit{Aristotle}.~\emph{The Art of rhetoric}, trad.~Freese J\,H., Michigan:
  2006.
\tit{\emph}{Comicorum Atticorum Fragmenta}~vol. 1., Ed. Kock,
  T\emph{.,~}Leipzig: Teubner, 1880.
\tit{Élio} Aristides.~\emph{Pròs Platona perì rethorikês,~}Ed. Dindorf, W\,  Leipzig: Reimer, 1829, Repr. 1964.
\tit{Eurípides}.~\emph{Bacantes}, trad. \versal{JAA} Torrano, São Paulo: Hucitec,
  1995.
\tit{Euripides}.~\emph{Hipólito}, trad. Miralles, C., Barcelona: Bosch,
  1977.
\tit{Euripides}. \versal{VII},~\emph{Fragments}:~\emph{Aegeus"-Meleager}, Ed.
  Christopher Collard \&Martin Cropp, Harvard University Press, 2008.
\tit{Gorgias} .~\emph{Reden, Fragmente und Testemonien}. Thomas Büchhein
  [Hrsg] Hamburg: Verlag, 1989.
\tit{Górgias}.~\emph{Elogio de Helena},~\emph{Tradado do não"-ente}, trad.
  Maria Cecília de M\,N\,Coelho, Cadernos de Tradução 4, São Paulo:
  Edusp, 1999.
\tit{Halicarnassus}, Dionysius of.~\emph{Critical Essays}, vol. 1, trad.
  Usher, S., Harvard University Press, 1974.
\tit{Hermógenes}.~\emph{Commentarium in Hermogenis librum}~perì staseôn,
  vol. 2, Ed.~Rabe, H\,Leipzig: Teubner, 1893.
\tit{Heródoto}.~\emph{História}, trad. Kury, M.G\,Brasília: UnB, 1985.
\tit{Hesiodi}~\emph{Theogonia; opera et dies};~\emph{Scutum}, Oxford: Oxford
  University Press, 1990.
\tit{Hesíodo}.~\emph{Teogonia}, a origem dos deuses, Trad. Torrano, \versal{JAA}.,
  São Paulo: Iluminuas, 1995.
\tit{Hipócrates}.~\emph{Aforismos}, trad. J\,M\,de Rezende, São Paulo:
  Unifesp, 2010.
\tit{Hipócrates}.~\emph{Da natureza do homem}, trad. Cairus, Henrique.
  F.,~\emph{História, Ciências, Saúde}~-- Manguinhos, Rio de Janeiro: \versal{VI}
  (2) 1999.
\tit{Hipócrates}.~\emph{Sobre o riso e a Loucura}, trad. Campos. R\,G\,de,
  São Paulo: Hedra, 2011.
\tit{Hipócrates}.~\emph{Tratados hipocráticos}, trad. Nava, Gual, Férez y
  Alvarez, Barcelona: Gredos, 2000.
\tit{Isocrates} Opera Omnia, vol. 1 \& 2, ed.~Mandilaras, B\,G., Germany:
  Verlag, 2003.
\tit{Isocrates}. I, Oratory of Classical Greece, trad. David C\,Mirhady \&
  Yun Lee Too, Texas: University of Texas Press, 2013.
\tit{Isocrates}. Vol. I trad. G\,Norlin, Harvard University Press, 1928.
\tit{Isocrates}'~\emph{Antidosis}, a commentary, Too, Y\,L., New York:
  Oxford University Press, 2008.
\tit{Laertius}, D.~\emph{Lives of Eminent Philosophers}, 2 Vols, trad.
  Hicks, R\,D., Harvard University Press, 1925.
\tit{Luciani} Samosatensis Opera, ed. Jacobitz, Karl Gottfried, Vol. 2,
  Leipzig: Teubneri, 1913.
\tit{Lysias}. Trad. W\,R\,M\,Lamb, Harvard University Press, 2006.
\tit{Menander} (Rhetor).~\emph{Division of Epideictic Speeches}, ed. Russel,
  D\,A\,\& Wilson, N\,G., Oxford: Oxford University Press, 1981.
\tit{Omero}.~\emph{Iliade,~}trad. Onesti, R\,C.,Torino: Einaudi, 1990.
\tit{Omero}.~\emph{Odissea}, trad. Onesti, R\,C.,Torino: Einaudi, 1989.
\tit{Pausanias}.~\emph{Description of Greece}, Leipzig: Teubner, 1967.
\tit{Photius}.~\emph{Bibliothèque}, Ed. Henry, R., Paris: Les Belles
  Lettres, 1977.
\tit{Pindar}.~\emph{Nemean Odes. Isthmian Odes. Fragments}. Trad. William H\,  Race, Harvard University Press, 1997.
\tit{Platão}.~\emph{Apologia de Sócrates, Eutifron,
  Críton,~}trad\emph{.}~Malta, André. Porto Alegre: L\&\versal{PM}, 2014.
\tit{Platão}.~\emph{Filebo}, trad. Muniz, F\,São Paulo: \versal{PUC} Rio \& Loyola,
  2015.
\tit{Platão}.~\emph{Górgias}. Trad. Lopes, Daniel. R\,N., São Paulo:
  Perspectiva, 2011.
\tit{Platão}.\emph{~Íon, Hípias menor,~}trad\emph{.}~Malta, André. Porto
  Alegre: L\&\versal{PM}, 2007.
\tit{Platão}.~\emph{O~Banquete, Apologia de Sócrates,}~trad.~Nunes, C\,A.,
  Pará: \versal{UFPA}, 2001.
\tit{Platão}.~\emph{Parmênides}, trad. Iglésias, M\,\& Rodrigues, F\,São
  Paulo: \versal{PUC} Rio \& Loyola, 2003.
\tit{Platão}.~\emph{Protágoras, Górgias, Fedão,}~trad.~Nunes, C\,A., Pará:
  \versal{UFPA}, 2002.
\tit{Platão}.~\emph{República,~}trad. Pereira, Maria H\,da R., Portugal:
  F.C\,Gulbenkian, 1993.
\tit{Platão}.~\emph{Teeteto, Crátilo,}~trad. Nunes, Pará: \versal{UFPA}, 2001.
\tit{Platão}.~\emph{Timeu, Crítias, Segundo Alcebíades, Hípias Menor},
  trad.~Nunes, Pará: \versal{UFPA}, 2001.
\tit{Platon}.~\emph{Lysis}, trad.~Croiset, A\,Paris: Les Belles Lettres,
  1999.
\tit{Platone}.~\emph{La Repubblica}, trad. Lozza, G\,Milano: Mondadori,
  1990.
\tit{Platone}.~\emph{Politico}, trad. Giorgini, G., Milano: \versal{BUR}, 2005.
\tit{Platone}.~\emph{Teeteto}, trad. Ferrari, F., Milano: \versal{BUR}, 2011.
\tit{Plotino}.~\emph{Sobre o Belo~}In~\emph{Tratados das Enéadas},
  trad.~Sommerman, A., São Paulo: Polar, 2000.
\tit{Safo} de Lesbos.~\emph{Hino a Afrodite e outros poemas}, trad. Ragusa,
  G., São Paulo: Hedra, 2011.
\tit{\emph}{Scholia graeca in aristophanem}, Dübner, F.\emph{,~}Paris:
  Didot, 1969.
\tit{Strabo}.~\emph{Geography}, Volume V: Books 10--12, trad. H\,L\,Jones,
  Harvard University Press, 1928.
\end{bibliohedra}

 

 

\subsection{Bibliografia Geral:}


\begin{bibliohedra}
\tit{Albert}, K.~\emph{Platonismo, caminho e essência do filosofar
  occidental}, São Paulo: Loyola, 2011.
\tit{Arnaoutoglou}, I.~\emph{Leis da Grécia Antiga}, São Paulo: Odysseus,
  2003.
\tit{Ast}, Friedrich\emph{. Lexicon Platonicum}, 3 vols. Lipsiae: Weidmann,
  1835.
\tit{Austin}, N.~\emph{Helen of Troy and her shameless phantom}. New York:
  Cornell University Press, 1994.
\tit{Barnabé}, A.~\emph{Platón y el orfismo}, Madrid: Agapea, 1990.
\tit{Benson}, H\,H\,[et. al.]~\emph{Platão}, trad. Zingano, M\,A., Porto
  Alegre: Artmed, 2011.
\tit{Berg}, R\,M\,van den.~\emph{Proclus' hymns}: essays, translations and
  commentary, Köln: Brill, 2001.
\tit{Bowra} C.M.~\emph{Stesichorus in the Peloponnese}, The Classical
  Quarterly, v. 28, n.2;1934.
\tit{Bowra} C.M.~\emph{The Two Palinodes of Stesichorus}, The Classical
  Review 13,03;1963.
\tit{Brisson}, L.~\emph{Orphée --- poèmes magiques et cosmologiques}, Paris:
  Les Belles Lettres, 1993.
\tit{Brisson}, L.~\emph{Proclus et l'orphisme}~In Proclus, lecteur et
  interprète des anciens, Paris: Les Belles Letres, 1987.
\tit{Bruce}, I\,A\,F\emph{. Athenian Embassies in the Early Fourth
  Century}~B.C\,Zeitschrift für Alte Geschichte, v. 15, n.3, Franz
  Steiner Verlag, 1966.
\tit{Cairus}, Henrique F\,\& Ribeiro Jr., W\,A.~\emph{Textos
  Hipocráticos}:~\emph{o doente, o médico e a doença}, Rio de Janeiro:
  Fiocruz, 2013.
\tit{Cambiano} G.~\emph{Dialettica, medicina, retorica nel Fedro platonico}.
  Rivista di Filosofia, 57:284--305, 1966.
\tit{Campbell}, David.~\emph{Greek Liric}~\versal{III}, Harvard University Press,
  1991.
\tit{Campos}, R\,G\,de~\emph{O~Fedro de Platão à luz da tríade de
  Estesícoro}, 2012.~Doutorado em Filosofia, São Paulo: \versal{FFLCH}"-\versal{USP}.
  Disponível em: . Acesso em: 2015--09--30.
\tit{Cardoso}, D.~\emph{A~alma como centro do filosofar de Platão}, São
  Paulo: Loyola, 2006.
\tit{Casadio}, Giovanni.~\emph{La metempsicosi tra Orfeo e Pitagora}, In
  Orphisme et Orphée, (Org.) Borgeaud, Philippe, Genève: Droz, 1991.
\tit{Cassin}, B.~\emph{O~efeito sofístico}, Brasil: Ed. 34, 2005.
\tit{Cerri}, G.~\emph{Il ruolo positivo della scrittura secondo Il Fedro di
  Platone}.~In Understanding the Phaedrus, (Org.) Rosseti, L., Sankt
  Augustin: Verlag, 1992.
\tit{Chantraine} P.~\emph{Dictionnaire Etymologique Langue Grecque: Histoire
  des mots}, Paris: Klincksieck 1968.
\tit{Cherniss}, H.~\emph{Selected Papers}, Leiden: Brill, 1977.
\tit{Colli}, G.~\emph{La Nascita della filosofia}, Milano: Adelphi, 1975.
\tit{Córdova}, P\,V.; Aquino, S.; Juárez, M\,G\,\& Vidal, G\,R\,  ;~\emph{Oratória griega y oradores áticos del primer periodo}; México:
  Unam, 2004.
\tit{Cornford}, F\,M.~\emph{Principium Sapientae, as origens do pensamento
  filosófico grego}, Lisboa: Calouste Gulbenkian, 1989.
\tit{Corrêa}, P.C.~\emph{Harmonia -- mito e música na Grécia antiga}, São
  Paulo: Humanitas, 2003.
\tit{D}'Alfonso, F.~\emph{La ``choreía'' astrale in un passo
  del~}Fedro\emph{~platônico}, Helikon, 33--34, 1993--1994.
\tit{D}'Alfonso, F.~\emph{Stesicoro et la performance}, Roma: \versal{GEI}, 1994.
\tit{Davies}, M.~\emph{Poetarum Melicorum Graecorum Fragmenta}, Oxford,
  1991.
\tit{Defradas}, J.~\emph{Les thèmes de la propagande delphique}, Paris:
  Librairie G\,Klincksieck, 1954.
\tit{Denniston}, J\,D.~\emph{Greek prose style}, London: Bristol, 2002.
\tit{Derrida}, J.~\emph{A~Farmácia de Platão}, São Paulo: Iluminuras, 1997.
\tit{Diels}, Herman \& Kranz, Walther.~\emph{Die fragmente der
  vorsokratiker}, 3 vol.~Berlin: Weidmann, 1989.
\tit{Dixsaut}, M.~\emph{Métamorphoses de la dialectique de Platon}, Paris:
  Vrin, 2001.
\tit{Dodds}, E.R.~\emph{The Greeks and the Irrational}, London: University
  of California Press, 1997.
\tit{Dover}, K\,J.~\emph{A~Homossexualidade na Grecia Antiga}, São Paulo:
  Nova Alexandria, 2007.
\tit{Dover}, K\,J.~\emph{Greek word order}, London: Bristol, 2001.
\tit{Entralgo}, P\,L.~\emph{La curación por la palabra en la antigüidad
  clásica}, Barcelona: Anthropos, 1987.
\tit{Fattal}, M.~\emph{L'alethès lógos du Phèdre
  en}~270c10~\emph{In}~Rosseti, L\,(Org.)~\emph{Understanding the
  Phaedrus}, Sankt Augustin: Verlag, 1992.
\tit{Ferrari}, G\,R\,F.~\emph{Listening to the Cicadas: A Study of Plato's
  Phaedrus}, Cambridge: Cambridge Classical Studies, 1990.
\tit{Festugière} A\,J\emph{. Platon et L'orient,~}In Études de philosophie
  grecque, Paris: J\,Vrin, 1971.
\tit{Fontes}, J\,B.~\emph{Eros, tecelão de mitos -- a poesia de Safo de
  Lesbos}, São Paulo: Estação Liberdade, 1991.
\tit{Friedländer}, P.~\emph{Platone}, Milano: Bompiani, 2004.
\tit{Furley} W\,D\,\& Bremer J\,M.~\emph{Greek Hymns}, vol.1, Tübingen: Mohr
  Siebeck, 2001.
\tit{Gadamer}, H"-G.~\emph{A~ideia do Bem entre Platão e Aristóteles}, São
  Paulo: Martins Fontes, 2009.
\tit{Gaiser}, K.~\emph{La metafísica della storia in Platone}, Milano: Vita
  e Pensiero, 1991.
\tit{González}, J.~\emph{Psique y Eros en el~}Fedro, In~\emph{Platón: Los
  diálogos tardíos.}~Conrado, Eggers L\,(Org.) Actas del Symposium
  Platonicum, Sankt Augustin: Verlag, 1986.
\tit{Graves}, R.~\emph{The Greek Myths}, 2 vol.~Great Britain, 1960.
\tit{Grimal}, P.~\emph{Dicionário da Mitologia grega e romana}, Rio de
  Janeiro: Bertrand, 1997.
\tit{Guthrie}, W\,K\,C.~\emph{The Greek philosophers}, London: Harper
  Colophon, 1975.
\tit{Guthrie}, W.K.C.~\emph{Os sofistas}, Brasil: Paulus, 1991.
\tit{Heath}, M.~\emph{The unity of Plato's Phaedrus}, In Oxford Studies in
  Ancient Philosophy, volume \versal{VII}, Oxford, p.151--173, 1989.
\tit{Heath}, M.~\emph{The unity of the Phaedrus: a postscript}, In Oxford
  Studies in Ancient Philosophy, volume \versal{VII}, Oxford, p.189--191, 1989.
\tit{Hösle}, V.~\emph{Interpretar Platão}, São Paulo: Loyola, 2004.
\tit{Howland}, R\,L.~\emph{The attack on Isocrates in the Phaedrus}, The
  Classical Quarterly, 31:151--9; 1937.
\tit{Irwin} T.~\emph{Plato's ethics}, New York: Oxford, 1995.
\tit{Jaeger}, W.~\emph{Paideia}, México: Fondo de Cultura Economica, 2004.
\tit{Joly}, Robert.~\emph{La question hippocratique et le « Phèdre »,}~In
  \versal{REG}, \versal{LXXIV}, 1961.
\tit{Jouanna}, J.~\emph{Hippocrate}, Paris: Fayard,1992.
\tit{Käpel}, Lutz.~\emph{Paian, Studien zur Geschichte einer Gattung},
  Berlin: Walter de Gruyter, 1992.
\tit{Kastely}, J\,L.~\emph{Respecting the rupture: not solving the problem
  of Unity in Plato's Phaedrus}, In Philosophy and Rhetoric, Ed. Penn
  State University, Vol. 35, No 2, 2002.
\tit{Kelly}, A.~\emph{Stesikhoros and Helen}, Suisse: Museum Helveticum, v.
  64, 2007.
\tit{Kennedy}, G\,A.~\emph{Invention and Method: two rethorical treatises
  from the Hermogenic Corpus,}~Atlanta: Society of Biblical Literature,
  2005.
\tit{Kerényi}, K.~\emph{Arquétipos da religião grega}, Rio de Janeiro:
  Vozes, 2015.
\tit{Kern}, O.~\emph{Orphicorum Fragmenta}, Zurich: Weidmann, 1972.
\tit{Kirk}, G\,S, Raven, J\,E\,\& Schofield, M.~\emph{Os filósofos
  pré"-socráticos}, Lisboa: Calouste Gulbenkian, 2010.
\tit{Krämer}, Hans.~\emph{Platone e i fondamenti della metafísica}, Milano:
  Vita e pensiero, terza edizione, 1989.
\tit{Krämer}, Hans.~\emph{La nuova imagine di Platone}, Napoli: Bibliopolis,
  1986.
\tit{Kraut}, R\,(ed.)~\emph{The Cambridge companion to Plato}, Cambridge:
  Cambridge University Press, 1992.
\tit{Kucharski} P.~\emph{La rhétorique dans le Gorgias et le Phèdre}, In
  \versal{REG}, \versal{LXXIV}, 1961.
\tit{Lidell}, H.G., Scott, R\,\& Jones, H\,S.,~\emph{A~Greek"-English
  Lexicon}, Oxford, 1996.
\tit{Marques}, M\,P.~\emph{Platão, pensador da diferença --- uma leitura do
  Sofista}, Belo Horizonte: \versal{UFMG}, 2006.
\tit{Marrou}, H"-I.~\emph{Histoire de l'Education dans L'Antiquité}, Paris:
  Seuil, 1965.
\tit{Mattéi}, J"-F.~\emph{Platão}, São Paulo: Unesp, 2010.
\tit{Mazzara}, G.~\emph{Gorgia, la retorica del verosimile}, Verlag, 1999.
\tit{Moravcsik}, J.~\emph{Platão e platonismo}, São Paulo: Loyola, 2006.
\tit{Moss}, J.~\emph{Soul"-Leading: The unity of the Phaedrus, Again}, In
  Oxford Studies in Ancient Philosophy, volume 43, Oxford: 2012.
\tit{Mossé}, C.~\emph{Histoire d'une démocratie: Athènes}, Paris: Editions
  du Seuil, 1971.
\tit{Mossé}, C\emph{. Péricles, o inventor da democracia}, São Paulo:
  Estação Liberdade, 2008.
\tit{Nightingale}, Andrea W.~\emph{Genres in dialogue: Plato and the
  construct of philosophy}, Cambridge University Press, 1995.
\tit{Paleologos}, K\emph{. As modalidades esportivas}, In~\emph{Os Jogos
  Olímpicos na Grécia Antiga}, Vários Autores, 2004, São Paulo:
  Odysseus.
\tit{Pradeau} J"-F\,\& Brisson, L.~\emph{Dictionnaire Platon}, Paris:
  Elipses, 1998.
\tit{Pulquério}, M\,de O.~\emph{O~problema das duas Palinódias de
  Estesícoro}, Portugal: Humanitas -- 25/\allowbreak{}26, 1973.
\tit{Ragon}, E.~\emph{Grammaire grecque}, Paris: Gigord, 1961.
\tit{Reale}, G.~\emph{Por uma nova interpretação de Platão}, São Paulo:
  Loyola, 1997.
\tit{Rijksbaron}, A.~\emph{The syntax and Semantics of the verb in classical
  Greek}, Chicago: University of Chicago Press, 2002.
\tit{Robin}, L.~\emph{Théorie platonicienne de l'amour}, Paris: Felix Alcan,
  1908.
\tit{Robinson}, T\,M.~\emph{A~psicologia de Platão}, trad. Marques, Marcelo
  P., São Paulo: Loyola, 2007.
\tit{Robinson}, T\,M.~\emph{The argument for immortality in
  Plato's}~\emph{Phaedrus}, In Essays in ancient Greek Philosophy, Ed.
  John Anton \& George Kustas, \versal{NY} Press, 1971.
\tit{Rosseti}, L\,(Org.)~\emph{Understanding the Phaedrus}, Sankt Augustin:
  Verlag, 1992.
\tit{Rougier}, L.~\emph{La religion astrale des pythagoriciens}, Paris:
  Presses Universitaires de France, 1959.
\tit{Rowe}, C.~\emph{The unity of the Phaedrus: a replay to Heath}, In
  Oxford Studies in Ancient Philosophy, volume \versal{VII}, Oxford, 1989.
\tit{Schäfer}, Christian (Org.)~\emph{Léxico de Platão}, São Paulo: Loyola,
  2012.
\tit{Schefer}, C.~\emph{Rhetoric as part of an initiation into the
  mysteries: a new interpretation of the platonic Phaedrus}, In Plato as
  author: the rhetoric of philosophy, edited by Michelini, Ann N.,
  Leiden, Boston: Brill, 2003.
\tit{Slezák}, T.~\emph{Ler Platão}, São Paulo: Loyola, 2005.
\tit{Smith}, H\,W.~\emph{Greek Grammar}, 2\textsuperscript{nd}~ed. 1920
  [revised by Messing, 1956].
\tit{Sorel}, R.~\emph{Les cosmogonies grecques}, Paris: \versal{PUF}, 1994.
\tit{Trabattoni}, F.~\emph{Oralidade e escrita em Platão}, trad. Bolzani
  Filho, R\,\& Puente, F\,R., São Paulo: Discurso Editorial, 2003.
\tit{Trabattoni}, F.~\emph{Scrivere nell'anima. Verità, dialettica e
  persuasione in Platone}, Firenze: La Nuova Italia, 1994.
\tit{Trivigno}, F.~\emph{Putting Unity in its Place: Organic Unity in
  Plato's Phaedrus}, In Literature and Aesthetics, vol. 19, Sidney:
  Sidney Society of Literature \& Aesthetics, 2009.
\tit{Untersteiner}. M.~\emph{I sofisti}, Milano: Mondadori, 1996.
\tit{Urbina}, J\,M P.~\emph{Diccionario bilingüe manual Griego clássico-
  Español}, Madrid: Vox, 2012.
\tit{Vernant}, J"-P.~\emph{Mito e religião na Grécia antiga}, Campinas:
  Papirus, 1992.
\tit{Vernant}, J"-P.~\emph{Mito e sociedade na Grécia antiga}, Brasil: \versal{UNB},
  1992.
\tit{Vidal}, G\,R.~\emph{La dimensión política de la retórica griega},
  Rétor, 1, 1, Argentina: 2011.
\tit{Vidal}, G\,R.~\emph{Notas sobre la retórica de Isócrates}, Noua Tellus,
  Anuario del Centro de Estudios Clásicos, 24, 1, México: Unam, 2006.
\tit{Vürtheim}, J.~\emph{Stesichoros' Fragmente und Biographie}, Lieden:
  A.W\,Sijthoff's, 1919.
\tit{Watanabe}, Lygia A\emph{. Platão -- por mitos e hipóteses,}~São Paulo:
  Moderna, 1995.
\tit{Watanabe}, Lygia A\emph{. Vérité du mythe, vérité de l'histoire chez
  Platon,~}Paris: \versal{EHESS}, 1983.
\tit{West}, M\,L.~\emph{Greek Metre}, London: Oxford, 1982.
\end{bibliohedra}
