\section{Personagens: Sócrates e
Fedro}\label{personagens-suxf3crates-e-fedro}

{[}227a{]} S: Ó querido Fedro, de onde vens e para onde vais?

F: Venho de junto de Lísias, o filho de Céfalo, ó Sócrates. Atravessei o
passeio por fora dos muros (\emph{éxô} \emph{teíchous}) porque permaneci
sentado durante muito tempo, desde cedo. Persuadido pelo teu e meu amigo
Acúmeno\footnote{Médico, pai de Erixímaco, o qual faz um dos discursos
  em defesa do Amor no \emph{Banquete} 185c-189b. Pai e filho são
  mencionados novamente no \emph{Fedro} 268a-c como médicos ilustres,
  assim como no \emph{Protágoras} 315c2-3.}, percorri o passeio pelas
estradas (\emph{hodoùs}), pois, segundo ele, esse caminho é menos
cansativo que o realizado pelas vias (\emph{drómois}) do
pórtico.\footnote{Dois caminhos, um curto e um longo, servem de metáfora
  inicial das maneiras pelas quais se pode adquirir conhecimento, seja
  ele qual for. Os sofistas e a sofística são o símbolo do caminho
  curto, do atalho, meio pelo qual se poderia adestrar o discurso e o
  pensamento dos aprendizes, mas a filosofia platônica a esse caminho se
  contrapõe, defendendo um caminho longo de formação, o único capaz de
  conduzir discípulos à sabedoria e à verdade.}

{[}227b{]} S: É belo o que dizes companheiro. E parece que Lísias estava
na cidade.

F: Sim, com Epícrates\footnote{Epícrates sofreu processo público, para o
  qual há um epílogo no corpo de textos atribuídos a Lísias,
  \emph{Contra} \emph{Epícrates} (Lamb, 2006, p.576-587). Acusado de
  malversação do dinheiro público e de receber suborno em embaixadas que
  realizou, Epícrates é considerado ``rétor e demagogo'' na
  \emph{Scholia graeca in aristophanem} (Dübner, 1969, v.71,2), também é
  mencionado no trecho do poeta cômico Platão, que lembra o epíteto de
  Epícrates: \emph{sakesphóros} ou ``portador de escudo'', sendo que há
  um trocadilho pela homonímia entre ``escudo'' (\emph{sákos, eos, to})
  e ``bolsa'' (\emph{sákos, sákkos, ho}), pelo qual se lê também
  ``Epícrates portador de bolsa'', num jogo com o campo semântico da
  corrupção. (\emph{Comicorum Atticorum Fragmenta}, frg. 122, p. 633,
  vol. 1, Kock, T. Leipzig: Teubner, 1880). Ainda na \emph{Scholia
  graeca in aristophanem} \emph{in Eclesiazousa} temos um verso que
  ajuda nessa caracterização de Epícrates: ἄναξ ὑπήνης Ἐπίκρατες
  σακεσφόρε, ou seja, ``ó Epícrates, rei da barba porta-bolsa!''. Também
  em Plutarco, \emph{Pelopidas} 30,12, temos o epíteto de Epícrates:
  Ἐπικράτους γοῦν ποτε τοῦ σακεσφόρου, ou seja, ``Epícrates, o portador
  de bolsa/escudo {[}...{]}''.}, na casa de Morico\footnote{Moriquia ou
  casa de Morico, sinônimo de riqueza e ostentação. Morico era um
  \emph{bon-vivant} citado por Aristófanes (\emph{Pac}. 1008), conhecido
  pela glutonaria (\emph{Ach}. 887, \emph{Pax} 1008-9, \emph{Vesp}. 506)
  e pelos banquetes que oferecia, daí Sócrates dizer, em seguida, que
  Lísias servia ``um banquete'' (\emph{eistía}) de discursos. Em
  Aristófanes há ainda uma interessante caracterização de Morico e do
  seu ``nobre estilo de vida'' (\emph{zên bíon gennaîon}), um homem que
  ``pela manhã frequentemente se desgasta em processos de sicofanta''
  (\emph{orthrophoitosukophantodikotalaipṓrōn Vesp}. 505). Essa nobreza,
  \emph{gennaîon,} vista em Aristófanes, aparece logo em seguida, em
  227c9 do \emph{Fedro}, quando Sócrates diz ``ó nobre!'' (\emph{ô
  gennaîos}), de modo similar.}, aquela próxima ao templo de Zeus
Olímpico.

S: E qual era o debate (\emph{diatribḗ})? É certo que Lísias vos servia
um banquete (\emph{eistía}) de discursos!\footnote{Cf. Yunis, p.86-87,
  que, na sua edição comentada do \emph{Fedro}, ressalta o banquete
  (\emph{eistía}) de discursos de Lísias como um lugar-comum platônico,
  especialmente pela analogia recorrente entre o discurso direcionado
  para a alma e o alimento para o corpo.}

F: Saberás, se estás livre para prosseguir comigo e escutar.

S: O quê? Não sabes que para mim, como diz Píndaro, ``não há nada mais
elevado''\footnote{Píndaro, \emph{Ístmicas} 1,2 (Race, 1997, p.134).}
que escutar o debate (\emph{diatribḕn}) que tiveste com Lísias!

F: Avança, então.

{[}227c{]} S: Ao discurso.

F: Sócrates, de fato a audição interessa-te, o discurso acerca do qual
debatíamos (\emph{dietríbomen}) versava, não sei com que precisão,
acerca do amor (\emph{erôtikόs}). Lísias escreveu (\emph{gégraphe}) a
respeito da sedução (\emph{peirṓmenόn}) dos belos, mas não dos que estão
sob a tutela de um amante (\emph{ouch hyp'erastoû dé}), e aí mesmo
reside a sua habilidade, pois Lísias diz ser melhor agradar
(\emph{charistéon}) o não afetado pelo amor (\emph{mḕ erônti}) do que um
amante (\emph{erônti}).\footnote{Primeira enunciação da tese central do
  discurso de Lísias: ``é melhor agradar (\emph{charistéon}) alguém não
  afetado pelo amor (\emph{mḕ erônti}: não amante, não amoroso, não
  apaixonado) do que agradar um amante (\emph{erônti}: um amoroso, um
  erasta, um apaixonado, alguém afetado pelo amor)''. Esse será o tema
  das duas declamações iniciais do diálogo, enquanto a palinódia de
  Sócrates, ou terceira declamação, afirmará o contrário dessa tese.}

S: Ó nobre! Espero que ele possa escrever ser melhor a pobreza frente à
riqueza, a velhice frente à mocidade, sem contar outras coisas que
comigo e a muitos de nós acontecem.\footnote{Sócrates evidencia, com
  essa ironia, a capacidade do discurso de dizer o não-ser, ou seja, de
  jogar com as aparências de modo persuasivo. Na oposição entre a
  retórica e a filosofia, dizer o não- ser, como se ele realmente
  ``fosse'', seria o fundamento da ``arte de enganar''. A filosofia, que
  também deve ser persuasiva, será capaz de descobrir quando isso ocorre
  e, em seguida, dizer o ser sem enganos. Na fala de Sócrates esse
  discurso encantador aparece como útil para muitos, uma vez que poderia
  ser usado como fármaco para tantos males, um discurso apaziguador que
  afastaria os males. Antifonte era um mestre da arte de aliviar
  (\emph{téchnên alupías}, DK, 87, A6) por meio das palavras, fármacos
  encantadores, tal qual Górgias apresentava o seu \emph{lógos} soberano
  (\emph{lógos dynástes mégas estín,} Gor. \emph{Hel}. §8), e um dos
  temas do diálogo é também o poder do \emph{lógos}, só que da
  perspectiva filosófica. Há uma simetria dessa proposta claramente
  democrática em aplicar o discurso para o bem das multidões com a
  proposta de \textbf{Theuth} em distribuir a escrita a todos os
  egípcios como uma utilidade (274d7 \emph{ophelían}), tal qual
  encontramos aqui, ``útil ao povo'' (227d2 \emph{demôpheleîs}). Em
  ambos os trechos, a utilidade funciona como pseudo-fármaco, ou seja,
  como algo encantador, mas capaz especialmente de esconder a verdade,
  um fármaco poderoso cuja imagem no diálogo é também a retórica. Contra
  essa arte discursiva estabelecida, a filosofia platônica, por meio de
  dialética, procura ser também um fármaco, mais precisamente um
  antídoto. É a primeira vez que no diálogo o discurso (\emph{lógos}) é
  entendido como fármaco, bem como o problema filosófico do ser e da
  possibilidade de dizê-lo se desenha com certa precisão. Essa simetria
  entre a utilidade pública aqui em Lísias e na alegoria final de
  Thamous e Theuth seria um elemento, entre outros, em favor da unidade
  do diálogo.} Nesse caso, os discursos seriam agradáveis
(\emph{asteîoi})\footnote{O termo \emph{asteîoi}, derivado de
  \emph{ástu}, ``cidade'', refere o urbano, o civilizado e, por
  derivação, o bom gosto. Sócrates nem mesmo parece ser um citadino, mas
  como bem podemos observar, um pouco à frente, em 230c5, Sócrates diz
  que Fedro é o melhor dos guias para extrangeiros (\emph{exenágetai}),
  como se ele próprio não fosse um autóctone. Fedro caracteriza Sócrates
  como alguém que parece nunca ter saído da cidade (\emph{ásteos} 230d),
  e que nem mesmo parecia ter estado fora dos muros (\emph{éxo teíchous}
  230d1), alguém muito extravagante (\emph{atopótatós} 230c6) ou,
  literalmente, sem lugar. Em 229c6 o termo \emph{átopos} se repete.
  Nesse trecho inicial Sócrates dirá que ele é um amigo do aprendizado
  (\emph{philomathès} 230d2), e que não aprende nada com a natureza, só
  com homens da cidade.} e úteis ao povo (\emph{dêmôpheleĩs}).\footnote{O
  termo \emph{demopheleîs}, útil ao povo, configura um ataque à
  democracia, além de se extender à prática sofística, uma vez que o
  campo de ação mais eficaz do sofista é a democracia, regime em que,
  segundo Platão, predomina um discurso contrário à verdade, um discurso
  que ilude o povo (\emph{dêmos}) e que pode, como consequência, levar à
  tirania. Na \emph{palinódia} de Sócrates Platão usa o termo
  \emph{demokopikós} para aquele que, literalmente, ``fere'' ou ``lesa''
  o povo, personagem que ocupa o penúltimo lugar (8º grau) na escala das
  almas (248e2).} {[}227d{]} Então desejo escutá-lo, mesmo que tu
percorras o passeio até Mégara, segundo os preceitos de Heródico,
chegando até os muros e dali novamente (\emph{pálin}) retornando pelo
mesmo trajeto, nem mesmo assim eu te abandonaria.\footnote{Heródico de
  Mégara foi um dos primeiros mestres da educação física de que se tem
  notícia. Foi também o primeiro a prescrever a caminhada para a
  manutenção da saúde. Ele propunha a caminhada de ida e volta de Atenas
  a Mégara. A prescrição sugere também uma analogia com a vida de
  Lísias, que foi exilado em Mégara de 404 a 403 a.C. pelo governo dos
  Trinta Tiranos e só retornou à Atenas com a restauração da democracia
  em 403 a.C. A edição do \emph{Fedro} da Loeb (trad. Fowler, 2005) traz
  na p.414, n.2, um trecho de um escólio acerca de Heródico no qual
  lemos: ``era médico e realizava exercícios fora dos muros, começando
  por uma distância não muito grande, mas moderada, até que chegasse ao
  muro, daí voltava'' (\emph{iatròs ên kaì tà gumnásia éxô teíchous
  epoieîto, archómenos apó tinos diastématos ou makroû allà summétrou,
  áchri toû teíchous}, \emph{kaì anastréphôn}). Em 403 a.C. os muros de
  Atenas, que haviam sido construídos por Temístocles, foram derrubados
  e as trirremes queimadas, como condição da rendição de Atenas frente a
  Esparta (cf. Plutarco, \emph{Vida de Lisandro}, 15, 4-5). Desse modo,
  a narrativa do diálogo se localiza nessa época em que ainda os muros
  não haviam sido derrubados, sendo o próprio muro um símbolo
  melancólico que remete à medicina de Heródico, que prescrevia
  caminhadas por fora dos muros (\emph{exô teíchous}), e especialmente à
  história política ateniense entre esses dois momentos democráticos,
  período da guerra do Peloponeso. Cf. também menção pejorativa a
  Heródico na \emph{República} 406a6-c, passagem em que ele é descrito
  como um pedótriba que adoeceu, mas que a partir de dietas bastante
  rigorosas foi capaz de prolongar a sua vida e de chegar até a velhice,
  mas sem poder exercer seu trabalho como antes, aspecto pelo qual é
  detratado como alguém que aplicou seus conhecimentos em benefício
  próprio e não em função do bem comum. Nesse retrato breve da
  \emph{República} Heródico teria misturado ginástica e medicina. O
  termo \emph{pálin}, aqui empregado pela primeira vez, será retomado em
  diversos momentos, como em 227d, 238d, 241b, 247e, 249b, 254b, 254d,
  255c, 263e, 265e, 266a, 277b, além do termo palinódia, em 243b e 257a,
  sempre denotando a repetição, a volta, o novo percurso.}

F: O que dizes, ó querido Sócrates? O que Lísias compôs durante muito
tempo e com empenho (\emph{en pollôi chrόnôi katà scholḕn}), {[}228a{]}
ele que é hoje um dos mais hábeis (\emph{deinótatos}) a escrever
(\emph{grapheîn}), considera-me capaz de lembrar
(\emph{apomnêmoneúsein}) dignamente disso, eu que sou um simplório?
Falta-me muito ainda. Na verdade, almejaria mais isso do que ter muito
ouro.\footnote{Esse é um traço marcante da descrição da personalidade de
  Fedro, sempre valer-se da imagem do ouro, da riqueza material. Isso
  ocorre também em 235d e 236b, motivo pelo qual Sócrates responde em
  235e ironicamente a Fedro: ``És amicíssimo e de ouro verdadeiro''.
  Obviamente esse retrato está ligado às grandes somas acumuladas por
  esses professores itinerantes da arte da palavra.}

S: Ó Fedro, se não conheço Fedro, estaria esquecido de mim
mesmo\footnote{Esse trecho será imitado por Fedro em 236c4-6.}, mas não
é nada disso. Bem sei que, tendo ouvido o discurso de Lísias
(\emph{Lysíou lógon akoúôn}), não só uma, mas muitas vezes, tu
remontavas seus dizeres (\emph{légein}), persuadido de boa vontade
(\emph{epeítheto prothúmōs}). Mas isso não {[}228b{]} era ainda o
suficiente. Aposto que tomavas o livro, passando a investigar o que mais
te interessava (\emph{epethýmei}), e, tendo feito isso desde cedo,
tratou de repeti-lo ao redor do passeio (\emph{perípaton}), como bem
sei, e, pelo cão, decoraste o tal discurso, isso se não era muito
longo.\footnote{Assim como o próprio Fedro acabara de dizer ser incapaz
  de lembrar do discurso de Lísias, Sócrates reitera e duvida
  completamente da capacidade mnemônica de Fedro, depreciando sutilmente
  seu interlocutor, não só por ele estar fascinado por um escrito vulgar
  de Lísias, um logógrafo meteco, mas também por desejar memorizá-lo sem
  êxito, despendendo tempo nisso.} Depois de ter atravessado por fora
dos muros (\emph{ektòs teíchous}) para exercitar-te, encontraste aquele
que é doente para ouvir discursos (\emph{nosoûnti peri lógôn akoḗn}) e
que, ao vê-lo, alegra-se porque terá um companheiro coribântico que o
ordenará prosseguir. Sendo obrigado a falar pelo {[}228c{]} amante dos
discursos (\emph{lógon}), ficas enternecido\footnote{Esse trecho será
  citado, de forma não literal, em 236c6.} (\emph{ethrúpteto}) como se
não desejasses falar (\emph{hos de ouk epithymôn légein}), mas, no final
das contas, falarias, mesmo que à força, ainda que não houvesse alguém
para ouvir-te voluntariamente. Então, ó Fedro, obriga-te a fazer
imediatamente o que certamente farias, em seguida, de qualquer modo.

F: Na verdade, é muito melhor que eu possa falar (\emph{légein}) assim,
mas parece também que tu, seja como for, não me largarias antes que eu
falasse.

S: Verdadeira é a tua impressão.

{[}228d{]} F: Assim farei. Quanto ao próprio {[}discurso{]}, Sócrates,
eu não conheço completamente suas palavras (\emph{ta} \emph{rhémata}).
Pretendo expor o pensamento (\emph{diánoian}) quase completo, que difere
o amante do não afetado pelo amor, em cada um dos seus pontos capitais
(\emph{kephalaíois}), começando pelo primeiro.

S: Então, primeiramente mostra, ó querido, o que trazes na mão esquerda,
debaixo do manto. Desconfio de que tenhas o próprio discurso (\emph{tòn
lógon autón}). {[}228e{]} Se é assim, entendas uma coisa a meu respeito,
que eu não me prestaria ao teu exercício estando junto de Lísias
(\emph{paróntos dè kaì Lysíou})\footnote{A escrita aqui corresponde à
  ``presença'' de Lísias. Até o final do diálogo uma visão oposta será
  defendida, a de que a escrita é apenas um ídolo, uma imagem imperfeita
  da palavra falada.}, logo eu que o estimo (\emph{philô})\footnote{No
  desenrolar do diálogo ficará claro que Sócrates está sendo irônico,
  uma vez que não gosta efetivamente de Lísias e tampouco gostará do seu
  discurso, o qual Sócrates criticará à frente, em 234e5-235a8, depois
  de esboçar, em poucas linhas, um leve elogio.} tanto. Vai logo,
mostra.

F: Para! Retiraste a minha esperança, ó Sócrates, de exercitar-me
contigo.\footnote{Essa é a primeira imagem da mudança de curso figurado
  no discurso, o que chamarei de caráter antistrófico, ou protréptico,
  do diálogo, pois os planos de Fedro de improvisar algo de memória
  acabam frustrados e ele terá agora de simplesmente ler o discurso
  escrito por Lísias. O particípio médio futuro \emph{eggumnasómenos}
  introduz a ambiência dos ginásios, dos excercíos físicos e
  intelectuais.} Onde desejas tomar assento para lermos?

{[}229a{]} S: Desviemo-nos daqui, seguindo a direção do Ilisso, para nos
sentarmos em algum lugar que te pareça tranquilo.\footnote{Verbalização
  da mudança de curso dos personagens através do verbo \emph{ektrépo,}
  ambos alteram a direção tendo como baliza o rio Ilisso, buscando um
  ponto no qual se reclinarão para a realização da leitura. O rio terá
  outra função simbólica quando Sócrates o atravessar, desejando ir
  embora, depois do seu primeiro discurso (237a7-241d), momento em que,
  depois da travessia, será impelido pelo seu \emph{daímon} a mudar o
  sentido do seu discurso, realizando uma nova recitação, a palinódia
  para Eros (243e-257b6).}

F: Oportunamente (\emph{eis kairón}), ao que parece, tive a sorte
(\emph{étychon}) de estar descalço hoje. Quanto a ti, Sócrates, sempre
estás assim.\footnote{A expressão ``oportunamente'' (\emph{eis}
  \emph{kairón}) tem vasta aplicação na retórica, na dialética e na
  medicina, e é um elemento comum do \emph{Fedro}, desde o encontro no
  plátano, que acontece a seguir, até a menção de uma razão hipocrática
  em 270c9-10. O Aforismo 1 de Hipócrates é eloquente quanto ao valor do
  senso de oportunidade ou \emph{kairós}: ``A vida é curta, a arte é
  longa, a ocasião fugidia (\emph{kairòs oxùs}), a experiência enganosa,
  o julgamento difícil'' (trad. Rezende, 2010, p. 29). Para além da
  medicina, na retórica esse \emph{kairós} não será menos importante,
  pois o senso de oportunidade é uma marca da arte discursiva, e da
  maioria das artes (\emph{téchnai}), inclusive da navegação (cf.
  \emph{Pol}. 284e). A \emph{tyké} de \emph{étykon}, sorte ou acaso,
  também se associa a esse campo do \emph{kairós}, pois Sócrates é um
  homem favorecido, por um lado, pela sorte e, ao mesmo tempo, é dono de
  um senso de oportunidade ímpar. Por isso Fedro diz que ele está sempre
  assim, descalço, o que simboliza essa condição especial de Sócrates na
  cena.} Então, será facílimo, e nada desagradável, molharmos nossos pés
por esse fio d'água, especialmente nesta época do ano e nesta hora do
dia.

S: Prossegue e busca o lugar no qual descansaremos.

F: Vês aquele elevadíssimo plátano?\footnote{\emph{Platano orientalis} é
  um símbolo da medicina, conhecido com a árvore de Hipócrates, em torno
  da qual ele e seus discípulos se reuniam na ilha de Cós. Nesse momento
  o diálogo estabelece a eloquente interface da filosofia com a
  medicina, especialmente a hipocrática, sem contar que Sócrates acabara
  de dizer (228b) que ele próprio era doente para ouvir discursos
  (\emph{nosoûnti peri lógôn akoèn}).}

S: O que há nele?

{[}229b{]} F: Uma sombra, uma brisa moderada (\emph{pneûma
métrion})\footnote{Essa ``brisa moderada'' (\emph{pneûma métrion}) do
  lugar em que farão a leitura contrastará com a menção ao rapto de
  Bóreas, divinização do Vento Norte, um vento raptor de donzelas. Esse
  caráter aéreo é importante no diálogo, pela ambiência da descrição da
  alma (\emph{psyché}) e da psicagogia, bem como a noção de
  \emph{métrion}, medida, bastante recorrente, como veremos, não só pelo
  caráter apolíneo resgatado por Sócrates, mas especialmente pelas
  diversas medidas necessárias, aos dialéticos e a todos os verdadeiros
  conhecedores das artes. A brisa moderada do lugar é momentaneamente
  substituída simbolicamente por Bóreas, quando Fedro menciona o mito,
  e, em seguida, a moderação de movimentos aéreos é reabilitada, porque
  na écfrase do lugar (230b-c) ressalta-se mais uma vez o ar agradável e
  melodioso: ``uma doce e muito agradável brisa (\emph{eúpnoun}) do
  lugar, estival e melodiosa'' (230c1). Em Hipócrates, especialmente no
  tratado \emph{Da doença Sagrada} (§16, Jones), Bóreas personifica um
  vento saudável, que vem do mar e preserva o encéfalo seco, ao
  contrário de Noto, o Vento Sul, nocivo porque umedece e liquefaz o ar
  no encéfalo do fleumático, gerando o ataque epilético.}, uma relva
para nos sentarmos e reclinarmos se quisermos (\emph{boulṓmetha}).

S: Prossegue então.

F: Diga-me, ó Sócrates, não é desse lugar do Ilisso que contam ter
Bóreas raptado (\emph{harpásai}) Orítia?\footnote{O rapto na cultura
  grega é uma das grandes tópicas poético-literárias, a qual pode ser
  observada em diversos mitos, como no rapto de Peséfone por Hades, no
  duplo rapto de Helena, por Teseu e Páris, no rapto de Orítia por
  Bóreas (229b5-229d1) e até mesmo no rapto de Ganimedes por Zeus.
  Especialmente nesse caso, como o \emph{Fedro} retoma o mito de Eros e
  Psique, é importante lembrar do rapto de Psique pelo vento Zéfiro (cf.
  Apuleio \emph{Met}. IV,28-VI,24).}

S: Dizem.

F: Então é aqui! A água parece agradável, pura e diáfana, própria às
donzelas que brincam (\emph{paídzein}) nessas margens.\footnote{A
  brincadeira, \emph{paídzein,} constitui elemento tradicional de
  comparação com a arte discursiva, de modo que o trato ou labor com o
  discurso, \emph{lógos}, desde pelo menos Górgias (\emph{emòn dè
  paígnion, Elogio de Helena} §21), é também uma espécie de brincadeira
  ou jogo. Cf. no \emph{Fedro} outras ocorrências de \emph{paídzein} em
  229c8, 234d7, 234d8 e 276e2.}

{[}229c{]} S: Não é aqui, mas dois ou três estádios\footnote{Unidade de
  medida grega ligada à corrida nos jogos atléticos antigos e que
  corresponde a 600 pés, por isso, de acordo com Paleologos, essa
  unidade variava, sendo que em Olímpia era de 192,28 m, em Priene
  191,39, em Atenas 184,96, em Epidauro 181,30, em Delfos 177,55 e em
  Mileto 177,36 (Cf. Paleologos, \emph{Os Jogos Olímpicos na Grécia
  Antiga}, Vários Autores, 2004, São Paulo: Odysseus, p.177).} abaixo,
onde atravessamos na direção de Agra, onde há um altar para Bóreas.

F: Não lembro, mas diz, por Zeus, ó Sócrates, estás persuadido
(\emph{peíthêi})\footnote{Aqui (229c4-230a7) um ambiente semântico
  bastante controlado apresenta, de modo exímio e econômico, os
  elementos fundamentais da retórica: a persuasão (\emph{peithêi,
  peithómenos}), a crença (\emph{pístis}), nas suas formas negativas,
  designando o que Sócrates designa por incredulidade (\emph{apistoíên},
  \emph{apistôn}), assim como um modo sofisticado (\emph{sophizómenos})
  de tentar conduzir os mitos à verossimilhança (\emph{eikòs}), outro
  elemento fundamental da prática sofística. Nesse trecho os termos
  ``sábios'' e ``sabedoria'' (\emph{sophoí, sophíai}) são usados de modo
  irônico, referindo a sabedoria grosseira (\emph{agroikôi tini
  sophíai}) dos sofistas, para a qual Sócrates diz que não ter tempo
  livre (\emph{scholé}). A extravagância de Sócrates e dos seres
  formidáveis são designadas ambas pelo mesmo termo (\emph{átopos,
  atopiai}), literalmente ``sem lugar'', além do que, Sócrates evocará
  sua missão apolínea de buscar a si mesmo (\emph{gnônai}
  \emph{emautón}). O trecho finda quando encontram a árvore
  (\emph{déndron}) que buscavam.} de que esse mitologema seja
verdadeiro?

S: Se eu fosse um incrédulo (\emph{apistoíēn}), como fazem os sábios
(\emph{sophoí}), não seria nada extravagante (\emph{átopos}). Em
seguida, diria, com ar sofisticado (\emph{sophizómenos}), que ela foi
arrebatada das proximidades das pedras, lá debaixo, pelo sopro
(\emph{pneûma}) de Bóreas, quando brincava com Farmaceia. Alguns dizem
que ela teria morrido pelo rapto (\emph{anárpaston}) de Bóreas -- ou que
isso ocorrera no Areópago --, {[}229d{]} pois contam também essa outra
versão, de que ela foi (\emph{hērpásthē}) raptada de lá e não
daqui.\footnote{Segundo Acusilao (DK, 9, B, 35) o rapto de Orítia por
  Bóreas teria ocorrido na Acrópole, quando a distinta filha do rei
  Erecteu levava um cesto de oferenda à deusa Atena.} Eu considero
graciosas essas coisas, mas uma ocupação terrível, laboriosa e própria a
homens não muito felizes. E não por outra razão, senão por nos obrigar a
restaurar, necessariamente, a forma dos Hipocentauros, da Quimera, de
uma turba de Górgonas, Pégasos e muitos {[}229e{]} outros seres
formidáveis, por conta da extravagância (\emph{atopiai}) dessas
naturezas monstruosas. E se alguém, entre os incrédulos
(\emph{apistôn}), conduzisse cada um deles à verossimilhança
(\emph{eikòs}), valendo-se de uma sabedoria (\emph{sophíai}) grosseira,
precisaria de um bom tempo livre (\emph{scholés}). Eu não tenho nenhum
tempo livre (\emph{scholé}) para essas coisas e a causa disso, querido,
é que não fui ainda capaz de conhecer a mim mesmo (\emph{gnônai}
\emph{emautón}), de acordo com a inscrição délfica.\footnote{A inscrição
  délfica ``conhece-te a ti mesmo'' explica um forte aspecto da
  religiosidade de Sócrates, como uma medida (\emph{métrion}) apolínea
  que ele próprio buscava.} Pareceria risível {[}230a{]}, ainda
ignorante de mim mesmo, que eu examinasse alguma outra coisa. Esse é o
motivo pelo qual me agrada renunciar a tudo, como disse agora mesmo,
convencido (\emph{peithómenos}) do que foi considerado sobre o tema.
Nesse caso, não examino (\emph{skopô}) essas coisas em outro lugar senão
em mim mesmo (\emph{emautón}), quer seja uma fera mais complexa e
orgulhosa (\emph{epitethymménon}) que Tifon (\emph{Tuphônos})\footnote{Gerado
  da união entre o Céu e o Tártaro, esse monstro, cujos dedos eram de
  cem cabeças de serpente e cujos olhos expeliam fogo, atacou o Olimpo,
  e apenas Zeus e Atena resistiram. Zeus lançou sobre ele o monte Etna,
  meio pelo qual foi lançado no Tártaro, onde se juntou aos Titãs. Nesse
  sentido é o símbolo máximo do orgulho e da insolência, cuja
  representação se dá pelos vapores de sua natureza. Cf. Hes.
  \emph{Theo}. 820-880.}, quer seja o animal mais doméstico e simples,
partícipe de uma natureza em algo divina e sem nenhum orgulho
(\emph{atúphou}). A propósito, companheiro, não é esta a árvore para a
qual nos trazias?

{[}230b{]} F: A própria.

S: Por Hera, que bela pousada!\footnote{Encontramos aqui, em 230b-c, uma
  écfrase, descrição detalhada de ampla evidência (\emph{enárgeia}) que
  procura pôr diante dos olhos do leitor o lugar descrito.} Um plátano
corpulento e magnífico, com uma folhagem excelente e digna desse
ambiente sagrado. O vigor da floração oferece ao lugar o melhor dos
aromas e a fonte agradabilíssima sob o plátano nos traz água bem fria,
como comprovamos com os pés. Parece ser um templo para alguma Ninfa e
para Aqueloo, a julgar pelas estátuas votivas de argila e imagens de
mármore (\emph{korôn te kaì agalmátōn}).\footnote{Figuras de terracota e
  de mármore ofertadas às Ninfas e a Aqueloo.} {[}230c{]} Se desejas
algo mais, há ainda uma doce e muito agradável brisa (\emph{eúpnoun}) do
lugar, estival e melodiosa, que ecoa o coro das cigarras (\emph{tettígōn
chorôi}). A relva é certamente o maior dos requintes
(\emph{kompsótaton}), porque a escarpa suave naturalmente solícita que
reclinemos nossas cabeças de modo maravilhoso. Assim, ó querido Fedro,
és o melhor dos guias para estrangeiros (\emph{exenágētai}).

F: Parece-me muito extravagante (\emph{atopṓtatόs})\footnote{Sócrates
  havia dito há pouco que se ele não acreditasse nos mitos não seria
  extravagante (\emph{átopos}), mas agora é caracterizado por Fedro como
  extravagante, no superlativo (\emph{atopótatós}). Essa extravagância
  um pouco forçada de Sócrates faz com que Fedro afirme, em seguida, que
  ele parece um estrangeiro, não um ateniense, e, imediatamente depois,
  afirme o contrário, que ele parece nunca ter saído da cidade, nem
  atravessado os muros (230d1-2). Para além do fato de Sócrates ter
  saído muito pouco de Atenas, especialmente nas campanhas militares de
  que participou, a passagem revela esse jogo de presença e ausência de
  Sócrates na cidade, dessa travessia que faz fora da ciscunscrição da
  cidade, talvez por isso \emph{atopṓtatόs}, uma vez que ele parece, ao
  mesmo tempo, um estrangeiro e um autóctone. Sócrates é tão citadino,
  tão embrenhado nas questões da \emph{pólis}, que nem mesmo parece
  conhecer nada fora dela, nem mesmo em seu entorno, por fora dos muros.
  Sócrates se confunde com a \emph{pólis} a ponto de parecer um estranho
  para um conterrâneo, mas é claro que, de modo análogo, essa relação
  equivale à busca de si mesmo enunciada antes, a qual, por ainda estar
  incompleta, não permitia que se empenhasse na busca de nada que lhe
  fosse externo. A prece final de Sócrates no final diálogo revela essa
  condição, pois ela está direcionada a Pã, para que ele possa enfim
  conciliar essas duas naturezas, externa e interna: ``tudo que há fora
  de mim possa ser amigo do que está no meu interior'' (279b8-c2).}, ó
admirável. Pelo que dizes, sem nenhum artifício
(\emph{atéchnôs})\footnote{Essa \emph{atechnía} ou \emph{ausência de
  arte} será um dos temas do diálogo, na medida em que se buscará uma
  melhor visualização do que seria, de fato, \emph{a arte}
  (\emph{téchne}), especialmente a arte da palavra, e quem realmente a
  possuia, bem como acerca das outras artes a ela comparadas. Cf. em
  268a8-269a3 a diferença entre os rudimentos das artes e as artes
  propriamente ditas, em especial a medicina, a poesia trágica e a
  harmonia musical.}, dá-me a impressão de seres guiado como um
estrangeiro e não como um autóctone. {[}230d{]} Parece que jamais te
ausentaste da cidade rumo à terra estrangeira, nem mesmo saíste para
além dos muros (\emph{éxô teíchous}).

S: Perdoa-me, ó excelente, é que eu sou um amante do aprendizado
(\emph{philomathḕs}), nem os campos nem as árvores querem me ensinar,
somente os homens da cidade.\footnote{Essa posição citadina de Sócrates
  nesse momento ressalta que os homens na cidade seriam superiores no
  que concerne ao aprendizado, uma vez que da natureza Sócrates diz não
  aprender nada. Como acontece com outros temas, essa posição mudará,
  especialmente porque em 275b5-c1 Sócrates dirá que os antigos ouviam
  discursos divinatórios de um carvalho do templo de Zeus em Dodona, bem
  como outros ouviam pedras, desde que lhes dissessem a verdade, ou
  seja, a natureza (\emph{phýsis}) passará a ser vista, quase no final
  do diálogo, como fonte de conhecimento verdadeiro, passível de ser
  traduzido pelo homem.} E tu, realmente, pareces ter encontrado
(\emph{hēurēkénai}) o fármaco do meu êxodo.\footnote{Uma vez
  estabelecida a analogia entre \emph{lógos} e fármaco, o \emph{lógos}
  passa a ser observado em sua aplicação e em seus efeitos colaterais,
  bem como os remédios na medicina, de modo que a relação do fármaco com
  o \emph{lógos} se eleva ao primeiro plano. A alegoria de Thamous e
  Theuth, no final, resgatará a mesma metáfora da descoberta
  (\emph{heuresis}) das artes, sendo a escrita um \emph{fármaco da
  memória} na visão de Theuth e, por outro lado, um \emph{fármaco da
  recordação} na visão de Thamous. Curiosamente conhecemos tal alegoria
  (274c5-275b1) como alegoria ou mito de ``Theuth'', quando seria mais
  preciso nomeá-la como alegoria ou mito de ``Thamous'', pois somente
  este último revela a perspectiva do dialético, juiz das artes
  miméticas, enquanto Theuth apresenta-nos, ao contrário, a visão do
  sofista, do inventor, do produtor de espanto (\emph{thaumatopoiois})
  (\emph{Rep}.VII 514b).} Tal como os que agitam um ramo para uma
criatura faminta, ou algum fruto que os conduza, tu, do mesmo modo,
estendendo discursos provenientes de livros, parece que me conduzirás
por toda a Ática ou para qualquer outro lugar que queiras. {[}230e{]}
Agora, tendo chegado aqui, vou reclinar-me e encontrarás a posição que
te seja mais cômoda à leitura. Depois disso, lê.

F: Escuta:

``Já estás ciente acerca dos meus assuntos e creio que ouviste acerca do
que pode acontecer conosco.\footnote{Nessa recitação do discurso de
  Lísias encontramos a matéria prima de todo o diálogo, naquilo que será
  repetido, de uma maneira diversa, e, em seguida, no que será refutado.
  Nesse discurso ``Lísias'' toma somente o lado canhestro de Eros e o
  vitupera impiedosamente, afirmando que o Amor é uma doença e que seus
  efeitos colaterais devem ser evitados, de modo que ``é melhor agradar
  a alguém que não está tomado por Eros'' (227c5-7), pois só assim seria
  possível se resguardar, de modo pragmático, desses fortes e nocivos
  efeitos do Amor. O discurso defende uma relação amorosa que pretende
  extrair o máximo de prazer e evitar ao máximo os sofrimentos.}
{[}231a{]} Espero que não me advenha nenhum infortúnio
(\emph{atychêsai}) só porque me ocorreu (\emph{tygchánō}) de não estar
te amando (\emph{ouk} \emph{erastḕs})\footnote{A palavra \emph{erastès},
  amante, aquele que ama, poderia nesse caso e em outros próximos ser
  traduzida por apaixonado. Nossa escolha por ``amante'', ``estar
  amando'' e, algumas vezes, por ``amoroso'', ocorre pela distância
  entre o substantivo original, \emph{erastès}, e a raiz de apaixonado
  (\emph{páskô}), de paixão, afecção, doença, a qual nem sempre
  traduziria com precisão esse amante ou erasta, que aparece coordedado
  com o \emph{erômenos} ou amado, o objeto do amor. Essa coordenação é
  às vezes difícil de acompanhar até o final do discurso, ou seja, é
  preciso perceber que Lísias quer detratar a educação grega, fundada na
  pederastia e personificada no erasta, para oferecer outro tipo de
  educação, uma educação ligada aos metecos, uma educação sofística,
  democrática, de modo que essa nova educação é defendida por Lísias
  contra a tradição.}, como aqueles que, tão logo tenha cessado o seu
desejo (\emph{epithymías paúsōntai})\footnote{Expressão retomada em
  232e, 234a e em 264a, momento em que o início desse discurso será
  citado.}, arrependem-se (\emph{metamélei}) do que bem fizeram. Outros,
por outra parte, não tem tempo hábil para mudar o pensamento
(\emph{metagnônai}), e não é por coação, mas espontaneamente, que bem
fazem o que podem ao amado, desejando-lhes (\emph{bouleúsainto}) o
melhor nos assuntos pessoais. Os amantes (\emph{oi} \emph{men}
\emph{erôntes}) observam o que fizeram de bom e de mau pelo amor
(\emph{érôta}), e, pelos sofrimentos (\emph{pónon}) causados, consideram
{[}231b{]} antiquadas as gratificações (\emph{chárin}) endereçadas aos
seus amados (\emph{tois} \emph{erôménois}) de outrora.

Os que não amam (\emph{mè} \emph{erôsin}), por seu turno, não podem dar
tal pretexto para o abandono de assuntos pessoais, nem consideram os
sofrimentos (\emph{pónous}) passados, nem os desentendimentos causados
com os parentes. Ao despojarem-se de todos esses males, não lhes resta
nada, senão fazer voluntariamente (\emph{prothýmos}) aquilo que
consideram poder agradar (\emph{chareîsthai}) o companheiro. {[}231c{]}
E se por esse valor aos amorosos tanto se faz, o motivo é que estes
sobretudo gostam de dizer que estão amando, e ficam a ponto de
hostilizarem quem quer que seja com palavras e ações, só para agradarem
(\emph{charídzesthai}) seus amados. Aliás, é fácil saber se dizem a
verdade, pelo tanto de amor (\emph{erasthôsin}) que lhes passam a
dedicar, pois fariam de tudo para ele, e é evidentemente que a outros
hostilizariam, se isso lhes fosse requisitado {[}pelo amado{]}.
{[}231d{]} De algum modo, é verossímil (\emph{eikós esti})\footnote{Esse
  \emph{eikòs}, verossímil, é bastante presente no discurso de Lísias
  (229e1, 231c7, 231e2, 232c2, 233a2), especialmente porque o discurso
  escrito remete a essa natureza dupla, da imagem da palavra viva. Na
  discussão posterior acerca da natureza e aplicação do discurso escrito
  e falado, ficará claro que o verossímil é a base da sofística e da
  possibilidade de produzir o engano, porque o rétor parte de um ponto
  verdadeiro de onde gradativa e sutilmente deriva algo falso, quase que
  imperceptivelmente aos ouvintes e leitores menos avisados. Só o
  treinamento dialético faz com que se observe com clareza como e quando
  isso ocorre, sendo um antídoto, portanto, a esse tipo de aplicação do
  \emph{lógos}. Nesse contexto de crítica aos recursos e fundamentos da
  arte da palavra há a maior recorrência do termo (cf. 258c9, 266e2
  \emph{eikóta}, 267c1 \emph{eikonologían}, ``estilo imagético'', 269d2,
  270b10, 272e1, 272e3, 272e4, 273b1, 273d2, 276c10). No primeiro
  discurso de Sócrates o termo aparece apenas duas vezes, em 237c4 e
  238e2, enquanto na palinódia apenas uma vez, em 255e3, de modo que
  fica claro o uso bastante controlado do termo, localizado
  predominantemente no discurso de Lísias e na parte de crítica à arte
  retórica e aos seus fundamentos.} aceitar semelhante dificuldade
àquele que passou por esse infortúnio, afinal quem se livraria
(\emph{apotrépein})\footnote{O termo \emph{apotrepein} (231d1) ``livrar,
  desviar, afastar, evitar'', no caso de Eros, aparece nesse discurso
  mais duas vezes (\emph{apotrépousin} 232c5, \emph{apotrépein} 233c4).
  Esse registro apotropaico será retomado em 238d (\emph{apotrápoito}),
  no primeiro discurso de Sócrates, e, em um sentido ainda mais fonte e
  mais próximo da ambiência apolínea que lhe é característica, no
  segundo discurso de Sócrates (243e-257b6), lugar em que também o
  discurso terá forte caráter protréptico (e apotropaico), para uma
  mudança de direção da alma, só que agora na direção inversa. Nesse
  discurso traduzo as três ocorrências de \emph{apotrépein} por livrar,
  afastar e evitar, respectivamente, acomodando às três aplicações
  verbais sem desconsiderar a proximidade conceitual.} dele, mesmo sendo
experiente (\emph{émpeiros})? E eles mesmos concordam que estão mais
doentes (\emph{noseîn}) do que prudentes (\emph{sōphroneîn}), e sabem
que pensam (\emph{phronoûsin}) mal, mas não podem dominar-se
(\emph{krateîn}).\footnote{Aqui vemos o retrato inicial de Eros
  entendido como uma enfermidade. Nesse trecho a prescrição de aniquilar
  os efeitos do amor levaria a um autodomínio temperante
  (\emph{sophroneîn}), um estado de boa prudência, uma clareza de
  pensamento (\emph{eû} \emph{phronésantes}).} Como é que em estado de
bom pensamento (\emph{eû} \emph{phronésantes}), poderiam considerar
belas as decisões (\emph{bouleúontai}) tomadas naquele estado anterior?
E se tu procuras escolher o melhor entre os amantes (\emph{tôn
eróntôn}), a eleição pode ser feita entre poucos (\emph{ex olígôn}), mas
seria mais proveitosa para ti se abarcasse outros (\emph{ek tôn állôn})
entre muitos (\emph{ek pollôn}). {[}231e{]} Desse modo, é muito maior a
esperança de encontrar, na multidão (\emph{en tois polloîs})\footnote{Esse
  elogio da multidão é um importante aspecto democrático do discurso de
  Lísias.}, alguém que seja digno da tua amizade.

Se receares a lei (\emph{nómon}) estabelecida\footnote{O termo
  \emph{nómos} pode referir-se não necessariamente a uma lei escrita,
  mas a uma convenção, um costume. Acerca da pederastia como instituição
  educacional cf. Dover, K. J. \emph{A Homossexualidade na Grecia
  Antiga} (1994), Marrou, H-I. \emph{Histoire de l'Education dans
  L'Antiquite} (1948) e Pradeau J-F. e Brisson, L. \emph{Dictionnaire
  Platon} (2007), bem como Estrabão, \emph{Geografia}, X, 4, 21 acerca
  dos costumes amorosos (\emph{perì tous érotas nómimon}) cretenses.},
não te afetes pelas observações vergonhosas feitas pelos homens, pois é
verossímil (\emph{eikós esti}) {[}232a{]} que os amantes,
considerando-se honrados pelo amado, tal qual eles mesmos o veneram,
exaltem-se em discursos e honradamente (\emph{philotimouménous}) mostrem
a todos que não têm sofrido (\emph{pepónêtai}) em vão. Por outro lado,
os que não estão sob o efeito do amor (\emph{mḕ}
\emph{erôntas})\footnote{Optamos pela paráfrase ``aqueles não estão sob
  o efeito do amor'' para traduzir \emph{mè erôntas}, evitando a
  expressão ``não-amantes'', bem como ``não-apaixonado'', preservando,
  em detrimento da fluência, a precisão conceitual dessa condição de não
  estar sob o efeito de Eros.} são superiores (\emph{kreíttous}) ao
escolher o melhor (\emph{béltiston}) em vez da opinião (\emph{anti tês
dóxês}) dos homens.\footnote{Nesse discurso de Lísias ``o melhor'' é
  superior à ``opinião''. No primeiro discurso de Sócrates, a seguir,
  haverá uma opinião sem razão e uma outra atrelada à razão (238b-c). Só
  no segundo discurso de Sócrates (palinódia) a opinião passará a ser
  considerara como uma categoria inferior, como de costume na filosofia
  de Platão. Nota-se então uma variação bastante controlada do
  significado do termo ``opinião'', \emph{dóxa}, nos três discursos
  proferidos.} Ademais, necessariamente, todos observam os amantes
(\emph{erôntas}) com seus amados (\emph{erôménois}), bem como as ações
que praticam, e quando os vêem a conversar {[}232b{]} entre si, pensam
que já consumaram, ou que estão para consumar, o seu desejo
(\emph{epithymías}).\footnote{É possível perceber um predomínio, a
  partir daqui, do vocabulário do desejo (\emph{epithymía}), seguindo a
  ideia de Hermias, de que cada uma das recitações do diálogo tem uma
  ligação com uma das partes da alma (\emph{In Platonis Phaedrum
  Scholia} §64, 5-7). O discurso de Lísias é o discurso do
  indisciplinado (\emph{akolaston}) segundo Hermias, indisciplina que
  colherá obviamente dor e sofrimento, termos que se evidenciam no
  discurso (231a8 \emph{pónon}, 231b4 \emph{pónous}, 232c
  \emph{lypoûnta}, 233b \emph{lýpen}), obviamente no sentido de serem
  afastados, evitados. Eros, nesse registro, tem de ser evitado, pelo
  sofrimento que lhe é congênito, de modo que o discurso corresponde à
  parte apetitiva ou \emph{epithymetikós.} A presença do termo desejo
  (\emph{epithymía}) no discurso demarca essa ligação (231a2 \emph{tês
  epithymías}, 232b1 \emph{tês epithymías}, 232e2 \emph{epethýmesan},
  232e6 \emph{tês epithymías}, 233b \emph{tèn epithymían}, 233d2
  \emph{ex epitymías}, 234a7 \emph{pauómenoi} \emph{tês epithymías}). De
  modo geral, o discurso de Lísias é uma descrição dos desejos mais
  baixos, ainda que com um verniz de civilizado, especialmente porque
  aos olhos de Platão esse discurso descreve esse universo dos metecos,
  da democracia, da logografia, da sofística, ou seja, de tudo aquilo
  que ele combateu veementemente em sua filosofia.}

Por outro lado, aos que não estão sob o efeito do amor (\emph{mḕ
erôntas}), não se culpa de tentar algo só por causa da companhia,
sabendo que é necessário dialogar (\emph{dialégesthai}) para sedimentar
a amizade (\emph{philían}) e para qualquer outro deleite
(\emph{hēdonḗn}). Se considerares difícil a permanência da amizade, uma
vez que qualquer tipo de diferença pode trazer desagrado a ambos,
{[}232c{]} quando tudo aquilo que fazes de grandioso torna-se
prejudicial (\emph{blábên}), desse modo, é bem verossímil
(\emph{eikótôs}) que tenhas temor (\emph{phoboîo}) dos amantes
(\emph{erôntas}). Muitas são as aflições (\emph{lupoûnta}) desses
amantes, além de considerarem que tudo lhes traz prejuízo
(\emph{blábêi}). E é por isso que tentam afastar (\emph{apotrépousin})
os amados (\emph{erôménôn}) da companhia de outros, temendo
(\emph{phoboúmenoi}) que detentores de bens os superem no dinheiro ou
que lhes sejam superiores na educação. Protegem o amado de qualquer
outro que possa deter esses bens. {[}232d{]} Quando persuadem-no a
odiá-los, colocam-no apartado dos amigos, e no caso de considerarem a si
mesmos melhores que aqueles, acabam provocando desavenças.

Os que, pela sorte, não estão sob o efeito do amor (\emph{mḕ erôntes
étychon}), mas pela virtude (\emph{aretḕn})\footnote{A prudência será
  identificada à virtude, que nesse contexto será evitar escolher alguém
  tomado pelo Amor.} praticam o desejado, não sentem ciúmes
(\emph{phthonoîen}) dos acompanhantes do amado, e, certamente, não
querem odiá-los. Por considerarem-se desprezados, eles querem ser úteis
aos ditos amigos do amado. A partir dessa prática, é muito maior a
esperança de ter amizade (\emph{philían}) com eles em vez de aversão
(\emph{échthran}) {[}232e{]}.

Na verdade, muitos amantes (\emph{erṓntōn}) desejam (\emph{epethúmēsan})
o corpo antes de lhes conhecerem seus hábitos e de experimentarem outras
familiaridades, de modo que não é evidente que decidam tornar-se amigos
(\emph{boulḗsontai phíloi eînai}), tão logo tenha cessado o seu desejo
(\emph{epithymías paúsôntai}).\footnote{Período idêntico em 231a. Nesse
  discurso há uma repetição da tópica do ``pausar o desejo'', evitar o
  impulso amoroso para evitar seus efeitos colaterais.} {[}233a{]} Para
os que não estão amando (\emph{mḕ erôsin}) e que praticam primeiramente
entre si a amizade, não é verossímil (\emph{eikòs}), partindo daquilo
que foi bem realizado, que a amizade (\emph{philían}) diminua. Ao
contrário, esse modo produz uma memória (\emph{mnēmeĩa}) prévia que
alimenta o que ainda está por vir. Na verdade, é melhor que tu sejas
persuadido (\emph{peithoménôi}) por mim do que por um amante
(\emph{erastêi}).\footnote{Retomada da tese central desse discurso. O
  suposto autor do discurso, Lísias, tenta sobrepujar os erastas ou
  amantes cidadãos.} Os amantes elogiam as ações e as palavras mesmo que
estas não sejam as melhores, seja pelo receio de serem odiados (\emph{mḕ
apéchthôntai}), seja por terem se tornado débeis em seu discernimento
pelo desejo (\emph{epithymían}). {[}233b{]} São essas as coisas que o
amor (\emph{érôs}) manifesta: desafortunados (\emph{distychoûntas}) que
consideram molesto o que não proporciona sofrimento (\emph{mḕ}
\emph{lupén}) aos demais e afortunados (\emph{eutychoûntas}) para os
quais ocorre (\emph{tygchanein}) ter de elogiar (\emph{epaínou})
forçosamente o valor daquilo que não lhes é prazeroso (\emph{mḕ
hedonês}).\footnote{O trecho é desenhado através de uma coordenação
  \emph{mén} -- \emph{dè}, e manifesta uma imagem trágica dos amantes
  (erastas). Lísias é no \emph{Fedro} um estereótipo, um ``personagem''
  de Platão. O infortúnio seria o não-sofrimento de outros e a fortuna
  seria elogiar obrigatoriamente algo que não é agradável, como um
  efeito colateral desse modelo de relacionamento pragmático que Lísias
  defende, sem amor obviamente. Esses ``afortunados'' como Lísias e
  todos os sofistas sempre foram obrigados, pela própria natureza da
  atividade, a elogiar aquilo que não necessariamente gostavam. Talvez
  esse seja um trecho em que fica quase que evidente que o discurso de
  Lísias é de fato uma mimese de Platão, posto que carrega elementos
  bastante comprometidos com a ``preocupação'' filosófica.} Assim, é
muito mais conveniente apiedar-se dos amantes (\emph{eleeîn toîs
erôménois}) do que invejá-los (\emph{zeloûn}). E se por mim fores
persuadido (\emph{peíthêi}), em primeiro lugar, não serei o teu guardião
do prazer (\emph{hêdonḕn} \emph{therapeúôn}), mas das utilidades
(\emph{ôphelían}) futuras, {[}233c{]} não sendo vencido pelo amor
(\emph{hyp'érôtos hēttṓmenos}), mas permanecendo senhor de mim mesmo
(\emph{emautoû} \emph{kratôn}). Não me arrastarei a um ódio extremo por
motivos fúteis e terei pouca ira em função de motivos maiores,
desculpando as faltas involuntárias e tentando evitar
(\emph{apotrépein}) as voluntárias. Esse é o testemunho
(\emph{tekmḗria}) de uma amizade (\emph{philían}) que durará muito
tempo. Se tu pensas que a mais forte das amizades não ocorreria sem a
presença do amor (\emph{erôn tygchánêi}), {[}233d{]} é necessário ainda
considerar (\emph{enthymeîsthai}) que nem aquela amizade aos muitos
filhos haveria, nem aos pais ou às mães, nem a fiel amizade dos amigos
teríamos adquirido (\emph{pistoùs phílous ekektḗmetha}), por isso a
amizade não provém daqueles desejos (\emph{epithymías}) antes
mencionados, mas de outras práticas.

Em seguida, se é necessário agradar (\emph{charídzesthai}) sobretudo a
quem precisa, convém que os beneficiados não sejam os melhores
(\emph{beltístous}), mas os muito isentos em recurso
(\emph{aporôtátous}), pois, livrados dos maiores males, esses terão
ampla gratidão (\emph{chárin}). Certamente então nos banquetes
particulares não vale a pena convidar os amigos (\emph{phílous}), mas os
que clamam e os que necessitam saciar-se, pois estes se tornarão
carinhosos (\emph{agapésousin}), companheiros (\emph{akolouthḗsousin}),
e virão a nossa porta sabendo comprazer-se e, não com pouca gratidão
(\emph{chárin}), desejar-nos-ão boas coisas.\footnote{Nesse ponto o
  argumento beira o cômico, uma vez que leva às últimas consequências a
  visão utilitarista acerca do amor. Não se escolhe os amigos para um
  banquete, mas os necessitados, pois estes últimos terão a gratidão
  (\emph{chárin}) que se espera conquistar, portanto tudo gira em torno
  da expectativa da recompensa.} Da mesma maneira, convém não agradar
(\emph{charídzesthai}) aos muito necessitados, mas aos que especialmente
podem oferecer gratidão (\emph{chárin}), não só aos que clamam
{[}234a{]}, mas aos dignos dessas práticas, não tanto aos que desfrutam
da juventude, mas aos que na velhice repartirão contigo os benefícios,
não aos que, por terem realizado seu intento, passam logo a dedicar-se
aos outros, mas aos que se envergonham e calam diante de todos, não aos
que se dedicam por um curto tempo, mas aos que serão amigos por toda a
vida, não aos que, cessado o desejo (\emph{oi pauómenoi tês
epithymías})\footnote{Encontramos novamente a expressão ``cessar do
  desejo'' (\emph{pauómenoi tês epithymías}) como uma marca desse
  discurso de Lísias (Cf. 231a, 232e e 234a).}, buscarão pretexto para o
ódio, enquanto os outros tendo {[}234b{]} passado a juventude
mostrar-lhes-ão a virtude (\emph{hoì pausaménou tês hṓras tόte tḕn
hautõn aretḕn epideíxontai}).

Então, recorda-te do que foi dito e põe no teu ânimo (\emph{enthymoû})
que os amigos advertem aos seus amantes (\emph{tous mèn erôntas oi
phíloi nouthetoûsin}) por seu mau comportamento, ao passo que aos que
não estão sob o efeito do amor (\emph{mḕ} \emph{erôsin}), nem mesmo seus
familiares lhes fazem censuras de qualquer tipo, pois eles decidem
(\emph{bouleuoménois}) os seus próprios males.

Talvez, então, tu me perguntes se eu te aconselho a agradar
(\emph{charídzesthai}) todos aqueles que não estão sob o efeito de Eros
(\emph{mḕ} \emph{erôsi}). Eu não considero que o amante (\emph{ton}
\emph{erônta}), em todo o caso, te incentivasse a essa maneira de pensar
(\emph{diánoian}) com relação aos amorosos (\emph{tous} \emph{erôntas}).
{[}234c{]} Nem aquele que pelo discurso recebe semelhante honra graciosa
(\emph{cháritos}), nem tu, se quisesses manter-te escondido dos outros,
poderias agir de modo semelhante. É necessário que disso não surja
nenhum dano (\emph{bláben}), mas que ocorra o proveito
(\emph{ôphéleían}) a ambos.\footnote{Essa seria a síntese do discurso,
  uma ação que não causa nenhum dano (\emph{bláben}), mas em que ocorre
  o proveito, a utilidade (\emph{ôphéleian}) a ambos os partícipes da
  relação. Esse dano (\emph{bláben}) e essa utilidade, de alguma forma,
  fecham um ciclo semântico que se inicia naquele \emph{demopheleîs}
  227d2 irônico de Sócrates e se reencontra nessa utilidade,
  \emph{ôphéleian}, em 234c3, no final da peroração. Em 274d7 na
  alegoria de Thamous, essa \emph{ôphelían}, no caso agora da escrita,
  será mais uma vez evocada por Theuth.} Eu considero o que foi dito
suficiente. Se desejares saber algo que foi negligenciado, pergunta
(\emph{erṓta})''.\footnote{A última palavra do discurso, \emph{eróta},
  verbo \emph{erotáô}, perguntar, tem em si, por homonímia, o nome de
  Eros. Veremos o nome de Eros se confundindo também com
  \emph{pt\emph{éros}}, alado, nos versos dos homéridas (252b9).}

Como te parece o discurso (\emph{ho} \emph{lógos}), ó Sócrates? Não é
maravilhoso (\emph{huperphýôs}), entre tantas outras razões,
especialmente no vocabulário (\emph{onómasin}) empregado?

{[}234d{]} S: Divino mesmo, ó companheiro, a ponto de eu estar atordoado
(\emph{ekplagênai}). E essa minha afecção (\emph{épathon}) foi gerada
por ti, ó Fedro, pois a visão (\emph{apoblépôn}) que tive de ti foi
radiante durante a leitura do discurso (\emph{toû} \emph{lógou}).
Considero-te melhor do que eu para apanhar esses discursos e sigo-te,
cabeça divina (\emph{theías kephalês}), como em um cortejo báquico
(\emph{sunebákcheusa}).\footnote{Sócrates simula um estado de afecção,
  pela impiedade do discurso e pela sua monotonia, mas Sócrates atribui
  esse estado à visão que manteve de Fedro durante a leitura e, em
  seguida, mimetiza uma saída, um livramento, evocando os mistérios e
  versos dionisíacos (\emph{sunebákcheusa}), considerendo Fedro como um
  guia inspirado (\emph{theías kephalês}). Tanto é uma afecção
  evidentemente excessiva que imediatamente Fedro o acusará de estar com
  brincadeiras (\emph{paídzein}).}

F: Já estás brincando (\emph{paídzein}), não é?

S: Pareço por acaso brincar (\emph{paídzein}) e não me esforçar
(\emph{espoudakénai})?

{[}234e{]} F: De modo algum, ó Sócrates, mas como dizes a verdade,
diante de Zeus protetor da Amizade (\emph{alêthôs eipe pros Diòs
philíou}), considera-te na iminência de pronunciar entre os helenos, a
respeito do mesmo assunto, outro, melhor e mais extenso {[}discurso{]}
que este?

S: O quê? É preciso que eu e tu elogiemos (\emph{epainethênai}) o
discurso (\emph{tòn lógon}) desse criador que disse o que devia? Ele não
é somente claro (\emph{saphê}) e perfeito (\emph{stroggúla}), mas também
exato (\emph{akribôs}) em cada uma das palavras entalhadas
(\emph{onomátôn apotetórneuetai})? Se for preciso agradeceremos
(\emph{chárin}) o autor, apesar da minha ignorância me obscurecer
(\emph{elathen}) {[}235a{]}. Tomando exclusivamente o pensamento
retórico dele (\emph{rhêtorikôi autoû mónôi tón noûn proseîchon}),
acredito que nem mesmo o próprio Lísias consideraria suficiente. E me
pareceu, ó Fedro, se não queres dizer outra coisa, que ele afirmou o
mesmo duas ou três vezes, como se não tivesse muitos recursos adicionais
para fazê-lo, acerca do mesmo assunto, sem nenhum interesse. E
pareceu-me ainda uma demonstração juvenil (\emph{neanieúesthai
epideiknúmenos}) de quem quer falar tanto de uma perspectiva quanto de
outra e, em ambos os casos, da melhor maneira possível.\footnote{Esse
  trecho passa do elogio ao vitupério, sendo esse movimento de caráter
  antistrófico, como o do coro, uma contraposição de duas opiniões. Ele
  começa elogiando levemente (234e5-235a2) e acaba detratando com
  veemência (235a2-235a8), como se estivesse desejando esconder algo,
  mas que, no fluxo discursivo, acaba dizendo o que pensa de fato,
  contra a polidez inicial, atravessando para o outro lado do discurso.
  O pensamento retórico de Lísias é posto à prova, e sua técnica é
  descrita como insuficiente na visão de Sócrates.}

{[}235b{]} F: Não é como dizes, ó Sócrates, uma vez que isso é o que
justamente o discurso (\emph{ho lógos}) tem de melhor, o fato de não ter
negligenciado nenhum dos assuntos convenientes em sua performance. Desse
modo, junto a este discurso, ninguém seria capaz de proferir
(\emph{eipeîn}) outro, mais extenso e mais digno (\emph{pleíô kaì
pleíonos áxia}).\footnote{Expressão idêntica à de 236b42.}

S: Nesse ponto, eu jamais poderia ser persuadido (\emph{pithésthai}) por
ti. Os antigos sábios\footnote{Nesse uso do termo \emph{sophós}, Platão
  atribui sabedoria aos poetas líricos que escreveram. Na menção
  anterior, ``sábio'' referia ironicamente os sofistas, que se ocupavam
  de reduzir o mito ao verossímil. Apesar da conhecida crítica de Platão
  aos poetas e sua arte mimética, fica claro que os poetas líricos
  estão, nessa gradação, acima dos sofistas e logógrafos, de modo que o
  termo ``sábio'' atribuído aos poetas nesse passo carrega uma ironia
  menor que na aplicação anterior do termo. Anacreonte será chamado de
  sábio novamente, logo à frente, em 235c3.} (\emph{sophoì}), homens e
mulheres que proferiram discursos e escreveram (\emph{eirêkotes kaì
gegraphótes}), refutar-me-iam (\emph{exelégksousí}) se, para agradar-te
(\emph{charidzómenos}), eu concordasse contigo.

{[}235c{]} F: Quais são eles e onde ouviste algo superior?

S: De imediato, assim, não posso dizer. Mas é evidente que ouvi ou da
bela Safo ou do sábio (\emph{sophoû}) Anacreonte, ou de algum outro
escritor (\emph{suggraphéôn}). De onde tiro os testemunhos de que falo?
De certa plenitude (\emph{plêrés}), ó divino, que sinto no peito e pela
qual poderia dizer outras tantas coisas nada inferiores. Bem sei que não
é por mim que tenho em mente (\emph{ennenóêka}) essas coisas, pois
conheço minha própria ignorância (\emph{emautôi amathían}). Deixemos
isso de lado, creio que são outras as fontes que, pela audição, me
encheram (\emph{peplêrôsthai}) como a uma vasilha.\footnote{Esse
  vocabulário da plenitude (\emph{plerés} 235c5\emph{, peplerôsthai}
  235d1) introduz a tópica do entusiasmo, coordenada com a contrapartida
  da ignorância (\emph{amathían}) e do esquecimento
  (\emph{epilélêsmai}), a partir da qual Sócrates fará o seu discurso. A
  plenitude/inspiração, simulada ou não, ameniza a responsabilidade de
  quem fala, posto que é impulsionada por força alheia.} {[}235d{]} E
foi por estupidez que me esqueci (\emph{epilélêsmai}) dessas coisas, bem
como de quem as ouvi.

F: Mas, ó excelentíssimo, disseste do melhor modo possível. De quem e
como ouviste, eu não te ordeno que digas, desde que cumpras o seguinte.
Melhor que este livro, promete (\emph{hypéschêsai}) dizer outro
{[}discurso{]} em nada inferior, ficando dele afastado. E eu prometo
(\emph{hypischnoûmai})\footnote{A promessa (\emph{hypischnéomai}) é o
  primeiro recurso para estimular Sócrates a falar, sendo que Fedro
  promete erigir duas estátuas (ícones) deles próprios como pagamento
  por um discurso que Sócrates faria sobre o mesmo tema, defendendo a
  mesma tese de Lísias, só que com maior propriedade, já que dizia que o
  discurso de Lísias era insuficiente. O segundo recurso que Fedro usará
  para Sócrates falar será um juramento (\emph{hórkos}) em 236d9.}, tal
qual os nove arcontes, oferecer-te um ícone (\emph{eikóna})\footnote{O
  ícone (\emph{eikóna}) nesse trecho antecipa um dos grandes temas do
  diálogo, bem como da filosofia platônica, localizado na tópica criada
  por Estesícoro do ídolo (\emph{eídôlon}) de Helena (cf. Davies,
  \emph{PMG} Frg. 192-193), ao mesmo tempo que remete ao juramento dos
  nove arcontes atenienses de não transgredir as leis, sob a pena de
  oferecer uma estátua em ouro (cf. {[}Arist.{]} \emph{Ath. Pol.} 7.1)}
dourado em tamanho natural (\emph{isométrêton}) no templo de Delfos, não
só o meu, mas também o teu.

{[}235e{]} S: És amicíssimo e de ouro verdadeiro, ó Fedro, se julgas que
eu poderia dizer todas as coisas em que Lísias se enganou {[}no
discurso{]} e que seria preciso proferir outro junto ao dele. Creio que
isso não abateria nem o mais inábil dos escritores (\emph{suggraphéa}).
Começando pelo discurso (\emph{ho lógos}), quem considera dizer que é
melhor e necessário agradar (\emph{charídzesthai}) o não afetado pelo
amor em vez do amante (\emph{mḕ erônti mâllon ḕ erônti}) {[}236a{]} quer
evitar o encômio do prudente (\emph{to phónimon egkômiádzein}) e o
vitupério do insensato (\emph{to áphon pségein}), e, sendo assim,
necessariamente, terá ainda ele algo a dizer?\footnote{Aqui Sócrates
  discorda claramente da tese central do discurso de Lísias, segundo a
  qual é melhor agradar alguém não afetado pelo amor, em vez de agradar
  um amante (amoroso), pois declara que isso equivaleria a não elogiar o
  prudente e não censurar o insensato. Mesmo assim ele defenderá nesse
  discurso a mesma opinião de Lísias, porque esse é o desafio proposto,
  só que fará isso com a cabeça coberta, prevendo a possível ira do Amor
  ou de Afrodite. Haverá um segundo discurso de Sócrates, sua palinódia
  (243e-257b6), em que irá desculpar-se dessa falta discursiva contra
  Eros.} Mas penso que é necessário livrar e desculpar o orador, e
elogiar (\emph{epainetéon}) a sua disposição (\emph{diáthesin}), não a
invenção (\emph{heúresin}), e quando a sua disposição (\emph{diathései})
não é suficiente ou difícil de encontrar (\emph{heureîn}), elogiamos,
para além dela, a invenção (\emph{heúresin}).\footnote{São aspectos
  fundamentais da retórica a invenção (\emph{heúresis}) e a disposição
  (\emph{taxis}). Na verdade, a crítica de Sócrates incide nesses dois
  fundamentos, pois Lísias não tem boa \emph{inventio} e nem boa
  \emph{dispositio}, especialmente porque o trecho é uma peroração, como
  se Lísias tivesse começado seu discurso pelo final, não respeitado
  regras básicas da arte discursiva, ``nadando de costas'' como Sócrates
  diz. As duas ocorrências do substantivo ``invenção'', ``encontro'',
  ``achado'', ``descoberta'' (\emph{heúresis}), ao lado do verbo
  ``inventar'', ``encontrar'', ``achar'', ``descobrir''
  (\emph{heureîn}), não implicam necessariamente uma padronização dos
  termos da tradução. A maioria dos tradutores, observando essa
  proximidade, padroniza, como por exemplo Santa-Cruz y Crespo
  (invención, inventar e invención, p.79), Rowe (invention, invent e
  invention, p.39) e Pucci (invenzione, inventare e invenzione, p.23),
  mas, nesse caso, optamos por matizar os termos, traduzindo os
  substantivos por ``invenção'' e o verbo no infinitivo por
  ``encontrar'', por parecer mais adequado à passagem e não ferir o peso
  desse \emph{heurisko}. Sigo Porrati que utilizou resolução similar
  (invención, hallar, invención, p.111). Howland, R.L., associa a
  passagem a uma crítica antecipada à Isocrates, que no seu
  \emph{Helena} (Isoc. \emph{Hel}. 211) diz que vai falar acerca do
  tema, deixando de lado o que outros já escreveram, notadamente
  Górgias. Estesícoro é citado (Isoc. \emph{Hel}. 243 a4) apenas no
  final do seu tratado, o que mostra que Isócrates não prestigia
  Górgias, não considerando que ele também manejou as mesmas tópicas
  retóricas arcaicas estesicóricas. Segundo Howland, nesse trecho há uma
  menção direta a Isócrates, quando fala acerca da ``composição''
  (\emph{tèn súnthesin}) dos sofistas e da invenção
  (\emph{heurískontai}) (Isoc. \emph{Hel}. 211), contanto que se entenda
  a disposição (\emph{diáthesin}) de Platão como correspondente da
  composição (\emph{súnthesin}) de Isócrates (Howland, 1937, p.154)}

F: Concordo com o que dizes e bem medido (\emph{metríôs}) me parece.
Farei então o seguinte, estabelecerei como base para ti que o amante
está mais doente que o não afetado pelo amor (\emph{tòn erônta toû mḕ
erôntos mâllon noseîn}), {[}236b{]} deixando o resto de lado, ao dizer
outro {[}discurso{]} mais extenso e mais digno (\emph{pleíô kaì pleíonos
áxia})\footnote{Expressão idêntica à de 235b4-5} que o de Lísias,
estarás em ouro maciço junto a oferenda dos Cipsélidas\footnote{Cípselo
  (657-627a.C.), tirano de Corinto, pai de Periandro, um dos sete
  sábios. Aqui há uma ligação do ícone (imagem) em ouro e a expressão
  anterior de Sócrates, quando diz, em 235e1, a Fedro que ele é
  ``amicíssimo e de ouro verdadeiro''. A referência aos Cipsélidas,
  decendentes de Cípselo, se justifica pela fama que tinham de oferecer
  estátuas votivas bastante custosas em Delfos e Olímpia (cf. Ar.
  \emph{Pol}. 1313b22 \emph{tà anathémata tôn Kupselidôn}).} em Olímpia.

S: Tomaste a sério, ó Fedro, porque eu brinquei acerca do teu favorito.
Consideras mesmo que eu verdadeiramente vá falar, contra aquela
sabedoria (\emph{sophían})\footnote{O termo sabedoria (\emph{sophían}),
  ligado a Lísias, mais uma vez é usado de modo irônico.}, outro
{[}discurso{]} em algo mais variado.

F: Nesse caso, ó querido, chegas a colher o mesmo. {[}236c{]} É
necessário que digas, tanto quanto lhe seja possível, um {[}discurso{]}
melhor em tudo, para que não nos obriguemos a realizar uma grosseria de
comediantes, trocando mutuamente de papéis, e que não queiras ainda
forçar-me a dizer-te aquele ``ó Sócrates, se não conheço Sócrates,
estaria esquecido de mim mesmo''\footnote{Fedro imita a fala de Sócrates
  em 228a5-6.}, ou aquele ``desejas dizer, mas, no entanto, fica
enternecido''.\footnote{Menção não literal de 228c1-2, quando Sócrates
  disse: ``ficas enternecido (\emph{ethrýpteto}) como se não desejasses
  falar''.} Põe na tua cabeça que não sairemos daqui antes que digas o
que tens a dizer. Estamos sozinhos num lugar ermo, além do que sou mais
forte e mais jovem, de modo que, de toda maneira, ``vê se entende o que
te digo''\footnote{Verso de Píndaro (frag. 105.1 Snell-Maehler) repetido
  no \emph{Mênon} 76d e em Aristófanes, \emph{Aves} v. 945.}, não
queiras falar à força (\emph{pròs bían boulêthêis}), mas de bom grado.

{[}236d{]} S: Ó bem-aventurado Fedro, eu seria risível se, de improviso,
um inábil como eu procurasse equiparar-me a um bom compositor
(\emph{agathòn poiêthḕn}) nesses assuntos.

F: Perceba que é melhor parar (\emph{paûsai}) de vangloriar-te para cima
de mim, pois eu tenho algo a dizer para justamente forçar-te a falar.

S: Não pode ser como dizes.

F: Não? Pois então eu digo, será um juramento (\emph{hórkos}) a minha
fala (\emph{lógos}), prometo a ti -- e a alguém mais talvez, a algum
deus daqui? -- {[}236e{]} e até mesmo a esse plátano.\footnote{Esse
  juramento diante do plátano soa com tópica médica, da enunciação de um
  universo hipocrático ou, por contiguidade, apolíneo.} Se não disseres
outro discurso no lugar deste (\emph{enantíon autês taútes}), nunca mais
te comunico exposição nenhuma, de quem quer que seja.

S: Que infâmia, ó abominável, bem encontraste (\emph{eũ
anēũres})\footnote{Continuidade do uso do campo semântico de
  \emph{heurisko}: ``inventar'', ``encontrar'', ``descobrir''. Cf. nota
  63.} um meio para forçar um homem que ama discursos
(\emph{philológôi})\footnote{Sócrates é amante desse \emph{lógos}, de
  todo tipo de discurso, falado e escrito, bem como obviamente do
  pensamento que ele porta.} a fazer o que dizes.

F: Então porque te esquivas (\emph{stréphēi})?\footnote{O termo
  \emph{strepheîn} aqui ilustra, tal qual o do movimento do coro, seja
  estrófico ou antistrófico, uma mudança de curso, um movimento.}

S: Em absoluto, uma vez que tu prometeste isso, como seria possível
apartar-me desse banquete (\emph{thoínês})?

{[}237a{]} F: Diz então.

S: Sabes como farei?

F: A respeito de quê?

S: Falarei encoberto (\emph{enkalypsámenos})\footnote{Cobrir a cabeça,
  supostamente com um manto, é um indício de que a tese que proferirá é
  ímpia e que Sócrates teme, antecipadamente, algum tipo de punição por
  parte de Eros ou de Afrodite. Em 243b5-7 Sócrates fará menção à essa
  atitude.}, para que rapidamente percorra o discurso (\emph{tòn lógon})
e para que não te veja, temendo vacilar de vergonha.

F: Simplesmente fala, de resto faz como queiras (\emph{boúlei}).

S: Vinde, ó Musas\footnote{Platão inicia a frase mimetizando o bordão de
  chamamento às Musas, inspiradoras da poesia. Cf. doxografia de
  Estesícoro (Eusthathius \emph{Sobre a} \emph{Ilíada} frag. 240,
  Campbell, \emph{Greek Lyric} III, p.166), onde vemos: \emph{deur'áge,
  Kalliópeia lígeia}. Alcman também utiliza similar fórmula dedicada à
  Calíope: \emph{Mus'age Kalliópa thýgater Dìos} (fr. 84 \emph{PMG},
  Davies). Há uma evocação explícita do universo da lírica na simulação
  de Sócrates.}, tanto na forma de odes melodiosas (\emph{lígeias}),
como na dos músicos de Ligure (\emph{Ligúôn}), ambas são epônimas,
``toma da minha'' palavra (\emph{toû mýthou}) e me obriga a falar
(\emph{legeîn}) da melhor forma possível (\emph{béltistos}), para que o
teu companheiro, parecendo-lhe ser sábio (\emph{sophòs}) num primeiro
momento, tenha agora melhor reputação (\emph{dóxêi}) ainda. {[}237b{]}
Era uma vez um menino, na verdade um jovem, que era muito belo
(\emph{kalós}), e que por isso tinha muitos amantes (\emph{erastaì}). Um
entre os seus aduladores (\emph{aimúlos}), não menos amoroso
(\emph{erôn}) que outros, foi capaz de convencê-lo (\emph{epepeíken}) de
que não o amava (\emph{ouk} \emph{erôiê}), e em seguida de que era
preciso antes agradar a quem não ama em vez dos amantes (\emph{hôs mḕ
erônti prò toû erôntos déoi charídzesthai})\footnote{Retomada da tese
  central do discurso de Lísias. Curiosamente agora isso ocorre a partir
  de uma pequena cena introdutória acerca de um amante capaz de forjar
  um discurso contrário aos seus próprios sentimentos, dizendo que não
  ama ninguém e que amar não é bom. A partir dessa indiferença como
  estratégia que capta a atenção do cortejado, o amante faz justamente o
  contrário do que o cortejado espera, depreciando o seu verdadeiro alvo
  do desejo. Essa fingidez premeditada apresenta a tentativa de Sócrates
  em defender através do discurso a mesma tese ímpia contra Eros, só que
  com maiores artifícios retóricos.}, ele dizia o seguinte:

``Acerca de tudo isso, ó menino, só há um princípio (\emph{mía archḕ})
aos que desejam bem deliberar (\emph{bouleúsesthai}), {[}237c{]} saber
necessariamente acerca do que se trata em cada deliberação
(\emph{boulé}) ou então será forçoso que haja engano (\emph{hamartánei})
em todo o resto. Muitos se esquecem (\emph{lélêthen}) de que não
conhecem cada uma das essências (\emph{ousían}). E como não sabem
acordarem-se no princípio de uma verificação (\emph{diomologoûntai en
archêi tês} \emph{sképseôs}), eles pagam o preço verossímil
(\emph{eikós}) de não concordarem (\emph{homologoûsin}) nem consigo
mesmos, nem com outros. Não soframos, eu e tu, daquilo que censuramos
neles, mas é necessário conhecermos o discurso (\emph{ho lógos}) que
propicia a melhor amizade (\emph{philían}), se é o direcionado ao amante
ou ao que não o é, para sabermos, em seguida, as potencialidades do amor
e de sua natureza, estabelecendo uma definição acordada (\emph{homología
thémenoi hóron}) {[}237d{]} para uma verificação (\emph{sképsin}), de
modo a vê-lo e referi-lo no que oferece de útil (\emph{ôpheleían}) e de
dano (\emph{blabên}).\footnote{Essa segunda introdução marca a
  racionalidade necessária acerca de uma discussão, que se mantenham os
  interlocutores atentos ao mesmo princípio, ou essência, a partir da
  qual teçam com segurança uma argumentação coerente. Aqui Sócrates
  retoma a dicotomia de Lísias entre utilidade e dano, que corresponde
  ao elogio e ao vitupério, mas apresenta elementos novos, como o engano
  (\emph{hamartánein} 237c2), a verificação (\emph{sképseos} 237c4,
  \emph{sképsin} 237d2), a essência (\emph{tèn ousían} 237c3) e o acordo
  (\emph{homologoúsin} 237c5, \emph{homología} 237d1). Um novo campo
  semântico surge e nessa concepção há uma disputa entre os desejos e a
  opinião, sendo que opinião se identificará com uma determinada razão.}

É claro para todos que o amor é um desejo (\emph{epithymia}) e que
também os que não estão sob o efeito do amor (\emph{mḕ erôntes}) desejam
(\emph{epithymoûsin}) os belos, isso nós sabemos. Mas como discerniremos
o amante do não afetado pelo amor (\emph{tòn erônta te kai mḕ
krinoûmen})? É preciso entender (\emph{noêsai}) que há, em cada um de
nós, duas formas (\emph{idéa}) que nos presidem e nos conduzem: uma
delas seguimos onde quer que nos leve, é a do desejo dos prazeres inatos
(\emph{emphytos oûsa epithymía hedonôn}), a outra é a da opinião
(\emph{dóxa}) adquirida, que tende para o melhor (\emph{toû arístou}).
{[}237e{]} Algumas vezes, essas duas tendências em nós estão de acordo
(\emph{homonoeîton}), outras vezes em conflito (\emph{stasiádzeton}), de
modo que algumas vezes predomina uma delas, outras vezes a outra.

A opinião (\emph{dóxes}) da melhor razão domina e conduz pelo poder do
que é chamado de prudência (\emph{sophrosýne}), ao passo que o desejo
irracional arrasta para os prazeres (\emph{epithymías dè alógôs
helkoûsês epi hêdonàs}), deflagrando aquilo que no início denominamos
desmesura (\emph{hýbris}).\footnote{Sócrates divide a ação humana em
  duas forças inerentes, por um lado, o desejo dos prazeres inatos e,
  por outro, a da opinião (\emph{dóxa}) adquirida, que tende para o
  melhor. Quando há acordo entre elas fica tudo bem, quando não, uma
  delas vence, se for a opinião, isso será sinal de prudência
  (\emph{sophrosýne}), se forem os prazeres, isso será um sinal da
  desmesura (\emph{hýbris}). Assim como na \emph{República}, em que a
  dicotomia entre os impulsos racionais e apetitivos são gradativamente
  substituídos pela tripartição, no \emph{Fedro} acontece um movimento
  similar, uma vez que a dicotomia desse primeiro discurso de Sócrates
  (237a7-241d) será ``substituída'' pela alegoria da tripartição da alma
  na palinódia (243e-257b6).} {[}238a{]} A desmesura (\emph{hýbris}) tem
muitos nomes, muitos membros e partes, e se por acaso (\emph{týchêi})
ela tomar alguma dessas formas, oferece o epônimo, nem belo e nem digno,
a todos que a trazem consigo. Quando a ingestão de alimentos domina a
razão (\emph{lógou}) e o melhor (\emph{arístou}), entre outros desejos
(\emph{epithymiôn}), o desejo (\emph{epithymía}) é o de um glutão
(\emph{gastrimargía}), que conferirá esse título a quem o possua.
{[}238b{]} Sobre a tirania (\emph{tyranneúsasa}) das bebidas, conduzindo
aquele que a tem, produz aquela óbvia denominação. Da mesma maneira, as
coisas a estas aparentadas, bem como o surgimento de desejos correlatos,
sempre são soberanos (\emph{dynasteuoúses}) e tem cada qual o seu nome.

De acordo com tudo o que foi dito antes, é quase evidente, podemos
dizer, ou não dizer, da forma mais clara possível (\emph{saphésteron}):
sem a razão (\emph{aneu lógou}), o desejo (\emph{epithymía}) domina a
opinião (\emph{doxês}) e conduz diretamente ao prazer da beleza
(\emph{hedonèn achtheîsa kállous}), {[}238c{]} então os desejos
(\emph{epithymiôn}) congêneres com força se lançam à beleza dos belos
corpos, vencendo todos, eles adquirem a força do seu nome, sendo
chamados de amor.''\footnote{Esse amor aquivale ao puro desejo dos
  prazeres, sem intervenção da razão, a qual levaria à opinião e à
  prudência. Nessa interrupção do discurso, Sócrates, mais uma vez, se
  gaba de uma suposta inspiração e procura uma confirmação da parte de
  Fedro, se ele percebe algo divino nele, ao que Fedro confirma,
  ressaltando a não usual fluência de Sócrates. Esse seria o retrato
  mimético do rapsodo inspirado que vemos no \emph{Íon}, obviamente
  Sócrates (Platão) está depreciando Fedro, bem como de todos que
  acreditam nessa suposta inspiração, que aos olhos de Sócrates (Platão)
  não passa de uma recitação técnica, uma mimese. Mesmo assim, é
  necessário ressaltar, que essa inspiração forjada e forçada que
  Sócrates mimetiza, e que será alvo de detratação posterior, ainda
  assim permanece superior à logografia de Lísias, pois o discurso
  escrito não trazia nenhum elemento reflexivo acerca dos desejos, nem
  da natureza humana.}

E então, ó querido Fedro, pareço-te, como a mim mesmo, afetado por algo
divino (\emph{theion páthos peponthénai})?

F: Completamente, ó Sócrates, foste tomado por uma não usual fluência.

S: Cala-te agora e escuta-me: realmente esse lugar parece divino
(\emph{theîos éoiken ho tópos eînai}) e não te espantes se eu, muitas
vezes, no discurso, for tomado pelas Ninfas. {[}238d{]} Agora mesmo, ao
falar, não estive longe de entoar um ditirambo.

F: Dizes a mais pura verdade.

S: E isso é por tua causa. Mas escuta o restante, pois talvez eu possa
me apartar (\emph{apotrápoito}) dessa ameaça. Deixemos isso ao cuidado
do deus (\emph{theôi}) e ocupemo-nos novamente do discurso (\emph{pálin
tôi logôi}) dirigido ao jovem.

``Que seja assim, ó meu bravo amigo. O assunto da deliberação
(\emph{bouletéon}) já foi mencionado e definido. Vejamos, agora, o que
nos falta ainda dizer sobre ele, qual é a utilidade (\emph{ôphelía}) ou
o dano (\emph{blábê}) que pode, de modo verossímil (\emph{ex eikótos}),
advir a quem agrada (\emph{charidzoménoi}) ao amante e a quem
{[}agrada{]} ao não afetado pelo amor.\footnote{Essa retomada do
  discurso ressalta a definição do tema no começo do discurso como uma
  regra básica da arte discursiva não cumprida por Lísias, mas agora
  cumprida por Sócrates. Essa dicotomia entre a utilidade
  (\emph{ôphelía}) e o dano (\emph{blábê}) perpassa todas as recitações
  do diálogo, chegando até mesmo ao mito final de Thamous, sendo que
  está diretamente ligada ao vitupério e ao elogio.} {[}238e{]} Aos que
começam a ser escravizados pelo desejo (\emph{epithymías}) é necessário
que, de alguma maneira, pelo prazer (\emph{hedonêi}), busquem no amado o
que lhes dá esse prazer (\emph{hḗdiston}). Ao doente (\emph{nosoûnti}) é
necessário que tudo lhe seja agradável (\emph{hedù}) e que nada o
contrarie (\emph{antiteînon}), pois causa aborrecimento tudo o que é
mais forte ou de força similar ao seu desejo. O amante não admite, da
mesma forma, nem superioridade nem igualdade ao seu predileto
(\emph{paidikà}), {[}239a{]} e sempre procura rebaixá-lo, tornando-o
inferior. O ignorante (\emph{amathḕs}) é inferior ao sábio
(\emph{sophoû}), assim como o covarde é inferior ao corajoso, o inábil
na fala inferior ao homem da retórica (\emph{rhētorikoũ}) e o lento
inferior ao sagaz.

Esses males e ainda outros maiores surgem no pensamento
(\emph{diánoian}) do amante com relação ao amado (\emph{erastḕn
erôménôi}), e, naturalmente, são {[}males{]} ligados ao prazer
(\emph{hḗdesthai}), podendo até mesmo provocá-los
(\emph{paraskeuázdein}), quando {[}o amante{]} fica momentaneamente
apartado do seu prazer (\emph{hêdéos}). Necessariamente o amante é
ciumento e afasta {[}o amado{]} de todas as outras companhias que lhes
sejam proveitosas (\emph{ôphelímôn}), {[}239b{]} sobretudo das que
enriqueçam o homem. O dano que isso causa é enorme, mas o maior deles é
o de evitar que {[}o amado{]} torne-se prudentíssimo
(\emph{phronimṓtatos}).\footnote{Nesse trecho o desejo é o grande mal a
  ser controlado, quando isso acontece é porque a prudência se torna
  mais forte, bem como a divina filosofia, mencionada a seguir, a que se
  deve esse suposto domínio dos desejos. Nesse sentido, o trecho tem um
  endereço certo, contra qualquer tipo de defesa dos prazeres enquanto
  bens supremos, e pode ser entendido como uma passagem anti-hedonista.}

Quando o amante mantém o afastamento necessário com relação ao seu
predileto, temendo por este ser depreciado, isso ocorre
(\emph{tugchánei}) devido à divina filosofia (\emph{theía
philosophía}).\footnote{A filosofia nessa perspectiva proporcionaria uma
  precaução com relação aos amados, cautela com ares de racionalidade.
  Na palinódia essa cautela será descrita por Sócrates como uma conduta
  ímpia e a filosofia será descrita de outro modo. Nesse novo quadro a
  filosofia dependerá necessariamente de Eros e de seus efeitos, mesmo
  porque a loucura erótica (\emph{erotikèn manían}) é revelada como a
  melhor (\emph{arístên}) das loucuras em 265b5.} Outras coisas são
maquinadas para que {[}o amado{]} possa ficar ignorante em tudo
(\emph{pant' agnoôn}) e só tenha olhos (\emph{apoblépôn}) para o seu
amante (\emph{erastén}), como se fosse o mais agradável
(\emph{hédistos}), entretanto isso pode ser o mais danoso
(\emph{blablerṓtatos}) a si mesmo. {[}239c{]} Então, segundo esse
pensamento (\emph{diánoian}), quer para o tutor, quer para o
companheiro, em nenhuma parte é proveitoso ao homem sentir amor
(\emph{érôta}).\footnote{Reafirmação da tese ímpia de Lísias contra o
  amor.}

Depois dessas coisas é preciso conhecer a disposição e os cuidados
(\emph{héxin te kaì therapeían}) relativos ao corpo, de como ele deve
ser cuidado (\emph{therapeúsei}) pelo seu comandante e em que medida é
forçado a perseguir (\emph{diókein}) o prazer antes do que é o melhor
(\emph{hedù prò agathoû}). Veremos um amante a perseguir um jovem
delicado e não muito valente, alguém que não foi bem nutrido na pureza
do sol, mas na companhia da sombra, inexperiente nas fadigas e suores
dos labores masculinos, mas experiente no estilo de vida delicado dos
afeminados, {[}239d{]} embelezando-se com cores e adornos incomuns, vida
acompanhada as outras práticas que dessas derivam, práticas evidentes e
indignas até mesmo para avançarmos no comentário, mas delimitemos a
questão primordial (\emph{kephálaion}) acerca disso para passarmos para
o outro assunto. Esse corpo, seja na guerra ou em outros afazeres,
proporciona, por um lado, coragem aos inimigos e, por outro, medo aos
amigos e mesmo aos amantes.

Deixemos de lado o que é evidente, pois é necessário que delimitemos
agora {[}239e{]} qual dessas atitudes nos é útil (\emph{ôphelían}) ou
danosa (\emph{blábên}) ao recebermos a companhia e a tutela de um
amante. É claro para todos, e especialmente para o amante (\emph{tôi
erástêi}), que ele se regozije de que o amado possa ser privado de tudo
aquilo que é mais querido, mais amistoso e diviníssimo. Prefere que ele
fique afastado de pai, mãe, parentes e amigos, {[}240a{]}
considerando-os todos como empecilhos e censores à sua prazerosa
(\emph{hêdístes}) convivência. Mas se {[}o amado{]} possui ouro ou
qualquer outro tipo de posse, não será da mesma maneira fácil ser
capturado e mantido sob controle. Por conta disso, é forçoso que o
amante (\emph{erastḕn}) sinta ciúmes de jovens que têm recursos, e
gostem (\emph{chaírein}) que eles sejam arruinados. E ele ainda
preferiria que seu amado viesse a ficar sem se casar (\emph{ágamon}),
sem filhos (\emph{ápaida}) e sem casa (\emph{áoikon}), tanto tempo
quanto fosse possível, para que pudesse colher o seu doce desejo
(\emph{epithymôn}) no máximo tempo possível.

Existem outros males ainda, mas algum \emph{daímôn} os misturou
(\emph{émeixe}) aos maiores prazeres (\emph{hêdonén}) momentâneos,
{[}240b{]} como no caso do adulador, terrível fera de enorme prejuízo
(\emph{kólaki, deinôi thêríôi kaì blábêi megálêi}), para o qual, ao
mesmo tempo, a natureza mesclou (\emph{epémeixen hê phýsis}) algum tipo
de prazer requintado (\emph{hêdonén tina ouk ámouson}).\footnote{O
  sofista é espécie perigosa de adulador que aparece também como um ser
  de múltiplas cabeças no diálogo \emph{Sofista} 240c (\emph{ho
  polyképhalos sophistès}), bem como na \emph{Politeia} 588c, onde a
  parte apetitiva da alma, parte em que predominam os desejos mais
  fortes, é descrita como uma fera variada e de múltiplas cabeças
  (\emph{theríou poikílou kaì polykephálou}).} Como os {[}prazeres{]} de
uma cortesã (\emph{hetaíran}), que poderia ser censurada como danosa
(\emph{blaberòn pséxeien}), assim como outras similares criaturas e
ocupações, prazerosas (\emph{hedístoisin}) no início, pelo menos durante
um dia. O amado, então, com relação ao seu amante, torna-se danoso, bem
como o convívio prolongado torna-se o maior desprazer possível
(\emph{aêdéstaton}). {[}240c{]} Como diz o antigo ditado, ``cada idade
agrada aos da mesma idade''\footnote{O mesmo provérbio encontra-se
  subentendido na \emph{Politeia} 329a e mencionado como ``o antigo
  ditado'' (\emph{ho palaiòs lógos}).}, pois considero que
\protect\hypertarget{setaDir}{}{}idades similares conduzem a uma
similaridade de prazeres (\emph{hedonàs}) e proporciona uma semelhante
amizade (\emph{philían}), da mesma maneira como a convivência contínua
entre eles causa a saciedade. E dizem que o constrangimento é pesado em
todos os casos e para todos, especialmente aos amantes (\emph{erastḕs})
que tem essa diferença etária com relação ao amado (\emph{pròs
paidikà}). O velho que convive com um jovem, nem de dia nem de noite
abandona voluntariamente seu amado, {[}240d{]} mas é conduzido pela
necessidade e pelo aguilhão daquele que sempre lhe oferece prazer
(\emph{hêdonàs}), vendo, escutando, tocando, com todos os sentidos
percebendo o amado, como se ele servisse justamente aos seus prazeres
(\emph{hêdonês}). Que tipo de exortação ou de prazeres (\emph{hêdonàs})
o amante oferece ao amado durante o tempo de convívio, para que não
cheguem a extremos do desprazer (\emph{aêdías})? Uma vez que o jovem vê
aquele olhar que já não está na flor da idade, acompanhado de outras
coisas desse tipo, que nem são agradáveis de ouvir falar, {[}240e{]}
para não ser obrigado sempre a estar disposto a essa ocupação, sendo
vigiado com suspeita constante do seu guardião em meio a todos, ouvindo
elogios (\emph{epaínous}) hiperbólicos e inoportunos (\emph{akaírous}),
bem como censuras (\emph{psógous}) inadmissíveis a um sóbrio
(\emph{nḗphontos}). E quando ele está entregue à bebida, todas essas
coisas, além de intoleráveis, passam a ser vergonhosas, especialmente
pela tagarelice excessiva e pelo atrevimento empregado.

Esse amante, danoso e desagradável (\emph{erôn mèn blaberós te kaì
aêdḗs}), quando deixa de amar, logo se torna indigno de confiança
(\emph{ápistos}), pois todos os juramentos (\emph{hórkôn}) e todas as
súplicas professadas mantinham a companhia com dificuldade, {[}241a{]}
uma vez que a relação já era penosa de suportar, mesmo quando havia a
esperança de lhe trazer coisas boas (\emph{agathôn}). Quando é
necessário mudar (\emph{metabalôn}) sua própria conduta, o amante passa
a dominar a si mesmo e a estar preparado, com a inteligência e a
prudência em vez do amor e da loucura (\emph{noûn kaì sôphrosýnên ant'
érôtos kaì manías}), e dessa forma ele esquece o seu predileto
(\emph{lélêthen ta paidiká}). Enquanto o amado demanda as graças
{[}prometidas{]}, relembrando (\emph{hypomimnḗiskon}) os feitos e ditos,
como se pudesse dialogar ainda com ele, o amante, por outro lado, por
vergonha não diz a ninguém o que ocorreu, e de nenhum modo confirma os
juramentos impensados do início (\emph{anoḗtou archês horkômósiá}) e as
promessas realizadas, {[}241b{]} pois agora está em sua plena
inteligência (\emph{noûn})\footnote{A dicotomia entre \emph{noûn} e
  \emph{anoétou} (241a8-b1) que aparece nesse trecho é similar à
  dicotomia de \emph{anoétôi} e \emph{noûn}, logo a seguir (241b7-c1),
  bem como entre \emph{noûs} e \emph{anoías} em 270a5.} e salvo pela
prudência (\emph{sesôphronêkós}), que o impede de agir de maneira
semelhante ou fazer aquelas coisas novamente (\emph{pálin}). Ele foge de
tudo isso, tendo cometido uma falta pela força do amor anterior
(\emph{prìn erastḗs}), e sendo alterada (\emph{metapesóntos}) a concha
de lado, ele se retira na direção alternada (\emph{híetai phugêi
metabalṓn}).

Aquele que, por outro lado, é levado agora a perseguir a irritação e a
imprecação contra os deuses, desconhece tudo desde o início, de que não
devia agradar (\emph{charídzesthai}) ao amoroso (\emph{erônti}), forçado
pela falta de intelecto (\emph{anoétôi}) {[}241c{]}, mas que seria muito
melhor não estar sob o efeito do amor (\emph{m}ḕ \emph{erônti}) e manter
o intelecto (\emph{noûn}).\footnote{Eros aqui aparecerá pela última vez
  como destruidor do intelecto.} Caso contrário, seria obrigado a
entregar-se a alguém sem crença (\emph{apístôi}), mal-humorado,
invejoso, desagradável (\emph{aêdeî}), danoso (\emph{blaberôi}) para a
essência (\emph{pròs ousían}), danoso (\emph{blaberôi}) para a
disposição do corpo e, sobretudo, muitíssimo danoso
(\emph{blaberôtátôi}) para a educação da alma (\emph{psychês
paídeusin}), a qual em verdade é a mais honrada (\emph{timiṓteron}), e
não haverá nada no futuro tão honrado entre homens ou deuses. É preciso
conhecer tais coisas, ó criança, e saber que a amizade do amante
(\emph{t}ḕ\emph{n erastoû philían}) não surge entre favores
(\emph{eunoías}), mas como alimento, para agradar (\emph{chárin}) a
saciedade, {[}241d{]} pois os amantes amam (\emph{philoûsin}) seus
prediletos como os lobos amam (\emph{agapôsin}) os
cordeiros''.\footnote{Segundo Hermias (§90.9-10 Couvreur = 61,12-13
  Lucarini e Moreschini) esse provérbio retoma \emph{Ilíada} XXII, v.
  263.}

Isso é tudo, ó Fedro, e nada mais ouvirás de meu discurso, pois esse é o
seu fim.

F: E eu considerei que estavas no meio e que dirias semelhantes coisas
também acerca do que não está sob o efeito do amor (\emph{m}ḕ
\emph{erôntos}), de como é melhor agradá-lo (\emph{charídzesthai}),
mencionando o quanto isso tem de bom, mas agora, ó Sócrates, porque
interrompeste (\emph{apopaúêi})?

{[}241e{]} S: Não percebeste, ó bem-aventurado, que eu proferia há pouco
um épico, não um ditirambo, o qual é mais conveniente ao vitupério
(\emph{pségôn})? E se eu começasse a elogiar (\emph{epaineîn}) o outro,
o que pensas que eu faria? Consideras que sob a influência das Ninfas,
as quais tu me colocaste premeditadamente, eu estaria obviamente
inspirado (\emph{enthousiásô})? Digo, então, que por meio de um só
discurso recusamos o outro, o qual nos oferece os benefícios contrários.
E para que, então, um discurso tão extenso (\emph{makroû lógoû})? Acerca
de ambos é suficiente o que foi dito, que a narrativa (\emph{mythos})
que te ofereci sofra (\emph{páschein}) o que for.\footnote{Suspeita, já
  sugerida no começo da recitação, quando Sócrates diz que falará com a
  cabeça coberta, de que algo pode acontecer contra ele ao proferir uma
  tal narrativa ímpia.} {[}242a{]} Quanto a mim, vou partir e atravessar
esse rio, antes que seja obrigado por ti a algo mais grave.

F: Não ainda, ó Sócrates, pelo menos antes que esse calor se vá, ou não
vês que é quase meio-dia, aquilo que chamamos de sol a pino? Vamos
permanecer e dialogar acerca do que foi dito, partiremos assim que o
tempo esteja mais fresco.

S: És divino em matéria discursiva, ó Fedro, e espantoso por ser isento
de arte (\emph{atechnôs}).\footnote{A palavra \emph{atechnôs} pode ser
  entendida como se Fedro estivesse inspirado, tomado pelas Musas, sem
  arte pela dádiva de participar do divino ou como se Fedro não
  detivesse mesmo nenhuma arte, ou seja, fosse um inábil. Há um jogo
  entre elogio e vitupério, Sócrates faz com que um aparente elogio seja
  verdadeiramente um vitupério, na medida que uma mesma sentença porta
  dois tipos de mensagens, apenas detectadas por aqueles que tem as
  ferramentas para decodificá-las com precisão. Nesse mesmo sentido,
  Sócrates atribui a Fedro de modo não menos irônico um dom quase que
  maiêutico, pois ele teria a capacidade de fazer com que outros
  falassem, mas isso é mais uma vez uma ironia, justamente por que
  Fedro, na visão de Sócrates, não detém nenhuma arte.} {[}242b{]} Por
conta dos discursos que ocorreram por tua força, creio que ninguém é
melhor que tu ao pronunciá-los ou forçando outros a proferi-los --
exceto o discurso de Símias de Tebas\footnote{Cf. \emph{Fédon} 59c.},
que é o mais forte (\emph{krateîs}) entre todos. E agora tu me parece
ser a causa (\emph{aitiós}) mesma de outro discurso (\emph{lógôi}) que
vou proferir (\emph{rhêthênai}).

F: Então é uma guerra o que anuncias! Mas diz como foi e a qual deles te
referes?

S: Quando decidi, ó meu caro, atravessar o rio, um \emph{daimon} que me
é familiar surgiu e me gerou um sinal (\emph{sêmeîón}) -- ele sempre me
impede quando estou prestes a fazer algo --, e parecia que eu ouvia a
sua voz, ela não me deixava partir antes de me purificar
(\emph{aphosiṓsōmai}), como se tivesse cometido alguma falta contra o
divino (\emph{hēmartēkόta eis to theîon}).\footnote{Acerca desse sinal
  costumeiro do \emph{daímon} de Sócrates cf. \emph{Euthyd}. 273e2-3,
  \emph{Theag}. 129b8, \emph{Pol}. VI 496c3, \emph{Apol}. 31d1, 40a4,
  \emph{Alc}.I 103a5, \emph{Theaet}. 151a4 entre outras passagens. Cf.
  outras ocorrências de \emph{daímon} no \emph{Lexicon Platonicum} de
  Friedrich Ast (1835) vol.1 p.415-417.} {[}242c{]} É que eu sou um
adivinho (\emph{mántis}), mas não muito aplicado (\emph{ou pánu dè
spoudaîos}), talvez como aqueles que são ruins na escrita (\emph{ta
grámmata phaûloi}), no entanto para mim isso já é suficiente.\footnote{Quando
  Sócrates diz que é um advinho não muito aplicado, tal qual os que não
  são muito bons na escrita, parece visar Lísias ou Isócrates, que
  tinham reputação de bons escritores (logógrafos).} Aí então compreendi
com clareza a minha falta (\emph{hamártēma}). É certo agora, ó
companheiro, que a alma (\emph{hē psykhḗ}) tem alguma adivinhação
(\emph{mantikón gé ti}), pois algo também me inquietou, ao proferir o
discurso passado, e temi, do mesmo modo que Íbico, com relação à sua
falta contra os deuses:

que ela não altere minha honra (\emph{timàn}) junto aos homens
\footnote{Cf. Campbell frg. 310, verso de Íbico também legado por
  Plutarco (\emph{Qu. Conv.} 748c, ix 15.2) que ilustra a distorção que
  até então há nos discursos proferidos, vituperando Eros, defendendo
  com isso que as honras e benefícios entre os homens seriam mais
  importantes do que agradar aos deuses. O tema será abordado com mais
  clareza em 273e4-274a3. Santa-Cruz y Crespo (p. 101 n. 245) chamam a
  atenção para outro poema de Íbico aludido por Parmênides no diálogo
  homônimo de Platão (\emph{Parm}. 137a), quando Parmênides se compara
  ao cavalo velho afetado por Eros, descrito por Íbico. Destaco os
  efeitos na visão provocados por Eros e Afrodite, meio pelo qual Eros
  comumente ataca, bem como Cípris (Afrodite) que lança sua rede
  infalível, e seus poderes são implacáveis, mesmo ao atingir um velho.
  A imagem da biga é arquetípica e os versos de Íbico remetem a um amplo
  campo de imagens às quais Platão retomará: \emph{Eros novamente
  obscurece-me, com seus olhos que dissolvem a vista, num encanto de
  Cípris, a que lança por todas as partes sua rede na direção dos
  desavisados. Temo agora sua aproximação, como um velho cavalo
  vitorioso, sendo conduzido contra a vontade aos rápidos jugos das
  carruagens que adentram a arena.} (Campbell frg. 287).}

{[}242d{]} Só agora percebi a minha falta (\emph{ḗisthēmai tò
hamártēma}).

F: Diz então, qual é {[}a falta{]}?

S: Terrível, ó Fedro, terrível foi o discurso (\emph{lógon}) que
trouxeste e o que me forçaste a dizer (\emph{eipeîn}).

F: Como é?

S: Foi uma tolice e uma espécie de impiedade (\emph{ti
asebê}).\footnote{A impiedade aqui consiste em não perceber a ligação
  que existe entre o homem e o divino Eros, preocupando-se, como Íbico,
  mais com sua reputação junto aos homens do que junto aos deuses. A
  falta cometida é teológica, uma impiedade, e logo em seguida Sócrates
  reavalia sua posição acerca de Fedro, dizendo que ele foi causador da
  falta, passando do elogio ao vilipêndio de Fedro no que tange aos seus
  supostos poderes de estimular discursos alheios.} Pode haver algo mais
terrível (\emph{deinóteros})?

F: Não, se é que dizes a verdade.

S: O quê? Não consideras que o Eros é um deus, filho de Afrodite?

F: Assim dizem.

S: Mas não segundo Lísias, nem pelo teu discurso, aquele que saiu da
minha boca envenenada (\emph{katapharmakeuthéntos}) pelo que tu disseste
(\emph{eléchtê}). {[}242e{]} Se é assim, tal qual sabemos, Eros é um
deus ou alguma divindade, e de nenhuma forma poderia ser mau, mas ambos
os discursos proferidos falaram dele como se assim ele fosse. Dessa
maneira, cometeram uma falta (\emph{hēmartanétēn}) contra Eros, além
disso, pretenderam-se bondosos e muito civilizados
(\emph{asteía})\footnote{Cf. nota 19 acerca de \emph{asteía}.}, mas não
foram discursos saudáveis (\emph{hygiès}) tampouco verdadeiros
(\emph{alêthès}), embora tenham disso se gabado, {[}243a{]} uma vez que,
ao enganar alguns homens, tornaram-se bem reputados entre eles. Ó
querido, eu preciso me purificar (\emph{kathérasthai}). Há uma
purificação arcaica (\emph{katharmòs archaîos}) para os que cometem
faltas em mitologia, Homero não a conheceu, mas Estesícoro\footnote{A
  Estesícoro de Himera (séc. VII a.C.) atribui-se a invenção do terceiro
  momento do coro na lírica arcaica, parte chamada de epodo, elemento
  pelo qual os movimentos do coro, que seguiam coordenados entre estrofe
  e antístrofe (movimento e contra-movimento, pergunta e resposta),
  passaram ao desenho triádico, composto por (a) estrofe, (b) antístrofe
  e (c) epodo. Esse novo desenho é conhecido como \emph{pattern} AAB. O
  epodo passou a ser usado na performance citaródica e/ou monódica, uma
  vez que seria a hora do canto solo, uma performance instrumental fixa
  combinada com os movimentos contrapostos do coro. A ``tríade de
  Estesícoro'' indica exatamente essa disposição (\emph{táxis}) poética,
  a mesma pela qual Píndaro compôs seus epinícios, seguindo o mesmo
  \emph{pattern} estesicórico AAB. Cf. D'Alfonso F. \emph{Stesicoro et
  la performance}, Roma: GEI, 1994.} sim. Privado da visão pela
linguagem abusiva contra Helena, não ignorou a causa como Homero, mas,
conhecendo a causa (\emph{égnô tèn aitían}), o músico de Himera compôs:

Esse não é um discurso verdadeiro,

nem embarcaste em naves bem assentadas,

nem foste à cidade de Troia.

{[}243b{]} E ao compor toda a obra, chamada de \emph{Palinódia},
imediatamente recuperou a visão (\emph{anéblepsen}).\footnote{A
  \emph{Palinódia} de Estesícoro, assim como esta \emph{palinódia} de
  Sócrates, são como peãs ou hinos purificadores e curativos.} Eu,
então, agora me torno mais sábio (\emph{sôph}ṓ\emph{teros}) que eles,
pelo menos nesse ponto, pois, antes de sofrer (\emph{patheîn}) algo pela
linguagem abusiva contra Eros, trato de ofertar-lhe uma \emph{palinódia}
com a cabeça descoberta (\emph{gymnêi têi kephalêi}), e não como havia
ocorrido, por vergonha, com a cabeça velada
(\emph{egkekalumménos}).\footnote{Cf. nota 72, referente 237a4, momento
  em que Sócrates diz: ``Falarei encoberto (\emph{enkalypsámenos})''.}

F: Nada poderia ser mais prazeroso (\emph{hedíô}), ó Sócrates, do que
isso que tu afirmas.

{[}243c{]} S: Pois então, ó bom Fedro, põe na tua mente (\emph{ennoeîs})
a falta de pudor (\emph{anaidôs}) com que os discursos foram proferidos,
tanto esse meu {[}discurso{]} como o teu, a partir do livro lido. Se,
por acaso (\emph{týchoi}), o caráter (\emph{êthos}) dos que nos ouvem
for nobre e gentil, eles pensariam que estão sendo levados a escutar
marinheiros e que de nenhum modo veriam um amor entre homens livres?
Quer seja o amado ou o amante, se os mencionarmos simplesmente como
enamorados, não seria através das pequenas coisas que ambos poderiam ser
tomados por grandes ódios (\emph{échthras}), ciúmes (\emph{phthonerôs})
e danos (\emph{blaberôs}) aos seus prediletos (\emph{ta paidikà}).
{[}243d{]} Eles, nesse caso, não concordariam plenamente conosco ao
vituperarmos o Amor (\emph{hêmîn homologeîn há pségomen tòn Érota})?

F: É possível, ó Sócrates, por Zeus!

S: Desse mesmo homem eu me envergonho e temo pelo próprio Amor, a ponto
de desejar, com um discurso potável, lavar-me dessa audição salgada
(\emph{epithymô potímôi lógôi oîon halmuràn akoḕn
apoklúsasthai}).\footnote{O \emph{lógos} tem o poder catártico e de
  cura, uma vez que lava a alma e o corpo de uma anterior audição
  salgada, bem como de uma falta discursiva. Essa audição salgada é
  aproximada ao amor dos marinheiros, como algo vulgar.} Diga a Lísias
para escrever (\emph{grápsai}) o mais depressa possível que partindo de
condições semelhantes, é melhor agradar (\emph{charídzesthai}) ao amante
(\emph{erastêi}) em vez do não afetado pelo amor (\emph{mè
erônti}).\footnote{Resgate e inversão da tese central da
  \emph{palinódia} de Sócrates. Agora Sócrates fará elogio pleno do
  amor. Esse é um dos temas que o \emph{pattern} AAB estesicórico
  elucida como elemento do diálogo.}

F: E veja bem que será assim mesmo, pois tu, ao fazer o elogio do amante
(\emph{tòn toû erastoû épainon}), gerarás em Lísias a necessidade de
escrever (\emph{grápsai}), impelido por mim, {[}243e{]} sobre esse mesmo
discurso (\emph{lógon}).

S: Eu acredito.

F: Diz agora com bravura.

S: Onde está o jovem com o qual eu falava? Quero que ele ouça também
isso, e que não se antecipe, por não ter ainda escutado, agradando
(\emph{charisámenos}) ao não afetado pelo amor (\emph{mḕ erônti}).

F: Ele está junto a ti, muito perto e sempre a acompanhar-te, quando tu
quiseres (\emph{boúlêi}).

S: ``Deste modo, ó bela criança, compreende (\emph{ennóêson}) que o
primeiro discurso foi o de Fedro, filho de Pítocles, homem de
Mirrinunte, {[}244a{]} e o discurso seguinte será o de Estesícoro, filho
de Eufemo, natural de Himera.\footnote{Nesta passagem demarca-se o
  terreno sagrado do novo discurso. Schefer identifica Pítocles,
  Mirrinunte e Eufemo, como índices do registro discursivo ligado aos
  mistérios, na medida em que \emph{kléos}, de Pítocles, é a glória do
  iniciado, a \emph{myrrínê}, ou mirra, de Mirrinunte, é usada nos
  rituais de revelação dos mistérios e \emph{euphémos}, \emph{euphemía},
  de Eufemo, está ligada ao hino místico e ao silêncio que se guarda
  nessas ocasiões (cf. Schefer, 2003, p. 194-195). Além disso Himera, a
  cidade natal de Estesícoro, corresponderá ao desejo (\emph{hímeros})
  de Zeus, o que mostra que há um vocabulário novo para enunciar um
  desejo diverso. Cf. também \emph{Crátilo} 418b e 419e.} Que seja dito
que \emph{não é um discurso verdadeiro} (\emph{ouk ést'étymos
lógos})\footnote{Retomada do primeiro verso de Estesícoro, em
  \emph{Fedro} 243 a-9. Traduzi \emph{étymos} por verdadeiro sem evitar
  homonímia com a tradução de \emph{aletheia}, ``verdade'', que
  aparecerá a seguir.}, aquele que diz, perto de um amante
(\emph{erastoû}), ser melhor agradar (\emph{charídzesthai}) a quem não
está afetado pelo amor (\emph{mḕ erônti}), porque um está louco
(\emph{maínetai}) e o outro sóbrio (\emph{sôphoneî}). Se a loucura
(\emph{manían}) fosse simplesmente (\emph{aploûn}) má, este seria um
belo discurso, mas os maiores bens (\emph{agathôn}) nos surgem por
intermédio da loucura (\emph{dià manías}), a qual seguramente é um
presente divino (\emph{theíai}). Tanto a profetisa do oráculo em Delfos,
quanto as sacerdotisas em Dodona, executaram para a Hélade muitas e
belas coisas, sejam particulares ou públicas, tomadas pela loucura
(\emph{maneîsai}), {[}244b{]} ao passo que sóbrias (\emph{sôphronoûsai})
elas pouco ou nada fizeram.\footnote{O outro lado do discurso aparece
  com o exemplo da Sibila e das sacerdotisas em Dodona que, quando
  sóbrias ou sem a loucura, pouco ou nada fizeram aos gregos. Aqui, ao
  contrário do exemplo anterior, em que hipoteticamente os homens se
  agradavam, trocavam favores entre si sobriamente, sem a loucura, é
  somente pela loucura que os homens obtêm dádivas. Essa divisão entre
  conjuntos aparentemente excludentes entre si mostra que Platão busca
  justamente a área em que há a intersecção entre loucura e sabedoria.}
E se dissermos que a Sibila e tantos outros, valendo-se da adivinhação
entusiástica (\emph{mantikêi ch}ṓ\emph{menoi enthéoi}), muitas vezes e
para tantos, predisseram um futuro correto, estaríamos nos alongando
sobre o que é evidente para todos.

Eis um testemunho digno (\emph{áxion epimartúrasthai}), que os antigos
instituidores dos nomes não consideravam a loucura (\emph{manían}) nem
ruim (\emph{aischròn}) nem vergonhosa (\emph{óneidos}), pois não a
teriam misturado à arte mais bela, a que interpreta o futuro, {[}244c{]}
designando-a pelo nome de \emph{maniké}. Julgaram-na bela porque a
loucura surgia por parte da divindade (\emph{theíai}). Nossos
contemporâneos, inexperientes em beleza, enfiando o ``\emph{tau''} no
meio, chamam-na de \emph{mantiké}. E os sóbrios (\emph{tôn
emphrónôn})\footnote{Essa sobriedade está relacionada à arte que surge
  entre os homens em 244d3: \emph{sôphrosýnes}.} que buscam o futuro
pelos pássaros e por outros sinais (\emph{diá orníthôn poiouménôn kai
tôn állôn semeíôn}), os mesmos que partem da reflexão e abrem o caminho
da suposição humana, do pensamento e da observação (\emph{ek dianoías
poridzoménôn anthrôpínei oiḗsei noûn te kaí historían}), esses
chamaram-na de \emph{oionoïstikḗn}. Hoje em dia os jovens imponentemente
dizem \emph{oiônistiké}, com um ``ô'' longo. {[}244d{]} Quanto mais
perfeita e honrada é a \emph{adivinhação oionística} (\emph{mantiké
oiônistiké}), e o nome da primeira atividade com relação ao nome da
segunda, mais bela é a loucura (\emph{manían}) em vista da sobriedade
(\emph{sôphrosýne}), testemunham os antigos (\emph{marturoûsin oi
palaioì}), pois uma surge por intermédio do deus (\emph{ek theoû}) e a
outra junto aos homens (\emph{par'anthópôn}).\footnote{Cf. oposição
  entre adivinhação e prudência também em Sófocles, \emph{Édipo Rei},
  pontualmente na fala de Tirésias: \emph{pháskein em' êde mantikei
  meden phronein,} v.462.}

Com efeito, a loucura (\emph{manían}) surgiu para algumas famílias que
necessitavam, profetizando as maiores enfermidades (\emph{nósôn}) e
dores (\emph{pónôn}) vindas de antigos ressentimentos, e elas
encontraram (\emph{heúreto}) refúgio em preces e cultos aos deuses.
{[}244e{]} Daí então surgiram purificações e iniciações (\emph{katharmôn
te kai teletôn}) praticadas para suas próprias isenções, tanto para o
tempo presente quanto para os tempos vindouros, sendo assim encontrado
(\emph{heuroméne}) o correto (\emph{orthôs}) afastamento dos males
coetâneos na loucura e na possessão (\emph{manéti te kai
kataschoménoi}).\footnote{A loucura isenta da mácula, dos
  ressentimentos, revela as causas e as remove. É nesse sentido que o
  discurso de Sócrates é como um peã apotropaico, um modo discursivo de
  evitar, de afastar, uma possível retaliação de Eros ou Afrodite.}

{[}245a{]} A terceira possessão e loucura (\emph{katokochḗ te kai
manía}) vem das Musas, as quais se apoderam da alma delicada e
inviolada, despertando e tornando-a báquica por meio de odes e outras
poesias, as quais ordenam (\emph{kosmoûsa}) inúmeras obras dos antigos e
educam (\emph{paideúei}) os pósteros. Aquele que chegar às portas da
poética sem a loucura das Musas (\emph{aneu manías Mousôn}), acreditando
(\emph{peistheìs}) que somente por força da arte seria poeta perfeito
(\emph{ék téchnês ikanòs poietês}), está incompleto (\emph{atelḕs}), sem
contar que a poesia (\emph{poíesis}) dos enlouquecidos
(\emph{mainoménôn}) ofusca (\emph{êphanísthê}) a dos sóbrios
(\emph{sôphronoûntos}).\footnote{Conservei nesta aparição de
  \emph{sophronoûntos} a tradução ``sóbrio'', bem como na seguinte
  aparição de \emph{sóphrona.} Seria possível também traduzir o termo
  por ``prudente'', mas sóbrio demarca melhor a oposição com o estado de
  loucura.}

{[}245b{]} Tenho dito a ti acerca da grandeza das belas obras repletas
da loucura que vem dos deuses, de tal maneira que dela não fujamos
(\emph{phobómetha}), nem nos perturbe algum discurso que amedronte o
amante na direção da necessidade da escolha (\emph{proaireîsthai}) do
amigo sóbrio (\emph{tón sṓphrona phílon}). Aquele {[}amante{]} leva a
honra da vitória antes deste {[}sóbrio{]}, mostrando que não é pela
utilidade (\emph{ôpheléiai}) que o amor (\emph{érôs}) é enviado, pelo
deus, ao amante e ao amado. Então, é preciso que demostremos o contrário
(\emph{apodeiktéon aû tounantíon})\footnote{É necessário admitir o
  contrário do que foi proferido nos discursos anteriores, ou seja, que
  era melhor agraciar o amante (afetado por Eros). Aqui é retomado o
  início da palinódia, momento em que foi explicitada a visão ímpia
  sobre o Amor e, uma vez fechado o círculo, há a mudança discursiva, a
  inversão protréptica pela qual Sócrates passa a louvar o Amor enviado
  pelo deus a ambos, amante e amado. Górgias, que usa a tópica
  estesicórica do rapto de Helena, indica também a necessidade de dizer
  o contrário do que se dizia comumente acerca de Helena no seu
  \emph{Elogio de Helena} § 9: \emph{pròs} \emph{állon} \emph{ap'állou}
  \emph{metastô} \emph{lógon}. Górgias não menciona Estesícoro em nenhum
  momento, talvez porque a tópica fosse bastante conhecida. O termo
  demonstração que ocorre aqui e logo a seguir (\emph{apodeiktéon,
  apódeixis, apodeíxeôs}) tornou-se especialmente importante em
  Aristóteles, para o qual, diferentemente de Platão, o tipo de
  argumentação científica e mais rigorosa será designado por
  \emph{apódeixis} (cf. \emph{A.po} 71a-b).}: como pela maior das sortes
essa loucura é dádiva dos deuses. {[}245c{]} Essa demonstração
(\emph{apódeixis}) não será persuasiva (\emph{ápistos}) aos terríveis
(\emph{deinoîs}), mas será persuasiva (\emph{pistḗ}) aos sábios
(\emph{sophoîs}).\footnote{Aqui \emph{sophoí} é usado sem ironia, ao
  contrário do emprego em 229-c5, em que \emph{sophoí} equivale a
  \emph{deinoîs}, ou seja, aos terríveis, aos hábeis, principalmente na
  engenhosidade e sofisticação da explicação racional dos mitos.} Será
necessário primeiramente tratarmos da natureza da alma, divina e humana,
vislumbrando suas paixões e ações (\emph{páthê te kaì érga}), entendendo
a verdade (\emph{talêthès noêsai}). O princípio (\emph{archḕ}) da
demonstração (\emph{apodeíxeôs}) é o seguinte.

Toda alma é imortal. Tudo aquilo que está sempre em movimento é imortal,
ao passo que o que move outro, ou por outro é movido, ao cessar do
movimento (\emph{kinéseos}) cessa também a vida (\emph{dzōẽs}). Somente
o auto-movido não se desliga de si mesmo, visto que nunca cessa seu
movimento, e para todas as coisas que são movidas essa é a fonte e o
princípio do movimento (\emph{pegḗ kai archḗ kinḗseos}). Princípio é sem
geração (\emph{archḕ dè agénêton}). {[}245d{]} É necessário que todo o
gerado advenha de um princípio (\emph{ex archês}), e ele mesmo não
{[}advenha{]} de nenhum, pois se ele surgisse de um princípio
(\emph{archè}) não mais poderia ser considerado um princípio. De modo
que é necessário que {[}o mesmo princípio{]} seja sem geração e sem
corrupção, pois nem se corrompe, nem é gerado, se é que todas as coisas
necessariamente surgem (\emph{gígnesthai}) de um princípio. Neste caso o
movimento tem um princípio que lhe é idêntico, dele não podendo sofrer
corrupção ou geração, ou todas as coisas e toda a gênese do céu estariam
conjuntamente perecendo e nunca teriam recebido movimento a partir de
algo. {[}245e{]} Esclarecida a imortalidade daquilo que é movido por si
mesmo, a essência da alma (\emph{psychês ousían}) e sua explicação não
foram enunciadas (\emph{legôn}) de modo vergonhoso. De todos os corpos,
os que recebem movimento externo (\emph{éxôthen}) são inanimados, ao
passo que os que de dentro de si e por si {[}recebem o movimento{]} são
animados, tal é a essência natural da alma (\emph{hos taútês oúsês
phýseôs psychês}). Sendo assim, a alma não é outra coisa senão aquilo
que move a si mesmo, necessariamente a alma não é gerada, o que a torna
imortal.

{[}246a{]} Acerca da imortalidade é o suficiente. Falemos {[}agora{]}
acerca dessa ideia (\emph{idéas}).\footnote{Forma, imagem, conceito
  bastante importante na filosofia platônica, especialmente porque no
  trecho anterior foi utilizado um argumento lógico-racional, mas agora
  Platão usará a imagem para descrever a natureza da alma dos deuses e
  dos homens, suas afinidades e suas diferenças. Segundo Sócrates esse
  lugar supraceleste nunca foi cantado dignamente por nenhum poeta e nem
  será (247c), pois talvez os limites da forma poética sejam muitos,
  isso não quer dizer necessariamente que acredite que a forma em prosa
  filosófica seja capaz de descrever tudo. Na filosofia platônica há um
  lugar nobre, não passível de alegorização, lugar do inefável, do
  ofuscamento e dos limites da própria linguagem.} Quanto ao que ela é
teríamos uma grandiosa e divina (\emph{theías}) exposição
(\emph{diegḗseos}), e, ao que parece, uma exposição inferior,
conveniente aos homens, falemos por meio desta última. Convencionemos
que ela tenha uma potência (\emph{dynámei}) e uma natureza similar
(\emph{sumphýtôi}) a uma biga alada e seu auriga. Os cavalos e cocheiros
dos deuses são essencialmente todos bons e vindos do que é bom
(\emph{agathoì kaì ex agathôn}), ao passo que os dos outros são
misturados (\emph{mémiktai}). {[}246b{]} Primeiramente dirige a biga
aquele que para nós é o comandante, em consequência disso um dos cavalos
é bom e belo, enquanto o outro é o seu contrário, sendo ele mesmo um
contrário. Entre nós, portanto, o ofício de auriga (\emph{henióchêsis})
é necessariamente penoso e adverso.

Experimentemos dizer o motivo pelo qual a vida foi enunciada como mortal
e imortal (\emph{thnetón te kaì athánaton}).\footnote{cf. Heráclito, DK,
  12, B 62: \emph{athánatoi thnêtoí, thnêtoì athánatoi} {[}...{]}.} Toda
alma ocupa-se inteiramente do que é inanimado, circula por todo o céu
tomando, algumas vezes, outros aspectos (\emph{eídesi}). {[}246c{]}
Estando em sua perfeição {[}a alma{]} é alada, atravessa as alturas
(\emph{meteôroporeî}) e habita todo o cosmo (\emph{pánta tòn kósmon
dioikeî})\footnote{Escolhi para traduzir \emph{dioikeî} por ``habita'',
  porque a tradução mais usada, que é ``governa'', me pareceu excessiva
  ao referir uma alma particular. À frente haverá um
  \emph{katoikistheîsa} que os tradutores não exitam em traduzir por
  \emph{instalar-se, apoderar-se, estabelecer-se}. Além disso, em 247e6,
  Zeus será enunciado como grande poder hegemônico do cosmo, que
  atravessa tudo e tudo dirige, sendo que ele, Zeus, é quem governa
  tudo, não a alma particular. Porrati, nessa mesma linha, também traduz
  \emph{dioikeî} por ``habita'' (Porrati, p.137). Muito provavelmente a
  maioria dos tradutores é demovido frente a autoridade de De Vries, que
  considera mais apropriada a tradução de \emph{dioikeî} por
  \emph{controls} (De Vries, 1969, p.128).}, mas quando é levada à perda
das asas, então, de algum sólido se apodera e ali se instala (\emph{ou
katoikistheîsa}), tomando corpo terrestre, o qual parece mover a si
mesmo devido àquela potência. Como viventes que são enunciados
conjuntamente, alma e corpo, fixados, ganham o epíteto de mortal\emph{.}
O imortal não é deduzido por um raciocínio, mas modelamos
(\emph{pláttomen}) o deus, não somente pela visão (\emph{idóntes}), nem
somente pelo pensamento (\emph{noḗsantes}), como um vivente imortal que,
tendo alma e corpo, mantém-se assimilado para sempre no tempo.
{[}246d{]} Que essas coisas assim sejam e que tenham sido expostas ao
agrado do deus.\footnote{A imagem é plasmada (\emph{pláttomen}) da
  divindade, não é derivada de um raciocínio (\emph{lelogisménou}), nem
  de uma visão (\emph{idóntes}), tampouco de um pensamento
  (\emph{noésantes}).}

Tomemos agora a causa da queda das asas, motivo pelo qual a alma se
perde. É da natureza da potência alada levar o que é pesado para o alto,
alcançando a casa do gênero divino, por onde ela se põe em comum
(\emph{kekoinṓnêke}), no mais alto grau corpóreo, com a alma do deus. O
divino é belo, sábio, bom e tudo o que é dessa mesma classe, {[}246e{]}
e justamente por essas coisas que são mais bem acrescidas e alimentadas
as asas da alma. As coisas contrárias a estas são corruptíveis e perecem
pela maldade e pelo vício.

Zeus é o grande condutor no céu com seu carro alado (\emph{ptênòn
hárma}), adianta-se em primeiro lugar, zelando por todas as coisas
através do cosmo (\emph{diakosmôn}). Ele é seguido por um exército de
deuses e \emph{daimônes} (\emph{theôn te kaì daimónôn}) ordenados
(\emph{kekosmêménê}) em onze partes, permanecendo Héstia sozinha na casa
dos deuses (\emph{ménei gar Hestía en theôn oíkôi mónê}). {[}247a{]}
Dentre os outros tantos deuses, em sua formação de doze partes, são
conduzidos pelo chefe, seguindo a composição que lhes foi
atribuída.\footnote{\emph{tetagménoi, etáchthe} 247a2-3, sucessão de
  formas do verbo \emph{tásso}, ``ordenar'', ``dispor'', ``colocar em
  ordem'', bem com do substantivo correspondente \emph{táxis}.} Então,
muitas divindades bem-aventuradas seguem trajetos no interior do céu
(\emph{entós ouranoû}) e circulam (\emph{epistréphetai}) no gênero feliz
dos deuses, cada uma delas fazendo o que lhes é próprio (\emph{práttôn
hékastos autôn tò autoû}). Seguem sempre que querem e podem, uma vez que
a inveja (\emph{phthónos}) permanece fora do coro dos deuses (\emph{éxô
theíou choroû}).\footnote{Coro dos deuses, sendo que o verbo
  \emph{koréo} refere, a um só tempo, o canto e a dança. Essa imagem é
  especialmente importante nessa leitura do \emph{Fedro}, uma vez que
  está na origem da mimese que o homem faz do universo material,
  especialmente do movimento dos astros.} Quando vão ao cume para um
festim ou banquete, atravessam para o ápice das escarpas que sustentam o
céu (\emph{ákran epí ten hypouránion apsîda poreúontai}), de modo que as
carruagens dos deuses, estando num dócil equilíbrio, {[}247b{]}
ultrapassam facilmente, já as outras, {[}ultrapassam{]} com dificuldade.
O cavalo que partilha do mal é pesado, inclina-se para a terra e impede
o trajeto do auriga que não foi bem-educado.

Ali mesmo fica o último (\emph{éskatos}) grau de sofrimento e disputa a
que a alma se dispõe\footnote{Cf. Plotino, \emph{Acerca do Belo} 7.31.};
as almas dos imortais, quando chegam ao extremo, atravessando
exteriormente (\emph{éxô poreutheîsai}), estabelecem-se sob o dorso do
céu (\emph{epì tôi toû ouranoû nôtôi}), sendo levadas e trazidas ao seu
redor; {[}247c{]} as outras {[}almas dos mortais{]} contemplam ali as
coisas fora do céu (\emph{theoroûsi ta éxô toû ouranoû}).

Esse lugar supraceleste (\emph{hyperouránion tópon}) ainda não foi
cantado por nenhum dos poetas e nunca será cantado de forma digna. É
necessário ousar dizer a verdade (\emph{alêthès eipeîn}), sobretudo ao
falarmos da verdade (\emph{alêtheías légonta}). \emph{eidôlon}
(\emph{ousía óntôs oûsa}) não tem cor, é sem figura, intangível e
somente contemplada pelo pensamento do piloto da alma (\emph{psykês
kybernḗtei monôi theatḕ nôi}), região na qual tem lugar o gênero
verdadeiro do conhecimento (\emph{tò tês alêthoûs epistḗmês génos}).
{[}247d{]} Então, bem como pensamento do deus (\emph{theoû diánoia}),
{[}a alma{]} é nutrida pela pureza do intelecto (\emph{nôi}) e do
conhecimento (\emph{epistḗmêi}), como todas as almas que possam vir a
mostrar tal preocupação, tendo visto o ser através do tempo
(\emph{idoûsa dià chrónou tò ón}). Ela é nutrida por ter contemplado a
verdade (\emph{theôroûsa talêthê}), sentindo-se completa, até que possa
chegar, pelo ciclo, ao ponto inicial do trajeto pelo qual foi levada.
Nesse período veriam a própria justiça (\emph{dikaiosúnên}), a prudência
(\emph{sôphrosúnên}), o conhecimento (\emph{epistémên}), {[}247e{]} não
aquilo que pertence à gênese, nem o que está em outras coisas, em outros
que agora chamamos seres (\emph{óntôn}), mas conheceriam a própria
essência que está no ser (\emph{ho estin òn óntôs epistémên oûsan}). E,
do mesmo modo, tendo contemplado (\emph{theasaméne}) a essência dos
seres em seu posto (\emph{estiatheîsa}), mergulham de volta (\emph{dûsa
pálin}) para o interior do céu e chegam a casa.\footnote{Héstia, única
  deusa a ficar na casa dos deuses, é uma referência central do
  universo. O conjunto \emph{estiatheîsa}, \emph{eíso toû ouranoû} e
  \emph{oíkade} devem ser vistos como elementos constitutivos da
  cosmografia, onde a casa, lugar do fogo central, recebe as almas
  unidas aos corpos após a contemplação do ser.} O auriga chega ao
estábulo, coloca ali os cavalos, oferece-lhes ambrosia e lhes dá néctar
para beber.

{[}248a{]} Esta é a vida dos deuses. Quanto às outras almas, a que
melhor acompanha e se assimila (\emph{epoménê kaì} \emph{eikasménê}) ao
deus\footnote{A assimilação entre homens e deuses aparece também em
  \emph{Teeteto} 176b: \emph{phygè dè homoíosis theôi kata to dynaton.
  homoíosis dè díkaion kai hósion metà phronéseos.}}, eleva a cabeça do
auriga até o lugar exterior (\emph{tòn éxô tópon}) e acompanha a volta
circular, perturbadas pelos cavalos e com muita fadiga veem do alto os
seres (\emph{kathorôsa ta ónta}). Já a {[}alma{]} que ora se eleva, ora
mergulha, tendo forçado os cavalos, algumas coisas vê, outras não.
Outras {[}almas{]} ainda, apegando-se a tudo o que é do alto (\emph{toû
ano épontai}) são incapazes (\emph{adynatoûsai}) de ter êxito, mas
seguem submersas na volta circular, em pisoteio e confronto mútuos,
tentando se adiantarem frente as outras. {[}248b{]} Então ocorre
tumulto, luta e suor extremos, é quando muitas almas claudicam pela
maldade do auriga (\emph{kakíai henióchôn}) e destroçam suas asas. Todas
estas, tendo muita fadiga, sem chegarem à contemplação do ser (\emph{toû
óntos théas}), afastam-se e servem-se do alimento da opinião
(\emph{trophêi doxastêi chrôntai}).\footnote{Como de costume na
  filosofia platônica, a opinião (\emph{dóxa}) serve de alimento às
  almas que não contemplaram o ser. No primeiro discurso de Sócrates a
  opinião foi valorizada como guardiã da prudência. (\emph{Phdr}. 237d8,
  237e2 e 238b8)} Eis o grande empenho que há para ser capaz de ver a
planície da verdade onde ela está (\emph{tò alêtheías ideîn pedíon oû
estin}), pois o pasto que convém ao melhor da alma provém desse prado e
a natureza do alado (\emph{pteroû phýsis}), que eleva a alma, {[}248c{]}
ali é alimentada.

Eis a lei de Adrasteia: A alma que tenha se tornado acompanhante do deus
(\emph{theôi sunopadòs}) e que tenha visto algo das verdades
(\emph{katídêi ti tôn alêthôn}) fica ilesa (\emph{apḗmona}) até o outro
percurso, e se puder fazer isso sempre, fica sempre intacta
(\emph{ablabê}). Quando não lhe é possível gerir-se, não se vale da
visão nem do sucesso, e, ao aplicar muito peso, perde as asas,
despencando por terra em função do fardo do esquecimento (\emph{léthês})
e da maldade. Então é lei, na primeira geração, não nascer em nenhuma
natureza de fera {[}248d{]}. Os que viram o máximo do gênero humano
tornar-se-ão filósofos, amigos do belo, músicos ou algum dentre os
eróticos. Em segundo lugar estão o rei na lei, o guerreiro ou o
comandante; no terceiro lugar um político, economista ou administrador;
na quarta posição um amigo das fadigas, da ginástica ou alguém para
curar o corpo; na quinta um adivinho (\emph{mantikón}) ou alguém que
pode cuidar das iniciações (\emph{telestikòn}); {[}248e{]} na sexta um
poeta, alguém que se ocupa da mimese ou outras {[}atividades{]}
concordes; na sétima um demiurgo ou homem do campo; na oitava um sofista
ou aquele que fere o povo (\emph{demokopikós})\footnote{\emph{Kóptô}:
  golpear, ferir, abater, derrubar, devastar, assolar, forjar.
  \emph{Kopé}: incisão, corte, golpe, ferida. \emph{Demokopikós},
  literalmente ``aquele que lesa (fere) o povo'', termo que designa algo
  muito próximo de ``demagogo'', palavra escolhida por quase todos os
  tradutores na falta de outro termo similar, mesmo que ferir
  (\emph{kóptô}) seja conceitualmente bastante diferente de conduzir
  (\emph{ágô}).}; e na nona um tirano.

Em todas elas, os que se conduzem com justiça tomam o melhor destino
(\emph{moíras}), os que o fazem injustamente, o pior.\footnote{Essa
  passagem é importante depois dessa apresentação das possibilidades ou
  tipos de almas, a partir do que fica claro que elas não são exatamente
  estanques, com um destino inexorável, mas que cabe, a cada uma dessas
  almas, realizar um esforço para alçar um novo lugar nessa escala,
  nessa e na próxima vida, pois estão intercaladas, ou seja, cada uma
  delas é parcialmente responsável pela consecutiva ascensão ou queda
  nessa escala.} Cada uma das almas não chega ao mesmo ponto de onde
saiu antes de dez mil anos, pois não criam asas antes desse tempo,
exceto aquela que foi, de maneira honrada, amante do saber
(\emph{philosophḗsantos}) ou {[}249a{]} amante dos jovens de acordo com
a filosofia (\emph{paiderastḗsantos metà philosophías}).\footnote{Esse
  amor à sabedoria e aos jovens segundo a filosofia se opõe
  diametralmente à proposta de Lísias, quando dizia para não agradar a
  um amante (\emph{erástes}) tomado pelo amor, ou seja, não buscar um
  tutor cidadão, mas agradar a um tutor que não participava dessa
  instituição tradicional da educação ateniense. Esse é de fato um dos
  pontos centrais da proposição de Lísias quando diz que não se deve
  agradar um amante, pois esse amante ao qual se opõe veementemente é um
  tutor ateniense.} Estas, na terceira volta de mil anos, se conduziram
este tipo de vida por três vezes seguidas, no terceiro milênio se
afastam de modo alado. Quanto as outras, há as que ao término da
primeira vida ocorre uma separação (\emph{kríseôs étychon}) e há
julgamento no tribunal subterrâneo, no qual elas prestam contas, ao
passo que há também as que chegam a algum lugar do céu, elevam-se pela
justiça e são levadas à dignidade da vida humana que tiveram. {[}249b{]}
Tanto umas como outras, no milésimo ano, sorteiam (\emph{klḗrôsín}) e
escolhem (\emph{aíresin}) a próxima vida, sendo que cada uma escolhe
(\emph{airoûntai}) a que quiser.\footnote{Com relação aos destinos das
  almas e às suas possibilidades de escolha, é interessante,
  conjuntamente com esse trecho, retomar os passos finais da
  \emph{República}, em que há alegoria de Er, o qual descreve através de
  uma alegoria cósmica e escatológica esse processo de sorteio
  (\emph{klérous} 617e4, \emph{klérou} 619d6, \emph{klêros} 619e1)
  combinado com o processo de escolha (\emph{aireîsthai} 618c5,
  \emph{diairomúmena} 618c6, \emph{aireîsthai} 618d6, \emph{aíresis}
  618e3, \emph{aireîsthai} 619a4, além de 619c2, 619d4, 619d6, 620a1,
  a3, a4 e c3 ) pelo que as almas passam, junto às Moiras (Parcas),
  Láquesis, Cloto e Átropo. Nas duas passagens o destino não é absoluto,
  havendo uma combinação dele com uma dimensão de arbítrio para cada uma
  das almas.} Ali mesmo, os homens que foram feras serão novamente
(\emph{pálin}) homens, e se a alma não atingir tal figura
(\emph{schêma}) é por não ter visto a verdade (\emph{mḗpote}
\emph{idoûsa tḕn alḗtheian}).

É necessário ao homem atingir (\emph{suniénai}) a ideia (\emph{eîdos})
que vai do múltiplo sensível ao uno, tomado conjuntamente pelo
raciocínio (\emph{logismôi}). {[}249c{]} Isso é a reminiscência
(\emph{anámnêsis}) daquilo que nossa alma viu, atravessando com o deus,
vendo além do que agora nos é dito e levantando a cabeça
(\emph{anakúpsasa})\footnote{cf. \emph{anakúptôn}, ``levantar a cabeça''
  na \emph{República} 529b1, dentro do mesmo vocabulário da contemplação
  de padrões invisíveis, \emph{aóraton} (\emph{Rep}. 529b5), além de no
  \emph{Fédon} as seguintes ocorrências, \emph{anakúptontes, anakúpsas}
  e \emph{anakúpsanta,} ressoarem esse mesmo verbo (\emph{Phaed}.
  109d-e).} para o verdadeiro ser (\emph{tò ón óntôs}). É por isso que,
justamente, só cria asas o pensamento (\emph{diánoia}) do filósofo, para
o qual sempre há, na medida do possível, memória (\emph{mnḗmêi}), e para
o qual os deuses são divinos. Homens de tal valor servem-se corretamente
da recordação (\emph{hypomnḗmasin})\footnote{Servir-se das recordações
  corretamente, como veremos, será importante na alegoria final de
  Thamous e Theuth, 274c-275b, momento em que essas memórias
  (\emph{mnéme}) internas serão superiores aos apoios à memória
  (\emph{hypomnéma}), que são sempre externos.}, sempre se iniciam
corretamente em mistérios e tornam-se os únicos perfeitos. Mudam a
dignidade dos homens ao tornarem-se próximos aos deuses (\emph{pròs tôi
theôi gignómenos}), e são advertidos por muitos que ao seu lado se
moviam. Pelo seu entusiasmo {[}249d{]} (\emph{enthousiádzôn}) eles são
esquecidos (\emph{lélêthen}) pela maioria.

Até aqui temos o discurso todo a respeito da quarta loucura, quando {[}a
alma{]} vê alguma dessas belezas, rememorando o verdadeiro
(\emph{alêthoûs anamimneiskómenos}) e tem as asas crescidas, momento em
que a alma está alada e cheia de disposição, entretanto, quando não pode
voar, ela parece um pássaro que vê o que há acima dele, mas descuida do
que está embaixo, por isso é vista como uma alma louca (\emph{manikôs}).
{[}249e{]} Essa é a melhor das coisas entusiásticas
(\emph{enthousiáseôn}) e provém das melhores, quem dela vier a ser
possuidor ou dela participar é chamado de amante do belo (\emph{kalôn
erastès}), porque aquele que ama é partícipe (\emph{metéchôn}) da
loucura (\emph{manías}). De acordo com o que foi dito, é da natureza de
toda alma humana ter contemplado os seres (\emph{tetheáthai tà ónta}),
ou não chegariam a essa vida.

{[}250a{]} Relembrar (\emph{anamimnḗiskesthai}) aquilo, a partir destas
coisas, não é fácil para todas as almas, nem para aquelas que tiveram
uma breve visão, nem para as que caíram, infortunadas dirigidas pela
injustiça da multidão, almas que esqueceram a visão sagrada
(\emph{léthên ôn tóte eîdon hierôn}) que outrora tiveram. Poucas
(\emph{olígai}) são as {[}almas{]} deixadas com suficiente memória
(\emph{mnḗmês} \emph{ikanôs}). Estas mesmas, quando têm visão de algo
semelhante (\emph{homoíoma ídôsin}), ficam fora de si
(\emph{ekplḗttontai}) e de nenhum modo voltam a si. Outras ignoram a
afecção (\emph{tò páthos}) por não a ter percebido com força suficiente
(\emph{mè ikanôs} \emph{diaisthánesthai}).

{[}250b{]} Justiça, prudência (\emph{dikaiosýnês mèn oûn kaì
sôphrosýnês}) e tantas outras preciosidades da alma não resplandecem
(\emph{phéggous})\footnote{Cf. Plotino, \emph{Acerca do Belo}, 4.9
  \emph{phéggous}, 5.11 \emph{phéggos}.} em nenhuma das semelhanças
(\emph{homoiómasin}) daqui, mas poucos, através de órgãos obscuros, com
fadiga, contemplam o gênero da similitude (\emph{theôntai to toû
eikasthéntos génos}) partindo dos ícones (\emph{epí tas eikónas
ióntes}).

Era de se ver a luminosa beleza quando, outrora, juntamente com o feliz
cortejo, {[}as almas{]} visualizavam e contemplavam a bem-aventurança
(\emph{makarían} \emph{ópsin te kaì théan}). Nós somos seguidores de
Zeus, outros seguem outros deuses. Vimos e nos completávamos nas
iniciações que, com justiça, são ditas as mais felizes que celebramos,
íntegros e indiferentes aos males que nos surgiriam em tempos
posteriores. Nas completas, simples, {[}250c{]} calmas e felizes
aparições estávamos iniciados e havíamos chegado à essa
revelação\footnote{Aqui a palavra revelação não refere o sentido comum
  de visão, como \emph{ópsis}, \emph{ommata}, mas está inserida no
  vocabulário das iniciações, ligado ao mais alto grau da iniciação dos
  mistérios de Eleusis (\emph{epopteúô, epoptikós, tà epoptiká}), daí
  que \emph{epopteúontes} tem o sentido de revelação ou visão sagrada.
  Cf. Schefer, 2001 e De Vries 1969, p.149-151.} pela mais pura luz
(\emph{epopteúontes en augêi katharâi}), estando puros (\emph{katharoì
óntes}), pois não havia a marca que nos é trazida pelo que agora
chamamos corpo, motivo pelo qual permanecemos atados {[}a ele{]} como
uma ostra {[}a sua concha{]}. Que estas coisas sejam agraciadas
(\emph{kecharísthô}) pela memória (\emph{mnḗmêi}),\footnote{Essa dupla,
  Graça e Memória, reaparece em 267a5 quando Sócrates menciona uma das
  invenções retóricas de Eveno de Paros, o para-vitupério em versos para
  ``auxiliar (agradar) a memória (\emph{mnémes chárin})''. É preciso
  ainda mencionar que o verbo da tese chave de Lísias, tese que aliás
  conduz o diálogo, é \emph{charídzesthai} (agradar, agraciar), assim
  como o desfecho do diálogo será um discernimento dialético acerca do
  valor da memória verdadeira.} a qual pela ausência de outrora foi
agora longamente enunciada.

Sobre a beleza, como dissemos, sendo em cada um de nós luminosa,
chegamos aqui {[}250d{]} tomando-a com a máxima clareza de nossos
sentidos, com o mais radiante brilho. A visão (\emph{ópsis}) é a mais
aguda das sensações que nos chega pelo corpo, mas por ela a prudência
(\emph{phrónêsis}) não é vista.\footnote{A prudência mostra a
  necessidade da cautela do sábio com relação ao visível, como em muitos
  casos ocorre em Platão, entretanto cabe aqui indicar também o papel de
  visão no \emph{Timeu} 47a-c, passagem em que a visão (\emph{ópsis,
  ommátôn}) é louvada como o maior dos bens (\emph{mégiston agathón}),
  sem o qual nem mesmo a filosofia existiria, uma vez que ela surgiu
  dessa observação do intelecto nos astros celestes. Se por um lado a
  visão pode ser perigosa, seja por poder levar ao engano, como outros
  sentidos, seja pelos deleites que a beleza quando vista proporciona,
  por outro lado é também fonte de conhecimento verdadeiro, de modo que
  Platão propõe, como em outros casos, uma aplicação cuidadosa desse
  recurso valioso, para que não caiamos nem nos deleites incomensuráveis
  daquilo que é belo pela visão, não negligenciemos essa ferramenta
  valiosa de conhecimento e tampouco acreditemos inadvertidamente em
  tudo aquilo que acolhemos pela visão. Nesse sentido, encontramos nessa
  passagem tanto o Platão crítico com relação às artes miméticas, o qual
  é bastante conhecido, além do Platão que propõe uma reforma
  epistemológica, naquilo que está além da visão, propondo paradigmas
  novos que sejam capazes de salvar os fenômenos, sem desprezar os
  fenômenos, ou seja, que expliquem o céu do ponto de vista matemático e
  geométrico, especialmente as anomalias dos movimentos planetários.}
Cairíamos em terríveis amores, se algum ídolo (\emph{eídôlon}) de tal
classe, por sua própria evidência, fosse enviado e desejado pela visão
(\emph{ópsin}), assim como tantas outras coisas amáveis. Só a beleza
(\emph{kállos} \emph{mónon}) teve este destino (\emph{moîran}), ser a
mais evidente (\emph{ekphanéstaton}) e a mais amada
(\emph{erasmiṓtaton}).

{[}250e{]} Um recém-iniciado ou alguém que veio a corromper-se não é
rapidamente trazido daqui para lá, para a beleza mesma (\emph{pròs autò
tò kállos}), contemplando o mesmo que aqui leva seu nome. Não venera ao
olhar, mas entregue ao prazer (\emph{hêdonêi}), põe-se a andar na lei de
um quadrúpede, produz filhos e, familiarizado com a desmesura
(\emph{hýbrei}), não teme nem se envergonha, perseguindo um prazer
contrário à natureza (\emph{parà phýsin hêdonḕn}).

{[}251a{]} O recém-iniciado (\emph{artitelḗs}) que contemplou muitas
coisas (\emph{polytheámon}), quando vê um rosto de forma divina
(\emph{theoeidés}), bem imitando o belo (\emph{kállos eû memimêménon})
ou alguma forma corpórea (\emph{sṓmatos idéan}), primeiro estremece
(\emph{éphrixe}), enquanto algo dos medos de outrora chega até ele,
depois de ter visto, venera-o como a um deus e, se não temesse a fama de
uma excessiva loucura (\emph{sphódra manías dóxan}), sacrificaria
(\emph{thúoi}) ao predileto (\emph{paidikoîs}) como a uma imagem e como
a um deus (\emph{agálmati kai theôi}). Depois dessa visão, surge nele
uma mudança do tremor (\emph{tês phríkês} \emph{metabolḗ}), pois um suor
e um calor atípico o toma e o aquece, tendo recebido {[}251b{]} o fluxo
(\emph{aporroḕn}) da beleza pelos olhos (\emph{ommátôn}), motivo pelo
qual a natureza alada é irrigada (\emph{pteroû phýsis ádretai}).
Aquecida a asa, dissolve-se uma natureza que há muito tempo não
germinava, por endurecimento, aprisionamento e clausura conjunta.
Túrgido de alimento, o caule da asa incha e começa a brotar da raiz em
todas as formas de almas, pois toda {[}alma{]} era anteriormente alada
(\emph{pterôtḗ}). Nesse momento, ela ferve toda e irrita-se
(\emph{anakêkíei}), {[}251c{]} como quando sofremos pelo surgimento dos
dentes, que recém-saídos raspam e irritam por toda a passagem, e o mesmo
sofre a alma no começo do crescer das asas (\emph{pterophyeîn}), ferve e
irrita-se com as cócegas provocadas pelo brotar das asas (\emph{phýousa
tà pterá}). Quando, ao olhar para a beleza do amado, e dele recebendo
parte do fluxo que sobrevém -- o qual precisamente é chamado de desejo
(\emph{hímeros})\footnote{Aqui \emph{hímeros} é o termo que designará
  esse desejo lícito, iniciático, ligado a Zeus, mais uma vez demarcado
  o ``terreno'' sacro desse discurso. Logo abaixo se confirma que é o
  mesmo desejo (\emph{hímeros}) capaz de irrigar as asas da alma, de
  fazer com que elas, antes secas, voltem a sua natureza inicial e
  cheguem, novamente, à sua potencialidade máxima.} --, é irrigada e
aquecida, recompondo-se da dor ela fica alegre. {[}251d{]} Quando ficam
separadas e áridas, as vias que deságuam onde crescem (\emph{hormâi}) as
asas, ficam secas, fechando e obstruindo o germinar das asas, as quais,
em seu interior, após terem sido fechadas ao fluxo do desejo
(\emph{himérou}), ficam agitadas, arranhando cada uma das vias de saída,
justamente porque a alma enfurece todas as feridas ao redor, causando
dor.

Por outro lado, {[}a alma{]} alegra-se tendo a memória (\emph{mnḗmen})
da beleza. Nessa mistura de ambos ela se atormenta pela estranheza da
afecção (\emph{atopíai toû páthous}), não conseguindo saída pela fúria,
e enlouquecida (\emph{emmanès}), nem a noite pode dormir, nem durante o
dia permanece num só lugar. {[}251e{]} Corre (\emph{theî}) ansiosa para
onde considera ver (\emph{ópsesthai}) o possuidor da beleza
(\emph{kállos}). Tendo visto e canalizado o desejo (\emph{hímeron}),
libera o que até então estava conjuntamente obstruído, tomando fôlego,
tendo apaziguado as picadas e dores. Este é o dulcíssimo prazer
(\emph{hedonḕn}) de que, no momento, ela desfruta. Por isso não é
voluntariamente afastada e a ninguém mais atende senão ao belo,
{[}252a{]} esquece de todos: mãe, irmãos, companheiros. Sendo arruinada
pela negligência, não realiza nada e, quanto aos hábitos e conveniências
com as quais antes se embelezava, a todos passa a desprezar, pronta a
escravizar-se e deleitar-se onde lhe permitam, o mais próximo possível
de seu desejo (\emph{póthou}). Além de adorar aquele que porta a beleza,
nele encontra único médico para os seus maiores sofrimentos
(\emph{iatrón hêurêke mónon tôn megístôn pónôn}).

{[}252b{]} Essa afecção (\emph{tò páthos}), ó bela criança a quem se
dirige meu discurso (\emph{lógos}), os homens denominam Amor
(\emph{Erôta}), mas se ouvires como os deuses o designam tu
possivelmente rirás, por conta de sua juventude. Alguns homéridas,
segundo penso, em dois de seus versos secretos (\emph{apothéton epôn})
falam o seguinte sobre o Amor (\emph{Érôta}) -- o segundo verso é
excessivo (\emph{hybristikòn}) e não precisamente na métrica --, eles
cantam assim:

\begin{quote}
os mortais o designam por Eros (\emph{Érôta}) alado (\emph{potênón}),

e os imortais por \emph{Ptérôta} (Alado), pela força do brotar das asas
(\emph{pterophýtor}).\footnote{A designação dos homens para o Amor,
  ``Érota'', está inserida na palavra que designa o Amor na perspectiva
  dos deuses, ``Pt\emph{\emph{érota}}'', além de todo o campo semântico
  dos versos estar marcado pelas asas de Eros, especialmente em
  \emph{potenón}, \emph{Ptérota}, \emph{pterophýtor}. Em 251c já
  havíamos encontrado expressões similares como \emph{pterophyeîn} e
  \emph{phýousa tà pterá}, bem como em 252c encontramos
  \emph{pteronúmou}, como epíteto de Zeus.}
\end{quote}

É possível ser persuadido por estes homens, assim como é possível não
ser. Apesar disso, a causa (\emph{aitía}) e a afecção (\emph{to páthos})
para aqueles que amam (\emph{tôn eróntôn}) são estas mesmas. {[}252c{]}
Dentre os acompanhantes, o que foi tomado com mais força pode carregar o
fardo de Zeus, aquele que é denominado como alado (\emph{pterônúmou}).
Quanto aos que foram servidores de Ares e com este circularam, quando
estão tomados por Eros e consideram que foram injustiçados pelo amado,
prontificam-se ao homicídio, querem sacrificar a si mesmos e aos seus
prediletos (\emph{tà paidiká}). E assim, cada qual sendo coreuta para
cada deus, honra-o e imita-o (\emph{mimoúmenos}) na vida o quanto podem.
{[}252d{]} Durante algum tempo, por não se corromperem, vivem aqui nesta
primeira geração, sendo levados a reunirem-se aos amantes e a outros por
esse modo. Então, cada um elege (\emph{eklégetai}) a sua maneira no que
concerne ao Amor dos belos, e, sendo aquele amado como um deus
(\emph{theòn}), {[}os amantes{]} a si mesmos fabricam
(\emph{tektaínetaí}) e adornam (\emph{katakosmeî}) uma imagem
(\emph{ágalma}), para lhes honrar e celebrar (\emph{timḕsôn te kaì
orgiásôn}). {[}252e{]} Os que acompanham alguém tal qual o divino Zeus,
buscam que a alma do seu amado seja similar a dele, daí observam se sua
natureza (\emph{phýsin}) é de filósofo (\emph{philósophós}) ou de
comandante (\emph{hêgemonikòs}), tornam-se amantes dele quando o
encontram (\emph{heuróntes}) e tudo fazem para que permaneça
assim.\footnote{Reitera-se nessa passagem que o filósofo é a categoria
  ``psíquica'' máxima entre os eróticos (248d2), bem como o vocabulário
  do comando de Zeus (\emph{mégas hegemôn en ouranôi} \emph{Zeús}
  246e4), de modo que se aproximam Zeus e o filósofo, como seu seguidor,
  alguém que o imita ou deve imitá-lo. De Vries indica que a expressão
  \emph{diôn}, ``celestial'', ``divino'', em algumas edições é grafada
  com maiúscula (\emph{Diôn}, \emph{Díion}) e que isso talvez fosse uma
  maneira de referir Dion de Siracusa.} Caso anteriormente eles não
tenham se empenhado nestas ocupações, logo atentamente aprendem, a
partir de onde for possível, se acercam dos seus amados e como
rastreadores eles mesmos descobrem (\emph{aneurískein}) a natureza do
deus que lhes é próprio, prosperam assim através do severo esforço em
olhar para o deus (\emph{pròs tòn theòn blépein}).\footnote{Mais uma vez
  a visão é importante no processo de assimilação com a imagem/ideia do
  deus. Nesse caso, o esforço do amante seria observar e reconhecer as
  similitudes entre o deus e o amado, uma habilidade dialética.}
{[}253a{]} Sendo eles apoderados pela memória (\emph{têi mnémei}),
tomam, em entusiasmo (\emph{enthousiôntes}), os hábitos (\emph{tà éthê})
e ocupações a partir daquele {[}deus{]}, tanto quanto é possível um
homem partilhar (\emph{matascheîn}) da divindade.\footnote{A visão é uma
  força motriz capaz de ativar a memória dos padrões, deuses e formas
  anteriormente conhecidas pela alma, de modo que a visão (\emph{ópsis,
  ommátôn}) é também veículo para a revelação (\emph{epoptia}).}

E essa é a causa de tanto amarem (\emph{agapôsin}) seus amantes, tanto
os que pertencem a Zeus, como as bacantes que atingem a alma do amado e
adotam a máxima semelhança (\emph{homoiótaton}) com relação ao seu deus
{[}Dioniso{]}, {[}253b{]} como os que são seguidores de Hera e buscam
{[}um amado de{]} natureza real (\emph{basilikòn}), tendo-o encontrado
(\emph{heuróntes}) fazem com ele tudo do mesmo modo. {[}253b{]} Os que
são de Apolo, bem como de cada um dos deuses, avançando com o deus,
buscam que seu amado seja de tal natureza, e, depois de o enredarem,
persuadem (\emph{peíthontes}) e disciplinam (\emph{rhythmídzontes}) o
predileto (\emph{tà paidiká}) a imitá-lo (\emph{mimoúmenoi}),
conduzem-no assim à ocupação e ao aspecto (\emph{idéian}) daquele
{[}deus{]} tanto quanto é possível a cada um, não por inveja ou
mesquinha hostilidade para com o predileto (\emph{tà paidiká}), mas
tentando, em tudo, levá-lo a maior semelhança (\emph{homoióteta})
possível consigo mesmo e, portanto, com o deus honrado, assim eles
fazem. {[}253c{]} A boa vontade (\emph{prothymía}) e a iniciação
(\emph{teletḗ}) dos verdadeiros amorosos (\emph{alêthôs erṓntôn}), caso
realizem essa benevolência que digo, é bela e feliz na loucura amorosa
do amante para com o amado, se ele foi mesmo arrebatado pelo amor. O
eleito é dessa maneira tomado.\footnote{Aqui, depois de uma sucessão de
  períodos especialmente longos, entre 251b e 253c, para os quais muitas
  vezes foi necessária uma pontuação artificial, que tornasse mais
  palatável e mais claro o que Sócrates diz, temos uma espécie de
  quebra, a partir da qual não mais se falará da mimese que o amante
  propõe, ao cultuar o amado, depois de capturado, tal qual a um deus. A
  partir daqui, de 253c7, Sócrates retoma a analogia da biga alada e da
  tripartição da alma, e somente em 225a, depois de reapresentada a
  alegoria da biga, retorna à relação entre amante e amado.}

Desde o começo dessa narrativa (\emph{mýthou}) dividimos a alma em três
partes, duas delas na forma de cavalos e a terceira na forma do auriga.
Então, agora, vamos manter isso. {[}253d{]} Entre os cavalos, dissemos
que um é bom e o outro não, mas não explicamos ainda qual é a virtude do
bom e o vício do mau. Façamos isso agora. Um deles tem uma bela postura,
uma forma correta e articulada, altivo, nariz adunco, branco, olhos
negros, amante da honra (\emph{timês erastḕs}) de acordo com a
temperança e o pudor (\emph{sôphosýnês te kaì aidoûs}).\footnote{Na
  analogia entre discursos e partes da alma, especialmente sugerida por
  Hermias (§12.6-10; §11.4-7 Couvreur), fica evidente que esse cavalo
  branco representa a parte irascível ou \emph{thymoeides}, descrita
  também na \emph{República} (580d-581c), lugar da amizade pela vitória
  (\emph{philónikon}) e das honras (\emph{philótimon}), ao mesmo tempo
  que tem afinidade com o primeiro discurso de Sócrates (237a7--241d). A
  manutenção dessas tendências domadas constitui a imagem do homem
  prudente (\emph{sôphrosúne} 237e2), aquele que domina seus desejos e
  prazeres, aquele que se controla de um modo ``racional''. Daí Hermias
  ter identificado essa recitação primeira de Sócrates como ligada à
  prudência (\emph{sóphrona}).} Ele é companheiro da opinião verdadeira
(\emph{alêthinês dóxês hetaîros}), ele não insulta, obedece a um só e é
conduzido pela palavra (\emph{lógoi}). {[}253e{]} Já o outro é oblíquo,
vulgar, levado ao acaso, tem pescoço forte e curto, nariz achatado, é
negro, tem olhos acinzentados, é sanguíneo, companheiro da desmesura
(\emph{hýbreôs}) e da jactância (\emph{aladzoneías}), orelhas peludas,
surdo (\emph{kôphós}), e só obedece com dificuldade ao açoite e ao
aguilhão. Então, quando o auriga vê o olhar do amante (\emph{tò erôtikòn
ómma}), sente toda alma aquecer-se, enchendo-se de prurido e dos
aguilhões do desejo (\emph{póthou}). {[}254a{]} Então, o cavalo que é
bem persuadido pelo auriga e sempre constrangido pela força do pudor
(\emph{aidoî}), permanece sob seu próprio domínio e não é levado para a
direção do amado. Já o outro, nem pelo aguilhão do auriga nem pelo
açoite recua, saltando e sendo conduzido pela força. Esse oferece todo
tipo de apuros ao companheiro de jugo e ao auriga, forçando-os na
direção do predileto, fazendo com que ele rememore (\emph{mneían})
gracejos afrodisíacos (\emph{aphrodisíôn cháritos}).

Ambos, desde o começo, opõem-se de modo irritadiço, {[}254b{]} uma vez
que são forçados a coisas terríveis e violentas. Mas ao final, quando
nem mesmo conseguem evitar a maldade, atravessam e seguem, agindo como
se concordassem (\emph{homologḗsante}) em fazer o que lhes foi ordenado.
Ao chegarem diante do amado e observarem a face (\emph{ópsin}) luminosa
do predileto, a memória (\emph{hê mnḗmê}) do auriga, pela visão, é
levada à natureza do belo, momento em que novamente vê (\emph{pálin
eîden}) aquela beleza, de acordo com a prudência (\emph{sôphrosýnês}),
estabelecida num sagrado pedestal.

Ao vê-lo, ele teme e o sente vergonha, a ponto de cair de costas, e ao
mesmo tempo o constrange ao puxar as rédeas com tanta força que ambos os
cavalos se assentam sobre os próprios quadris {[}254c{]}, um por vontade
própria, sem oferecer oposição, mas o outro, rebelde (\emph{hubrístḕn}),
faz isso muito a contragosto. Chegando a um lugar mais afastado, um por
estar com vergonha e estupefato, banha toda a alma com suor, o outro,
estando apaziguado da dor, causada pelo freio e pela queda, toma fôlego
e, com ímpeto, vitupera os muitos abusos do auriga e do companheiro de
jugo, como se por timidez ou covardia eles houvessem abandonado a ordem
(\emph{táxin}) e o acordo (\emph{homologían}).\footnote{Aqui há uma
  descrição importante de uma ocasião em que esse modelo tripartido da
  alma se desequilibra, e o cavalo irascível (\emph{thymoeides}) quem
  reclama do descontrole propiciado pelo auriga e pelo cavalo apetitivo
  em mútuo excesso, o que levou ao descontrole total da biga. O auriga
  cede, às vezes, aos desejos do cavalo apetitivo, até que o
  \emph{thymoeides} volte ao equilíbrio.} {[}254d{]} E novamente
(\emph{pálin}), não desejando ser conduzido à força, às duras penas ele,
{[}o cavalo negro{]}, aceita o que lhe foi demandado, adiando sua nova
investida (\emph{hyperbalésthai}). Chegado o tempo determinado, como se
estivesse esquecido (\emph{amnêmoneîn}), ele, {[}o cavalo negro{]}, é
levado à rememoração (\emph{anamimnḗiskôn}), e usando toda sua energia,
relinchando, puxa fortemente para o lado contrário, arrastando
(\emph{hélkôn}) para onde está o favorito e oferecendo-lhe os mesmos
discursos (\emph{lógous}). Logo em seguida, quando ele se aproxima,
agacha e estica a calda, morde o freio e arrasta (\emph{hélkei}) sem
nenhum pudor (\emph{anaideías}).

{[}254e{]} O auriga sofre a maior dessas afecções (\emph{páthos
pathón}), como se estivesse impedido por uma corda, uma vez que há a
desmesura (\emph{hybristoû}) do cavalo que é arrastado com força pelo
freio dos dentes, tendo a língua maledicente e a mandíbula
ensanguentadas, aí então suas patas e sua anca são lançadas à terra e
expostas ao sofrimento. Quando esse malvado sofre todas essas coisas,
cessa sua desmesura (\emph{hýbreôs}),\footnote{Cf. simetria em
  expressões ligadas ao ``cessar do desejo'': \emph{pauómenoi tês
  epithymías\textbf{. }}Cf. nota 52.} submetendo-se à condução e à
intenção (\emph{pronoíai}) do auriga, e quando vê o belo {[}novamente{]}
ele é aniquilado pelo medo (\emph{phóbôi}). A partir daí, ocorre que a
alma do amante (\emph{erastoû}) passa a seguir o predileto com pudor e
respeito (\emph{aidouménên te kaì dediuîan}). {[}255a{]} Por conta de
todos os cuidados dispensados pelo amante, que eram similares aos
dispensados a um deus, não por que o amante fingia, mas por sentir-se
assim de verdade, fizeram com que o próprio amado, naturalmente, se
tornasse amigo do seu servidor, mesmo que anteriormente ele tivesse sido
reprovado pelos companheiros ou por quaisquer outros que diziam ser
vergonhoso associar-se a um amante\footnote{Essa atitude de rechaço ao
  amante era a base da tese anteriormente defendida, mas agora essa tese
  aqui já aparece refutada.}, e que por isso tivesse repelido o amante,
mas com o passar do tempo, {[}255b{]} a idade e a necessidade fazem com
que aquele seja aceito em sua companhia. Não quis o destino nem que o
malvado fosse amigo do malvado, nem que o bondoso não fosse amigo do
bondoso. Tendo oferecido o discurso e o recebido em sua companhia, a
proximidade e a benevolência do amante provoca no amado a sensação de
perturbação, uma vez que nem outros amigos, nem familiares, ninguém
frente ao amigo entusiasmado (\emph{éntheon})\footnote{Hermias
  classifica a palinódia de Sócrates como um discurso entusiástico (cf.
  \emph{In Platonis} Couvreur §64.6 {[}p.11{]} ἐνθουσιαστικοῦ, §64.9
  {[}p.12{]} ἔνθουν).} oferece parcela alguma de amizade. E quando por
muito tempo age assim, aproximando-se dele para tocá-lo nos ginásios e
em outras ocasiões, {[}255c{]} aí então {[}surge{]} a fonte
(\emph{pêgḗ}) daquele fluxo (\emph{rheúmatos}), o qual Zeus, amante de
Ganimedes, denominou de desejo (\emph{hímeron}), que chega em abundância
no amante, preenchendo-lhe e, uma vez preenchido, transborda para o
exterior. Tal qual um sopro (\emph{pneûma}) ou algum eco (\emph{êchô})
que numa superfície lisa ou sólida é levado novamente (\emph{pálin}) ao
ponto de partida, assim o fluxo da beleza é novamente (\emph{pálin})
direcionado ao belo, através dos olhos (\emph{ommátôn}), por onde a alma
é acessada e tem as asas acrescidas (\emph{anapterôsan}). {[}255d{]}
Então, as vias das asas são irrigadas (\emph{pterôn árdei}), iniciando o
seu brotar (\emph{pterophyeîn}), enquanto o amor preenche (\emph{érôtos
enéplêsen}) a alma do amado (\emph{erôménou}). Ele ama, mas não sabe o
quê. Não sabe o que sofre e não tem como expressar isso. Tal qual uma
oftalmia (\emph{ophthalmías})\footnote{A oftalmia aqui reaviva a tópica
  do poeta Estesícoro, curado da cegueira pela \emph{Palinódia}.}
adquirida de outrem, ele não tem como expressar a causa, uma vez que lhe
escapa (\emph{lélêthen}) que vê a si mesmo no seu amante, como se fosse
em um espelho (\emph{katóptrôi}). E quando está junto dele, cessa o seu
sofrimento, tal qual no amante, mas quando está separado, ele deseja e é
também desejado (\emph{potheî kaì potheîtai}), pois adquire um ídolo do
amor, um Ânteros (\emph{eídôlon érôtos antérôta échôn}).\footnote{\emph{Eidôlon}
  pode ser traduzido também por ``imagem'', ou, no caso, por ``imagem
  reflexa''. Ânteros, irmão de Eros, é conhecido como ``Amor
  contrário'', ``recíproco'' ou ``vingativo''. Algumas vezes é entendido
  como aquele que pune quem não retribui o amor de outrem. De acordo com
  Pausânias havia uma ligação entre Ânteros e os metecos de Atenas:
  ``Antes da entrada da Academia havia um altar a Eros com um epigrama
  de Carmos, que foi o primeiro ateniense a dedicar um altar a esse
  deus. Dizem que o altar é de fato para Ânteros e dedicado aos
  residentes metecos da cidade, porque o ateniense Meles, rejeitando o
  amor de Timágoras, um residente meteco, o exortou a subir em uma pedra
  e se jogar de lá de cima. Timágoras, sem cuidado com a sua alma, e
  disposto a agradar o seu predileto em tudo que ele dissesse,
  efetivamente se atirou. Meles, quando viu que Timágoras havia morrido,
  mudou seu pensamento e, com remorso, se atirou também da mesma pedra e
  morreu. Por isso os metecos cultuam o espírito de Timágoras a partir
  desse \emph{daimon} Ânteros'' (Pausânias, \emph{Descrição da Grécia},
  1,30,1, Leipzig: Teubner, 1967).} {[}255e{]} A este nomeia e considera
não como amor (\emph{érota}), mas como amizade (\emph{philían}). O seu
desejo (\emph{epithymeî}) é quase o mesmo daquele, só que menos intenso,
o de ver, tocar, beijar, deitar-se ao seu lado, ações que, como é
verossímil (\emph{eikós}), não tardará a realizar. Então, quando
partilha o mesmo leito do amante, o cavalo indisciplinado
(\emph{akólastos}) tem algo a dizer ao auriga, esperando depois de todos
os sofrimentos tirar-lhe um pequeno benefício. {[}256a{]} O predileto
não tem nada a dizer e, pleno de desejo, perplexo, acerca-se do amante e
o adora, amigavelmente saudando a quem bem lhe quer, como quem não pode
recusar os gracejos (\emph{charísasthai}) do amante quando está a seu
lado, se por acaso (\emph{tycheîn}) ele demandar. O companheiro de jugo
se opõe a isso, juntamente com o auriga, seguindo seu pudor e sua razão
(\emph{aidoûs kaì lógou}). E se, por acaso, predominarem as melhores
partes do pensamento (\emph{dianoías}), as que conduzem a um regime de
vida ordenado e amante da sabedoria (\emph{philosophían}), {[}256b{]}
são felizes e conduzem uma vida de concórdia, estando eles senhores de
si (\emph{egkrateîs}) e disciplinados (\emph{kósmioi}), subjugam aquilo
que faz nascer a maldade na alma e libertam aquilo que nela gera a
virtude (\emph{areté}). Então, no fim da vida, ganham asas e leveza
(\emph{hypópteroi kaì elaphroì}), pois venceram um dos três combates
verdadeiramente olímpicos, o qual é o maior bem (\emph{agathòn}), não
podendo ser alcançado pelo homem, nem pela prudência
(\emph{sôphrosýnê}), nem pela loucura divina (\emph{theía manía}). Mas
se, pelo contrário, levarem um regime de vida mais vulgar e sem amor
pela sabedoria (\emph{aphilosóphôi}), {[}256c{]} valendo-se do amor pela
honra (\emph{philotímôi}), então, rapidamente, nas bebedeiras
(\emph{méthais}) ou em outras ocasiões de despreocupação
(\emph{ameleíai}), os dois {[}cavalos{]} libertinos (\emph{akolástô}),
sob o mesmo jugo, tomam as almas desprevenidas (\emph{aphroúrous}),
unindo-se ambos para o mesmo fim, escolhendo (\emph{airesin}) o que a
maioria (\emph{tôn pollôn}) toma por excelente e assim
praticam.\footnote{A multidão insensata é agora explicitamente
  detratada, especialmente na sua percepção distorcida acerca da
  felicidade, pois a visão democrática, já bastante desgastada a esta
  altura, já não é acompanhada da ironia do início do diálogo.} Tendo
realizado isso, valem-se desse comportamento em ocasiões futuras, embora
raramente isso ocorra, visto que a praticam sem a aprovação plena da
reflexão (\emph{dianoíai}). Estes são obviamente amigos, mas em menor
grau que os anteriores, {[}256d{]} e, o amor lhes é recíproco e até
mesmo depois dele acabar, (\emph{písteis}) acreditam terem oferecido e
recebido mutuamente as melhores coisas, o que torna ilícito que fiquem
apartados a ponto de serem hostis entre si. No final da vida, sem asas
(\emph{ápteroi}), mas desejosos de as terem adquirido, eles saem do
corpo, e não é pequena a sua recompensa (\emph{âthlon}) advinda dessa
loucura amorosa (\emph{erôtikês manías}). Não há uma lei que designe que
aqueles que iniciam seu trajeto sob o céu devam passar pela travessia
escura e pelo subterrâneo, mas sim que atravessem celebrando entre si
uma vida luminosa (\emph{phanòn bíon}) e feliz (\emph{eudaimoneîn}),
{[}256e{]} e que sejam agraciados pelo amor com asas semelhantes
(\emph{homópteros érôtos chárin}), quando chegar o momento de seu
surgimento. São essas então, ó jovem, as coisas divinas (\emph{theîa})
que lhes são entregues pela amizade para com o amante (\emph{par'
erastoû philía}).\footnote{Ao contrário do que havia sido defendido, de
  que haveria uma certa prudência ao evitar os desejos amorosos, aqui a
  associação dos cavalos libertinos do amante e do amado, nesse
  contexto, aparece como mais louvável, do ponto de vista ético, do que
  a tese anteriormente defendida.}

A familiaridade com o não afetado pelo amor (\emph{mè erôntos}),
mesclada com a prudência (\emph{sôphrosýnêi}) mortal na administração de
bens mortais e miseráveis (\emph{thnêtá te kaì pheidôlà}), é uma
servilidade (\emph{aneleutherían}) elogiada por muitos (\emph{plḗthous
epainouménen}) como uma virtude (\emph{aretḕn}) gerada pela amizade na
alma, {[}257a{]} mas que faz com que ela gire por nove mil anos, ao
redor e debaixo da terra, num percurso sem intelecto
(\emph{ánoun}).\footnote{Aqui a velha tese de Lísias, repetida por
  Sócrates, acerca de agradar o não afetado por Eros, é cabalmente
  refutada, pois aquela prudência e virtude antes defendidas agora não
  são mais sustentadas naqueles termos.}

Esta é, ó querido Eros, dentro das nossas possibilidades, a mais bela e
melhor palinódia que eu poderia oferecer-te como pagamento, entre tantas
outras razões, mas especialmente no vocabulário\footnote{Retomada das
  palavras de Fedro em 234c.} poético a que fui forçado por Fedro.
Desculpe-me (\emph{suggnṓmên}) pelos primeiros {[}discursos{]} e que
este último o tenha agradado (\emph{chárin}), seja para mim benévolo e
propício na arte de amar (\emph{tḕn erôtikḗn téchnên}) que me destes,
que eu não seja dela subtraído nem incapacitado pelo impulso
(\emph{orgén}), e que me seja concedido ser ainda mais honrado
(\emph{tímion}) junto aos belos. {[}257b{]} E, se com os dois primeiros
discursos eu e Fedro fomos dissonantes a ti, o causador foi Lísias, o
pai do discurso (\emph{tòn toû lógoû patéra}). Então, interrompe
(\emph{paûe}) nele tais discursos (\emph{lógôn}) e o conduz
(\emph{trépson}) para a filosofia, como foi conduzido (\emph{tétraptai})
o seu próprio irmão Polemarco\footnote{Polemarco, irmão de Lísias,
  condenado sem julgamento pelos 30 tiranos. Na \emph{República}
  331d-336a Polemarco conversa com Sócrates, herdando a conversa inicial
  que Sócrates tem com Céfalo. \emph{Contra Eratóstenes} (Lysias, 2006,
  p.221-277) parece ser o único discurso que Lísias teria proferido ele
  mesmo como cidadão acusando Eratóstenes de ser o responsável pela
  morte de seu irmão.}, para que este seu amante (\emph{erastès}) aqui
não fique mais entre dois caminhos (\emph{epamphoterídzêi}), justamente
como agora, mas tenha a vida devotada somente para o Amor, de acordo com
discursos filosóficos''.

F: Junto minhas preces às tuas, ó Sócrates, e, se isso for o melhor para
nós, que assim seja. {[}257c{]} O teu discurso há tempo que admiro
(\emph{thaumásas}), tanto mais belo que o anterior o fizeste. Assim,
receio que Lísias me apareça inferior mesmo, especialmente se queres
(\emph{ethelêsêi}) contra ele competir
(\emph{antiparateînai}).\footnote{Fedro reconhece somente aqui a
  superioridade de Sócrates frente a Lísias na composição de discursos.}
Pois é algo assim, ó admirável (\emph{thaumásie}), agora mesmo um dos
políticos insultava e censurava Lísias, e entre os insultos proferidos o
designava por logógrafo (\emph{logográphon}). Talvez, então, tenha sido
o amor pela honra (\emph{philotimías}) o motivo pelo qual ele se absteve
de nos escrever (\emph{gráphein}).\footnote{Nesse ponto o diálogo ganha
  novo aspecto, pois pela primeira vez abandonará as recitações, sem,
  entretanto, abandonar completamente seus temas principais. Muitos
  comentadores acreditam ser esse o início de uma segunda parte do
  diálogo, como um comentário às três recitações anteriores, mas apesar
  de agora o movimento discursivo ser diferente, especialmente em seu
  andamento, não há elementos suficientes para entender esse trecho em
  diante como completamente desligado das recitações anteriores. Daqui
  em diante os principais temas permanecem sendo discutidos, com outro
  tratamento, é verdade, através de pontes para temas correlatos ou
  subtemas, o principal deles é, nessa nova navegação, o próprio
  discurso falado e escrito. A questão da escrita e do \emph{lógos}
  ganham força sob nova perspectiva, pois Sócrates e Fedro agora
  realizarão um diálogo, sem nenhuma interrupção de recitações longas,
  referindo muitas vezes discursos anteriores, mas ampliando a discussão
  para um juízo da aplicação do discurso, falado e escrito. A primeira
  pergunta desse trecho é se a arte de escrever, a logografia, é uma
  honra ou uma vergonha.}

S: Engraçado, ó jovem, o parecer (\emph{dógma}) que proferes, pois sobre
o teu companheiro (\emph{hetaírou}) estás completamente enganado
(\emph{diamartáneis}), se o consideras como alguém tímido. {[}257d{]}
Talvez aquele que o insultava considerasse censurável dizer o que disse
(\emph{légein hà élegen}).

F: É o que parece, ó Sócrates. Tu sabes como os poderosos e
reverenciados nas cidades envergonham-se de escrever discursos
(\emph{lógous te gráphein}) e de deixar composições suas
(\emph{kataleípein suggrámmata heautôn}), temerosos da reputação
(\emph{dóxan}) que, com o tempo, pode atingi-los, sendo designados por
sofistas (\emph{mè sophistaì kalôntai}).\footnote{Essa visão de que é
  vergonhoso deixar escritos (\emph{suggrámmata}) aproxima logografia e
  sofística. Ambas são construídas a partir de uma reputação
  (\emph{dóxan}), e trabalham mesmo no campo da opinião (\emph{dóxa}),
  mas há um vitupério implícito em ser chamado de sofista (\emph{mè
  sophistaì kalôntai}). Aqui observa-se uma proximidade muito grande
  entre o logógrafo, o político e o sofista, proximidade que perpassa
  toda a filosofia platônica.}

S: Doce rodeio, ó Fedro, mas esqueces ainda do grande rodeio mencionado
pelos que descem o Nilo. {[}257e{]} E além desse rodeio, esqueceste que
os maiores amantes da logografia (\emph{erôsi logographías}), bem como
do legado de composições escritas (\emph{kataleípseôs suggrammátôn}),
são os grandes e os mais notáveis políticos (\emph{oi mégiston
phronoûntes tôn politikôn}).\footnote{Observa-se aqui uma menção aos
  temas egípcios que serão retomados na alegoria de Theuth e Thamous
  (274c5-275b2). Aqui vemos um exemplo de uma passagem discursiva que
  vai do vitupério ao elogio, se por um lado o termo ``logógrafo'' é
  usado como um vitupério, contra Lísias, também pode ser um elogio,
  especialmente quando refere uma habilidade de um grande político. A
  logografia em si não pode ser avaliada, pois ela sempre está atrelada
  aos seus autores, os quais podem, sem dúvida, serem louvados ou
  censurados tanto em suas ações como nos seus discursos.} Em seguida,
em discurso escrito (\emph{gráphôsi lógon}), eles agradam aos seus
panegiristas (\emph{epainétas}), uma vez que estes são os primeiros a
elogiá-los (\emph{epainôsin}), previamente e em qualquer
situação.\footnote{Os que elogiam e os elogiados nos discursos
  proferidos estabelecem uma conexão entre pólos que alimentam
  reciprocamente desejos e opiniões similares, o que revela essa conexão
  capaz de imantar amante e amado, além de impulsionar essa relação
  recíproca que está em jogo, no fluxo do desejo, inerente a toda e
  qualquer busca intelectual e discursiva.}

F: Como dizes isso? Não compreendo.

{[}258a{]} S: Não compreendes porque os políticos, no início das suas
composições escritas, inscrevem (\emph{gégraptai}) primeiramente o nome
dos seus panegiristas (\emph{epainétês}).

F: Como?

S: ``Foi resolvido'', como ele diz, ``pelo conselho'' ou ``pelo povo'',
ou por ambos, e ao dizer ``aquele que'', refere-se ao seu próprio
discurso, no que há de mais sagrado e elogiável (\emph{egkômiádzôn}) no
escritor (\emph{suggrapheús}). Depois de tudo isso, mostra aos seus
panegiristas (\emph{epainétais}) a sua própria sabedoria
(\emph{sophía}), por vezes redigindo composições escritas
(\emph{sýggramma}) bastante longas. Que outra coisa te parece isso,
senão uma composição de discurso escrito (\emph{lógos syggegramménos})?

F: Não me parece outra coisa.

S: Então, se for bem recebido, o poeta deixa o teatro com júbilo, se for
rejeitado, é privado da logografia e da dignidade (\emph{áxios}) de
escrever (\emph{suggráphein}), lamentando-se ele e os seus companheiros.

F: E muito.

S: Parece que não desprezam essa ocupação, mas a admiram.

F: Perfeitamente.

S: O quê? Quando alguém vem a ser um rétor ou um rei, tal qual Licurgo,
Sólon ou Dario, não é possível que ele venha a se tornar um logógrafo
imortal da cidade? {[}258c{]} Enquanto está vivo, ele é visto como um
deus\footnote{Mais uma vez, a ideia de ser considerado um deus vem à
  tona, mas não deve ser confundida com a reciprocidade entre amado e
  amante (255c), pois aqui esse louvor está mais próximo do que veremos
  a seguir, quando esses artífices da palavra adquirem uma reputação
  (\emph{dóxa}) de divinos sem o serem de fato.} e, depois, os
subsequentes cidadãos não o considerariam da mesma maneira, ao
contemplarem suas composições escritas (\emph{suggrámmata})?

F: E como.

S: Consideras que um desses, qualquer um, com qualquer tipo de desavença
contra Lísias, poderia censurá-lo (\emph{oneidízein}) porque escreveu
(\emph{suggráphei})?

F: Não é verossímil (\emph{eikós}) pelo que dizes. Pois seria, como
parece, uma censura (\emph{oneidízoi}) contra o próprio desejo
(\emph{epithymíai}).

{[}258d{]} S: E isso é claro para todos, que não é vergonhoso por si só
escrever discursos (\emph{grapheîn lógous}).

F: Como?

S: Considero vergonhoso falar (\emph{légein}) e escrever
(\emph{gráphein}) sem nenhuma beleza, além de uma vergonha é algo
malvado.

F: É claro.

S: Qual é então a maneira (\emph{trópos}) de escrever com beleza ou sem?
Precisamos, ó Fedro, examinar (\emph{exetásai}) esse assunto junto a
Lísias ou a qualquer outro que tenha escrito (\emph{gégraphen}) ou que
ainda vá escrever (\emph{grápsei}), seja sobre um escrito político
(\emph{politikòn súggramma}) ou um assunto particular, seja na métrica
como poeta ou sem, como um prosador?\footnote{Esse exame
  (\emph{exetásai}) é usado nessa passagem assinalando a busca
  intelectual acerca dos discursos, conferindo assim um certo
  distanciamento necessário nessa análise.}

{[}258e{]} F: Perguntas (\emph{erôtais}) se precisamos? Que motivo teria
alguém para viver senão em vista, por assim dizer, desses mesmos
prazeres (\emph{hêdonôn})?\footnote{Pergunta similar à do início do
  diálogo (227b9), quando Sócrates, citando Píndaro, diz que ``não há
  nada mais elevado'' do que ouvir o que Fedro tem a dizer acerca do que
  escutou de Lísias, só que agora é Fedro quem ``imita'' Sócrates,
  fazendo essa pergunta exagerada. Essa imitação funciona como um índice
  de que uma nova busca se inicia, agora circunscrita em dizer e em
  fazer um juízo mais elaborado de como se escreve bem ou mal, com ou
  sem beleza.} Pois não são daqueles que necessitam de sofrimento prévio
(\emph{prolupêthênai}), sem o que nem mesmo o prazer (\emph{hêsthênai})
haveria, mas estão entre os poucos (\emph{olígou}) que fornecem todos os
prazeres corpóreos (\emph{sôma hêdonaì}), motivo pelo qual, justamente,
são designados por servis (\emph{andrapodṓdeis}).

S: Temos tempo livre (\emph{scholḕ}), como parece. Enquanto isso as
cigarras cantoras\footnote{Mais um lugar-comum (\emph{tópos}) da poesia
  de Estesícoro, quando ele diz que ``é preciso evitar a desmesura
  (\emph{hýbris}), para que as cigarras não cantem no chão'', ou seja,
  para que tudo ao redor não seja destruído, não sobrando nem mesmo as
  árvores, onde normalmente elas habitam. Segundo Aristóteles
  (\emph{Rhet}. 1395a) esse é um exemplo do uso de sentença ou máxima
  (\emph{gnômes}).} conversam entre si nesse calor e nos observam
(\emph{kathorân}) lá de cima. {[}259a{]} Se elas nos vissem, como a
maioria, ao meio-dia e sem dialogarmos (\emph{mḕ dialegoménous}), quase
dormindo, encantados pela preguiça da reflexão (\emph{dianoías}), elas
justamente nos desprezariam, considerando-nos como criaturas cativas que
chegaram a um recanto, como ovelhas, ao meio-dia, a dormir junto à fonte
(\emph{tḕn krḗnen eúdein}). Mas se elas nos vissem a dialogar
(\emph{dialegoménous}) e a evitá-las, como {[}quem evita{]} as Sirenas,
sem nos deixarmos encantar (\emph{akêlḗtous}), então rapidamente nos
admirariam e conceder-nos-iam as dádivas divinas atribuídas aos
homens.\footnote{Aqui Platão reccore pela primeira vez no \emph{Fedro} a
  um expediente importante, mencionado por Slezák (1993, p.161-163), o
  do interlocutor imaginário. Sócrates convida Fedro a imaginar o que as
  cigarras contoras diriam deles se os vissem preguiçosos no discurso
  ou, ao contrário, ávidos por ele.}

{[}259b{]} F: Quais são essas dádivas? Não ouvi, como parece, acerca de
nenhuma delas?

S: Não é adequado (\emph{prépei}) a um homem amigo das Musas
(\emph{philómouson}) não ter ouvido falar nisso. Dizem que, antes do
tempo das Musas, as cigarras eram homens e que, quando estas {[}Musas{]}
surgiram e lhes mostraram os cantos (\emph{phaineísês} \emph{oidês}),
alguns deles foram tomados por esse prazer (\emph{hêdonês}). Envolvidos
com o canto (\emph{áidontes}), eles, sem perceber, acabaram descuidando
da comida e da bebida, sendo levados à morte. Deles é que a família das
cigarras descende, pois, junto às Musas, tendo recebido essa dádiva
(\emph{géras}), elas não têm necessidade de alimentos, mas vivem a
cantar (\emph{aidein}) ininterruptamente, sem comer e sem beber, até a
morte e, depois disso, para as Musas relatam (\emph{apaggéllein}) quais
foram aqueles que as honraram (\emph{timâi}) aqui. Terpsicore
(Alegra-coro) é venerada (\emph{tetimêkótas}) nas danças
(\emph{choroîs}), relato que proporciona maior benevolência aos seus
realizadores.\footnote{Hermias ao comentar o trecho, associa Calíope e
  Urânia, respectivamente, à audição e à visão, retomando uma ideia que
  há no \emph{Timeu} 47c (\emph{en hemîn peplanêménas katastêsaímetha}),
  segundo a qual a visão dos astros celestes, além de servirem para o
  surgimento da filosofia, servem de paradigma para ajustarmos o nosso
  interior, salvarmo-nos da errância e ajustando nossa falta de razão
  (\emph{en hemîn peplanêménon sôizomen...táttomen tò en hemîn álogon}
  {[}...{]} \emph{en hemîn álogon} \emph{katatáttein}) (Hermias
  §180.10-21 Couvreur).} {[}259d{]} Érato (Amorosa) com a {[}poesia{]}
erótica (\emph{erôtikoîs}) é venerada, assim também em outras ocasiões,
segundo cada forma de honra (\emph{timês}). As mais velhas delas são
Calíope (Belavoz) e em seguida Urânia (Celeste), para aqueles que se
dedicam à filosofia e que estimam (\emph{timôntas}) a música, pois
especialmente as Musas enviam bela-voz acerca do céu, dos discursos dos
deuses e dos homens.\footnote{Deixei o nome grego praticamente
  transliterado de cada Musa para não esconder as raízes gregas,
  importantes nessa leitura, e mantive cada nome acompanhado pela versão
  de Torrano, tal qual na sua tradução da \emph{Teogonia} de Hesíodo,
  v.77-8 (Torrano, 1995, p. 109).} Muitas são as razões para que falemos
ao meio-dia e não cochilemos.

F: Falemos então.

{[}259e{]} S: Vamos agora estabelecer uma verificação
(\emph{sképsasthai}) sobre o discurso, verifiquemos (\emph{skeptéon}) em
que medida é possível falar e escrever (\emph{légein te kaì gráphein})
de modo belo (\emph{kalôs}) ou não.

F: Claro.

S: Não é necessário àqueles que desejam falar bem e de modo belo, que o
pensamento (\emph{diánoian}) de quem fala conheça a verdade
(\emph{eiduîan tò alêthès}) acerca do que será tratado?\footnote{Nesse
  passo, o campo filosófico aparece delineado na conexão possível entre
  pensamento, linguagem e verdade (\emph{alêthès}).}

{[}260a{]} F: Acerca disso ouvi o seguinte, ó querido Sócrates: aquele
que deseja tornar-se rétor não necessita compreender (\emph{manthánein})
o que é verdadeiramente justo (\emph{tôi ónti díkaia}), mas o que parece
ser para aqueles muitos (\emph{tà dóxant' an plḗthei}), nem o verdadeiro
bom e belo, mas o que lhes parecer assim (\emph{tà óntos agathà ê kalà
all' hósa dóxei}). Disso deriva a persuasão (\emph{peíthei}), e não da
verdade (\emph{alêtheías}).\footnote{Nesse parágrafo temos delineado o
  campo semântico principal dessa parte do diálogo que segue, além de um
  eloquente retrato da retórica, através de um lema, de que a retórica
  não trabalha com a verdade, mas apenas com o verossímil, com a
  aparência (\emph{tà dóxant', dóxei}), de onde deriva a persuasão
  (\emph{peíthei}), não da verdade (\emph{alêtheías}). Nessa perspectiva
  retórica, a verdade não valeria nada, apenas o verossímil, o aparente.
  Se na \emph{palinódia} Sócrates descreve a importância da planície da
  verdade no trajeto das almas, agora há um empenho em mostrar que a
  verossimilhança -- que é a base da retórica --, está pautada também em
  algo verdadeiro, de onde surge, de fato, o diastema entre o verdadeiro
  e o verossímil. A partir daí Sócrates apresentará os fundamentos da
  dialética, a arte de discernir acerca dessa diferença.}

S: ``Palavras nada desprezíveis''\footnote{Retomada da expressão da
  \emph{Ilíada}, 2.361; 3.65: \emph{oú} \emph{toi} \emph{apóbleton}
  \emph{épos}}, ó Fedro, essas que os sábios (\emph{sophoí}) proferem,
mas vamos examinar (\emph{skopeîn}) se elas nos dizem algo. Certamente o
que foi dito não deve ser abandonado.

F: Dizes bem.

S: Examinemos.

F: Como?

{[}260b{]} S: Se eu quisesse convencer-te e ajudá-lo na aquisição
(\emph{ktêsámenon}) de um cavalo de combate, ambos desconhecendo
(\emph{agnooîmen}) o que é um cavalo, mas, se alguma coisa, entretanto,
eu soubesse sobre você, que Fedro considera que ele é o animal doméstico
que tem a maior orelha.

F: Seria engraçado, ó Sócrates.

S: Nem tanto. Mas na ocasião de ocupar-me da tua persuasão
(\emph{peíthoimi}), colocando o discurso elogioso (\emph{épainon}) no
asno, designando-o por cavalo, falando acerca de todas as qualidades da
criatura no uso doméstico, na aquisição, na guerra, defendendo sua
utilidade (\emph{ôphélimon}) com bagagens e outras tantas tarefas.

{[}260c{]} F: Isso seria realmente engraçado.

S: Mas então não seria melhor o engraçado (\emph{geloîon}) e o amistoso
(\emph{phílon}) do que o terrível (\emph{deinón}), ou o hostil
(\emph{echthròn})?

F: Parece.

S: Mas, quando o rétor desconhece (\emph{agnoôn}) o bom (\emph{agathòn})
e o mau (\emph{kakón}), tomando uma cidade pela a persuasão
(\emph{peíthêi}), não faria um elogio (\emph{épainon}) da sombra
(\emph{skiâs}) de um asno como se fosse de um cavalo, mas elogiaria o
mau como sendo o bom, e, exercitado na opinião da maioria (\emph{dóxas
dè plḗthous memeletêkòs}), ele poderia persuadi-los (\emph{peísêi}) a
fazer o mau e não o bom. Considerando isso tudo, que tipo de fruto
(\emph{karpòn}) a retórica poderia colher dessa semeadura?

F: Um fruto não muito agradável.

S: Então, ó bondoso, fomos mais grosseiros que o necessário ao
detratarmos a arte dos discursos (\emph{lógôn téchnên})? E ela talvez
nos dissesse: ``Ó admiráveis, porque dizeis tais bobagens? Eu não obrigo
ninguém que desconheça a verdade a aprender a falar (\emph{agnooûnta
talêthès anagkádzô manthánein légein}), mas, se em algo vale o meu
conselho (\emph{sumboulḗ}), que adquiram (\emph{ktêsámenon}) aquela
{[}verdade{]} antes de me tomar (\emph{lambánein}). Eis então o que digo
veementemente: que, sem mim, aquele que conhece a verdade nunca
alcançará a arte de persuadir (\emph{peíthein téchnei}).''\footnote{Agora
  será a própria arte discursiva que interpela ambos, e diz não ser
  culpada por que alguns não sabem a verdade antes de aprenderem a
  falar, ou seja, ela aconselha a fazer o caminho contrário, conhecer a
  verdade antes de aprender a falar, para evitar que acabem falando
  coisas não verdadeiras. Esse seria o flanco aberto da arte discursiva,
  ser capaz de dizer tanto a verdade quanto a mentira, sendo essa
  segunda aplicação também motivo pelo qual a arte era detratada. A arte
  da palavra aparece personificada e defendendo-se dessa suposta culpa
  como interlocutora imaginária.}

{[}260e{]} F: E ela não diria coisas justas ao proferir isso?

S: É o que digo, se os discursos apresentados testemunham que ela é uma
arte. É como se eu ouvisse a aproximação de alguns contestadores da arte
do discurso a dizer que ela é falsa (\emph{pseúdetai}), que ela não é
uma arte, mas uma ocupação isenta de arte (\emph{ouk esti téchne alla
atechnos tribé}). Os Lacônios afirmam que ``não existe uma arte
verdadeira (\emph{étymos téchnê}) sem estar atada à verdade
(\emph{alêtheías}), nem mesmo poderá existir no futuro.''

{[}261a{]} Precisamos desses discursos, ó Sócrates, traze-os agora para
junto de nós para examinarmos (\emph{exétaze}) o quê e como eles falam.

S: Vinde, nobres criaturas, persuadi (\emph{peíthete}) Fedro de belos
filhos de que, quem não filosofar suficientemente (\emph{ikanôs
philosophḗsêi}), não será nunca capaz de falar sobre coisa alguma.
Responde agora Fedro.

F: Pergunta.

S: Então, o todo da retórica não seria a arte da condução das almas
(\emph{téchnê} \emph{psychagôgía}) por meio das palavras (\emph{dià
lógôn}), não só nos tribunais (\emph{dikastêríois}) e em outras
assembleias públicas (\emph{dêmósioi súllogoi}), mas também nas questões
particulares (\emph{idíois}), naquelas insignificantes e nas grandiosas,
e que não há nada de mais honrado (\emph{entimóteron}) que o seu
emprego, quando correto, seja nos assuntos sérios ou nos banais?
{[}261b{]} Ou como ouviste falar disso tudo?

F: Não, por Zeus, não foi assim absolutamente, mas que especialmente nos
tribunais (\emph{dikás}) fala-se e escreve-se com arte, bem como nas
assembleias públicas (\emph{dêmêgorías}). Não ouvi mais do que isso.

S: Mas então apenas ouviste sobre as artes discursivas de Nestor e de
Odisseu, as quais foram escritas (\emph{sunegrapsátên}) por eles em
Troia, nas horas vagas (\emph{scholázontes}), e nem mesmo chegaste a
ouvir aquela composta por Palamedes?

{[}261c{]} F: Por Zeus, nem mesmo ouvi a de Nestor, a não ser que
consideres Górgias uma espécie de Nestor, ou Trasímaco e Teodoro
distintos tais quais Odisseu. \footnote{Em Homero, Nestor é um lendário
  senhor sábio, justo e hábil na eloquência. Na doxografia ligada a
  Antifonte, o rétor, há indícios de que este também carregou o epíteto
  de ``Nestor'' (Plut. \emph{Vit. decem oratorum}, 832e5; Phil.
  \emph{Vitae sophistarum} 1.15). Nesse caso, haveria uma justaposição
  entre Górgias e Antifonte, bastante justificada aliás, pela técnica
  discursiva que ambos exerceram. Antifonte, embora não mencionado,
  permanece bastante próximo desse campo do \emph{Fedro} que refere a
  sofística, especialmente por conta da sua ``arte de aliviar''
  (\emph{téchnen alupías}), de ``tratar dos sofredores com discursos''
  (\emph{tous lupouménos dià lógon therapeúein}) (Plut. \emph{Vit. decem
  oratorum}, 833c7-d2, Fócio, \emph{Bib}. 486a), temas correlatos à cura
  discursiva.}

S: Talvez, mas deixemos estes aí por hora. E tu dize-me o que fazem
(\emph{tí drôsin}) os que disputam nos tribunais (\emph{dikastêríois oi
antídikoi}), eles não entram em litígio (\emph{antilégousin})? Ou o que
diremos?

F: Isso mesmo.

S: Acerca do justo e do injusto (\emph{dikaíou te kaì
adíkou})?\footnote{A partir daqui especialmente será possível observar o
  jogo entre expressões polares, muitas delas, como veremos,
  referenciadas com alfa privativo, como nesse caso (\emph{dikaíou te
  kaì adíkou}).}

F: Sim.

S: Então, quem lançar mão dessa atividade com arte fará as mesmas coisas
parecerem justas às mesmas pessoas, e, por outro lado, quando quiser
(\emph{boúlêtai}), parecerem injustas?

{[}261d{]} F: O que tem isso?

S: E, também, nas assembleias públicas (\emph{dêmêgoríai}) da cidade,
fará parecer as mesmas coisas ora boas ora o seu contrário
(\emph{tanantía})?

F: É assim.

S: Então não conhecemos os dizeres com arte do eleático Palamedes, por
meio do qual mostrava aos ouvintes as mesmas coisas como semelhantes e
dessemelhantes (\emph{hómoia kaì anómoia}), unas e múltiplas (\emph{hén
kaì pollá}), em repouso e em movimento (\emph{ménontá te aû kaì
pherómena})?\footnote{Eleático Palamedes é um epônimo de Zenão de Eleia.
  No \emph{Parmênides} de Platão, Zenão lê um texto seu, segundo
  Sócrates, defendendo a mesma tese de Parmênides, de que o todo é um,
  só que por seu lado inverso, de que não exista a pluralidade (Pl.
  \emph{Parm}. 128a-c). No diálogo \emph{Parmênides} as expressões
  \emph{hómoia kaì anómoia}, ligadas à filosofia de Zenão, em diversos
  momentos são mencionados (127e2, 135e5, 139e6, 147c1, 148c1-2).}

F: E como!

{[}261e{]} S: Então, não só no tribunal e nas assembleias públicas
existe a antilogia (\emph{antilogikḕ}), mas, como parece, em todas as
coisas que são ditas há uma só arte (\emph{mía tis téchnê}), se é que
existe, aquela que é capaz de assemelhar tudo a todas as coisas
possíveis (\emph{pân pantì homoíoûn tôn dynatôn}), na medida do
possível, e também de trazer à luz (\emph{eis phôs agein}) o que outros,
operando essas semelhanças (\emph{homoioûntos}), tentam dissimular
(\emph{apokruptôménou}).\footnote{Depois de confirmar a aplicação da
  antilogia nos tribunais e nas assembleias públicas, Sócrates amplia
  essa aplicação, para em seguida afirmar que há uma só arte, a que é
  capaz de assemelhar através do discurso os seres, na medida do
  possível. Observar essas assimilações e dessemelhações, bem como
  trazê-las à luz quando necessário será apresentado posteriormente como
  o fundamento da dialética, que traz esse movimento duplo e contraposto
  de síntese e análise, de assemelhação e dessemelhação. Embora pareça
  que simplesmente o rétor dissimula e o dialético desvenda, é preciso
  perceber que a arte de assemelhar e dessemelhar é uma técnica do
  pensamento e do discurso, usada por ambos, pelo rétor e pelo filósofo,
  com finalidades diferentes. Todo esforço de Platão está em mostrar a
  superioridade do filósofo frente ao sofista, uma vez que este último
  não saberia definir, com propriedade, o fundamento da arte que
  acredita poder ensinar e praticar (\emph{Phdr}. 269b4-c4), enquanto o
  filósofo seria capaz de definir, ensinar e aplicar no discurso os
  fundamentos da dialética (\emph{Phdr}. 271d-272b).}

F: Como é que dizes isso?

S: Procuro mostrar-te isso que buscamos. O engano nasce
predominantemente naquilo que difere muito ou pouco?

{[}262a{]} F: No que difere pouco.

S: Mas então seria melhor para transportar às ocultas (\emph{metabaínôn
mâllon lḗseis}), conduzindo os discursos ao seu contrário, guiar-se pelo
que difere pouco (\emph{smikròn}) ou, ao contrário, pelo que difere
muito.

F: Como não seria assim?

S: É necessário àquele que se prontifica a enganar (\emph{apatḗsein})
outros, sem enganar a si mesmo (\emph{autòn dè mḕ apatḗsesthai}), que
conheça exatamente as semelhanças e as dessemelhanças entre os seres
(\emph{tèn homoiótêta tôn óntôn kaì anomoiótêta akribôs
dieidénai}).\footnote{A arte de enganar, nesse retrato, tem êxito ao
  dominar ``as semelhanças e as dessemelhanças entre os seres'' e fazer
  com que, de modo sutil, o rétor passe em seu discurso, a partir das
  semelhanças, a ser capaz de dizer o Ser e o seu contrário, o não-Ser.
  Para enganar é preciso amenizar semelhanças e ressaltar diferenças ou,
  ao contrário, amenizar diferenças e ressaltar semelhanças, sempre de
  acordo com objetivos predeterminados. Diante dessa constatação, fica
  claro que para saber enganar através das semelhanças e dessemelhanças
  discursivas é preciso conhecer a verdade, ainda que parcialmente, ou
  aspectos dela, tese apresentada por meio do dito dos lacônicos em
  260e5-6.}

F: É necessário.

S: E será possível a este mesmo homem, desconhecendo a verdade,
reconhecer as semelhanças (\emph{homoiótêta}) menores e maiores em
outros seres?

{[}262b{]} F: Impossível.

S: Dessa forma, para os que opinam contra {[}a existência{]} dos seres e
são enganados, é evidente como sua afecção (\emph{páthos}) foi arrastada
por aquelas semelhanças (\emph{di'homoiotḗtôn}).

F: É assim mesmo que acontece.

S: Como o artífice (\emph{technikòs}) irá transladar
(\emph{metabibázein}), seguindo as menores semelhanças
(\emph{homoiotḗtôn}) entre os seres, levando cada um deles ao seu
contrário (\emph{tounantíon})? E como ele poderia esquivar-se
(\emph{diapheúgein}) desse mesmo efeito sem reconhecer o que são cada um
dos seres (\emph{hékaston} \emph{tôn óntôn})?

F: Não poderia.

{[}262c{]} S: Então a arte do discurso (\emph{lógôn téchnên}), ó
companheiro, sem o conhecimento do que é verdadeiro (\emph{tèn alétheian
mè eidôs}), é como uma caça das opiniões (\emph{dóxas dè tethêreukós}),
ocupação risível e, como bem parece, desprovida de arte (\emph{átechnon
paréxetai}).

F: É bem possível.

S: Sobre esse discurso de Lísias que trazes consigo ou esses que
pronunciamos em seguida, pretendes observar o que neles há desprovido de
arte (\emph{atéchnôn}), bem como o que está de acordo com a arte
(\emph{entéchnôn})?

F: É o melhor a fazer, especialmente porque até agora só falamos no
vazio, sem paradigmas suficientes.

S: E foi por sorte (\emph{tychên}), como bem parece, que nós temos dois
discursos como paradigmas, o que mostra que aquele que conhece a verdade
(\emph{eidôs tò alêthès}), brincando com as palavras (\emph{prospaídzôn
en lógois}), pode demover os ouvintes. {[}262d{]} E eu, ó Fedro, atribuo
isso aos deuses desse lugar (\emph{entopíous theoús}). Talvez tenham
sido as cigarras, intérpretes das Musas, que, sobre nossas cabeças,
cantam e inspiram-nos essa honra, pois eu não partilho (\emph{métochos})
de nenhuma arte no meu discurso (\emph{téchnês toû légein}).\footnote{Sócrates
  agora mimetiza a fala de Fedro, quando disse, em 229a3, que por
  ``sorte'' estava desclaço, como Sócrates sempre estava. Conhecer a
  verdade nessa passagem já é entendido como uma condição para ser capaz
  de brincar (\emph{prospaidzein}) com discursos e conduzir os ouvintes
  (\emph{parágoi toùs akoúntas}). É notável a citação à brincadeira
  (\emph{paígnion}) discursiva, uma tópica encontrada em Górgias também.}

F: Que assim seja, mas apenas mostra o que dizes.

S: Lá vai então. Lê-me o início do discurso de Lísias.

{[}262e{]} F: ``Já estás ciente acerca dos meus assuntos e creio que
ouviste acerca do que pode acontecer conosco. Espero que não me advenha
nenhum infortúnio (\emph{atychêsai}) só porque me ocorreu
(\emph{tygcháno}) de não estar te amando (\emph{ouk} \emph{erastès}),
como aqueles que (...) arrependem-se (\emph{metamélei}) {[}...{]}''.
\footnote{Citação de 230e-231a, início da declamação de Lísias. Aqui é a
  segunda vez que os discursos anteriores são retomados, mas sem
  profundidade, pois, logo em seguida, o tema se desloca para uma
  explanação teórica acerca dos fundamentos da arte discursiva.
  Obviamente, onde há mais controvérsia a retórica pode ser aplicada com
  maior vigor, como é o caso do Amor (Eros), tema naturalmente
  controverso.}

S: Para. Tratemos agora do que ele errou (\emph{hamartánei}) e no que
procedeu sem arte (\emph{átechnon}), não é mesmo?

{[}263a{]} F: Sim.

S: Mas isso não é evidente para todos, que acerca de algumas coisas nós
concordamos (\emph{homonoêtikôs}) e de outras discordamos
(\emph{stasiôtikôs})?

F: Perece que entendo o que dizes, mas explica ainda de modo mais claro.

S: Quando dizes um nome como ferro (\emph{sidḗrou}) ou prata
(\emph{argúrou}), não entendemos (\emph{dienoḗthemen}) todos nós a mesma
coisa?

F: E como.

S: E quanto ao justo ou ao bom? Não ocorre que nos dirijamos uns para um
lado e outros para outro, fazendo como que entremos em controvérsia
mútua e até conosco mesmo (\emph{amphisbêtoûmen allélois te kaì hêmîn
autoîs})?

F: É assim mesmo.

{[}263b{]} S: Então, existem coisas a respeito das quais nós chegamos a
um acordo (\emph{sumphônoûmen}) e outras não?

F: De fato.

S: Em quais delas é mais fácil nos enganarmos e em qual delas a retórica
(\emph{rhetorikḕ}) tem maior poder (\emph{meîdzon dýnatai})?

F: É evidente que naquelas em que nós somos errantes
(\emph{planṓmetha}).\footnote{Temas em que as pessoas podem facilmente
  pender para a opinião contrária, sendo convencidas, seja pelo poder da
  retórica comum, seja pela filosofia. É importante frisar que o
  movimento do intelecto, da alma e do discurso são descritos de acordo
  com o vocabulário dos \emph{planetas}, pois as almas são animadas como
  os planetas e o verbo ``errar'', \emph{planesthai}, é aplicado
  igualmente a esses campos\emph{.} Em Górgias os meteologistas e
  filósofos são igualmente hábeis manipuladores das opiniões
  contrapostas (\emph{dóxan anti dóxês}, \emph{dóxês}), inclusive acerca
  de coisas incríveis e invisíveis (\emph{ápista kaì adêla}) (Gorg.
  \emph{Hel}. §13).}

S: Então, para aquele que deseja seguir a arte retórica (\emph{téchnên
rhetorikḕn}), primeiramente, é preciso que diferencie esses dois
caminhos e que detecte os caracteres de cada um deles, onde
necessariamente a multidão erra (\emph{to} \emph{plêthos}
\emph{planâsthai}) e onde não.\footnote{Uma definição provisória de arte
  retórica é aqui apresentada, pois aquele que conhece discursivamente
  onde a multidão erra (\emph{to} \emph{plêthos} \emph{planâsthai})
  conhece o poder da retórica ou seu principal artifício motor de almas
  (psicagógico). O rétor conhece como aplicar e jogar com as opiniões de
  modo persuasivo, muitas vezes promovendo essa mudança, essa
  ``errança'' no ouvinte.}

{[}261c{]} F: Belo seria, ó Sócrates, deter essa forma de apreensão
(\emph{katanenoêkṑs}).

S: Em seguida, creio que não devemos deixar de observar cada um dos
assuntos que surgem, mas percebê-los com agudeza (\emph{oxéôs
aisthánesthai}), bem como o gênero daquilo que falaremos.

F: Sem dúvida.

S: E o que diremos do amor? É algo controverso (\emph{amphisbêtêsímôn})
ou não?

F: Presumo que seja algo controverso (\emph{amphisbêtêsímôn}), ou
consideras ser possível admitir o que há pouco disseste acerca dele, que
é danoso (\emph{blábê}) ao amante e ao amado (\emph{tôi erôménôi kaì
erônti}) e, logo depois, que é o maior dos bens ocorridos (\emph{agathôn
tugchánei})?

{[}263d{]} S: Dizes muito bem, mas também diz-me isso -- pois, pelo meu
entusiasmo (\emph{enthousiastikòn}), não me lembro bem (\emph{ou pánu
mémnêmai}) --, se defini o amor no início do discurso.

F: Sim, por Zeus, e com extraordinária precisão.

S: Ah! Proclamas superiores na arte (\emph{technikôtéras}) as Ninfas,
filhas de Aqueloo, e Pã, filho de Hermes, comparados a Lísias, filho de
Céfalo, em seu discurso. Talvez eu esteja errado, mas não é verdade que
Lísias, no início do seu discurso erótico (\emph{toû erôtikoû}),
forçou-nos a entender o amor tal qual ele desejou (\emph{eboulḗthê}), e
a partir disso compôs (\emph{suntaxámenos}) tudo o que veio depois,
levando o discurso a seu termo? {[}263e{]} Queres novamente que leiamos
o seu começo (\emph{boúlei pálin anagnômen tèn archèn autoû})?

F: Se te parece conveniente. Mas o que procuras (\emph{dzêteîs}) não
está aí.

S: Diz, para que eu possa ouvi-lo dele mesmo.

F: ``Já estás ciente acerca dos meus assuntos e creio que ouviste acerca
do que pode acontecer conosco. Espero que não me advenha nenhum
infortúnio (\emph{atychêsai}) só porque me ocorreu (\emph{tygcháno}) de
não estar te amando (\emph{ouk} \emph{erastḕs}), {[}264a{]}como aqueles
que tão logo tenha cessado o seu desejo (\emph{epithymías paúsôntai}),
arrependem-se (\emph{metamélei}) do que bem fizeram.''\footnote{Citação
  de 230e-231a e 262e1-4. Nessa passagem, citada novamente, há a
  inclusão da frase ``tão logo tenha cessado o seu desejo
  (\emph{epithymías paúsontai})'', omitida na primeira citação.}

S: Falta muito ainda, ao que parece, para que ele realize isso que
procuramos. Ele nem começa pelo começo, mas pelo final, empreendendo seu
discurso como alguém que nada de costas e para trás, iniciando pelas
coisas que o amante diria ao seu predileto somente no final. Ou não é
como digo, Fedro, querida cabeça.\footnote{Cf. a mesma expressão
  homérica da \emph{Ilíada} 8, 281, em Platão, \emph{Fedro} 234d,
  \emph{Eutidemo} 293e, \emph{Górgias} 513a. Mais uma vez o texto de
  Lísias é citado quase que para mostrar que dele não se extrai nada, ou
  quase nada, pois jamais Sócrates analisa o texto escrito de Lísias.
  Nessa citação aparece obviamente um primeiro aceno a Isócrates, grande
  citador de si mesmo, retrato exato daquilo que Sócrates descreverá à
  frente como alguém que só tem seus próprios escritos para trabalhar e
  não faz nada senão passar em revista seus próprios textos. Sócrates
  não provoca nenhuma busca mais profunda, apenas aponta para o aspecto
  vivo da linguagem presencial falada. Dizer que Lísias nada de costas
  se deve ao fato desse discurso ser uma peroração, ou seja, uma parte
  final de discurso, de modo que Lísias não segue a prescritiva
  retórica, pois não saberia nem mesmo organizar um discurso a partir de
  um proêmio.}

{[}264b{]} F: É assim mesmo, ó Sócrates, uma peroração (\emph{teleutḗ})
em torno da qual é realizado o discurso.

S: E quanto ao resto? Não perece que foi lançado indiscriminadamente no
discurso? Ou o que veio depois do discurso deveria ser de fato colocado
depois por alguma necessidade, ou alguma outra coisa entre as que foram
ditas? Pois a mim me parece, como não sei de nada, que não é vil o que
foi proferido pelo escritor (\emph{tôi gráphonti}). E tu conheces alguma
necessidade logográfica pela qual ele dispôs assim o discurso de modo
sucessivo, uma coisa ao lado da outra?

F: És muito gentil, uma vez que me considera suficientemente capaz de
assim discerni-lo com precisão (\emph{akribôs diideîn}).

{[}264c{]} S: Mas te considero capaz de mostrar isso, que é necessário
que todo discurso esteja combinado como um ser vivo (\emph{hṓsper
dzôion}), tendo corpo próprio, não sendo acéfalo nem ápodo, e que tenha
tronco e membros convenientes entre si e com relação ao todo do escrito
(\emph{tôi hólôi gegramména}).

F: Como não?

S: Verifica (\emph{sképsai}) esse discurso do teu companheiro, seja ele
assim ou de outra maneira. Não encontras (\emph{heurḗseis}) no escrito
nada de diferente do epigrama (\emph{epigrámmatos}) da tumba de Midas da
Frígia, segundo alguns relatos escritos (\emph{epigegráphthai}).

F: Como é e do que ele trata?

S: É assim:

Eu sou a virgem de bronze que jaz sobre a tumba de Midas,

\begin{quote}
enquanto a água fluir e grandes árvores florescerem,

eu permaneço sobre este túmulo tão chorado,

e anuncio aos que passam que Midas está aqui sepulto.\footnote{Cf.
  Laércio, D. \emph{A} \emph{vida} \emph{dos} \emph{ilustres}
  \emph{filósofos}, § 89.}
\end{quote}

Suponho que percebestes como não há diferença entre o que vem dito antes
ou depois.

{[}264e{]} F: Tu zombas do nosso discurso, ó Sócrates!

S: Deixemo-lo então para não te irritar. Ainda que ele me pareça um
exemplo (\emph{paradeígmata}) àqueles que podem observá-lo com algum
proveito, sem, contudo, imitá-lo na performance, mas vamos para outros
discursos, pois neles há algo, como me parece, que diz respeito aos que
querem conhecer (\emph{ideîn}) e examinar discursos (\emph{perì lógôn
skopeîn}).

{[}265a{]} F: A que tipo de coisas te referes?

S: Meus dois discursos eram como que opostos (\emph{enantíô}), pois um
dizia que é necessário agraciar (\emph{charídzesthai}) ao amoroso
(\emph{tôi erônti}) e o outro ao que não é amoroso.

F: E com que virilidade.

S: Considerei que tu dirias ``com que loucura'' (\emph{manikôs}), que
seria o termo verdadeiro (\emph{talêthès}). Era ele que de fato eu
procurava. Pois dizemos ser o amor (\emph{erôta}) uma loucura
(\emph{manían}) ou não?

F: Sim.

S: Mas há duas (\emph{eîde dúo}) espécies de loucura (\emph{manías}): a
que afeta os homens como uma enfermidade (\emph{nosêmátôn}) e a que os
transporta (\emph{exallagês}) das normas habituais sob a influência da
divindade (\emph{hypò theías}).\footnote{Cf. o discurso de Pausânias no
  \emph{Banquete} 180e, onde duas Afrodites, Pandemia e Urânia, originam
  duas espécies de Eros, um vulgar e outro nobre.}

{[}265b{]} F: Exato.

S: No que diz respeito aos deuses, são quatro as divindades e quatro as
partes pelas quais foram divididas. A Apolo atribui-se a inspiração da
adivinhação (\emph{mantikḕn}), a Dioniso as iniciações
(\emph{telestikḗn}), às Musas a poética (\emph{poêtikḗn}), e a loucura
amorosa (\emph{erôtikḕn manían}), a quarta, que dizemos ser a melhor
(\emph{arístên}), atribui-se à Afrodite e ao Amor. E não sei como, mas
ao apresentar (\emph{apeikádzontes}) a afecção amorosa (\emph{erôtikòn
páthos}), talvez por atingirmos algo verdadeiro (\emph{alethoûs tinos}),
talvez por chegarmos a outros lugares, forjamos um discurso não
totalmente isento de força persuasiva (\emph{apíthanon}) e celebramos
com uma espécie de hino, em algo mítico, {[}265c{]} bem medido e
respeitoso (\emph{metríôs te kaì euphḗmôs}) em nome desse meu e teu
senhor, o Amor, ó Fedro, o guardião dos belos jovens.

F: E a audição não me desagradou.

S: Captemos a partir disso como o discurso muda (\emph{metabênai}) do
vitupério (\emph{pségein}) ao elogio (\emph{epaineîn}).

F: Como dizes?

S: Parece-me tudo isso ser uma brincadeira de criança, mas nessas
afirmações proferidas ao acaso (\emph{ek týchês}) há dois aspectos, e
não será desagradável se deles pudermos captar algo ligado à arte
(\emph{téchnêi}).

{[}265d{]} F: Quais deles?

S: Levar a uma só ideia, a uma visão de conjunto (\emph{sunorônta}), as
muitas coisas que estão dispersas (\emph{diesparména}), para que se
possa tornar evidente (\emph{dêlon}), pela definição
(\emph{horizómenos}), cada tema, sempre que pretendemos ensinar
(\emph{didáskein}), como agora mesmo foi feito com o Amor -- que foi
definido --, quer tenha sido bem ou mal definido, e que proporcionou, ao
mencionarmos o discurso, certa clareza (\emph{saphès}) e concordância
(\emph{homologoúmenon}) consigo mesmo.

F: E o outro aspecto (\emph{eîdos}) de que falas, ó Sócrates?

{[}265e{]} S: Poder novamente (\emph{pálin}) discernir
(\emph{diatémnein}) em espécies (\emph{eídê})\emph{,} segundo as
articulações naturais, procurando não causar roturas em nenhuma parte,
ao modo do cozinheiro inexperiente. Mas que sirvam de exemplo os dois
discursos anteriores, que reuniram a insanidade do pensamento
(\emph{áphron tês dianoías}) a uma ideia comum (\emph{koinêi}
\emph{eidos}). Tal como de um só corpo nascem membros duplos e homônimos
(\emph{diplâ kaì homṓnuma}), chamados sinistros e destros, {[}266a{]}
assim também o discurso nos apresentou uma ideia do desvio do intelecto
(\emph{paranóias}). Um deles, discernindo (\emph{temnómenos}) a sua
parte esquerda, não cessou de novamente (\emph{pálin}) discerní-la
(\emph{témnôn}) enquanto não descobriu (\emph{epheurṑn}) aí uma espécie
de amor denominado sinistro, a quem com toda a razão encheu de censuras
(\emph{eloidórêsen}), e o outro nos levou para a parte destra da loucura
(\emph{manías})\emph{,} homônima àquela (\emph{homônumon mèn ekeínôi}),
mas divina (\emph{theîon}) foi essa parte descoberta (\emph{epheurṑn})
do amor, apresentando-a diante dos nossos olhos e cantando-lhe elogios
(\emph{epḗinesen}), como sendo a causa dos nossos maiores bens
(\emph{agathôn}).\footnote{Aqui vemos uma apresentação da dialética
  patônica, como arte intelectual de reunir e separar (cortar) ideias,
  bem como apresentada no \emph{Sofista} 253b-254c. Nesse trecho os
  elementos formais da lírica estesicórica são relevantes, pois ilustram
  os movimentos complementares do coro, estrofe e antístrofe, aplicados
  a esses contrapostos movimentos intelectuais da dialética.}

{[}266b{]} F: É verdade o que dizes.

S: Eu mesmo sou um amante (\emph{erastḗs}), ó Fedro, dessas divisões e
sínteses (\emph{diairéseôn kaì sunagôgôn}), meio pelo qual é possível
falar e pensar (\emph{légein te kaì phroneîn}). Se considero qualquer
outra pessoa capaz de observar a natureza do uno e do múltiplo, este eu
persigo, ``seguindo seus passos como os de um deus''. Os que são capazes
disso, quer tenha eu os designado bem ou não, deus o sabe, até agora os
referi como dialéticos (\emph{dialektikoús}). {[}266c{]} Mas aos que
aprendem (\emph{mathóntas}) junto a ti e a Lísias, como é necessário que
os designemos? Ou essa não é aquela arte discursiva (\emph{lógôn
téchnê}), segundo a qual Trasímaco e outros sábios manejavam o falar,
proporcionando que outros assim também o fizessem, aqueles que queriam
presenteá-los como se fossem reis (\emph{basileûsin})? \footnote{Cf.
  Homero, \emph{Odisseia} VII, v. 38. A menção a ``Trasímaco e outros
  sábios'', ressalta que eram cultuados e presenteados como reis. A
  metáfora do rei é importante e se completará na conversa entre Theuth
  e Thamous, um deus e um rei. Trasímaco da Calcedônia será citado
  novamente em 267c7-d.}

F: Há de fato uma realeza (\emph{basilikoì}) nesses homens, embora não
conheçam isso que perguntas. Parece-me correta essa forma de dizer, que
esses são chamados de dialéticos (\emph{dialektikòn}), mas parece-me,
todavia, que a retórica (\emph{rhêtorikòn}) ainda nos escapa
(\emph{diapheúgein}).

{[}266d{]} S: Como dizes? Onde poderia existir algo belo que, mesmo
afastado dessas mesmas características, fosse adquirido como uma arte
(\emph{téchnêi})? Em todo caso é preciso que não a desprezemos, tu e eu,
mas que falemos o que ficou de lado sobre a retórica
(\emph{rhêtorikês}).

F: E é bastante vasto, ó Sócrates, aquilo que foi escrito nos livros
acerca da arte discursiva (\emph{en toîs biblíois toîs perì lógôn
téchnês gegramménois}).

S: Bem me recordaste (\emph{hupémnêsas}) disto. Segundo creio,
primeiramente é necessário proferir no início dos discursos o
``proêmio''. É a isso que te referes ou não? A esses refinamentos da
arte?\footnote{Aqui começa um ataque sistemático às concepções
  superficiais dos sofistas acerca da arte da palavra e, ao mesmo tempo,
  uma quantidade de informações relevantes acerca das \emph{téchnai} ou
  tratados especializados de retórica. Podemos considerar o trecho
  266c-267d como uma pequena história da retórica, pois apesar da
  perspectiva crítica de Platão, há nele elementos importantes de
  reconstrução histórica. A expressão ``refinamentos da arte'' (\emph{tà
  kompsà tês téchnês}, 266d9) é um indício que Sócrates referirá
  rudimentos, nada refinados na verdade, como por exemplo o fato de
  começar um discurso pelo ``proêmio'', coisa que Lísias, nesse retrato,
  não fez, e nem poderia ter feito, pois Fedro trazia apenas a
  peroração, o trecho final, daquele discurso. Nas \emph{Leis} 722b-723d
  o ateniense esboça alguns princípios necessários às leis,
  especialmente considerando a importância dos proêmios.}

{[}266e{]} F: Sim.

S: Em segundo lugar vem a ``narração'' (\emph{diḗgesín}) e alguns
``testemunhos'' (\emph{marturías}) que lhes dizem respeito, em terceiro
lugar a ``prova'' (\emph{tekmḗria}) e em quarto as ``verossimilhanças''
(\emph{eikóta})\emph{.} Também existe, segundo creio, a ``confirmação''
(\emph{pístôsin}) e a ``confirmação suplementar'' (\emph{epipístôsin}),
nos dizeres do excelente burilador de discursos
(\emph{logodaídalon})\footnote{O epíteto de \emph{logodaídalon}, nesse
  passo, ganha diferentes traduções, entre as mais comuns estão:
  \emph{cinzelador de discursos, burilador de discursos, Dédalo da
  palavra}. Cf. Hesíodo \emph{Trabalhos e Dias}, v.64, lugar em que a
  expressão \emph{poludaídalon} (polidedáleo) refere a tecedura de
  Atena.}, o homem de Bizâncio.

F: Mencionas o auspicioso Teodoro?\footnote{Em Aristóteles, Teodoro é
  considerado um inovador da linguagem, um criador de novidades
  (\emph{tà kainà}) (Arist. \emph{Rh}. III,11,6). O diagrama de
  conceitos aplicados por Teodoro à arte discursiva não coincide
  totalmente em Platão e Aristóteles, uma vez que o último menciona
  apenas as seguintes ``invenções'' de Teodoro: a narração
  (\emph{diegesis}), a narrativa suplementar (\emph{epidiegesis}), a
  narrativa prévia (\emph{prodiegesis}), a refutação (\emph{elegchos}) e
  a refutação suplementar (\emph{epeksélegchos}) (Arist. \emph{Rh}.
  III,13,5).}

{[}267a{]} S: Quem mais senão ele, o qual disse haver nas composições
uma ``refutação'' (\emph{elegchón}) e uma ``refutação suplementar''
(\emph{epexélegchon}), tanto na acusação como na defesa
(\emph{katêgoríai te kaì apologíai}). E o belíssimo Eveno de Paros, não
o traremos para o debate? Ele foi quem primeiro inventou (\emph{hêuren})
as ``insinuações'' (\emph{hypodḗlosin}) e os ``para-elogios''
(\emph{parepaínous})\emph{.} Dizem que ele compôs ``para-vitupérios''
(\emph{parapsógous}) em versos, para auxiliar a memória (\emph{mnḗmes
chárin}). Foi, portanto, um homem sábio (\emph{sophòs gàr
anḗr}).\footnote{Na \emph{Apologia} (Pl. \emph{Ap}. 20a-c), Sócrates
  nega que ele próprio cobrasse para ensinar, para tanto lança mão dos
  mais ilustres sofistas como exemplos, dizendo que esses homens sim
  persuadiam os jovens a não mais frequentarem encontros públicos, nos
  quais habitualmente aprendiam, e a passarem a realizar seus estudos no
  âmbito particular. Sócrates satiriza Cálias, filho de Hipônico, por
  ter gasto somas incríveis com o sofista Eveno de Paros, que cobrava
  cinco minas por suas aulas. No \emph{Fédon} (Pl. \emph{Phd}. 60d-61d),
  Cebes diz a Sócrates que Eveno de Paros o havia indagado acerca do
  motivo pelo qual Sócrates havia transposto para o metro os contos de
  Esopo e o proêmio ao {[}hino{]} a Apolo, depois de ter sido
  encarcerado. Sócrates pede a Cebes que diga a verdade a Eveno, que não
  havia feito aquilo por desejar fazer-lhe concorrência, uma vez que
  Eveno era considerado poeta e rétor, mas por seguir um sonho que teve
  que o exortava a compor música.} E Tísias\footnote{Juntamente com
  Córax, Tísias é frequentemente lembrado como um dos inventores da
  retórica siciliana. No \emph{Fedro}, Tísias tem um homônimo famoso, o
  próprio Estesícoro de Himera, que, embora não mencionado como
  ``Tísias'', cumpre papel especial no \emph{Fedro.} Embora Estesícoro,
  ``o estabelecedor do coro'', não seja tão conhecido como Tísias, seu
  nome original, Platão, a partir da homonímia, estabelece um jogo
  alusivo, quase imperceptível, que faz com que poética e retórica
  apareçam atreladas, também no nome de seus patronos, Estesícoro
  (Tísias) para a poética e Tísias para a retórica. Cf. Suda,
  \emph{Lexicographi Graeci,} sigma, 1095 (iv 433 Adler): ``Ele foi
  chamado de Estesícoro, porque foi o primeiro a estabelecer coro na
  citaródia, aquele que antes era designado como Tísias'' (ἐκλήθη δὲ
  Στησίχορος, ὅτι πρῶτος κιθαρῳδίᾳ χορὸν ἔστησεν· ἐπεί τοι πρότερον
  Τισίας ἐκαλεῖτο).} e Górgias\footnote{Capítulo à parte na história do
  pensamento ocidental, Górgias de Leontino (485-385 a.C) foi rétor,
  sofista, sendo uma imagem arquetípica da atividade, juntamente com
  Córax, Tísias, Protágoras, Trasímaco, Hípias e outros. Talvez não seja
  exagero considerá-lo como um dos mais famosos, especialmente porque
  dele, ao contrário de outros sofistas, temos, além do retrato legado
  por Platão, tratados exemplares da sua própria ``arte'' discursiva (DK
  82 p. 271-307). Cf. também o \emph{Górgias} de Platão, em que um
  retrato desse sofista é traçado (Pl. \emph{Gor}. 449c9-461b2).}, vamos
deixá-los dormindo, eles que souberam honrar mais a verossimilhança do
que a verdade, que pela força discursiva fizeram o grande parecer
pequeno e o pequeno parecer grande, {[}267b{]} o novo parecer arcaico,
bem como o seu contrário, o arcaico parecer novo, e que acerca de todos
os assuntos encontraram (\emph{anêuron}) a concisão discursiva e seu
prolongamento indefinido? \footnote{Cf. sobre Górgias o estudo de
  Untersteiner \emph{I sofisti} (Milano: Mondadori, 1996, p.141-306), o
  de Mazzara, G. (\emph{Gorgia, la retorica del verosimile}, Verlag,
  1999) e o de Cassin (\emph{L'Effet sofistique}, Paris: Galiimard,
  1995). Quanto ao fazer parecer pequeno o grande e vice-versa pelo
  discurso, cf. Isócrates (\emph{Panegírico} 8), quanto à braquiologia e
  à macrologia, cf. Platão (\emph{Górgias}, 449b10-c6, \emph{Protágoras}
  335b-c) e quanto ao fazer parecer velho o novo e vice-versa, cf.
  Isócrates (\emph{Panegírico} 8), Aristófanes (\emph{Nuvens} v. 896) e
  Aristóteles (\emph{Rh}. III, 6).} Ouvindo isso de mim, outrora,
Pródico sorriu e disse que somente ele havia descoberto
(\emph{heurêkénai}) o que é preciso na arte discursiva, discursos que
não sejam nem longos nem curtos, mas na medida
(\emph{metríôn}).\footnote{Pródico de Ceos (470/60-395 a.C.), famoso
  sofista, amplamente citado em Platão (Pl. \emph{Prt}.315c-d,
  \emph{Hp.Ma}.282c, \emph{Tht}.151b, \emph{Ap}.19e, \emph{Cra}.384b,
  \emph{Prt}.337a, 340a, \emph{Euthd}.277e-278a), escreveu um discurso
  em prosa chamado \emph{Horai} (\emph{Estações}), trecho que devemos a
  Xenofonte uma paráfrase (\emph{Mem}. 2,1, 21-34), na qual narra uma
  cena do jovem Héracles sendo visitado pela Virtude e pela Maldade
  personificadas, sendo que a última lhe promete uma vida de prazeres,
  uma imagem da felicidade suprema, enquanto a primeira, ao contrário,
  defenderá um caminho mais longo e uma felicidade atrelada à virtude,
  conquistada pela disciplina e pela moderação (cf. também DK 84,
  p.308-319). Essa medida (\emph{metríon}) de que se gabava Pródico,
  encontrada já em Protágoras (\emph{métron} DK, 80, B1), é bastante
  presente na abordagem filosófica de Platão.}

F: Ó Pródico, sapientíssimo!

S: Não falamos ainda de Hípias?\footnote{Hípias de Élis (460-400 a.C.),
  sofista de saber enciclopédico, foi embaixador, pregava a autarquia e
  desenvolveu uma menemotécnica espantosa. Platão dedicou dois diálogos
  a esse sofista (\emph{Hípias Maior} e \emph{Hípias Menor}). Cf. DK 86,
  p.326-334.} Creio eu que o estrangeiro de Élis votaria também com
Pródico.

F: E por que não?

S: E o que diremos do \emph{Museu Discursivo} de Polo? {[}267c{]} Com
sua duplicação discursiva (\emph{diplasiologían})\emph{,} coleção de
máximas (\emph{gnômologían}) e estilo imagético
(\emph{eikonologian}).\footnote{Polo de Agrigento figura como discípulo
  de Górgias no diálogo homônimo de Platão e trava um segundo embate com
  Sócrates (Pl. \emph{Grg}. 461b3-481b5). Cf. a excelente tradução ao
  \emph{Górgias} feita por Lopes, D. R. N., bem como a introdução que
  faz ao efeito cômico que Polo cumpre no diálogo, o qual chama de 2º
  Ato (\emph{Górgias}, trad. Lopes, D., São Paulo: Perspectiva, 2011,
  p.41-79).} E do \emph{Vocabulário} que Licímnio\footnote{Licímnio de
  Quios é citado em Aristóteles (Arist. \emph{Rh}. III, 1405b13; 1413b2
  e 1414b5) como conhecedor de sons e significados aplicados às
  metáforas, compositor de ditirambos e de uma arte retórica (τῇ τέχνῃ),
  na qual tratou de procedimentos como extender o discurso, realizar
  divagação (\emph{apoplánêsin}) e ramificação (Arist. \emph{Rh}.
  1414b5).} havia presenteado Polo, em vista da composição do seu belo
falar?

F: E de Protágoras\footnote{Protágoras de Abdera (492/0-422 a.C.), um
  dos mais ilustres sofistas, tem um pensamento de complexa reconstrução
  através da doxografía (DK 80, p. 253-271). Autor da célebre frase ``o
  homem é a medida de todas as coisas, das que são, enquanto são, das
  que não são, enquanto não são'' (Pl.\emph{Tht}.152a1-4), foi acusado
  de ateísmo por ter dito que acerca dos deuses ``não poderia dizer se
  existem ou não'' (Pl.\emph{Tht}.162d5-e2). Sustentou que em todos os
  assuntos existem dois argumentos que se contrapõem (D.L. 9,51), além
  de, segundo Aristóteles, ter desenvolvido o argumento mais forte e o
  mais fraco e ter ensinado que a verossimilhança aparente
  (\emph{phainómenon eikós}) é usada na retórica e na erística (A.
  \emph{Rh}. 1402a11).}, ó Sócrates, não há nada desse tipo?

S: A \emph{Dicção Correta} (\emph{Orthoépeia}) é uma delas, ó jovem,
entre muitas outras e belas composições. E dos discursos piedosos
escritos sobre a velhice e a pobreza, o que me parece dominar pela arte
é o do grande Calcedônio\footnote{Trasímaco da Calcedônia (459-400 a.
  C.), sofista muito conhecido, especialmente por Platão ter dedicado a
  ele a famosa conversação com Sócrates no início da \emph{República}
  acerca da justiça, a qual segundo ele seria ``a conveniência do mais
  forte'' (Pl. \emph{Rep}. 338c1-2). Segundo Aristóteles, Trasímaco
  teria inovado ao aplicar ritmos nas suas composições logográficas (Ar.
  \emph{Rhet}. 1409a1-27), assim como na aplicação de discursos
  ``\emph{Piedosos}'' (\emph{oîon Thrasúmachos en toîs Eléois} - Ar.
  \emph{Rhet}. 1404a15).}, homem terrível que a muitos enfurecia e, em
seguida, novamente, conduzia pelos encantos da palavra (\emph{epáidôn
kêleîn}), a dissipar essa fúria, como ele mesmo dizia. {[}267d{]}
Fortíssimo ele era em gerar e destruir qualquer tipo de calúnia. Quanto
à conclusão dos discursos parecem estar todos em comum acordo, embora
alguns a chamem de peroração (\emph{epánodon}) e outros estabeleçam
outro nome.

F: Aludes à recordação (\emph{hypomnêsai}) de cada um dos pontos
capitais, no final do que foi dito aos ouvintes?

S: Falo disso mesmo, e se tens algo mais a dizer acerca da arte
discursiva...

F: Insignificâncias indignas de menção.

{[}268a{]} S: Deixemos de lado as insignificâncias e, à luz do sol,
vejamos melhor que potencialidade (\emph{dúnamin}) tem quem detém essa
arte.

F: Muita é sua força, ó Sócrates, sobretudo nas reuniões populares
(\emph{plḗthous} \emph{sunódois}).

S: Tem mesmo, mas, ó divino, vê também se essa urdidura não te parece
frouxa, como a mim me parece.

F: Mostra-me.

S: Diz-me, se alguém chegasse a teu amigo Erixímaco ou a seu pai Acúmeno
afirmando o seguinte: ``Eu conheço aplicações para aquecer o corpo ou,
se desejar, resfriá-lo, e se me parecer adequado fazê-lo vomitar ou, ao
contrário, evacuar, além de muitos outros tantos efeitos semelhantes.
{[}268b{]} Tendo conhecimento disso, considero-me um médico capaz de
fazer com que outros assim procedam, transmitindo tais saberes''. O que
pensas que os seus ouvintes, nesse caso, diriam?

F: Que outra coisa, senão perguntar se ele sabe em quem e quando é
preciso aplicar cada um deles, e também em que quantidade?

S: Se então dissesse: ``De modo algum, mas considero que aquele que
junto a mim aprender essas coisas, poderá fazer o que
perguntas''.\footnote{Nesse trecho em que Sócrates propõe esses diálogos
  imaginários, fica claro que pretende mostrar que os pseudo-artífices,
  que manejavam apenas rudimentos das artes, mesmo afirmando que não
  conheciam a arte, nesse caso a medicina, ainda assim eram capazes de
  afirmar que outros poderiam com eles aprender algo. Se por um lado,
  esse é um recurso que tem como objetivo mostrar a que ponto eram
  ignorantes esses ``homens imaginários'', ao interpelarem os
  verdadeiros mestres das artes com tamanha grosseria, por outro lado
  introduz também elementos muito valiosos que serão retomados à frente,
  relativos à medicina hipocrática e, por analogia, à dialética
  filosófica, especialmente em quem se deve aplicar uma arte ou fármaco,
  em que ocasião e em que quantidade.}

{[}268c{]} F: Poderiam dizer, creio eu, que esse homem estivesse louco,
pois só por ter colhido de algum livro\footnote{Ao expor a pseudo-arte,
  ou arte inacabada, Sócrates dá um indício de uma de suas causas, senão
  da principal, algo que será retomado na alegoria de Thamous, o fato
  desses conhecimentos superficiais terem sido colhidos nos livros, sem
  a transmissão que só o discurso vivo é capaz.} ou por calhar de
conhecer alguns fármacos, considera-se um médico, sem nenhum
conhecimento da arte.

S: E o que diria, se alguém chegasse junto a Sófocles e Eurípides,
dizendo saber compor grandes falas sobre temas insignificantes e sobre
temas grandiosos falas curtas, e quando quisesse falas piedosas ou, ao
contrário, terríveis e ameaçadoras, e tantas outras desse tipo. E ainda
que, com tais ensinamentos, considerava-se capaz de transmitir a própria
arte da composição de tragédias?

{[}268d{]} F: E esses também, ó Sócrates, creio que ririam, se alguém
considerasse ser a tragédia outra coisa senão a adequação
(\emph{prépousan}) dos elementos entre si e com o todo (\emph{tôi
holôi}) da composição.\footnote{Nesse segundo diálogo imaginário, cujos
  mestres agora são Sófocles e Eurípides, alguns recursos da retórica
  são mencionados, como as falas piedosas, \emph{oiktrás}, ligadas a
  Trasímaco (\emph{oiktrogóôn}, 267c7). Aristóteles usa o termo
  \emph{eléois,} ``piedosos'' como um título de uma obra de Trasímaco
  (cf. Ar. \emph{Rhet}.1404a15). Outro fundamento dessas artes é a
  adequação (\emph{prépousan}) entre as partes e todo, que será também
  importante na medicina hipocrática (270c9-d7) e na retórica
  (271a4-b5).}

S: Penso eu que tais censuras não seriam grosseiras, tal qual o músico
que encontra uma pessoa que se considera um harmonista, só porque lhe
ocorreu aprender a fazer soar uma corda aguda e grave. O músico não
seria grosseiro dizendo: ``Ó desgraçado, melancólico'', mas por ser
músico, gentilmente diria: ``Ó meu querido, para quem deseja vir a ser
um harmonista, esses conhecimentos são necessários, mas ninguém adentra
nem aprende o mínimo da harmonia, só por possuir essa tua disposição.
Conheces então os saberes prévios necessários à harmonia (\emph{prò}
\emph{harmonías}), mas não a harmonia propriamente dita (\emph{tà
harmoniká})''.

F: Corretíssimo.

{[}269a{]} S: Então Sófocles e Eurípides mostrariam que aquilo era
apenas um rudimento para a composição da tragédia (\emph{prò
tragoidías),} mas não a arte trágica propriamente dita (\emph{tà
tragiká),} bem como Acúmeno diria que apresentavam os saberes prévios
necessários à medicina (\emph{prò} \emph{iatrikês}), mas não detinham a
medicina mesma (\emph{tà iatriká}).\footnote{Esse último diálogo
  imaginário, em que o músico recebe alguém que acredita ser um
  harmonista, desvela a dicotomia desse e de todos os casos anteriores,
  entre alguém que tem rudimentos prévios (\emph{pró}) das artes, mas
  ainda não é capaz de compreender os fundamentos completos das
  respectivas artes, da medicina, da arte trágica (arte de compor
  tragédias) e da harmonia musical. Na sequência Asdrasto e Péricles
  serão mencionados, também Sócrates cogitando o que estes diriam para
  ambos, também num diálogo imaginário, mas nesse caso outros elementos,
  para além dessa dicotomia fundamental entre rudimentos das artes e
  artes completas, serão aventados.}

F: Sem dúvida.

S: E o que pensaremos de ``Adrasto voz de mel'' ou de Péricles, se
ouvissem o que agora mesmo percorremos sobre todas as artes, acerca dos
discursos breves (\emph{brachylogiôn})\emph{,} dos estilos imagéticos
(\emph{eikonologiôn}) e tantos outros a que chegamos, os quais já
mencionamos e foram verificados à luz do dia? {[}269b{]} Qual deles
seria o mais cruel, assim como eu e tu, ao falar mal, pela própria
rusticidade (\emph{hup' agroikías}), desses escritores e professores da
arte retórica? Ou, sendo mais sábios que nós, eles nos reprovariam
dizendo: ``Ó Fedro e Sócrates, não é preciso ser odioso para com eles,
mas desculpá-los, se alguns que não conheceram a dialética se tornaram
incapazes de definir o que é a retórica, e por essa mesma condição
consideraram ter descoberto a arte retórica, quando apenas eram
detentores de conhecimentos prévios necessários à arte. {[}269c{]} Por
ensinarem tais coisas a outros, eles se consideram perfeitos professores
de retórica, embora dizer em cada ocasião o persuasivo e arranjar o todo
no discurso não lhes pareça um trabalho especial, de modo que seus
aprendizes precisam por si próprios adquirir essas habilidades nos
discursos''. \footnote{Cf. em 267c1 \emph{eikonologían} ou ``estilo
  imagético'' atribuído a Polo e, em 267b4, o termo \emph{brachéôn}
  negado por Pródico em nome daquilo que chama de \emph{metríôn} ou
  medida adequada de um discurso no que tange à extensão. Os recursos da
  retórica referidos aqui foram especialmente descritos, ainda que de
  forma realtivamente superficial e irônica em 266d7-267d6. Adrasto e
  Péricles, agora como interlocutores imaginários de Fedro e Sócrates,
  seriam de alguma maneira complacentes com os retores vulgares ou
  ignorantes, mesmo que isso se dê com ironia, ao supostamente afirmarem
  que, por não conhecerem a dialética, aqueles retores vulgares deveriam
  ser desculpados, pois não seriam capazes de definir a retórica, e esse
  era o motivo pelo qual acreditavam, erroneamente, conhecer seus
  fundamentos, e dessa mesma ignorância derivava também a crença de
  serem capazes de ensinar isso a outros. O resultado dessa constatação
  retoma a conclusão dos diálogos imaginários anteriores, na medida que
  esses retores vulgares também só teriam os rudimentos da arte, não a
  arte completa. A imagem que Platão sugere é similar à descrita no
  \emph{Górgias}, ou seja, eles realizavam apenas um experimento e um
  treino, mas supunham ensinar uma arte, a qual, na verdade, nem mesmo
  conheciam em seus fundamentos (cf. Pl. \emph{Górgias} 463b3). A prova
  disso está no final do trecho, quando, agravando ainda mais a situação
  precária desses infelizes ignorantes, estes nem mesmo aparecem como
  capazes de ensinar os ditos rudimentos da retórica, nem na eficácia
  persuasiva e nem no arranjo (\emph{dispositio}) do discurso, uma vez
  que seus discípulos teriam de aprender ambas as coisas por si mesmos.
  Adrasto parece referir o rei de Argos, que em Tirteu tem a mesma
  característica do trecho, um discurso doce, uma voz doce, remetendo ao
  universo da poética (cf. West 12,8 \emph{glôssan d'Adréstou
  meilichógêrun}). Péricles é também exemplo da eficácia oratória e será
  referido a seguir como discípulo de Anaxágoras de Clazômenas.}

F: Ó Sócrates, isso parece ser aquela arte retórica que os homens
ensinaram e sobre a qual escreveram, e me parece ser proclamada
verdadeira. Mas a arte da retórica propriamente dita e sua
credibilidade, {[}269d{]} como e a partir de onde seria possível
alcançá-la?

S: Para ser capaz, ó Fedro, de se tornar um competidor
(\emph{agonistḕn}) perfeito é verossímil (\emph{eikós}) -- talvez
necessário --, que detenhas outras coisas. Se está na tua natureza
(\emph{phýsei}) ser um retórico, serás um rétor consumado, acrescendo a
isso conhecimento (\emph{epistḗmen}) e exercício (\emph{melétên}). Se
deixares de lado qualquer um desses elementos, serás
imperfeito.\footnote{Esses três aspectos necessários ao orador,
  \emph{phýsis}, \emph{epistémen} e \emph{meleten} (natureza,
  conhecimento e exercício) repercutem séculos à frente em Siriano, que
  faz uma sinopse da retórica platônica através da passagem referida em
  seu comentário às \emph{Stasis} de Hermógenes (\emph{Commentarium in
  Hermogenis librum} perì staseôn, vol. 2, Ed. Rabe, H. Leipzig:
  Teubner, 1893, p.4). O mesmo trecho é citado no \emph{Pròs Platona
  perì rethorikês} de Élio Aristides (Ed. Dindorf, W. Leipzig: Reimer,
  1829, Repr. 1964. p.114).} Tal é a arte, a qual não me parece evidente
atingi-la através do método de Lísias e de Trasímaco.

F: E por que modo então?

S: É possível, ó querido, que Péricles tenha se tornado de modo
verossímil (\emph{eikótôs}) o maior perito na arte retórica.\footnote{Sutil
  e repetido jogo entre retórica e verossímil, especialmente porque a
  retórica no \emph{Fedro} é, muitas vezes, explicada como a arte de
  persuadir o auditório a partir do verossímil (\emph{eikótôs}), a qual
  seria uma definição praticamente canônica. Além disso, não será demais
  lembrar que, para Isócrates, Péricles foi o mais democrata de todos os
  democratas.}

F: E por quê?

S: Todas as grandes artes requerem a tagarelice e a meteorologia
(\emph{adoleschías} \emph{kai} \emph{meteôrologías phýseos péri}) acerca
da natureza, pois parece que é justamente delas que se pode adquirir
sublime pensamento (\emph{hypselónoun}) e perfeição. Péricles foi capaz
de adquiri-las, além da sua inclinação natural (\emph{euphyḕs}), e me
parece que envolto com Anaxágoras, pleno daquela meteorologia, chegou à
natureza do intelecto e da sua ausência (\emph{noû te kaì
anoías}),\footnote{Alguns tradutores preferem a transcrição de Burnet
  para o trecho (\emph{noû te kaì dianoías}), proveniente do códice V
  (Vindobonensis 109), que determinaria a tradução por ``intelecto e
  pensamento'' ou algo similar. Na verdade, os códices B (Bodleianus,
  MS. E.D. Clarke 39) e T (Venetus Append. Class. 4, cod.1) trazem um
  par de opostos, ``\emph{noû te kaì anoías''}, muito mais compatível
  com o \emph{Fedro}, diálogo em que essas contraposições são
  recorrentes (cf. \emph{noûn} e \emph{anoétou} em 241a8-b1;
  \emph{anoétôi} e \emph{noûn} em 241b7-c1). Também no \emph{Timeu} 92c2
  há uma oposição idêntica (\emph{noû} \emph{kaì} \emph{anoías}), mas
  nunca foi considerada para justificar a escolha pela transcrição que
  traz o par de opostos em \emph{Phdr}. 270a5. Sigo os tradutores que
  optam por essa oposição, como Nunes ``compenetrado da natureza do que
  é inteligente e do que carece de inteligencia'', Chambry ``pénétra la
  nature de ce qui est intelligent et de ce qui ne l'est pas'', Rowe
  ``arrived at the nature of mind and absence of mind'', Santa Cruz y
  Crespo ``la naturaleza de la inteligencia y de la falta de ella'' e
  Poratti ``la naturaleza de la inteligencia y de la falta de
  inteligência''. Rowe, concordando com De Vries, afirma que a correção
  de Burnet deve ser desconsiderada, por introduzir uma redundância
  insustentável (Cf. Rowe, \emph{Commentary} \emph{In Phaedrus} 270a-c,
  p. 205, e De Vries, \emph{A commentary on the Phaedrus of Plato}, p.
  234). Hermias (§194.15-18 Couvreur = Lucarini e Moreschini 244,20-24)
  foi quem primeiramente explicou o sentido desse par de opostos: ``o
  \emph{intelecto e a ausência de intelecto}. Tudo que foi realizado e
  trazido pelo intelecto constitui o que é belo, as demais coisas estão
  sob a ação da matéria, por isso ele diz ausência de intelecto
  (\emph{ánoian}) para o que é material e para o que tem essa
  necessidade. Acerca disso diz, sobretudo, que essa busca visava
  atingir os inteligíveis e as questões materiais''. (Τὸ δὲ
  \textless{}νοῦ τε καὶ ἀνοίας\textgreater{}· πάντα γὰρ τὰ καλῶς
  δεδημιουργημένα ὑπὸ τὸν νοῦν ἀνέφερε, τὰ δὲ ὡς ἄλλως ὑπὸ τὴν ὕλην, ἵνα
  ἄνοιαν λέγῃ τὴν ὕλην καὶ τὸ κατηναγκασμένον· μᾶλλον δὲ τοῦτο λέγει ἐπὶ
  τὴν ἔρευναν τῶν νοητῶν καὶ ὑλικῶν πραγμάτων
  \textless{}ἀφικόμενος\textgreater{}).} tal qual Anaxágoras apresentava
em muitos dos seus discursos, e a partir disso Péricles forjou
utilidades para sua arte discursiva (\emph{enteûthen eílkysen epì tèn
tôn lógôn téchnên tò prósphoron autêi}).\footnote{Essa tagarelice,
  \emph{adoleschías} (\emph{garrulitas}), aparece associada a um tipo de
  discursividade, muitas vezes considerada excessiva, acerca das coisas
  que vêm do céu (\emph{tà metéora}), dos fenômenos celestes, de modo
  que o par \emph{adoleschías} \emph{kai} \emph{meteorologias} não é
  exclusivo a esse trecho do \emph{Fedro} 270a. (cf. \emph{Polit.} 299b7
  μετεωρολόγον, ἀδολέσχην τινὰ σοφιστήν e \emph{Crat}. 401b7
  μετεωρολόγοι καὶ ἀδολέσχαι τινές; \emph{Resp}. 488e-489a μετεωροσκόπον
  τε καὶ ἀδολέσχην, nesse caso a especulação acerca do que vem do céu,
  \emph{metéora}, não é uma \emph{logia}, um estudo, mas um
  \emph{skopia}, uma observação). Péricles parece cumprir os requisitos
  antes mencionados como necessários ao rétor, pois tem uma natureza
  propícia (\emph{euphyès}) e, além disso, parece ter sido capaz de
  realizar exercícios e de obter conhecimentos. O mais interessante é
  que foi capaz de transpor para a arte discursiva o que havia aprendido
  da astronomia de Anaxágoras. Um ponto obscuro é como seria essa
  transposição, essa forja, esse empréstimo, entre as artes, como se
  transportariam conhecimentos de uma área para outra, da meteorologia
  para a retórica. Acerca disso temos apenas algumas pistas lexicais,
  como por exemplo o verbo \emph{élkô}, de onde vem o aoristo
  \emph{eílkysen}, que descreve o sulco, a ferida, a forja, a marca de
  um camino dilacerado, de um novo corredor escavado entre um campo do
  conhecimento e outro, efetivando a extensão entre dois universos antes
  incomunicáveis. Themistio, parafraseando o mesmo trecho, mantém o
  verbo \emph{élkô} ``\emph{ek tês Anaxagoroû adoleschías tauta
  proseilkýsato eis tèn téchnen''} (Cf. Them. \emph{Or}.
  \textless{}\emph{Hyper toû légein e pôs tô philosophô
  lektéon}\textgreater{} In \emph{Themistii orationes quae supersunt,}
  vol. 2, Ed. Schenkl, H., Downey, G., Norman, A.F. Leipzig: Teubner,
  1971, p. 329 c7).}

F: Como dizes?

S: {[}270b{]} É que o recurso técnico (\emph{trópos téchnês}) da
medicina é similar ao da retórica.\footnote{Fedro não entende bem o que
  Sócrates diz acerca dessa relação entre meteorologia e retórica, e a
  analogia que Sócrates apresentará em seguida não responderá
  explicitamente à analogia anterior, uma vez que a analogia entre
  medicina e retórica estará pautada na dialética. É possível
  subentender que a dialética seria o fundamento da analogia anterior,
  especialmente porque o pensamento hipocrático, evidenciado aqui,
  também já havia estabelecido uma relação bastante marcante entre os
  fenômenos climáticos, astronômicos e a saúde do corpo (cf. Hipp.
  \emph{Aër}). Ao apresentar uma versão mais palatável dos fundamentos
  da dialética a partir da medicina, fica evidente que Fedro não
  acompanha as especulações mais ``elevadas'', colhidas na meteorologia
  e astronomia de Anaxágoras.}

F: Como assim?

S: Em ambas é preciso dividir a natureza (\emph{phýsin}), numa delas a
do corpo, noutra a da alma. Se pretendes, por um lado, fornecer fármacos
e alimento para a saúde e para a força, e, por outro, discursos e
estudos prescritos como úteis à persuasão desejada e à virtude
transmitida, isso não só aconteceria por treino e experiência, mas por
arte (\emph{mḕ tribêi mónon kai empeiríai allà téchnêi}).\footnote{Cf.
  Pl. \emph{Górgias} 463b3 expressão similar acerca da retórica, segundo
  a qual ela não seria uma arte, mas apenas experimento e treino
  (\emph{oúk estin téchnê all'empeiría kaì tribé}), além de \emph{Gorg}.
  465a3, 500e5-501a1, \emph{Phil}. 55e6, \emph{Leis} 720b-c5, 938a.}

F: É verossímil (\emph{eikós}) que seja assim, ó Sócrates.

{[}270c{]} S: E tu consideras ser possível compreender o valor da
natureza da alma sem, contudo, compreender a natureza do todo (\emph{tês
toû hólou phýseôs})?

F: Se devemos acreditar em Hipócrates, que pertence aos Asclepíades,
quando diz que sem esse método nem mesmo o corpo seria possível
conhecer.

S: Belo dizer, ó companheiro, mas é necessário, todavia, examinarmos
isso em vista do discurso (\emph{tòn lógon}) de Hipócrates e
verificarmos se estamos de acordo.

F: É o que digo.

S: Verifica então agora o que diz Hipócrates e a verdadeira razão
(\emph{alêthḕs lógos}) acerca da natureza (\emph{perì phýseos}). Não é
preciso, para compreender a natureza de qualquer coisa, primeiramente
verificar se ela é simples ou de múltiplas formas, sobretudo se
desejaremos nós mesmos sermos os artífices (\emph{technikoì}) e capazes
de transmitir isso a outros, em seguida, se forem simples, verificar sua
potencialidade, saber em que tipo de relação sua natureza produz uma
determinada ação ou pelo que é afetada por algo externo. Se são
múltiplas as suas formas, estas devem ser também enumeradas e, tal qual
a unidade, devem ser observadas, cada uma delas, em que sua natureza
pode produzir ou ser afetada por algo externo? \footnote{A expressão
  \emph{alêthès lógos} foi estudada especialmente por Fattal, M.
  \emph{L'alethès lógos du Phèdre en} 270c10 \emph{In} Rosseti (1992, p.
  257-60). Esse \emph{acerca da natureza} (\emph{perì phýseos}) parece
  retomar o \emph{da natureza do todo} (\emph{tês toû hólou phýseôs}) de
  270c2, momento em que a compreensão da totalidade foi apresentada como
  fundamental para a compressão do particular, seja do corpo, seja da
  alma. Parece haver aqui uma menção ao tratado de Hipócrates \emph{Da
  natureza do homem} (\emph{Perì Anthópou phýseôs}), pelo menos Galeno,
  em seu comentário a esse mesmo tratado de Hipócrates (\emph{In
  Hippocratis de natura hominis commentarius} 15.4-5) parafraseia esse
  trecho do \emph{Fedro} 270c-d como exemplo do conceito de todo
  (\emph{tò hólon}), que Platão, segundo Galeno, teria aludido na
  passagem. Littrè (1839, pp. 294-300), ao contrário, acreditava ser uma
  menção ao \emph{Da medicina antiga,} ao passo que Jouanna (1992, pp.
  88-9, 97) não concorda com essa interpretação de Littrè. Para detalhes
  acerca desse ponto cf. Cairus, H. \emph{Da natureza do homem Corpus
  hippocraticum} (1999, p. 395-430) e Cambiano G. \emph{Dialletica,
  medicina, retorica nel Fedro platônico} (1966, p. 284-305). Para
  Hermias, questão principal do trecho é ``o método da divisão''
  (\emph{têi diairetikêi methódôi}, §194,14), um dos movimentos da
  dialética.}

F: É possível, ó Sócrates.

{[}270e{]} S: Sem esse método pareceríamos fazer uma travessia de cego
(\emph{typhloû poreíai})\footnote{Mais uma vez, a oftalmia de
  Estesícoro, curado pelo próprio canto, deve ser entendida como uma
  cura intelectual, uma salvação poética (recitativa) para a alma, um
  encanto que move, que libera de um miasma, que afasta da doença.}, mas
não devemos comparar um cego ou um surdo àquele que persegue com arte
alguma coisa, como é evidente a quem ofereça a arte discursiva e mostre,
de forma acurada, a essência da natureza para quem os discursos serão
oferecidos. E essa será, sem dúvida, a alma.

F: Seguramente

{[}271a{]} S: Então, o combate se estende por toda alma, pois a
persuasão (\emph{peithṑ}) é produzida nela, ou não?

F: Sim.

S: É evidente que Trasímaco e tantos outros dos que zelosamente nos
legaram tratados da arte retórica, inicialmente e com muito cuidado
inscreverão e produzirão uma visão da alma, em seguida observavão se sua
natureza é una e semelhante, ou se, como o corpo, tem múltiplas formas.
Dizemos que mostrar a natureza de algo é isso.

F: É assim mesmo.

S: Em segundo lugar, é preciso entender o que sua natureza pode produzir
e pelo que é afetada.

F: Seguramente.

{[}271b{]} S: Em terceiro lugar, é preciso realizar uma disposição entre
gêneros de discursos e almas, bem como todas as causas que as afetam,
harmonizar cada qual ao seu correspondente e ensinar por quais causas,
necessariamente, alguns são persuadidos por determinados discursos e
outros não são persuadidos.\footnote{Há uma sutil ligação entre as
  ocorrências ligadas à persuasão e à crença (\emph{peíthêi} 299c5,
  \emph{apistoíên} 299c6, e \emph{apistôn} 299e) do começo do diálogo
  com esse trecho, no qual transparece também a oposição persuasivo e
  não-persuasivo (\emph{peíthetai, apeitheî} 271b4-5). Cf. Chantraine
  (1968, p. 868-869) a descrição do campo semântico de \emph{peíthomai}
  e \emph{pístis}. É possível entender esse trecho como o início de um
  êxodo do diálogo, pois aqui se apresenta uma espécie de síntese de
  todas as facetas da arte discursiva, agora já apropriada da dialética,
  embora ambas, a retórica vulgar e a retórica filosófica, tenham em si
  a potencialidade, mais ou menos apurada, de conhecer e, portanto, de
  conduzir as almas. Esses três passos elencados, (1) inscrever uma
  ideia da alma, (2) avaliar suas potencialidades e passividades, bem
  como (3) ajustar discursos e almas, formando uma espécie de tipologia,
  seria um fundamento da arte discursiva, ou seja, entender com certa
  precisão os valores que mobilizam ou não cada tipo de alma e aplicar o
  fármaco discursivo de acordo com objetivos claros.}

F: Parece que essa é a mais bela maneira de agir.

S: Não haverá então, ó querido, outra forma de demonstrar ou proferir
com arte, pela qual se dirá ou se escreverá algo, seja acerca desse ou
de outro tema. {[}271c{]} Os que hoje escrevem, como tu ouviste, são
hábeis em todas essas artes discursivas e as dissimulam, tendo muito
conhecimento acerca da alma. Antes de falarem e escreverem com tais
recursos, não nos deixemos persuadir de que eles escrevem com arte.

F: Quais são esses recursos?

S: Dizer com todas as letras não é fácil. Mas o modo pelo qual é preciso
escrever, se queres mesmo dominar a arte, tanto quanto se possa admitir,
acerca disso eu quero falar.

F: Diz então.

S: Uma vez que a potência do discurso está na condução das almas
(\emph{psychagôgía}), {[}271d{]} aquele que pretende ser rétor deve
necessariamente conhecer as formas (\emph{eíde}) que a alma tem. Há
tantas e tantos tipos de almas, motivo pelo qual existem umas de um modo
e outras de outro. Tendo assim realizado tal distinção, existem
discursos correspondentes para cada uma delas e cada uma tem o seu
específico. Por causa disso, umas {[}almas{]} são levadas à plena
persuasão (\emph{eupeitheîs}) pelo efeito do discurso, e outras, por
outro lado, pelo mesmo discurso, são levadas à desconfiança
(\emph{duspeitheîs}). É preciso ter pensado (\emph{noḗsanta})
suficientemente nessas coisas, ter contemplado os próprios seres em ação
e ter praticado {[}271e{]}, bem como ser capaz de acompanhar com agudeza
pela sensação tudo isso, ou não haverá plenitude dos saberes que outrora
ouviu nos discursos que trazia consigo. Quando for capaz de dizer como e
pelo que há persuasão, quando for capaz de, junto a alguém, perceber e
mostrar a si mesmo qual é a natureza {[}272a{]} acerca da qual versavam
os discursos de outrora, depois disso, junto a eles tendo trabalhado, é
preciso administrar os discursos pelos quais serão persuadidos. É
necessário ser detentor disso tudo, escolhendo também o momento oportuno
(\emph{kairós}) de falar e o de calar, os discursos breves
(\emph{brachylogias})\footnote{Cf. expressão similar,
  \emph{brachylogiôn\textbf{,}} em 269a7.}, os discursos piedosos
(\emph{eleinologias}) e cada uma das formas dos discursos veementes
(\emph{deinṓseos}) aprendidos, reconhecendo neles o momento oportuno
(\emph{eukairían}) e a falta de oportunidade (\emph{akairían}). Bela e
acabada estará, nesse ponto, a arte adquirida, antes disso não.
{[}272b{]} Mas se alguém deixar de lado qualquer um desses elementos ao
falar, ensinar ou escrever, dizendo que o faz com arte, não terá força
persuasiva. Daí talvez dissesse o escritor: ``O quê? Ó Fedro e Sócrates,
parece-vos mesmo assim? Não há então outro modo de conceber a arte dos
discursos?''\footnote{Bloco textual importante por definir as
  potencialidades do discurso aplicado. Agora considera-se a diversidade
  de almas e a diversidade de discursos que são passíveis ou não de
  persuasão, além disso há na passagem uma ênfase na formação do rétor,
  especialmente que ele tenha praticado, experimentado, esses efeitos
  discursivos nas almas, bem como tenha aprendido a adequação da
  oportunidade e dos recursos da arte discursiva. Antes desse processo,
  não poderá ser considerado rétor completo.}

F: Impossível, ó Sócrates, que seja de outro modo, e não me parece uma
tarefa de pouca monta.

S: Dizes a verdade. E, em consequência disso, é necessário percorrer de
cima abaixo todos os discursos para verificar se há, em alguma parte, um
caminho mais fácil e curto, {[}272c{]} a fim de que este não seja vão e
muito árduo, mas, se possível, curto e suave. Se em alguma ocasião
tiveste o auxílio da audição de Lísias ou de algum outro, procura
lembrar-te e diz.\footnote{Menção ao caminho curto pelo qual se poderia
  adquirir o conhecimento retórico é, obviamente, um lugar comum
  fundamental da sofísitica, de que para ser rétor há maneiras rápidas,
  atalhos.}

F: Eu poderia até tentar, mas assim, agora, não posso.

S: Queres que eu te exponha um discurso que ouvi de alguns, acerca desse
tema?

F: Como não?

S: Dizem, ó Fedro, que é justo mencionar também a razão do lobo.

{[}272d{]} F: Então faça isso.

S: Dizem alguns que não é preciso honrar tanto essas coisas, nem se
elevar tanto a altos rodeios. Foi perfeito o que dissemos no início
dessa discussão, que aquele que pretende ser um rétor pleno, não precisa
participar da verdade (\emph{alêtheías}), nem da justiça
(\emph{dikaíôn}), nem da bondade (\emph{agathôn}) acerca daquilo que
trata, quer tenham os homens tais naturezas, quer as tenham adquirido
pela educação. Ocorre que nos tribunais ninguém se preocupa com a
verdade (\emph{alêtheías}), mas com o persuasivo (\emph{pithanoû}), isto
é, com o verossímil (\emph{eikós}), questão na qual é preciso aplicar-se
quem pretenda falar com arte. {[}272e{]} Algumas vezes, nem mesmo os
fatos ocorridos devem ser mencionados, caso não sejam verossímeis
(\emph{eikótôs}), mas somente as coisas verossímeis (\emph{eikóta}),
seja na acusação seja na defesa, bem como tudo aquilo que se diz de
forma verossímil (\emph{eikòs}) é necessário perseguir, motivo pelo qual
muitas vezes é preciso renunciar à verdade (\emph{alêtheî}). {[}273a{]}
O verossímil surge ao longo dos discursos e proporciona toda a arte
(\emph{téchnên}).\footnote{Sócrates expõe a definição clássica de
  retórica que está combatendo, a retórica em que o verossímil
  (\emph{eikós}) vale mais que o verdadeiro.}

F: Detalhaste com propriedade, ó Sócrates, os dizeres daqueles que
professam serem detentores dessa arte dos discursos. Recordo-me que, no
começo dessa discussão, tocamos rapidamente nesse tema, o qual parece
ser muito importante para aqueles que se ocupam disso.

S: Mas tu certamente tens degustado bem do seu Tísias. {[}273b{]} Pois
então ele que nos diga, também, se o verossímil (\emph{tò eikòs}) é
outra coisa senão a opinião da maioria (\emph{tôi plḗthei dokoûn}).

F: E o que mais seria?

S: Como parece, esse sábio artista aqui encontrado (\emph{heurṑn})
escreveu que, se por acaso, um homem fraco e corajoso assaltasse um
homem forte e covarde, levando sua toga e outros pertences, ao serem
ambos levados ao tribunal, seria necessário que nenhum deles dissesse a
verdade, uma vez que o covarde diria que não havia sido assaltado
unicamente pelo corajoso, o qual por sua vez diria que estava sozinho,
{[}273c{]} usando aquele argumento: "Como eu, sendo assim fraco, poderia
ter executado o assalto contra ele que é forte?" E o outro não
expressaria sua própria maldade, mas usaria alguma outra mentira para,
rapidamente, oferecer refutação ao oponente. E acerca de outras tantas
coisas dessas, também ligadas à arte, não dizemos que é assim, ó Fedro?

F: Como não?

S: Ah! Parece-me terrivelmente escondida essa arte encontrada
(\emph{aneureîn}) por Tísias ou por outro qualquer, bem como o nome pelo
qual ela pode ser designada. Mas, ó companheiro, qual deles, dizemos ou
não --.

{[}273d{]} F: O qual?

S: Ó Tísias, antes de ti, os mais antigos já haviam dito que o
verossímil surge para a multidão pela semelhança que ele tem com o
verdadeiro (\emph{to eikòs toîs polloîs dia homoiótêta toû alethoûs
tygchánei eggignómenon}). E as semelhanças (\emph{homoiótêta}), como
dissemos há pouco, são encontradas (\emph{heurískein}) em toda parte e
da melhor maneira possível por aquele que conhece a verdade. Então, se
tens algo mais acerca da arte dos discursos, diz, pois te escutaremos,
caso contrário chegaremos ao mesmo ponto onde estávamos persuadidos
{[}273e{]}. Se alguém não enumera a natureza dos seus ouvintes, discerne
os seres segundo sua forma, não é capaz de levá-los a uma só ideia,
abarcando cada uma delas, e não será um artista do discurso, tanto
quanto isso é possível ao homem. Essas coisas não se podem adquirir sem
muito empenho (\emph{pragmateías}). E não é por causa do falar e do agir
com outros que o homem moderado precisa cultivar sua prudência
(\emph{sṓphrona}), mas para poder agradar aos deuses, ao falar, ao agir
e em tudo que seja possível. Não é assim, ó Tísias? Dizem-nos os mais
sábios que é necessário seguir e agradar o que possui intelecto
(\emph{tòn noûn échonta}), {[}274a{]} porque esse não é simples
acessório, mas senhor bondoso em tudo aquilo que é bom. Se essa estrada
é longa, não te espantes, pois é pelas grandes coisas que se faz esse
percurso, ao contrário do que tu supões. Como o discurso já afirmou, se
for algo que desejas, o que surge desse trajeto será
belíssimo.\footnote{Mais uma vez Platão faz Sócrates conversar com um
  personagem ``imaginário'', na verdade Sócrates evoca Tísias no lugar
  de Fedro, como se dialogasse diretamente com um dos patriarcas da
  retórica. É importante a ligação necessária do verossímil,
  \emph{eikós}, com o domínio das similaridades (\emph{homoióteta}), bem
  como com o conhecimento da verdade. Além disso, retoma-se no trecho a
  síntese e a análise da dialética, bem como a habilidade prática já
  explicitada de adequar discursos às almas, mas agora com um elemento
  novo, a questão de agradar os deuses, para além dos homens, com o
  \emph{lógos}. Sócrates menciona a prudência e o intelecto como
  aspectos importantes dessa arte discursiva, e, novamente, exalta o
  longo caminho para a sua aquisição. Esse longo caminho tem como índice
  primeiro no diálogo a escolha pelo passeio pelas estradas
  (\emph{hodoùs}) e não pelas vias (\emph{drómois}) do pórtico em
  227a5-6, além, é claro, do longo esforço em que se alça a alma fora do
  céu, no supracleste em 247b-c.}

F: Perece-me que afirmas algo magnífico, ó Sócrates, se for assim mesmo.

S: É belo ocupar-se das coisas belas, bem como suportar aquilo que lhes
advém.

F: É certo.

{[}274b{]} S: Então, acerca da arte e da sua ausência (\emph{téchnês te
kai atechnías}) nos discursos, dissemos o suficiente.

F: E o que mais haveria?

S: E acerca da conveniência ou inconveniência da escrita, como ela pode
ser bela ou inconveniente, omitimos isso?

F: Sim.

S: Conheces, pois, o meio pelo qual em matéria de discursos, devemos
agradar aos deuses, agindo ou falando?

F: Não conheço, e tu?

{[}274c{]} S: Escuta o que posso te contar dos antigos, pois eles
conheciam a verdade. Se nós a descobríssemos (\emph{heuroimen}), não nos
preocuparíamos com a opinião dos homens, não é mesmo?

F: Engraçada tua pergunta, mas conta o que ouviste.

S: Escutei que perto de Náucratis, no Egito, existia um dos deuses
antigos, cujo pássaro sagrado era chamado Íbis, o nome desse
\emph{daímon} era Theuth. Ele foi o primeiro a inventar (\emph{heureîn})
os números, o cálculo, a geometria, a astronomia, o jogo de tabuleiro, o
jogo de dados e especialmente a escrita (\emph{grámmata}).\footnote{Hermias
  interpreta esse início do mito identificando Theuth a Hermes, além de
  discernir entre três ordens: a dos deuses, dos \emph{daimons} e dos
  herois. Esse Theuth-Hermes no diálogo de Platão é um \emph{daimon},
  como se vê transliterado na tradução, mas Hermias, em seu comentário,
  usa inesperadamente o termo ``deus'' para Hermes, embora essa
  nomenclatura fosse evidentemente a mais comum. Vejamos o comentário
  completo de Hermias: ``O mito é claro. Conduziremos nossa busca de
  acordo com a letra e de acordo com as suas partes. \emph{Escutei} ele
  diz ou pelo costume de levar um discurso de si mesmo para outros, ou
  porque os antigos também desde sempre haviam feito demonstrações a
  partir desse meio. Quanto ao \emph{um dos deuses antigos} refere os
  que estão entre os primeiros, mais elevados, predominantes e das
  ordens mais altas. Pois em deuses, \emph{daimons} e heróis estão,
  respectivamente, a primeira, a segunda e a terceira -- e última --
  entre as ordens. Cultuam o mestre Hermes pela invenção dos discursos e
  o Egito. Em toda parte, no \emph{Timeu} inclusive, {[}Platão{]}
  diviniza os egípcios como os mais antigos, motivo pelo qual nem o
  dilúvio, nem a destruição pelo fogo eram desconhecidas dos egípcios,
  embora fossem desconhecidas de outros povos. Por isso essas invenções
  eram imortais para os egípcios, as quais para os gregos foram trazidas
  por Palamedes ou Prometeu, enquanto o filósofo traz do Egito esse
  guardião dos discursos, o \emph{deus} Hermes'' (§199.9-21 Couvreur =
  88, 8-22 Lucarini e Moreschini). Para Hermias não há dúvida que
  Thamous-Amon é deus e Theuth-Hermes, por seu turno, é um
  \emph{daimon}, mas, mesmo assim, ele não deixa de referir Hermes como
  deus, quando, seguindo sua própria exposição, seria mais preciso dizer
  \emph{daimon} Hermes. Um pouco à frente (§199.4-8), Hermias usa
  novamente a nomenclatura ``deus Hermes'' e ``\emph{daimon} Theuth''
  explicando que a natureza do \emph{daimon} é justamente a de ser um
  intermediário (\emph{mesótetos, metaxù}) entre o deus e os homens,
  distribuindo as invenções aos homens, daí ser Theuth chamado também de
  Hermes, diluindo assim a aparente confusão anterior entre essas
  diferentes ordens da realidade e suas nomenclaturas, pois Theuth não
  seria exatamente Hermes, mas um outro, um intermediário, um
  \emph{daimon}, do deus Hermes. Cf. também \emph{Filebo} 18b-d Theuth
  como inventor da gramática.} {[}274d{]} O rei de todo Egito nessa
época era Thamous, que vivia na grande cidade alta, a qual os gregos
chamavam de Tebas egípcia e onde o mesmo Thamous Amon era o
deus.\footnote{No \emph{Político} 257b6, quando Sócrates conversa com
  Teodoro de Cirene, o deus Amon é também evocado, porque Teodoro provém
  dessa região na qual havia um oráculo de Amon, deus que teve um culto
  oficial em Atenas (cf. Pausânias, \emph{Descrição} IX,16,1 e Píndaro,
  frg. 36). Em Tebas e no oásis de Siwah, na Líbia, haviam oráculos de
  Amon, sendo que este teria sido fundado ao mesmo tempo que o de
  Dodona, por uma sacerdotisa proveniente da Líbia (sigo aqui Santa-Cruz
  y Crespo, p. 199, notas 585). De Vries (1969, p.248) explica a
  controvérsia em torno desse deus (\emph{theòn})\emph{,} que seria uma
  variante ou corrupção de \emph{Amoûs}. Enquanto Postgate substitui
  \emph{tòn theòn Ammôna} por \emph{tòn Thamoûn Ammôna}, Scheidweiler
  sugere \emph{tòn ``Thamoûn'' theòn Ammôna}. Embora De Vries não
  considere estritamente necessária a leitura de Postgate, ainda assim
  julga preferível sua solução, defendida também por Gil, para o qual
  houve também um erro ao interpretar um \emph{theòn}, quando na verdade
  seria \emph{Thamoûn}. Rowe (1988, p.208-9) e Hackforth (1952, p.157)
  seguem essa leitura, enquanto Burnet (1901) e Moreschini (1998, p.143)
  mantém simplesmente o \emph{theòn}. Brisson (1989, p. 177 e 232),
  seguindo Scheidweiler, traduz ``le dieu (Thamous) Ammon''. Nossa
  tradução segue Scheidweiler/Brisson sem entusiasmo pela controvérsia.
  O ruído está na identificação ou não do rei Thamous como uma
  divindade, embora nisso não haja, de fato, alguma alteração brusca na
  interpretação da alegoria, pois não há dúvida de que a designação de
  Theuth como \emph{daímon} permanece inferior à de Thamous, seja ele ou
  não designado como deus, porque este último é superior na medida em
  que as artes e invenções de Theuth estão todas sob o crivo do
  discernimento dialético representado por Thamous. A famosa alegoria,
  se bem compreendida, deveria ser designada pela perspectiva do
  dialético, embora não seja estranho que tenha sido interpretada
  predominantemente pela perspectiva da retórica, sendo amplamente
  descrita e nomeada como mito ou alegoria de ``Theuth'', quando mais
  preciso seria designá-la, da perspectiva dialética, como alegoria ou
  mito de ``Thamous''.} Theuth veio junto ao rei para mostrar-lhe suas
artes, que segundo ele deveriam ser presenteadas a todos os egípcios.
Indagado, então, acerca das utilidades (\emph{ôphelían}) de cada uma
delas, ele as expunha, de modo que o rei dizia o que parecia, aos seus
olhos, ser belo ou feio em cada uma, ora elogiando ora
vituperando.\footnote{Cf. na Nota 10 o sentido do termo ``utilidade'' e
  sua aplicação no diálogo, especialmente em 227d2 \emph{demôpheleîs},
  233c1 \emph{ophelían}, 245b6 \emph{opheléiai,} bem como a explicação
  do tom democrático associado à expressão. Platão condena a
  distribuição equânime da \emph{téchne} da escrita para todos os
  egípcios devido aos seus efeitos colateriais, não observados por
  Theuth (Hermes), seu criador. O mecanismo dialético de Thamous aplica
  elementos fundamentais da retórica, na medida em que ele tem de expor
  o que lhe parece louvável e censurável na invenção de Theuth.}
{[}274e{]} Muitas foram as artes para as quais Thamous apresentou seu
comentário a Theuth, e todo o seu discurso seria muito longo para
referi-lo aqui, mas acerca da escrita, foi assim: "Ó rei, disse Theuth,
esse conhecimento tornará os egípcios mais sábios e com maior disposição
para a memória. Foi inventado (\emph{heuréthê}) então o fármaco da
memória e da sabedoria". Ao que o rei replicou: "Ó artificiosíssimo
Theuth, enquanto uns são capazes de criar uma arte, outros são capazes
de julgá-la, especialmente em que aspectos elas serão nocivas ou úteis
para quem poderá usá-las. {[}275a{]} Agora, aqui, tu, como pai da
escrita que és, por tua benevolência para com ela, dizes o contrário do
que ela é capaz. Ela produz esquecimento nas almas daqueles que aprendem
pela falta de cuidado com a memória, sendo então por escritos externos e
alheios que adquirem a crença (\emph{pístis})\emph{,} não adquirindo
mais a reminiscência internamente e por si mesmos. Portanto, não
inventaste (\emph{hêures}) o fármaco da memória (\emph{mnḗmês})\emph{,}
mas o da recordação (\emph{hypomnḗseos}). Ela oferece uma aparente
sabedoria aos discípulos, que não alcançam a verdade propriamente dita.
{[}275b{]} Muitos dos teus ouvintes, sem aprendizado, parecem
conhecedores de muitas coisas, quando na verdade são geralmente
ignorantes e difíceis no trato, tornando-se aparentemente sábios
(\emph{doxósophoi}) em vez de sábios (\emph{sophôn})''.\footnote{A
  alegoria sintetiza os elementos fundamentais do diálogo, como a
  diferença da memória (\emph{mnémes}) e da recordação
  (\emph{hypomnéseos}), no sentido de a escrita ser apenas uma memórica
  auxiliar e de provocar o efeito colateral do esquecimento nas almas
  dos aprendizes, quando na verdade o único conhecimento possível seria
  o re-conhecimento, ou seja, a reminiscência (anamnese) daquilo que a
  alma já contemplou na planície da verdade. Nesse sentido muitos
  adquirem crenças (\emph{pístis}) e delas se valem a partir de
  rudimentos colhidos externamente (\emph{exôthen}), nos livros, e não
  daquilo que investigaram internamente (\emph{endothen}) e por si
  mesmos se lembraram (\emph{autoùs huph' autôn anamimnêiskoménous}).
  Toda a construção de Platão em mostrar a diferença entre o sofista e o
  filósofo encontra eco na diferença entre esses aparentemente sábios
  (\emph{doxósophoi}) e os sábios (\emph{sophôn}) verdadeiros.}

F: Ó Sócrates, que facilidade tens para apresentar histórias egípcias e
de qualquer lugar que queiras.

S: Ó amigo, dizem que os antigos discursos divinatórios provinham de um
carvalho situado no templo de Zeus em Dodona. Naquele tempo os homens
não eram tão sábios quanto vós, os jovens, motivo pelo qual lhes
bastava, devido à sua simplicidade, ouvir um carvalho ou uma pedra,
desde que estes lhes dissessem somente a verdade. {[}275c{]} Tu talvez
possas discernir qual é o discurso e de onde ele provém. E não observes
somente se é assim ou não.\footnote{É importante observar a ligação
  entre os oráculos de Amon e de Dodona (244b1), bem como as
  manifestações oraculares do divino por intermédio da natureza, no caso
  apresentados pelo carvalho e pela pedra. Se antes Sócrates havia dito
  que não aprendia nada da natureza (230d2-5), apenas com os homens da
  cidade, aqui observa-se que a natureza passa a ser, ao contrário,
  portadora de um conhecimento verdadeiro. Mais uma vez o termo
  ``sábio'' é usado de modo irônico, especialmente porque a associação
  entre juventude e sabedoria contraria as anteriores menções aos
  antigos sábios.}

F: Correta é a tua repreensão e me parece que, acerca dos escritos,
ocorre o que tebano já havia afirmado.

S: Tanto aquele que supõe deixar alguma arte por meio da escrita
(\emph{en grámmasi}), quanto aquele que espera recebê-la por esse meio,
ambos consideram que a escrita (\emph{grammátôn}) porta algo de claro e
seguro, o que é muita ingenuidade e prova de desconhecimento
(\emph{agnooî}) do oráculo de Amon, {[}275d{]} segundo o qual os
discursos escritos (\emph{lógos gegramménous}) nada mais são do que um
meio de recordar (\emph{hypomnêsai}) aquele que já conhece
(\emph{eidóta}) os assuntos tratados nos escritos
(\emph{gegramména}).\footnote{O \emph{oráculo de Amon}, nesse caso,
  corresponde ao discernimento proferido por Amon Thamous acerca da
  diferença entre memória e recordação (274e6).}

F: Corretíssimo.

S: É terrível mesmo, ó Fedro, essa escrita (\emph{graphḕ}) e como tem
verdadeira semelhança (\emph{hómoion}) com a pintura
(\emph{zôgraphíai}). Os frutos desta são estabelecidos como vivos, mas
se lhe perguntas algo, ela permanece sempre num silêncio sagrado
(\emph{semnôs pánu sigâi}), e assim também acontece com os discursos
(\emph{oi lógoi}). Eles parecem dizer algo de sensato, mas, se alguém
que deseja aprender lhes pergunta algo sobre o que foi dito, eles só
significam a mesma coisa sempre. {[}275e{]} E a grafia (\emph{graphêi})
roda por todo lado conservando o mesmo discurso, seja para os que a
elogiam, seja para os que nela não têm nenhum interesse, pois ela não
conhece o momento de falar ou de calar. E se ela for atacada num
tribunal, sempre haverá a necessidade que o seu pai a socorra
(\emph{boêthoû}) das injúrias, pois ela não é capaz de defender ou
socorrer (\emph{amúnasthai oute boêthêsai}) a si mesma.

F: Também isso que dizes é corretíssimo.

{[}276a{]} S: O quê? Dizemos que há outro discurso, irmão legítimo
deste, mas surgido por outro modo, melhor quanto à natureza e mais
poderoso?\footnote{Cf. especialmente em 265e-266b a divisão
  (\emph{diatémnein, temnómenos, témnôn}) que discerniu amor sinistro do
  amor destro. Obviamente esse corte está relacionado à \emph{diairesis}
  ou divisão dialética.}

F: Sobre qual discurso te referes e como ele surge?

S: Sobre aquele que é inscrito na alma (\emph{gráphetai en têi psychêi})
daquele que aprende, segundo o conhecimento (\emph{met'epistḗme}), ele é
capaz de defender (\emph{amûnai}) a si mesmo, conhecedor da ocasião
frente a qual é preciso falar (\emph{légein}) ou calar (\emph{sigân}).

F: O discurso de quem efetivamente sabe, ao qual te referes, é vivo e
animado (\emph{zônta kaì empsuchon}), de modo que o discurso escrito
(\emph{gegramménos}), poderíamos dizer com justiça, é um ídolo
(\emph{eídôlon}) seu.\footnote{O \emph{eídôlon} é uma tópica
  estesicórica utilizada por Platão no \emph{Fedro} em 250d4-e1, quando
  Sócrates descreve os efeitos dos ídolos na visão, quando remetem à
  beleza, em 255d4-e1, quando descreve um ídolo (\emph{eídôlon}) do
  amor, um Ânteros ou Amor recíproco, aqui em 276a8-9 como ídolo da
  palavra viva, além de \emph{República} 586c1-5, lugar em que cita
  Estesícoro e o ídolo de Helena (\emph{Helénes eidôlon}) conjuntamente,
  motivo pelo qual se combatia em Troia, por simples desconhecimento da
  verdade (\emph{agnoíai toû alethoûs}).}

{[}276b{]} S: É assim mesmo, agora me diz, quanto ao agricultor
(\emph{geôrgós}) que tem inteligência (\emph{noûn}) e deseja cuidar das
suas sementes para que frutifiquem, o que ele faria? Haveria de
lançá-las, durante o verão, no jardim de Adônis (\emph{Adónidos
kêpous}), para homenagear a sua festa, para que floresçam em oito dias?
Ou isso ele poderia fazer só por brincadeira (\emph{paidiâs}) e
exclusivamente de bom grado para o festival, quando muito. Ou, quanto às
sementes que ele realmente despende atenção, valendo-se da arte da
agricultura (\emph{georgikêi}), ele semearia em local adequado,
felicitando-se em oito meses, quando as sementes atingem sua maturidade?

{[}276c{]} F: Ó Sócrates, num caso ele faria com atenção, no outro não,
como tu dizes.

S: O que dizemos daquele que tem conhecimento do justo, do belo e do
bom? Que ele tem menor inteligência (\emph{noûn}) que a do agricultor,
com relação às suas sementes?\footnote{Cf. \emph{Eutífron} 2d2-4
  metáfora da agricultura relacionada também à educação.}

F: De modo algum.

S: Então não vai cuidadosamente escrevê-las na água escura com uma pena,
compondo discursos incapazes de socorrerem-se (\emph{boêtheîn}) a si
mesmos, insuficientes para ensinar a verdade (\emph{adunátôn dè ikanôs
talêthḕs didáxai}).\footnote{A expressão ``escrever na água escura''
  refere a escrita à tinta por meio do cálamo, parte inferior e oca da
  pena. Impossível não relacionar esse ato de escrever, ou seja, de
  irrigar uma pena, com as imagens anteriores das asas da alma quando
  irrigadas e desobstruídas pelo fluxo do desejo, bem como todo léxico
  das asas (alado), especialmente em \emph{Phdr}. 246a-e, 248b-e,
  249a-d, 251b-d, 252b-c.}

F: Não é mesmo verossímil (\emph{eikós}).

{[}276d{]} S: Não? Mas nos jardins da escritura (\emph{en grámmasi
kḗpous}), como parece (\emph{eoike}), semeiam e escrevem pelo deleite da
brincadeira (\emph{paidiâs chárin spereî te kaì grápsei}), e quando
escrevem entesouram recordações (\emph{hypomnḗmata}) de si mesmos, para
o ``oblívio da velhice'' (\emph{lḗthes} \emph{gêras}), se ela
``chegar''. E todos que buscam seguir seus passos\footnote{Cf.
  \emph{Phdr}. 266b7 expressão similar: ``seguindo seus passos
  (\emph{íchnos}) como os de um deus'', sendo que essas ``pegadas'' ou
  ``marcas'' (\emph{íchnos, íchnion}) fazem parte também do campo
  semântico relativo à grafia, ao qual se agrega o sulco ou gravação
  presente em \emph{élkô} (\emph{eílkysen Phdr}. 270a7), bem como o
  golpe ou a ferida presente em \emph{kóptô}, de democópico
  (\emph{democopikós} 248e3).} serão agraciados pela contemplação dessas
delicadas plantas. Por outro lado, quando outros se valem de outras
diversões, bebendo nos simpósios, entregues a prazeres similares a este,
e, como parece, divertir-se-ão exatamente com as coisas referidas.

{[}276e{]} F: Boa diversão frente àquela frívola, ó Sócrates, essa de
poder brincar (\emph{paídzein}) com os discursos, sejam eles judiciais
ou outros que dizes nos quais possamos também narrar
(\emph{mythologoûnta})

S: É assim, ó querido Fedro, considero muito mais belo o empenho daquele
que pela arte da dialética toma uma alma para plantar e nela semear
discursos com conhecimento (\emph{met'epistḗmês}), {[}277a{]} aqueles
que são capazes de socorrer (\emph{boêtheîn}) quem os plantou. Então, os
discursos não são infrutíferos, mas têm sementes, pelas quais outros em
outros lugares se habituarão a crescer, tornando-as sempre imortais o
bastante, tornando felizes os homens, tanto quanto possível.

F: Muito mais belo é o que dizes agora.

S: Agora que chegamos a esse acordo, ó Fedro, somos então capazes de
julgar (\emph{krínein}).

F: Julgar o quê?

S: Aquilo que queríamos saber e que nos trouxe até aqui, justamente para
que pudéssemos examinar a censura endereçada a Lísias pelos seus
discursos escritos (\emph{tês tôn lógôn graphês péri}), {[}277b{]} e
para examinarmos os próprios discursos, se haviam sido escritos com arte
(\emph{téchnêi}) ou sem arte (\emph{aneu techné}). Os que estão de
acordo com a arte (\emph{éntechnon}) parecem-me terem sido expostos de
modo bem medido (\emph{metríôs})\emph{.}

F: Parece mesmo. Mas recorda-me (\emph{hypómnêson}) novamente
(\emph{palin}) como.

S: Antes, devemos saber a verdade acerca de cada coisa sobre o que se
fala e escreve, tudo deve poder ser definido por si mesmo, e uma vez
definido, devemos conhecer como dividi-lo (\emph{témnein}) novamente
(\emph{pálin}) até a forma indivisível (\emph{toû atmétou}). E a
respeito da natureza da alma, que se distinga tudo da mesma forma,
{[}277c{]} descobrindo (\emph{aneurískôn}) a forma natural que se
harmoniza com cada uma delas, para então estabelecer e ordenar o
discurso. Um discurso variegado é oferecido para uma alma variegada, um
simples para uma alma simples, antes disso não é possível haver um
gênero discursivo que faça uso natural da arte, nem para ensinar nem
para persuadir, como nos foi revelado pelo discurso anterior.

F: É tudo mesmo dessa forma, tal qual nos pareceu.

S: E a respeito do falar e do escrever discursos ser algo belo ou
vergonhoso, e de quando é possível dizer, com justiça, o que é
vergonhoso ou não. O que há pouco foi dito não ficou bem claro?

F: O quê?

S: Que Lísias ou qualquer outro que tenha escrito ou venha a escrever
leis particulares ou públicas, quando consideram o tratado escrito sobre
política algo grandioso, estável e claro, é nesse momento que eles podem
se envergonhar dos discursos, quer isso seja mencionado ou não. O fato
de alguém ignorar, sob o efeito do sono, {[}277e{]} o justo e o injusto,
o mau e o bom, não pode livrá-lo da verdade de ser censurado, ainda que
toda a turba o elogie.

F: Não mesmo.

S: É necessário que haja muito divertimento (\emph{paidián}) em cada um
desses discursos escritos, e que nenhum deles, em metro ou sem, mereça
grande esforço para ser escrito, ou mesmo lido como fazem os rapsodos,
sem preparo ou didática naquilo que é dito para persuadir. {[}278a{]} Os
melhores entre eles são os que, pela recordação (\emph{hypómnêsin}),
levam ao saber. Por outro lado, os que são feitos para ensinar,
discursos que agradam ao aluno e inscrevem na alma (\emph{graphoménois
en psychêi}) algo acerca do justo, do belo e do bom, somente estes são
visíveis, acabados e dignos de esforço. É preciso que tais discursos
sejam enunciados como filhos legítimos, {[}278b{]} primeiro por eles
mesmos, se eles os descobrirem (\emph{heuretheìs}) em si, e, em seguida,
se alguns desses seus descendentes e irmãos plantam concomitantemente em
outras almas, em outros lugares, de acordo com a dignidade. Quanto a
outros discursos, é melhor afastar-nos deles, ó Fedro, pois essa é a
atitude do homem que ambos, eu e tu, gostaríamos de ser.

F: Quanto a mim, desejo e faço votos para que seja assim, tudo da
maneira que dizes.

S: Então nós já nos divertimos (\emph{pepaísthô}) o bastante
(\emph{metríôs}) acerca dos discursos, e tu vai até Lísias e diz a ele
que nós dois descemos até a fonte das ninfas e ao santuário das Musas e
que escutamos um discurso {[}278c{]} para ser enviado a Lísias e para
qualquer outro que componha discursos, a Homero e a qualquer outro que
tenha composto poesia com ou sem acompanhamento musical (\emph{ôidêi}),
e em terceiro lugar a Sólon e aos que escreveram discursos políticos,
tratados que foram chamados de leis escritas: ``Se conheces a verdade
daquilo que está composto nesse escrito e és capaz de socorrê-lo
(\emph{boêtheîn}), nas refutações que lhes são endereçadas, e ainda és
capaz de mostrar o que é ineficiente no teu próprio escrito, então, na
verdade, pelo qual epônimo deverá ser designado, por esta atividade de
escrever ou por aquela atividade a qual se dedicou?''

F: Qual dos epônimos tu atribuis a ele?

S: O de sábio, ó Fedro, acredito parecer demasiado, conveniente somente
a um deus. O de filósofo ou outro desse tipo poderia ser mais ajustado e
adequado.

F: E de nenhum modo inapropriado.

S: Aquele que não tem, por outro lado, nada de mais honrado
(\emph{timiṓtera}) do que aquilo que outrora escreveu e passa o tempo a
percorrer (\emph{stréphôn}) seus escritos de cima abaixo, separando
trechos e trocando-os de lugar, {[}278e{]} é com justiça que o
designarás por poeta, compositor de discursos ou escritor de leis?

F: É certo.

S: E é isso mesmo que deves dizer ao teu companheiro.

F: E tu? Como farás? Não deves pôr de lado o teu companheiro.

S: Qual deles?

F: O belo Isócrates.\footnote{Hackforth, seguindo Wilamowitz, não
  acredita que a expressão ``belo Isócrates'' seja irônica (Hackforth,
  1952, p. 167). Rowe, por outro lado, entende como irônica a expressão
  (Rowe, p.215-216). Yunis também considera Isócrates um representante
  da cultura retórica combatida por Platão no diálogo (Yunis, p.
  243-246), bem como De Vries, que ressalta as profundas diferenças
  epistemológicas entre eles, além das possíveis alusões anteriores a
  Isócrates, nenhuma delas amigável (De Vries, p.15-18). Também
  entendemos, com Rowe, Yunis e De vries, esse ``belo Isócrates'' como
  ironia.} O que dirás a ele, ó Sócrates, e nós diremos o quê?

S: Isócrates é jovem ainda, ó Fedro, {[}279a{]} entretanto adivinho algo
sobre ele e quero dizer.

F: O quê?

S: Parece-me que ele é superior a Lísias quanto à natureza de seus
discursos, e ainda temperado por um caráter mais nobre (\emph{ḗthei
gennikôtéroi}), de modo que não seria espantoso se, com a idade, ele
superasse nessa prática os que hoje em dia se ocupam disso, tornando
infantis os que sempre se ocuparam de discursos. Se isso ainda não for
suficiente, ele será guiado por um impulso maior e mais divino. Pois há,
ó querido, certa filosofia no intelecto desse homem. {[}279b{]} É isso,
então, que eu vou, junto aos deuses, anunciar a Isócrates, o meu
favorito, e tu, por sua parte, faça o mesmo ao teu Lísias. \footnote{A
  equiparação de Isócrates como \emph{hetaîron} de Sócrates, de modo
  similar a Lísias com relação a Fedro, faz parecer, à primeira vista,
  que Sócrates (Platão) elogia Isócrates, especialmente porque diz
  adivinhar que sua natureza é superior à de Lísias, bem como que há
  algo de filosófico nele que poderá ainda vir à tona. Temos elementos
  suficientes para desconfiar desse elogio, pois Isócrates não era um
  novato, como Sócrates diz, e uma maneira de entender essa ideia é a
  comparação com Lísias, pois Isócrates parece ter ampliado sua
  atividade de logógrafo apenas quando voltou a Atenas durante o governo
  dos Trinta, período em que Lísias tinha já uma carreira consolidada,
  apenas nesse sentido Isócrates era um ``novato''. Nessa temporalidade
  instaurada por Platão, Isócrates tem mais de 60 anos, ou seja, não
  parece que Sócrates acredite, nem Platão, que Isócrates chegará à
  filosofia, no sentido platônico do termo. Apesar de defender uma
  retórica que chamava de filosofia, apesar de atacar também os
  sofistas, Isócrates não tem uma concepção filosófica similar a de
  Sócrates/Platão. Além disso, na oratória, não poderia ser um modelo,
  pois não tinha boa dicção, e se reservava, assim como Lísias, ainda
  que por motivos diferentes, aos discursos escritos. É preciso
  compreender que Isócrates é um exemplo de ausência para Platão. Lísias
  não era ateniense, ou seja, escrevia para outros usarem seus discursos
  no tribunal, enquanto Isócrates, apesar de ser ateneiense e poder
  participar das instâncias oficiais da cidade, nas quais naturalmente o
  discurso falado era importante, nunca foi capaz disso, primeiro porque
  esteve fora durante todo o período da guerra do Peloponeso, além da
  limitação performática de não falar em público e ficar restrito também
  aos discursos escritos, como modelos, exercícios didáticos e discursos
  sob encomenda. De alguma maneira Lísias e Isócrates se completam, em
  níveis diferentes, como contra-exemplos daquilo que Platão defende
  como filosofia ou como retórica filosófica, nas quais se deve falar e
  escrever dentro dessa nova prescritiva, agora dialética, das artes
  discursivas.}

F: Assim será, partamos agora que o calor se tornou ameno.

S: Não é adequado fazermos uma prece antes de partir?

F: Sim, é.

S: Ó querido Pã e outros deuses, concedam-me uma beleza interior. Que
tudo que há fora de mim possa ser amigo do que está no meu interior
{[}279c{]}\footnote{Sócrates nesse epílogo em forma de prece desfaz a
  tensão que perpassou todo o diálogo entre interno e externo, pedindo
  que Pã, filho de Hermes, proporcione a ``harmonia'' entre o seu
  interior e o exterior.} e que eu considere rico o sábio. Quanto à
quantidade de ouro, que eu possua tanto quanto o homem prudente seja
capaz de levar e trazer. Precisamos de algo mais, ó Fedro? Pois me
parece bem medida (\emph{metríôs}) a prece.

F: E eu partilho dessa súplica, pois tudo é comum entre
amigos.\footnote{Esse dito pitagórico é mencionado na \emph{República}
  424a1, 449c6, bem como em Aristóteles, na \emph{Política} 1263,30 e na
  \emph{Ética Nicomaqueia} 1159b31.}

S: Partamos.
