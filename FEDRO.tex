
\chapter{}

%\let\footnote=\endnote 


[227a] S: Ó querido Fedro, de onde vens e para onde vais?

 

F: Venho de junto de Lísias, o filho de Céfalo, ó Sócrates. Atravessei o
passeio por fora dos muros (\emph{éxô}~\emph{teíchous}) porque permaneci
sentado durante muito tempo, desde cedo. Persuadido pelo teu e meu amigo
Acúmeno, percorri o passeio pelas estradas (\emph{hodoùs}), pois,
segundo ele, esse caminho é menos cansativo que o realizado pelas vias
(\emph{drómois}) do pórtico.

 

[227b] S: É belo o que dizes, companheiro. E~parece que Lísias estava
na cidade.

 

F: Sim, com Epícrates, na casa de Morico, aquela próxima ao templo de
Zeus Olímpico.

 

S: E qual era o debate (\emph{diatribḗ})? É certo que Lísias vos servia
um banquete (\emph{eistía}) de discursos!

 

F: Saberás, se estás livre para prosseguir comigo e escutar.

 

S: O quê? Não sabes que para mim, como diz Píndaro, ``não há nada mais
elevado''~que escutar o debate (\emph{diatribḕn}) que tiveste com
Lísias!

 

F: Avança, então.

 

[227c] S: Ao discurso.

 

F: Sócrates, de fato a audição interessa"-te, o discurso acerca do qual
debatíamos (\emph{dietríbomen}) versava, não sei com que precisão,
acerca do amor (\emph{erôtik\gr{ό}s}). Lísias escreveu (\emph{gégraphe}) a
respeito da sedução (\emph{peirṓmen\gr{ό}n}) dos belos, mas não dos que estão
sob a tutela de um amante (\emph{ouch hyp'erastoû dé}), e aí mesmo
reside a sua habilidade, pois Lísias diz ser melhor agradar
(\emph{charistéon}) o não afetado pelo amor (\emph{mḕ~erônti}) do que um
amante (\emph{erônti}).

 

S: Ó nobre! Espero que ele possa escrever ser melhor a pobreza frente à
riqueza, a velhice frente à mocidade, sem contar outras coisas que
comigo e a muitos de nós acontecem.~Nesse caso, os discursos seriam
agradáveis (\emph{asteîoi})~e úteis ao povo
(\emph{dêmôpheleĩs}).~[227d] Então desejo escutá"-lo, mesmo que tu
percorras o passeio até Mégara, segundo os preceitos de Heródico,
chegando até os muros e dali novamente (\emph{pálin}) retornando pelo
mesmo trajeto, nem mesmo assim eu te abandonaria.

 

F: O que dizes, ó querido Sócrates? O que Lísias compôs durante muito
tempo e com empenho (\emph{en pollôi chr\gr{ό}nôi katà scholḕn}), [228a]
ele que é hoje um dos mais hábeis (\emph{deinótatos}) a escrever
(\emph{grapheîn}), considera"-me capaz de lembrar
(\emph{apomnêmoneúsein}) dignamente disso, eu que sou um simplório?
Falta"-me muito ainda. Na verdade, almejaria mais isso do que ter muito
ouro.

 

S: Ó Fedro, se não conheço Fedro, estaria esquecido de mim mesmo, mas
não é nada disso. Bem sei que, tendo ouvido o discurso de Lísias
(\emph{Lysíou lógon akoúôn}), não só uma, mas muitas vezes, tu
remontavas seus dizeres (\emph{légein}), persuadido de boa vontade
(\emph{epeítheto prothúmōs}). Mas isso não [228b] era ainda o
suficiente. Aposto que tomavas o livro, passando a investigar o que mais
te interessava (\emph{epethýmei}), e, tendo feito isso desde cedo,
tratou de repeti"-lo ao redor do passeio (\emph{perípaton}), como bem
sei, e, pelo cão, decoraste o tal discurso, isso se não era muito
longo.~Depois de ter atravessado por fora dos muros (\emph{ektòs
teíchous}) para exercitar"-te, encontraste aquele que é doente para ouvir
discursos (\emph{nosoûnti peri lógôn akoḗn}) e que, ao vê"-lo, alegra"-se
porque terá um companheiro coribântico que o ordenará prosseguir. Sendo
obrigado a falar pelo [228c] amante dos discursos (\emph{lógon}),
ficas enternecido~(\emph{ethrúpteto}), como se não desejasses falar
(\emph{hos de ouk epithymôn légein}), mas, no final das contas,
falarias, mesmo que à força, ainda que não houvesse alguém para ouvir"-te
voluntariamente. Então, ó Fedro, obriga"-te a fazer imediatamente o que
certamente farias, em seguida, de qualquer modo.

 

F: Na verdade, é muito melhor que eu possa falar (\emph{légein}) assim,
mas parece também que tu, seja como for, não me largarias antes que eu
falasse.

 

S: Verdadeira é a tua impressão.

 

[228d] F: Assim farei. Quanto ao próprio [discurso], Sócrates,
eu não conheço completamente suas palavras (\emph{ta}~\emph{rhémata}).
Pretendo expor o pensamento (\emph{diánoian}) quase completo, que difere
o amante do não afetado pelo amor, em cada um dos seus pontos capitais
(\emph{kephalaíois}), começando pelo primeiro.

 

S: Então, primeiramente mostra, ó querido, o que trazes na mão esquerda,
debaixo do manto. Desconfio de que tenhas o próprio discurso (\emph{tòn
lógon autón}). [228e] Se é assim, entendas uma coisa a meu respeito,
que eu não me prestaria ao teu exercício estando junto de Lísias
(\emph{paróntos dè kaì Lysíou}), logo eu que o estimo
(\emph{philô})~tanto. Vai logo, mostra.

 

F: Para! Retiraste a minha esperança, ó Sócrates, de exercitar"-me
contigo.~Onde desejas tomar assento para lermos?

 

[229a] S: Desviemo"-nos daqui, seguindo a direção do Ilisso, para nos
sentarmos em algum lugar que te pareça tranquilo.

 

F: Oportunamente (\emph{eis kairón}), ao que parece, tive a sorte
(\emph{étychon}) de estar descalço hoje. Quanto a ti, Sócrates, sempre
estás assim.~Então, será facílimo, e nada desagradável, molharmos nossos
pés por esse fio d'água, especialmente nesta época do ano e nesta hora
do dia.

 

S: Prossegue e busca o lugar no qual descansaremos.

 

F: Vês aquele elevadíssimo plátano?\\\\

S: O que há nele?

 

[229b] F: Uma sombra, uma brisa moderada (\emph{pneûma métrion}),
uma relva para nos sentarmos e reclinarmos se quisermos
(\emph{boulṓmetha}).

 

S: Prossegue então.

 

F: Diga"-me, ó Sócrates, não é desse lugar do Ilisso que contam ter
Bóreas raptado (\emph{harpásai}) Orítia?

 

S: Dizem.

 

F: Então é aqui! A água parece agradável, pura e diáfana, própria às
donzelas que brincam (\emph{paídzein}) nessas margens.

 

[229c] S: Não é aqui, mas dois ou três
estádios\textsuperscript{~}abaixo, onde atravessamos na direção de Agra,
onde há um altar para Bóreas.

 

F: Não lembro, mas diz, por Zeus, ó Sócrates, estás persuadido
(\emph{peíthêi})~de que esse mitologema seja verdadeiro?

 

S: Se eu fosse um incrédulo (\emph{apistoíēn}), como fazem os sábios
(\emph{sophoí}), não seria nada extravagante (\emph{átopos}). Em
seguida, diria, com ar sofisticado (\emph{sophizómenos}), que ela foi
arrebatada das proximidades das pedras, lá debaixo, pelo sopro
(\emph{pneûma}) de Bóreas, quando brincava com Farmaceia. Alguns dizem
que ela teria morrido pelo rapto (\emph{anárpaston}) de Bóreas -- ou que
isso ocorrera no Areópago \mbox{---,} [229d] pois contam também essa outra
versão, de que ela foi (\emph{hērpásthē}) raptada de lá e não daqui.~Eu
considero graciosas essas coisas, mas uma ocupação terrível, laboriosa e
própria a homens não muito felizes. E~não por outra razão, se não por nos
obrigar a restaurar, necessariamente, a forma dos Hipocentauros, da
Quimera, de uma turba de Górgonas, Pégasos e muitos [229e] outros
seres formidáveis, por conta da extravagância (\emph{atopiai}) dessas
naturezas monstruosas. E~se alguém, entre os incrédulos
(\emph{apistôn}), conduzisse cada um deles à verossimilhança
(\emph{eikòs}), valendo"-se de uma sabedoria (\emph{sophíai}) grosseira,
precisaria de um bom tempo livre (\emph{scholés}). Eu não tenho nenhum
tempo livre (\emph{scholé}) para essas coisas e a causa disso, querido,
é que não fui ainda capaz de conhecer a mim mesmo
(\emph{gnônai}~\emph{emautón}), de acordo com a inscrição
délfica.~Pareceria risível [230a], ainda ignorante de mim mesmo, que
eu examinasse alguma outra coisa. Esse é o motivo pelo qual me agrada
renunciar a tudo, como disse agora mesmo, convencido
(\emph{peithómenos}) do que foi considerado sobre o tema. Nesse caso,
não examino (\emph{skopô}) essas coisas em outro lugar se não em mim
mesmo (\emph{emautón}), quer seja uma fera mais complexa e orgulhosa
(\emph{epitethymménon}) que Tifon (\emph{Tuphônos}), quer seja o animal
mais doméstico e simples, partícipe de uma natureza em algo divina e sem
nenhum orgulho (\emph{atúphou}). A~propósito, companheiro, não é esta a
árvore para a qual nos trazias?

 

[230b] F: A própria.

 

S: Por Hera, que bela pousada!~Um plátano corpulento e magnífico, com
uma folhagem excelente e digna desse ambiente sagrado. O~vigor da
floração oferece ao lugar o melhor dos aromas e a fonte agradabilíssima
sob o plátano nos traz água bem fria, como comprovamos com os pés.
Parece ser um templo para alguma Ninfa e para Aqueloo, a julgar pelas
estátuas votivas de argila e imagens de mármore (\emph{korôn te kaì
agalmátōn}).~[230c] Se desejas algo mais, há ainda uma doce e muito
agradável brisa (\emph{eúpnoun}) do lugar, estival e melodiosa, que ecoa
o coro das cigarras (\emph{tettígōn chorôi}). A~relva é certamente o
maior dos requintes (\emph{kompsótaton}), porque a escarpa suave
naturalmente solicita que reclinemos nossas cabeças de modo maravilhoso.
Assim, ó querido Fedro, és o melhor dos guias para estrangeiros
(\emph{exenágētai}).

 

F: Parece"-me muito extravagante (\emph{atopṓtat\gr{ό}s}), ó admirável. Pelo
que dizes, sem nenhum artifício (\emph{atéchnôs}), dá"-me a impressão de
seres guiado como um estrangeiro e não como um autóctone. [230d]
Parece que jamais te ausentaste da cidade rumo à terra estrangeira, nem
mesmo saíste para além dos muros (\emph{éxô teíchous}).

 

S: Perdoa"-me, ó excelente, é que eu sou um amante do aprendizado
(\emph{philomathḕs}), nem os campos nem as árvores querem me ensinar,
somente os homens da cidade.~E tu, realmente, pareces ter encontrado
(\emph{hēurēkénai}) o fármaco do meu êxodo.~Tal como os que agitam um
ramo para uma criatura faminta, ou algum fruto que os conduza, tu, do
mesmo modo, estendendo discursos provenientes de livros, parece que me
conduzirás por toda a Ática ou para qualquer outro lugar que queiras.
[230e] Agora, tendo chegado aqui, vou reclinar"-me e encontrarás a
posição que te seja mais cômoda à leitura. Depois disso, lê.

 

F: Escuta:

 

``Já estás ciente acerca dos meus assuntos e creio que ouviste acerca do
que pode acontecer conosco.~[231a] Espero que não me advenha nenhum
infortúnio (\emph{atychêsai}) só porque me ocorreu (\emph{tygchánō}) de
não estar te amando (\emph{ouk}~\emph{erastḕs}), como aqueles que, tão
logo tenha cessado o seu desejo (\emph{epithymías paúsōntai}),
arrependem"-se (\emph{metamélei}) do que bem fizeram. Outros, por outra
parte, não tem tempo hábil para mudar o pensamento (\emph{metagnônai}),
e não é por coação, mas espontaneamente, que bem fazem o que podem ao
amado, desejando"-lhes (\emph{bouleúsainto}) o melhor nos assuntos
pessoais. Os amantes (\emph{oi}~\emph{men}~\emph{erôntes}) observam o
que fizeram de bom e de mau pelo amor (\emph{érôta}), e, pelos
sofrimentos (\emph{pónon}) causados, consideram [231b] antiquadas as
gratificações (\emph{chárin}) endereçadas aos seus amados
(\emph{tois}~\emph{erôménois}) de outrora.

Os que não amam (\emph{mè}~\emph{erôsin}), por seu turno, não podem dar
tal pretexto para o abandono de assuntos pessoais, nem consideram os
sofrimentos (\emph{pónous}) passados, nem os desentendimentos causados
com os parentes. Ao despojarem"-se de todos esses males, não lhes resta
nada, senão fazer voluntariamente (\emph{prothýmos}) aquilo que
consideram poder agradar (\emph{chareîsthai}) o companheiro. [231c]
E se por esse valor aos amorosos tanto se faz, o motivo é que estes,
sobretudo, gostam de dizer que estão amando, e ficam a ponto de
hostilizarem quem quer que seja com palavras e ações, só para agradarem
(\emph{charídzesthai}) seus amados. Aliás, é fácil saber se dizem a
verdade, pelo tanto de amor (\emph{erasthôsin}) que lhes passam a
dedicar, pois fariam de tudo para ele, e é evidentemente que a outros
hostilizariam, se isso lhes fosse requisitado [pelo amado].
[231d] De algum modo, é verossímil (\emph{eikós esti})~aceitar
semelhante dificuldade àquele que passou por esse infortúnio, afinal,
quem se livraria (\emph{apotrépein})~dele, mesmo sendo experiente
(\emph{émpeiros})? E eles mesmos concordam que estão mais doentes
(\emph{noseîn}) do que prudentes (\emph{sōphroneîn}), e sabem que pensam
(\emph{phronoûsin}) mal, mas não podem dominar"-se (\emph{krateîn}).~Como
é que em estado de bom pensamento (\emph{eû}~\emph{phronésantes}),
poderiam considerar belas as decisões (\emph{bouleúontai}) tomadas
naquele estado anterior? E se tu procuras escolher o melhor entre os
amantes (\emph{tôn eróntôn}), a eleição pode ser feita entre poucos
(\emph{ex olígôn}), mas seria mais proveitosa para ti se abarcasse
outros (\emph{ek tôn állôn}) entre muitos (\emph{ek pollôn}). [231e]
Desse modo, é muito maior a esperança de encontrar, na multidão
(\emph{en tois polloîs}), alguém que seja digno da tua amizade.

Se receares a lei (\emph{nómon}) estabelecida, não te afetes pelas
observações vergonhosas feitas pelos homens, pois é verossímil
(\emph{eikós esti}) [232a] que os amantes, considerando"-se honrados
pelo amado, tal qual eles mesmos o veneram, exaltem"-se em discursos e
honradamente (\emph{philotimouménous}) mostrem a todos que não têm
sofrido (\emph{pepónêtai}) em vão. Por outro lado, os que não estão sob
o efeito do amor (\emph{mḕ}~\emph{erôntas})~são superiores
(\emph{kreíttous}) ao escolher o melhor (\emph{béltiston}) em vez da
opinião (\emph{anti tês dóxês}) dos homens.~Ademais, necessariamente,
todos observam os amantes (\emph{erôntas}) com seus amados
(\emph{erôménois}), bem como as ações que praticam, e quando os veem a
conversar [232b] entre si, pensam que já consumaram ou que estão
para consumar o seu desejo (\emph{epithymías}).

Por outro lado, aos que não estão sob o efeito do amor
(\emph{mḕ~erôntas}), não se culpa de tentar algo só por causa da
companhia, sabendo que é necessário dialogar (\emph{dialégesthai}) para
sedimentar a amizade (\emph{philían}) e para qualquer outro deleite
(\emph{hēdonḗn}). Se considerares difícil a permanência da amizade, uma
vez que qualquer tipo de diferença pode trazer desagrado a ambos,
[232c] quando tudo aquilo que fazes de grandioso torna"-se
prejudicial (\emph{blábên}), desse modo, é bem verossímil
(\emph{eikótôs}) que tenhas temor (\emph{phoboîo}) dos amantes
(\emph{erôntas}). Muitas são as aflições (\emph{lupoûnta}) desses
amantes, além de considerarem que tudo lhes traz prejuízo
(\emph{blábêi}). E é por isso que tentam afastar (\emph{apotrépousin})
os amados (\emph{erôménôn}) da companhia de outros, temendo
(\emph{phoboúmenoi}) que detentores de bens os superem no dinheiro ou
que lhes sejam superiores na educação. Protegem o amado de qualquer
outro que possa deter esses bens. [232d] Quando persuadem"-no a
odiá"-los, colocam"-no apartado dos amigos, e no caso de considerarem a si
mesmos melhores que aqueles, acabam provocando desavenças.

Os que, pela sorte, não estão sob o efeito do amor (\emph{mḕ~erôntes
étychon}), mas pela virtude (\emph{aretḕn})~praticam o desejado, não
sentem ciúme (\emph{phthonoîen}) dos acompanhantes do amado, e,
certamente, não querem odiá"-los. Por considerarem"-se desprezados, eles
querem ser úteis aos ditos amigos do amado. A~partir dessa prática, é
muito maior a esperança de ter amizade (\emph{philían}) com eles em vez
de aversão (\emph{échthran}) [232e].

Na verdade, muitos amantes (\emph{erṓntōn}) desejam (\emph{epethúmēsan})
o corpo antes de lhes conhecerem seus hábitos e de experimentarem outras
familiaridades, de modo que não é evidente que decidam tornar"-se amigos
(\emph{boulḗsontai phíloi eînai}), tão logo tenha cessado o seu desejo
(\emph{epithymías paúsôntai}).~[233a] Para os que não estão amando
(\emph{mḕ~erôsin}) e que praticam primeiramente entre si a amizade, não
é verossímil (\emph{eikòs}), partindo daquilo que foi bem realizado, que
a amizade (\emph{philían}) diminua. Ao contrário, esse modo produz uma
memória (\emph{mnēmeĩa}) prévia que alimenta o que ainda está por vir.
Na verdade, é melhor que tu sejas persuadido (\emph{peithoménôi}) por
mim do que por um amante (\emph{erastêi}).~Os amantes elogiam as ações e
as palavras, mesmo que estas não sejam as melhores, seja pelo receio de
serem odiados (\emph{mḕ~apéchthôntai}), seja por terem se tornado débeis
em seu discernimento pelo desejo (\emph{epithymían}). [233b] São
essas as coisas que o amor (\emph{érôs}) manifesta: desafortunados
(\emph{distychoûntas}) que consideram molesto o que não proporciona
sofrimento (\emph{mḕ}~\emph{lupén}) aos demais e afortunados
(\emph{eutychoûntas}) para os quais ocorre (\emph{tygchanein}) ter de
elogiar (\emph{epaínou}) forçosamente o valor daquilo que não lhes é
prazeroso (\emph{mḕ~hedonês}).~Assim, é muito mais conveniente
apiedar"-se dos amantes (\emph{eleeîn toîs erôménois}) do que invejá"-los
(\emph{zeloûn}). E~se por mim fores persuadido (\emph{peíthêi}), em
primeiro lugar, não serei o teu guardião do prazer
(\emph{hêdonḕn}~\emph{therapeúôn}), mas das utilidades (\emph{ôphelían})
futuras, [233c] não sendo vencido pelo amor (\emph{hyp'érôtos
hēttṓmenos}), mas permanecendo senhor de mim mesmo
(\emph{emautoû}~\emph{kratôn}). Não me arrastarei a um ódio extremo por
motivos fúteis e terei pouca ira em função de motivos maiores,
desculpando as faltas involuntárias e tentando evitar
(\emph{apotrépein}) as voluntárias. Esse é o testemunho
(\emph{tekmḗria}) de uma amizade (\emph{philían}) que durará muito
tempo. Se tu pensas que a mais forte das amizades não ocorreria sem a
presença do amor (\emph{erôn tygchánêi}), [233d] é necessário ainda
considerar (\emph{enthymeîsthai}) que nem aquela amizade aos muitos
filhos haveria, nem aos pais ou às mães, nem a fiel amizade dos amigos
teríamos adquirido (\emph{pistoùs phílous ekektḗmetha}), por isso a
amizade não provém daqueles desejos (\emph{epithymías}) antes
mencionados, mas de outras práticas.

Em seguida, se é necessário agradar (\emph{charídzesthai}) sobretudo a
quem precisa, convém que os beneficiados não sejam os melhores
(\emph{beltístous}), mas os muito isentos em recurso
(\emph{aporôtátous}), pois, livrados dos maiores males, esses terão
ampla gratidão (\emph{chárin}). Certamente, então, nos banquetes
particulares não vale a pena convidar os amigos (\emph{phílous}), mas os
que clamam e os que necessitam saciar"-se, pois estes se tornarão
carinhosos (\emph{agapésousin}), companheiros (\emph{akolouthḗsousin}),
e virão a nossa porta sabendo comprazer"-se e, não com pouca gratidão
(\emph{chárin}), desejar"-nos-ão boas coisas.~Da mesma maneira, convém
não agradar (\emph{charídzesthai}) aos muito necessitados, mas aos que
especialmente podem oferecer gratidão (\emph{chárin}); não só aos que
clamam [234a], mas aos dignos dessas práticas; não tanto aos que
desfrutam da juventude, mas aos que, na velhice, repartirão contigo os
benefícios; não aos que, por terem realizado seu intento, passam logo a
dedicar"-se aos outros, mas aos que se envergonham e calam diante de
todos; não aos que se dedicam por um curto tempo, mas aos que serão
amigos por toda a vida; não aos que, cessado o desejo (\emph{oi
pauómenoi tês~epithymías}), buscarão pretexto para o ódio, enquanto os
outros tendo [234b] passado a juventude mostrar"-lhes-ão a virtude
(\emph{hoì pausaménou tês hṓras t\gr{ό}te tḕn hautõn aretḕn epideíxontai}).

Então, recorda"-te do que foi dito e põe no teu ânimo (\emph{enthymoû})
que os amigos advertem aos seus amantes (\emph{tous mèn erôntas oi
phíloi nouthetoûsin}) por seu mau comportamento, ao passo que aos que
não estão sob o efeito do amor (\emph{mḕ}~\emph{erôsin}), nem mesmo seus
familiares lhes fazem censuras de qualquer tipo, pois eles decidem
(\emph{bouleuoménois}) os seus próprios males.

Talvez, então, tu me perguntes se eu te aconselho a agradar
(\emph{charídzesthai}) todos aqueles que não estão sob o efeito de Eros
(\emph{mḕ}~\emph{erôsi}). Eu não considero que o amante
(\emph{ton}~\emph{erônta}), em todo o caso, te incentivasse a essa
maneira de pensar (\emph{diánoian}) com relação aos amorosos
(\emph{tous}~\emph{erôntas}). [234c] Nem aquele que pelo discurso
recebe semelhante honra graciosa (\emph{cháritos}), nem tu, se quisesses
manter"-te escondido dos outros, poderias agir de modo semelhante. É~necessário que disso não surja nenhum dano (\emph{bláben}), mas que
ocorra o proveito (\emph{ôphéleían}) a ambos.~Eu considero o que foi
dito suficiente. Se desejares saber algo que foi negligenciado, pergunta
(\emph{erṓta})''.

 

Como te parece o discurso (\emph{ho}~\emph{lógos}), ó Sócrates? Não é
maravilhoso (\emph{huperphýôs}), entre tantas outras razões,
especialmente no vocabulário (\emph{onómasin}) empregado?

 

[234d] S: Divino mesmo, ó companheiro, a ponto de eu estar atordoado
(\emph{ekplagênai}). E~essa minha afecção (\emph{épathon}) foi gerada
por ti, ó Fedro, pois a visão (\emph{apoblépôn}) que tive de ti foi
radiante durante a leitura do discurso (\emph{toû}~\emph{lógou}).
Considero"-te melhor do que eu para apanhar esses discursos e sigo"-te,
cabeça divina (\emph{theías kephalês}), como em um cortejo báquico
(\emph{sunebákcheusa}).

 

F: Já estás brincando (\emph{paídzein}), não é?

 

S: Pareço por acaso brincar (\emph{paídzein}) e não me esforçar
(\emph{espoudakénai})?

 

[234e] F: De modo algum, ó Sócrates, mas como dizes a verdade,
diante de Zeus protetor da Amizade (\emph{alêthôs eipe pros Diòs
philíou}), considera"-te na iminência de pronunciar entre os helenos, a
respeito do mesmo assunto, outro, melhor e mais extenso [discurso]
que este?

 

S: O quê? É preciso que eu e tu elogiemos (\emph{epainethênai}) o
discurso (\emph{tòn lógon}) desse criador que disse o que devia? Ele não
é somente claro (\emph{saphê}) e perfeito (\emph{stroggúla}), mas também
exato (\emph{akribôs}) em cada uma das palavras entalhadas
(\emph{onomátôn apotetórneuetai})? Se for preciso, agradeceremos
(\emph{chárin}) o autor, apesar da minha ignorância me obscurecer
(\emph{elathen}) [235a]. Tomando exclusivamente o pensamento
retórico dele (\emph{rhêtorikôi autoû mónôi tón noûn proseîchon}),
acredito que nem mesmo o próprio Lísias consideraria suficiente. E~me
pareceu, ó Fedro, se não queres dizer outra coisa, que ele afirmou o
mesmo duas ou três vezes, como se não tivesse muitos recursos adicionais
para fazê"-lo, acerca do mesmo assunto, sem nenhum interesse. E~pareceu"-me ainda uma demonstração juvenil (\emph{neanieúesthai
epideiknúmenos}) de quem quer falar tanto de uma perspectiva quanto de
outra e, em ambos os casos, da melhor maneira possível.

 

[235b] F: Não é como dizes, ó Sócrates, uma vez que isso é o que
justamente o discurso (\emph{ho lógos}) tem de melhor, o fato de não ter
negligenciado nenhum dos assuntos convenientes em sua performance. Desse
modo, junto a este discurso, ninguém seria capaz de proferir
(\emph{eipeîn}) outro, mais extenso e mais digno (\emph{pleíô kaì
pleíonos áxia}).

 

S: Nesse ponto, eu jamais poderia ser persuadido (\emph{pithésthai}) por
ti. Os antigos sábios~(\emph{sophoì}), homens e mulheres que proferiram
discursos e escreveram (\emph{eirêkotes kaì gegraphótes}),
refutar"-me"-iam (\emph{exelégksousí}) se, para agradar"-te
(\emph{charidzómenos}), eu concordasse contigo.

 

[235c] F: Quais são eles e onde ouviste algo superior?

 

S: De imediato, assim, não posso dizer. Mas é evidente que ouvi ou da
bela Safo ou do sábio (\emph{sophoû}) Anacreonte, ou de algum outro
escritor (\emph{suggraphéôn}). De onde tiro os testemunhos de que falo?
De certa plenitude (\emph{plêrés}), ó divino, que sinto no peito e pela
qual poderia dizer outras tantas coisas nada inferiores. Bem sei que não
é por mim que tenho em mente (\emph{ennenóêka}) essas coisas, pois
conheço minha própria ignorância (\emph{emautôi amathían}). Deixemos
isso de lado, creio que são outras as fontes que, pela audição, me
encheram (\emph{peplêrôsthai}) como a uma vasilha.~[235d] E foi por
estupidez que me esqueci (\emph{epilélêsmai}) dessas coisas, bem como de
quem as ouvi.

 

F: Mas, ó excelentíssimo, disseste do melhor modo possível. De quem e
como ouviste, eu não te ordeno que digas, desde que cumpras o seguinte:
melhor que este livro, promete (\emph{hypéschêsai}) dizer outro
[discurso] em nada inferior, ficando dele afastado. E~eu prometo
(\emph{hypischnoûmai}), tal qual os nove arcontes, oferecer"-te um ícone
(\emph{eikóna})~dourado em tamanho natural (\emph{isométrêton}) no
templo de Delfos, não só o meu, mas também o teu.

 

[235e] S: És amicíssimo e de ouro verdadeiro, ó Fedro, se julgas que
eu poderia dizer todas as coisas em que Lísias se enganou [no
discurso] e que seria preciso proferir outro junto ao dele. Creio que
isso não abateria nem o mais inábil dos escritores (\emph{suggraphéa}).
Começando pelo discurso (\emph{ho lógos}), quem considera dizer que é
melhor e necessário agradar (\emph{charídzesthai}) o não afetado pelo
amor em vez do amante (\emph{mḕ~erônti mâllon~ḕ~erônti}) [236a] quer
evitar o encômio do prudente (\emph{to phónimon egkômiádzein}) e o
vitupério do insensato (\emph{to áphon pségein}), e, sendo assim,
necessariamente, terá ainda ele algo a dizer?~Mas penso que é necessário
livrar e desculpar o orador, e elogiar (\emph{epainetéon}) a sua
disposição (\emph{diáthesin}), não a invenção (\emph{heúresin}), e
quando a sua disposição (\emph{diathései}) não é suficiente ou difícil
de encontrar (\emph{heureîn}), elogiamos, para além dela, a invenção
(\emph{heúresin}).

 

F: Concordo com o que dizes e bem medido (\emph{metríôs}) me parece.
Farei então o seguinte, estabelecerei como base para ti que o amante
está mais doente que o não afetado pelo amor (\emph{tòn erônta toû
mḕ~erôntos mâllon noseîn}), [236b] deixando o resto de lado, ao
dizer outro [discurso] mais extenso e mais digno (\emph{pleíô kaì
pleíonos áxia})~que o de Lísias, estarás em ouro maciço junto a oferenda
dos Cipsélidas~em Olímpia.

 

S: Tomaste a sério, ó Fedro, porque eu brinquei acerca do teu favorito.
Consideras mesmo que eu verdadeiramente vá falar, contra aquela
sabedoria (\emph{sophían}), outro [discurso] em algo mais variado.

 

F: Nesse caso, ó querido, chegas a colher o mesmo. [236c] É
necessário que digas, tanto quanto lhe seja possível, um [discurso]
melhor em tudo, para que não nos obriguemos a realizar uma grosseria de
comediantes, trocando mutuamente de papéis, e que não queiras ainda
forçar"-me a dizer"-te aquele ``ó Sócrates, se não conheço Sócrates,
estaria esquecido de mim mesmo'', ou aquele ``desejas dizer, mas, no
entanto, fica enternecido''.~Põe na tua cabeça que não sairemos daqui
antes que digas o que tens a dizer. Estamos sozinhos num lugar ermo,
além do que sou mais forte e mais jovem, de modo que, de toda maneira,
``vê se entende o que te digo'', não queiras falar à força (\emph{pròs
bían boulêthêis}), mas de bom grado.

 

[236d] S: Ó bem"-aventurado Fedro, eu seria risível se, de improviso,
um inábil como eu procurasse equiparar"-me a um bom compositor
(\emph{agathòn poiêthḕn}) nesses assuntos.

 

F: Perceba que é melhor parar (\emph{paûsai}) de vangloriar"-te para cima
de mim, pois eu tenho algo a dizer para justamente forçar"-te a falar.

 

S: Não pode ser como dizes.

 

F: Não? Pois então eu digo, será um juramento (\emph{hórkos})\emph{~}a
minha fala (\emph{lógos}), prometo a ti -- e a alguém mais talvez, a
algum deus daqui? -- [236e] e até mesmo a esse plátano.~Se não
disseres outro discurso no lugar deste (\emph{enantíon autês taútes}),
nunca mais te comunico exposição nenhuma, de quem quer que seja.

 

S: Que infâmia, ó abominável, bem encontraste (\emph{eũ anēũres})~um
meio para forçar um homem que ama discursos (\emph{philológôi})~a fazer
o que dizes.

 

F: Então porque te esquivas (\emph{stréphēi})?

 

S: Em absoluto, uma vez que tu prometeste isso, como seria possível
apartar"-me desse banquete (\emph{thoínês})?

 

[237a] F: Diz então.

 

S: Sabes como farei?

 

F: A respeito de quê?

 

S: Falarei encoberto (\emph{enkalypsámenos}), para que rapidamente
percorra o discurso (\emph{tòn lógon}) e para que não te veja, temendo
vacilar de vergonha.

 

F: Simplesmente fala, de resto faz como queiras (\emph{boúlei}).

 

S: Vinde, ó Musas, tanto na forma de odes melodiosas (\emph{lígeias}),
como na dos músicos de Ligure (\emph{Ligúôn}), ambas são epônimas,
``toma da minha'' palavra (\emph{toû mýthou}) e me obriga a falar
(\emph{legeîn}) da melhor forma possível (\emph{béltistos}), para que o
teu companheiro, parecendo"-lhe ser sábio (\emph{sophòs}) num primeiro
momento, tenha agora melhor reputação (\emph{dóxêi}) ainda. [237b]
Era uma vez um menino, na verdade um jovem, que era muito belo
(\emph{kalós}), e que por isso tinha muitos amantes (\emph{erastaì}). Um
entre os seus aduladores (\emph{aimúlos}), não menos amoroso
(\emph{erôn}) que outros, foi capaz de convencê"-lo (\emph{epepeíken}) de
que não o amava (\emph{ouk}~\emph{erôiê}), e em seguida de que era
preciso antes agradar a quem não ama em vez dos amantes (\emph{hôs
mḕ~erônti prò toû erôntos déoi charídzesthai}), ele dizia o seguinte:

 

``Acerca de tudo isso, ó menino, só há um princípio (\emph{mía archḕ})
aos que desejam bem deliberar (\emph{bouleúsesthai}), [237c] saber
necessariamente acerca do que se trata em cada deliberação
(\emph{boulé}) ou então será forçoso que haja engano (\emph{hamartánei})
em todo o resto. Muitos se esquecem (\emph{lélêthen}) de que não
conhecem cada uma das essências (\emph{ousían}). E~como não sabem
acordarem"-se no princípio de uma verificação (\emph{diomologoûntai en
archêi tês}~\emph{sképseôs}), eles pagam o preço verossímil
(\emph{eikós}) de não concordarem (\emph{homologoûsin}) nem consigo
mesmos, nem com outros. Não soframos, eu e tu, daquilo que censuramos
neles, mas é necessário conhecermos o discurso (\emph{ho lógos}) que
propicia a melhor amizade (\emph{philían}), se é o direcionado ao amante
ou ao que não o é, para sabermos, em seguida, as potencialidades do amor
e de sua natureza, estabelecendo uma definição acordada (\emph{homología
thémenoi hóron}) [237d] para uma verificação (\emph{sképsin}), de
modo a vê"-lo e referi"-lo no que oferece de útil (\emph{ôpheleían}) e de
dano (\emph{blabên}).

É claro para todos que o amor é um desejo (\emph{epithymia}) e que
também os que não estão sob o efeito do amor (\emph{mḕ~erôntes}) desejam
(\emph{epithymoûsin}) os belos, isso nós sabemos. Mas como discerniremos
o amante do não afetado pelo amor (\emph{tòn erônta te kai
mḕ~krinoûmen})? É preciso entender (\emph{noêsai}) que há, em cada um de
nós, duas formas (\emph{idéa}) que nos presidem e nos conduzem: uma
delas seguimos onde quer que nos leve, é a do desejo dos prazeres inatos
(\emph{emphytos oûsa epithymía hedonôn}), a outra é a da opinião
(\emph{dóxa}) adquirida, que tende para o melhor (\emph{toû arístou}).
[237e] Algumas vezes, essas duas tendências em nós estão de acordo
(\emph{homonoeîton}), outras vezes em conflito (\emph{stasiádzeton}), de
modo que algumas vezes predomina uma delas, outras vezes a outra.

A opinião (\emph{dóxes}) da melhor razão domina e conduz pelo poder do
que é chamado de prudência (\emph{sophrosýne}), ao passo que o desejo
irracional arrasta para os prazeres (\emph{epithymías dè
alógôs~helkoûsês~epi hêdonàs}), deflagrando aquilo que no início
denominamos desmesura (\emph{hýbris}).~[238a] A desmesura
(\emph{hýbris}) tem muitos nomes, muitos membros e partes, e se por
acaso (\emph{týchêi}) ela tomar alguma dessas formas, oferece o epônimo,
nem belo e nem digno, a todos que a trazem consigo. Quando a ingestão de
alimentos domina a razão (\emph{lógou}) e o melhor (\emph{arístou}),
entre outros desejos (\emph{epithymiôn}), o desejo (\emph{epithymía}) é
o de um glutão (\emph{gastrimargía}), que conferirá esse título a quem o
possua. [238b] Sobre a tirania (\emph{tyranneúsasa}) das bebidas,
conduzindo aquele que a tem, produz aquela óbvia denominação. Da mesma
maneira, as coisas a estas aparentadas, bem como o surgimento de desejos
correlatos, sempre são soberanos (\emph{dynasteuoúses}) e tem cada qual
o seu nome.

De acordo com tudo o que foi dito antes, é quase evidente, podemos
dizer, ou não dizer, da forma mais clara possível (\emph{saphésteron}):
sem a razão (\emph{aneu lógou}), o desejo (\emph{epithymía}) domina a
opinião (\emph{doxês}) e conduz diretamente ao prazer da beleza
(\emph{hedonèn achtheîsa kállous}), [238c] então, os desejos
(\emph{epithymiôn}) congêneres com força se lançam à beleza dos belos
corpos, vencendo todos, eles adquirem a força do seu nome, sendo
chamados de amor.''

 

E então, ó querido Fedro, pareço"-te, como a mim mesmo, afetado por algo
divino (\emph{theion páthos peponthénai})?

 

F: Completamente, ó Sócrates, foste tomado por uma não usual fluência.

 

S: Cala"-te agora e escuta"-me: realmente esse lugar parece divino
(\emph{theîos éoiken ho tópos eînai}) e não te espantes se eu, muitas
vezes, no discurso, for tomado pelas Ninfas. [238d] Agora mesmo, ao
falar, não estive longe de entoar um ditirambo.

 

F: Dizes a mais pura verdade.

 

S: E isso é por tua causa. Mas escuta o restante, pois talvez eu possa
me apartar (\emph{apotrápoito}) dessa ameaça. Deixemos isso ao cuidado
do deus (\emph{theôi}) e ocupemo"-nos novamente do discurso (\emph{pálin
tôi logôi}) dirigido ao jovem.

 

``Que seja assim, ó meu bravo amigo. O~assunto da deliberação
(\emph{bouletéon}) já foi mencionado e definido. Vejamos, agora, o que
nos falta ainda dizer sobre ele, qual é a utilidade (\emph{ôphelía}) ou
o dano (\emph{blábê}) que pode, de modo verossímil (\emph{ex eikótos}),
advir a quem agrada (\emph{charidzoménoi}) ao amante e a quem
[agrada] ao não afetado pelo amor.~[238e] Aos que começam a ser
escravizados pelo desejo (\emph{epithymías}) é necessário que, de alguma
maneira, pelo prazer (\emph{hedonêi}), busquem no amado o que lhes dá
esse prazer (\emph{hḗdiston}). Ao doente (\emph{nosoûnti}) é necessário
que tudo lhe seja agradável (\emph{hedù}) e que nada o contrarie
(\emph{antiteînon}), pois causa aborrecimento tudo o que é mais forte ou
de força similar ao seu desejo. O~amante não admite, da mesma forma, nem
superioridade nem igualdade ao seu predileto (\emph{paidikà}),
[239a] e sempre procura rebaixá"-lo, tornando"-o inferior. O~ignorante
(\emph{amathḕs}) é inferior ao sábio (\emph{sophoû}), assim como o
covarde é inferior ao corajoso, o inábil na fala inferior ao homem da
retórica (\emph{rhētorikoũ}) e o lento inferior ao sagaz.

Esses males e ainda outros maiores surgem no pensamento
(\emph{diánoian}) do amante com relação ao amado (\emph{erastḕn
erôménôi}), e, naturalmente, são [males] ligados ao prazer
(\emph{hḗdesthai}), podendo até mesmo provocá"-los
(\emph{paraskeuázdein}), quando [o amante] fica momentaneamente
apartado do seu prazer (\emph{hêdéos}). Necessariamente, o amante é
ciumento e afasta [o amado] de todas as outras companhias que lhes
sejam proveitosas (\emph{ôphelímôn}), [239b] sobretudo das que
enriqueçam o homem. O~dano que isso causa é enorme, mas o maior deles é
o de evitar que [o amado] torne"-se prudentíssimo
(\emph{phronimṓtatos}).

Quando o amante mantém o afastamento necessário com relação ao seu
predileto, temendo por este ser depreciado, isso ocorre
(\emph{tugchánei}) devido à divina filosofia (\emph{theía
philosophía}).~Outras coisas são maquinadas para que [o amado] possa
ficar ignorante em tudo (\emph{pant' agnoôn}) e só tenha olhos
(\emph{apoblépôn}) para o seu amante (\emph{erastén}), como se fosse o
mais agradável (\emph{hédistos}), entretanto, isso pode ser o mais danoso
(\emph{blablerṓtatos}) a si mesmo. [239c] Então, segundo esse
pensamento (\emph{diánoian}), quer para o tutor, quer para o
companheiro, em nenhuma parte é proveitoso ao homem sentir amor
(\emph{érôta}).

Depois dessas coisas é preciso conhecer a disposição e os cuidados
(\emph{héxin te kaì therapeían}) relativos ao corpo, de como ele deve
ser cuidado (\emph{therapeúsei}) pelo seu comandante e em que medida é
forçado a perseguir (\emph{diókein}) o prazer antes do que é o melhor
(\emph{hedù prò agathoû}). Veremos um amante a perseguir um jovem
delicado e não muito valente, alguém que não foi bem nutrido na pureza
do sol, mas na companhia da sombra, inexperiente nas fadigas e suores
dos labores masculinos, mas experiente no estilo de vida delicado dos
afeminados, [239d] embelezando"-se com cores e adornos incomuns, vida
acompanhada as outras práticas que dessas derivam, práticas evidentes e
indignas até mesmo para avançarmos no comentário, mas delimitemos a
questão primordial (\emph{kephálaion}) acerca disso para passarmos para
o outro assunto. Esse corpo, seja na guerra ou em outros afazeres,
proporciona, por um lado, coragem aos inimigos e, por outro, medo aos
amigos e mesmo aos amantes.

Deixemos de lado o que é evidente, pois é necessário que delimitemos
agora [239e] qual dessas atitudes nos é útil (\emph{ôphelían}) ou
danosa (\emph{blábên}) ao recebermos a companhia e a tutela de um
amante. É~claro para todos, e especialmente para o amante (\emph{tôi
erástêi}), que ele se regozije de que o amado possa ser privado de tudo
aquilo que é mais querido, mais amistoso e diviníssimo. Prefere que ele
fique afastado de pai, mãe, parentes e amigos, [240a]
considerando"-os todos como empecilhos e censores à sua prazerosa
(\emph{hêdístes}) convivência. Mas se [o amado] possui ouro ou
qualquer outro tipo de posse, não será da mesma maneira fácil ser
capturado e mantido sob controle. Por conta disso, é forçoso que o
amante (\emph{erastḕn})\emph{~}sinta ciúmes de jovens que têm recursos,
e gostem (\emph{chaírein}) que eles sejam arruinados. E~ele ainda
preferiria que seu amado viesse a ficar sem se casar (\emph{ágamon}),
sem filhos (\emph{ápaida}) e sem casa (\emph{áoikon}), tanto tempo
quanto fosse possível, para que pudesse colher o seu doce desejo
(\emph{epithymôn}) pelo maior tempo possível.

 
 

Existem outros males ainda, mas algum~\emph{daímôn}~os misturou
(\emph{émeixe}) aos maiores prazeres (\emph{hêdonén}) momentâneos,
[240b] como no caso do adulador, terrível fera de enorme prejuízo
(\emph{kólaki, deinôi thêríôi kaì blábêi megálêi}), para o qual, ao
mesmo tempo, a natureza mesclou (\emph{epémeixen hê phýsis}) algum tipo
de prazer requintado (\emph{hêdonén tina ouk ámouson}).~Como os
[prazeres] de uma cortesã (\emph{hetaíran}), que poderia ser
censurada como danosa (\emph{blaberòn pséxeien}), assim como outras
similares criaturas e ocupações, prazerosas (\emph{hedístoisin}) no
início, pelo menos durante um dia. O~amado, então, com relação ao seu
amante, torna"-se danoso, bem como o convívio prolongado torna"-se o maior
desprazer possível (\emph{aêdéstaton}). [240c] Como diz o antigo
ditado, ``cada idade agrada aos da mesma idade'', pois considero que
idades similares conduzem a uma similaridade de prazeres
(\emph{hedonàs}) e proporciona uma semelhante amizade (\emph{philían}),
da mesma maneira como a convivência contínua entre eles causa a
saciedade. E~dizem que o constrangimento é pesado em todos os casos e
para todos, especialmente aos amantes (\emph{erastḕs}) que tem essa
diferença etária com relação ao amado (\emph{pròs paidikà}). O~velho que
convive com um jovem, nem de dia nem de noite abandona voluntariamente
seu amado, [240d] mas é conduzido pela necessidade e pelo aguilhão
daquele que sempre lhe oferece prazer (\emph{hêdonàs}), vendo,
escutando, tocando, com todos os sentidos percebendo o amado, como se
ele servisse justamente aos seus prazeres (\emph{hêdonês}). Que tipo de
exortação ou de prazeres (\emph{hêdonàs}) o amante oferece ao amado
durante o tempo de convívio, para que não cheguem a extremos do
desprazer (\emph{aêdías})? Uma vez que o jovem vê aquele olhar que já
não está na flor da idade, acompanhado de outras coisas desse tipo, que
nem são agradáveis de ouvir falar, [240e] para não ser obrigado
sempre a estar disposto a essa ocupação, sendo vigiado com suspeita
constante do seu guardião em meio a todos, ouvindo elogios
(\emph{epaínous}) hiperbólicos e inoportunos (\emph{akaírous}), bem como
censuras (\emph{psógous}) inadmissíveis a um sóbrio (\emph{nḗphontos}).
E~quando ele está entregue à bebida, todas essas coisas, além de
intoleráveis, passam a ser vergonhosas, especialmente pela tagarelice
excessiva e pelo atrevimento empregado.

Esse amante, danoso e desagradável (\emph{erôn mèn blaberós te kaì
aêdḗs}), quando deixa de amar, logo se torna indigno de confiança
(\emph{ápistos}), pois todos os juramentos (\emph{hórkôn}) e todas as
súplicas professadas mantinham a companhia com dificuldade, [241a]
uma vez que a relação já era penosa de suportar, mesmo quando havia a
esperança de lhe trazer coisas boas (\emph{agathôn}). Quando é
necessário mudar (\emph{metabalôn}) sua própria conduta, o amante passa
a dominar a si mesmo e a estar preparado, com a inteligência e a
prudência em vez do amor e da loucura (\emph{noûn kaì sôphrosýnên ant'
érôtos kaì manías}), e dessa forma ele esquece o seu predileto
(\emph{lélêthen ta paidiká}). Enquanto o amado demanda as graças
[prometidas], relembrando (\emph{hypomimnḗiskon}) os feitos e ditos,
como se pudesse dialogar ainda com ele, o amante, por outro lado, por
vergonha não diz a ninguém o que ocorreu, e de nenhum modo confirma os
juramentos impensados do início (\emph{anoḗtou archês horkômósiá}) e as
promessas realizadas, [241b] pois agora está em sua plena
inteligência (\emph{noûn})~e salvo pela prudência
(\emph{sesôphronêkós}), que o impede de agir de maneira semelhante ou
fazer aquelas coisas novamente (\emph{pálin}). Ele foge de tudo isso,
tendo cometido uma falta pela força do amor anterior (\emph{prìn
erastḗs}), e sendo alterada (\emph{metapesóntos}) a concha de lado, ele
se retira na direção alternada (\emph{híetai phugêi~metabalṓn}).

Aquele que, por outro lado, é levado agora a perseguir a irritação e a
imprecação contra os deuses, desconhece tudo desde o início, de que não
devia agradar (\emph{charídzesthai}) ao amoroso (\emph{erônti}), forçado
pela falta de intelecto (\emph{anoétôi}) [241c], mas que seria muito
melhor não estar sob o efeito do amor (\emph{m}ḕ~\emph{erônti}) e manter
o intelecto (\emph{noûn}).~Caso contrário, seria obrigado a entregar"-se
a alguém sem crença (\emph{apístôi}), mal"-humorado, invejoso,
desagradável (\emph{aêdeî}), danoso (\emph{blaberôi}) para a essência
(\emph{pròs ousían}), danoso (\emph{blaberôi}) para a disposição do
corpo e, sobretudo, muitíssimo danoso (\emph{blaberôtátôi}) para a
educação da alma (\emph{psychês paídeusin}), a qual, em verdade, é a mais
honrada (\emph{timiṓteron}), e não haverá nada no futuro tão honrado
entre homens ou deuses. É~preciso conhecer tais coisas, ó criança, e
saber que a amizade do amante (\emph{t}ḕ\emph{n erastoû philían}) não
surge entre favores (\emph{eunoías}), mas como alimento, para agradar
(\emph{chárin}) a saciedade, [241d] pois os amantes amam
(\emph{philoûsin}) seus prediletos como os lobos amam (\emph{agapôsin})
os cordeiros''.

 

Isso é tudo, ó Fedro, e nada mais ouvirás de meu discurso, pois esse é o
seu fim.

 

F: E eu considerei que estavas no meio e que dirias semelhantes coisas
também acerca do que não está sob o efeito do amor
(\emph{m}ḕ\emph{~erôntos}), de como é melhor agradá"-lo
(\emph{charídzesthai}), mencionando o quanto isso tem de bom, mas agora,
ó Sócrates, porque interrompeste (\emph{apopaúêi})?

 

[241e] S: Não percebeste, ó bem"-aventurado, que eu proferia há pouco
um épico, não um ditirambo, o qual é mais conveniente ao vitupério
(\emph{pségôn})? E se eu começasse a elogiar (\emph{epaineîn}) o outro,
o que pensas que eu faria? Consideras que sob a influência das Ninfas,
as quais tu me colocaste premeditadamente, eu estaria obviamente
inspirado (\emph{enthousiásô})? Digo, então, que por meio de um só
discurso recusamos o outro, o qual nos oferece os benefícios contrários.
E~para que, então, um discurso tão extenso (\emph{makroû lógoû})? Acerca
de ambos é suficiente o que foi dito, que a narrativa (\emph{mythos})
que te ofereci sofra (\emph{páschein}) o que for.~[242a] Quanto a
mim, vou partir e atravessar esse rio, antes que seja obrigado por ti a
algo mais grave.

 

F: Não ainda, ó Sócrates, pelo menos antes que esse calor se vá, ou não
vês que é quase meio"-dia, aquilo que chamamos de sol a pino? Vamos
permanecer e dialogar acerca do que foi dito, partiremos assim que o
tempo esteja mais fresco.

 

S: És divino em matéria discursiva, ó Fedro, e espantoso por ser isento
de arte (\emph{atechnôs}).~[242b] Por conta dos discursos que
ocorreram por tua força, creio que ninguém é melhor que tu ao
pronunciá"-los ou forçando outros a proferi"-los -- exceto o discurso de
Símias de Tebas, que é o mais forte (\emph{krateîs}) entre todos. E~agora tu me parece ser a causa (\emph{aitiós}) mesma de outro discurso
(\emph{lógôi}) que vou proferir (\emph{rhêthênai}).

 

F: Então é uma guerra o que anuncias! Mas diz como foi e a qual deles te
referes?

 

S: Quando decidi, ó meu caro, atravessar o rio, um~\emph{daimon}~que me
é familiar surgiu e me gerou um sinal (\emph{sêmeîón}) -- ele sempre me
impede quando estou prestes a fazer algo \mbox{---,} e parecia que eu ouvia a
sua voz, ela não me deixava partir antes de me purificar
(\emph{aphosiṓsōmai}), como se tivesse cometido alguma falta contra o
divino (\emph{hēmartēk\gr{ό}ta eis to theîon}).~[242c] É que eu sou um
adivinho (\emph{mántis}), mas não muito aplicado (\emph{ou pánu dè
spoudaîos}), talvez como aqueles que são ruins na escrita (\emph{ta
grámmata phaûloi}), no entanto para mim isso já é suficiente.~Aí então
compreendi com clareza a minha falta (\emph{hamártēma}). É~certo agora,
ó companheiro, que a alma (\emph{hē psykhḗ}) tem alguma adivinhação
(\emph{mantikón gé ti}), pois algo também me inquietou, ao proferir o
discurso passado, e temi, do mesmo modo que Íbico, com relação à sua
falta contra os deuses:

 

que ela não altere minha honra (\emph{timàn}) junto aos homens

 

[242d] Só agora percebi a minha falta (\emph{ḗisthēmai tò
hamártēma}).

 

F: Diz então, qual é [a falta]?

 

S: Terrível, ó Fedro, terrível foi o discurso (\emph{lógon}) que
trouxeste e o que me forçaste a dizer (\emph{eipeîn}).

 

F: Como é?

 

S: Foi uma tolice e uma espécie de impiedade (\emph{ti asebê}).~Pode
haver algo mais terrível (\emph{deinóteros})?

 

F: Não, se é que dizes a verdade.

 

S: O quê? Não consideras que o Eros é um deus, filho de Afrodite?

 

F: Assim dizem.

 

S: Mas não segundo Lísias, nem pelo teu discurso, aquele que saiu da
minha boca envenenada (\emph{katapharmakeuthéntos}) pelo que tu disseste
(\emph{eléchtê}). [242e] Se é assim, tal qual sabemos, Eros é um
deus ou alguma divindade, e de nenhuma forma poderia ser mau, mas ambos
os discursos proferidos falaram dele como se assim ele fosse. Dessa
maneira, cometeram uma falta (\emph{hēmartanétēn}) contra Eros, além
disso, pretenderam"-se bondosos e muito civilizados (\emph{asteía}), mas
não foram discursos saudáveis (\emph{hygiès}) tampouco verdadeiros
(\emph{alêthès}), embora tenham disso se gabado, [243a] uma vez que,
ao enganar alguns homens, tornaram"-se bem reputados entre eles. Ó
querido, eu preciso me purificar (\emph{kathérasthai}). Há uma
purificação arcaica (\emph{katharmòs archaîos}) para os que cometem
faltas em mitologia. Homero não a conheceu, mas Estesícoro~sim. Privado
da visão pela linguagem abusiva contra Helena, não ignorou a causa como
Homero, mas, conhecendo a causa (\emph{égnô tèn aitían}), o músico de
Himera compôs:

 

\begin{quote}

\begin{verse}[\versewidth]
Esse não é um discurso verdadeiro,\\
 nem embarcaste em naves bem
assentadas,\\
 nem foste à cidade de Troia.\\!
\end{verse} 
\end{quote}

 

[243b] E ao compor toda a obra, chamada de~\emph{Palinódia},
imediatamente recuperou a visão (\emph{anéblepsen}).~Eu, então, agora me
torno mais sábio (\emph{sôph}ṓ\emph{teros}) que eles, pelo menos nesse
ponto, pois, antes de sofrer (\emph{patheîn}) algo pela linguagem
abusiva contra Eros, trato de ofertar"-lhe uma~\emph{palinódia}~com a
cabeça descoberta (\emph{gymnêi têi kephalêi}), e não como havia
ocorrido, por vergonha, com a cabeça velada (\emph{egkekalumménos}).

 

F: Nada poderia ser mais prazeroso (\emph{hedíô}), ó Sócrates, do que
isso que tu afirmas.

 

[243c] S: Pois então, ó bom Fedro, põe na tua mente (\emph{ennoeîs})
a falta de pudor (\emph{anaidôs}) com que os discursos foram proferidos,
tanto esse meu [discurso] como o teu, a partir do livro lido. Se,
por acaso (\emph{týchoi}), o caráter (\emph{êthos}) dos que nos ouvem
for nobre e gentil, eles pensariam que estão sendo levados a escutar
marinheiros e que de nenhum modo veriam um amor entre homens livres?
Quer seja o amado ou o amante, se os mencionarmos simplesmente como
enamorados, não seria através das pequenas coisas que ambos poderiam ser
tomados por grandes ódios (\emph{échthras}), ciúmes (\emph{phthonerôs})
e danos (\emph{blaberôs}) aos seus prediletos (\emph{ta paidikà}).
[243d] Eles, nesse caso, não concordariam plenamente conosco ao
vituperarmos o Amor (\emph{hêmîn homologeîn há pségomen tòn Érota})?

 

F: É possível, ó Sócrates, por Zeus!

 

S: Desse mesmo homem eu me envergonho e temo pelo próprio Amor, a ponto
de desejar, com um discurso potável, lavar"-me dessa audição salgada
(\emph{epithymô potímôi lógôi oîon halmuràn akoḕn apoklúsasthai}).~Diga
a Lísias para escrever (\emph{grápsai}) o mais depressa possível que,
partindo de condições semelhantes, é melhor agradar
(\emph{charídzesthai}) ao amante (\emph{erastêi}) em vez do não afetado
pelo amor (\emph{mè erônti}).

 

F: E veja bem que será assim mesmo, pois tu, ao fazer o elogio do amante
(\emph{tòn toû erastoû épainon}), gerarás em Lísias a necessidade de
escrever (\emph{grápsai}), impelido por mim, [243e] sobre esse mesmo
discurso (\emph{lógon}).

 

S: Eu acredito.

 

F: Diz agora com bravura.

 

S: Onde está o jovem com o qual eu falava? Quero que ele ouça também
isso, e que não se antecipe, por não ter ainda escutado, agradando
(\emph{charisámenos}) ao não afetado pelo amor (\emph{mḕ~erônti}).

 

F: Ele está junto a ti, muito perto e sempre a acompanhar"-te, quando tu
quiseres (\emph{boúlêi}).

 

S: ``Deste modo, ó bela criança, compreende (\emph{ennóêson}) que o
primeiro discurso foi o de Fedro, filho de Pítocles, homem de
Mirrinunte, [244a] e o discurso seguinte será o de Estesícoro, filho
de Eufemo, natural de Himera.~Que seja dito que\emph{~não é um discurso
verdadeiro}~(\emph{ouk ést'étymos lógos}),\textsuperscript{~}aquele que
diz, perto de um amante (\emph{erastoû}), ser melhor agradar
(\emph{charídzesthai}) a quem não está afetado pelo amor
(\emph{mḕ~erônti}), porque um está louco (\emph{maínetai}) e o outro
sóbrio (\emph{sôphoneî}). Se a loucura (\emph{manían}) fosse
simplesmente (\emph{aploûn}) má, este seria um belo discurso, mas os
maiores bens (\emph{agathôn}) nos surgem por intermédio da loucura
(\emph{dià manías}), a qual seguramente é um presente divino
(\emph{theíai}). Tanto a profetisa do oráculo em Delfos, quanto as
sacerdotisas em Dodona, executaram para a Hélade muitas e belas coisas,
sejam particulares ou públicas, tomadas pela loucura (\emph{maneîsai}),
[244b] ao passo que sóbrias (\emph{sôphronoûsai}) elas pouco ou nada
fizeram.~E se dissermos que a Sibila e tantos outros, valendo"-se da
adivinhação entusiástica (\emph{mantikêi ch}ṓ\emph{menoi enthéoi}),
muitas vezes e para tantos, predisseram um futuro correto, estaríamos
nos alongando sobre o que é evidente para todos.

Eis um testemunho digno (\emph{áxion epimartúrasthai}), que os antigos
instituidores dos nomes não consideravam a loucura (\emph{manían}) nem
ruim (\emph{aischròn}) nem vergonhosa (\emph{óneidos}), pois não a
teriam misturado à arte mais bela, a que interpreta o futuro, [244c]
designando"-a pelo nome de~\emph{maniké}. Julgaram"-na bela porque a
loucura surgia por parte da divindade (\emph{theíai}). Nossos
contemporâneos, inexperientes em beleza, enfiando o ``\emph{tau''}~no
meio, chamam"-na de~\emph{mantiké}. E~os sóbrios (\emph{tôn
emphrónôn})~que buscam o futuro pelos pássaros e por outros sinais
(\emph{diá orníthôn poiouménôn kai tôn állôn semeíôn}), os mesmos que
partem da reflexão e abrem o caminho da suposição humana, do pensamento
e da observação (\emph{ek dianoías poridzoménôn anthrôpínei oiḗsei noûn
te kaí historían}), esses chamaram"-na de~\emph{oionoïstikḗn}. Hoje em
dia os jovens imponentemente dizem~\emph{oiônistiké}, com um ``ô''
longo. [244d] Quanto mais perfeita e honrada é a~\emph{adivinhação
oionística~}(\emph{mantiké oiônistiké}), e o nome da primeira atividade
com relação ao nome da segunda, mais bela é a loucura (\emph{manían}) em
vista da sobriedade (\emph{sôphrosýne}), testemunham os antigos
(\emph{marturoûsin oi palaioì}), pois uma surge por intermédio do deus
(\emph{ek theoû}) e a outra junto aos homens (\emph{par'anthópôn}).

Com efeito, a loucura (\emph{manían}) surgiu para algumas famílias que
necessitavam, profetizando as maiores enfermidades (\emph{nósôn}) e
dores (\emph{pónôn}) vindas de antigos ressentimentos, e elas
encontraram (\emph{heúreto}) refúgio em preces e cultos aos deuses.
[244e] Daí então surgiram purificações e iniciações (\emph{katharmôn
te kai teletôn}) praticadas para suas próprias isenções, tanto para o
tempo presente quanto para os tempos vindouros, sendo assim encontrado
(\emph{heuroméne}) o correto (\emph{orthôs}) afastamento dos males
coetâneos na loucura e na possessão (\emph{manéti te kai
kataschoménoi}).

[245a] A terceira possessão e loucura (\emph{katokochḗ~te kai
manía}) vem das Musas, as quais se apoderam da alma delicada e
inviolada, despertando e tornando"-a báquica por meio de odes e outras
poesias, as quais ordenam (\emph{kosmoûsa})\textsuperscript{~}inúmeras
obras dos antigos e educam (\emph{paideúei}) os pósteros. Aquele que
chegar às portas da poética sem a loucura das Musas (\emph{aneu manías
Mousôn}), acreditando (\emph{peistheìs}) que somente por força da arte
seria poeta perfeito (\emph{ék téchnês ikanòs poietês}), está incompleto
(\emph{atelḕs}), sem contar que a poesia (\emph{poíesis}) dos
enlouquecidos (\emph{mainoménôn}) ofusca (\emph{êphanísthê}) a dos
sóbrios (\emph{sôphronoûntos}).

[245b] Tenho dito a ti acerca da grandeza das belas obras repletas
da loucura que vem dos deuses, de tal maneira que dela não fujamos
(\emph{phobómetha}), nem nos perturbe algum discurso que amedronte o
amante na direção da necessidade da escolha (\emph{proaireîsthai}) do
amigo sóbrio (\emph{tón sṓphrona phílon}). Aquele [amante] leva a
honra da vitória antes deste [sóbrio], mostrando que não é pela
utilidade (\emph{ôpheléiai}) que o amor (\emph{érôs}) é enviado, pelo
deus, ao amante e ao amado. Então, é preciso que demonstremos o contrário
(\emph{apodeiktéon aû tounantíon}): como pela maior das sortes essa
loucura é dádiva dos deuses. [245c] Essa demonstração
(\emph{apódeixis}) não será persuasiva (\emph{ápistos}) aos terríveis
(\emph{deinoîs}), mas será persuasiva (\emph{pistḗ}) aos sábios
(\emph{sophoîs}).~Será necessário primeiramente tratarmos da natureza da
alma, divina e humana, vislumbrando suas paixões e ações (\emph{páthê te
kaì érga}), entendendo a verdade (\emph{talêthès noêsai}). O~princípio
(\emph{archḕ}) da demonstração (\emph{apodeíxeôs}) é o seguinte.

Toda alma é imortal. Tudo aquilo que está sempre em movimento é imortal,
ao passo que o que move outro, ou por outro é movido, ao cessar do
movimento (\emph{kinéseos}), cessa também a vida (\emph{dzōẽs}). Somente
o auto"-movido não se desliga de si mesmo, visto que nunca cessa seu
movimento, e para todas as coisas que são movidas essa é a fonte e o
princípio do movimento (\emph{pegḗ~kai archḗ~kinḗseos}). Princípio é sem
geração (\emph{archḕ~dè agénêton}). [245d] É necessário que todo o
gerado advenha de um princípio (\emph{ex archês}), e ele mesmo não
advenha de nenhum, pois, se ele surgisse de um princípio,
(\emph{archè}) não mais poderia ser considerado um princípio. De modo
que é necessário que [o mesmo princípio] seja sem geração e sem
corrupção, pois nem se corrompe, nem é gerado, se é que todas as coisas
necessariamente surgem (\emph{gígnesthai}) de um princípio. Neste caso o
movimento tem um princípio que lhe é idêntico, dele não podendo sofrer
corrupção ou geração, ou todas as coisas e toda a gênese do céu estariam
conjuntamente perecendo e nunca teriam recebido movimento a partir de
algo. [245e] Esclarecida a imortalidade daquilo que é movido por si
mesmo, a essência da alma (\emph{psychês ousían}) e sua explicação não
foram enunciadas (\emph{legôn}) de modo vergonhoso. De todos os corpos,
os que recebem movimento externo (\emph{éxôthen}) são inanimados, ao
passo que os que de dentro de si e por si [recebem o movimento] são
animados, tal é a essência natural da alma (\emph{hos taútês oúsês
phýseôs psychês}). Sendo assim, a alma não é outra coisa se não aquilo
que move a si mesmo, necessariamente a alma não é gerada, o que a torna
imortal.

[246a] Acerca da imortalidade é o suficiente. Falemos [agora]
acerca dessa ideia (\emph{idéas}).~Quanto ao que ela é, teríamos uma
grandiosa e divina (\emph{theías}) exposição (\emph{diegḗseos}), e, ao
que parece, uma exposição inferior, conveniente aos homens. Falemos por
meio desta última. Convencionemos que ela tenha uma potência
(\emph{dynámei}) e uma natureza similar (\emph{sumphýtôi}) a uma biga
alada e seu auriga. Os cavalos e cocheiros dos deuses são essencialmente
todos bons e vindos do que é bom (\emph{agathoì kaì ex agathôn}), ao
passo que os dos outros são misturados (\emph{mémiktai}). [246b]
Primeiramente, dirige a biga aquele que para nós é o comandante, e em
consequência disso, um dos cavalos é bom e belo, enquanto o outro é o seu
contrário, sendo ele mesmo um contrário. Entre nós, portanto, o ofício
de auriga (\emph{henióchêsis}) é necessariamente penoso e adverso.

Experimentemos dizer o motivo pelo qual a vida foi enunciada como mortal
e imortal (\emph{thnetón te kaì athánaton}).~Toda alma ocupa"-se
inteiramente do que é inanimado, circula por todo o céu tomando, algumas
vezes, outros aspectos (\emph{eídesi}). [246c] Estando em sua
perfeição [a alma] é alada, atravessa as alturas
(\emph{meteôroporeî}) e habita todo o cosmo (\emph{pánta tòn kósmon
dioikeî}), mas quando é levada à perda das asas, então, de algum sólido
se apodera e ali se instala (\emph{ou katoikistheîsa}), tomando corpo
terrestre, o qual parece mover a si mesmo devido àquela potência. Como
viventes que são enunciados conjuntamente, alma e corpo, fixados, ganham
o epíteto de mortal\emph{.~}O imortal não é deduzido por um raciocínio,
mas modelamos (\emph{pláttomen}) o deus, não somente pela visão
(\emph{idóntes}), nem somente pelo pensamento (\emph{noḗsantes}), como
um vivente imortal que, tendo alma e corpo, mantém"-se assimilado para
sempre no tempo. [246d] Que essas coisas assim sejam e que tenham
sido expostas ao agrado do deus.

Tomemos agora a causa da queda das asas, motivo pelo qual a alma se
perde. É~da natureza da potência alada levar o que é pesado para o alto,
alcançando a casa do gênero divino, por onde ela se põe em comum
(\emph{kekoinṓnêke}), no mais alto grau corpóreo, com a alma do deus. O~divino é belo, sábio, bom e tudo o que é dessa mesma classe, [246e]
e justamente por essas coisas que são mais bem acrescidas e alimentadas
as asas da alma. As coisas contrárias a estas são corruptíveis e perecem
pela maldade e pelo vício.

Zeus é o grande condutor no céu com seu carro alado (\emph{ptênòn
hárma}), adianta"-se em primeiro lugar, zelando por todas as coisas
através do cosmo (\emph{diakosmôn}). Ele é seguido por um exército de
deuses e~\emph{daimônes}~(\emph{theôn te kaì daimónôn}) ordenados
(\emph{kekosmêménê}) em onze partes, permanecendo Héstia sozinha na casa
dos deuses (\emph{ménei gar Hestía en theôn oíkôi mónê}). [247a]
Dentre os outros tantos deuses, em sua formação de doze partes, são
conduzidos pelo chefe, seguindo a composição que lhes foi
atribuída.~Então, muitas divindades bem"-aventuradas seguem trajetos no
interior do céu (\emph{entós ouranoû}) e circulam (\emph{epistréphetai})
no gênero feliz dos deuses, cada uma delas fazendo o que lhes é próprio
(\emph{práttôn hékastos autôn tò autoû}). Seguem sempre que querem e
podem, uma vez que a inveja (\emph{phthónos}) permanece fora do coro dos
deuses (\emph{éxô theíou choroû}).~Quando vão ao cume para um festim ou
banquete, atravessam para o ápice das escarpas que sustentam o céu
(\emph{ákran epí ten hypouránion apsîda poreúontai}), de modo que as
carruagens dos deuses, estando num dócil equilíbrio, [247b]
ultrapassam facilmente, já as outras, [ultrapassam] com dificuldade.
O~cavalo que partilha do mal é pesado, inclina"-se para a terra e impede
o trajeto do auriga que não foi bem"-educado.

Ali mesmo fica o último (\emph{éskatos}) grau de sofrimento e disputa a
que a alma se dispõe; as almas dos imortais, quando chegam ao extremo,
atravessando exteriormente (\emph{éxô poreutheîsai}), estabelecem"-se sob
o dorso do céu (\emph{epì tôi toû ouranoû nôtôi}), sendo levadas e
trazidas ao seu redor; [247c] as outras [almas dos mortais]
contemplam ali as coisas fora do céu (\emph{theoroûsi ta éxô toû
ouranoû}).

Esse lugar supraceleste (\emph{hyperouránion tópon}) ainda não foi
cantado por nenhum dos poetas e nunca será cantado de forma digna. É~necessário ousar dizer a verdade (\emph{alêthès eipeîn}), sobretudo ao
falarmos da verdade (\emph{alêtheías
légonta}).~\emph{Eidôlon}\textsuperscript{~}(\emph{ousía óntôs oûsa})
não tem cor, é sem figura, intangível e somente contemplada pelo
pensamento do piloto da alma (\emph{psykês kybernḗtei monôi
theatḕ~nôi}), região na qual tem lugar o gênero verdadeiro do
conhecimento (\emph{tò tês alêthoûs epistḗmês génos}). [247d] Então,
bem como pensamento do deus (\emph{theoû diánoia}), [a alma] é
nutrida pela pureza do intelecto (\emph{nôi}) e do conhecimento
(\emph{epistḗmêi}), como todas as almas que possam vir a mostrar tal
preocupação, tendo visto o ser através do tempo (\emph{idoûsa dià
chrónou tò ón}). Ela é nutrida por ter contemplado a verdade
(\emph{theôroûsa talêthê}), sentindo"-se completa, até que possa chegar,
pelo ciclo, ao ponto inicial do trajeto pelo qual foi levada. Nesse
período veriam a própria justiça (\emph{dikaiosúnên}), a prudência
(\emph{sôphrosúnên}), o conhecimento (\emph{epistémên}), [247e] não
aquilo que pertence à gênese, nem o que está em outras coisas, em outros
que agora chamamos seres (\emph{óntôn}), mas conheceriam a própria
essência que está no ser (\emph{ho estin òn óntôs epistémên oûsan}). E,
do mesmo modo, tendo contemplado (\emph{theasaméne}) a essência dos
seres em seu posto (\emph{estiatheîsa}), mergulham de volta (\emph{dûsa
pálin}) para o interior do céu e chegam à casa.\textsuperscript{~}O
auriga chega ao estábulo, coloca ali os cavalos, oferece"-lhes ambrosia e
lhes dá néctar para beber.

[248a] Esta é a vida dos deuses. Quanto às outras almas, a que
melhor acompanha e se assimila (\emph{epoménê kaì}~\emph{eikasménê}) ao
deus, eleva a cabeça do auriga até o lugar exterior (\emph{tòn éxô
tópon}) e acompanha a volta circular, perturbadas pelos cavalos e com
muita fadiga veem do alto os seres (\emph{kathorôsa ta ónta}). Já a
[alma] que ora se eleva, ora mergulha, tendo forçado os cavalos,
algumas coisas vê, outras não. Outras [almas] ainda, apegando"-se a
tudo o que é do alto (\emph{toû ano épontai}), são incapazes
(\emph{adynatoûsai}) de ter êxito, mas seguem submersas na volta
circular, em pisoteio e confronto mútuos, tentando se adiantarem frente
as outras. [248b] Então ocorre tumulto, luta e suor extremos, é
quando muitas almas claudicam pela maldade do auriga (\emph{kakíai
henióchôn}) e destroçam suas asas. Todas estas, tendo muita fadiga, sem
chegarem à contemplação do ser (\emph{toû óntos théas}), afastam"-se e
servem"-se do alimento da opinião (\emph{trophêi doxastêi chrôntai}).~Eis
o grande empenho que há para ser capaz de ver a planície da verdade onde
ela está (\emph{tò alêtheías ideîn pedíon oû estin}), pois o pasto que
convém ao melhor da alma provém desse prado e a natureza do alado
(\emph{pteroû phýsis}), que eleva a alma, [248c] ali é alimentada.

Eis a lei de Adrasteia: A alma que tenha se tornado acompanhante do deus
(\emph{theôi sunopadòs}) e que tenha visto algo das verdades
(\emph{katídêi ti tôn alêthôn}) fica ilesa (\emph{apḗmona}) até o outro
percurso, e se puder fazer isso sempre, fica sempre intacta
(\emph{ablabê}). Quando não lhe é possível gerir"-se, não se vale da
visão nem do sucesso, e, ao aplicar muito peso, perde as asas,
despencando por terra em função do fardo do esquecimento (\emph{léthês})
e da maldade. Então é lei, na primeira geração, não nascer em nenhuma
natureza de fera [248d]. Os que viram o máximo do gênero humano
tornar"-se-ão filósofos, amigos do belo, músicos ou algum dentre os
eróticos. Em segundo lugar estão o rei na lei, o guerreiro ou o
comandante; no terceiro lugar, um político, economista ou administrador;
na quarta posição, um amigo das fadigas, da ginástica ou alguém para
curar o corpo; na quinta, um adivinho (\emph{mantikón}) ou alguém que
pode cuidar das iniciações (\emph{telestikòn}); [248e] na sexta, um
poeta, alguém que se ocupa da mimese ou outras [atividades]
concordes; na sétima, um demiurgo ou homem do campo; na oitava, um sofista
ou aquele que fere o povo (\emph{demokopikós}); e na nona, um tirano.

Em todas elas, os que se conduzem com justiça tomam o melhor destino
(\emph{moíras}), os que o fazem injustamente, o pior.~Cada uma das almas
não chega ao mesmo ponto de onde saiu antes de dez mil anos, pois não
criam asas antes desse tempo, exceto aquela que foi, de maneira honrada,
amante do saber (\emph{philosophḗsantos}) ou [249a] amante dos
jovens de acordo com a filosofia (\emph{paiderastḗsantos metà
philosophías}).\emph{~}Estas, na terceira volta de mil anos, se
conduziram este tipo de vida por três vezes seguidas, no terceiro
milênio se afastam de modo alado. Quanto as outras, há as em que, ao término
da primeira vida, ocorre uma separação (\emph{kríseôs étychon}) e há
julgamento no tribunal subterrâneo, no qual elas prestam contas, ao
passo que há também as que chegam a algum lugar do céu, elevam"-se pela
justiça e são levadas à dignidade da vida humana que tiveram. [249b]
Tanto umas como outras, no milésimo ano, sorteiam (\emph{klḗrôsín}) e
escolhem (\emph{aíresin}) a próxima vida, sendo que cada uma escolhe
(\emph{airoûntai}) a que quiser.~Ali mesmo, os homens que foram feras
serão novamente (\emph{pálin}) homens, e se a alma não atingir tal
figura (\emph{schêma}) é por não ter visto a verdade
(\emph{mḗpote}~\emph{idoûsa tḕn alḗtheian}).

É necessário ao homem atingir (\emph{suniénai}) a ideia (\emph{eîdos})
que vai do múltiplo sensível ao uno, tomado conjuntamente pelo
raciocínio (\emph{logismôi}). [249c] Isso é a reminiscência
(\emph{anámnêsis}) daquilo que nossa alma viu, atravessando com o deus,
vendo além do que agora nos é dito e levantando a cabeça
(\emph{anakúpsasa})~para o verdadeiro ser (\emph{tò ón óntôs}). É~por
isso que, justamente, só cria asas o pensamento (\emph{diánoia}) do
filósofo, para o qual sempre há, na medida do possível, memória
(\emph{mnḗmêi}), e para o qual os deuses são divinos. Homens de tal
valor servem"-se corretamente da recordação (\emph{hypomnḗmasin}), sempre
se iniciam corretamente em mistérios e tornam"-se os únicos perfeitos.
Mudam a dignidade dos homens ao tornarem"-se próximos aos deuses
(\emph{pròs tôi theôi gignómenos}), e são advertidos por muitos que ao
seu lado se moviam. Pelo seu entusiasmo [249d]
(\emph{enthousiádzôn}) eles são esquecidos (\emph{lélêthen}) pela
maioria.

Até aqui temos o discurso todo a respeito da quarta loucura, quando [a
alma] vê alguma dessas belezas, rememorando o verdadeiro
(\emph{alêthoûs anamimneiskómenos}) e tem as asas crescidas, momento em
que a alma está alada e cheia de disposição. Entretanto, quando não pode
voar, ela parece um pássaro que vê o que há acima dele, mas descuida do
que está embaixo, por isso é vista como uma alma louca (\emph{manikôs}).
[249e] Essa é a melhor das coisas entusiásticas
(\emph{enthousiáseôn}) e provém das melhores, quem dela vier a ser
possuidor ou dela participar é chamado de amante do belo (\emph{kalôn
erastès}), porque aquele que ama é partícipe (\emph{metéchôn}) da
loucura (\emph{manías}). De acordo com o que foi dito, é da natureza de
toda alma humana ter contemplado os seres (\emph{tetheáthai tà ónta}),
ou não chegariam a essa vida.

[250a] Relembrar (\emph{anamimnḗiskesthai}) aquilo, a partir destas
coisas, não é fácil para todas as almas, nem para aquelas que tiveram
uma breve visão, nem para as que caíram, infortunadas dirigidas pela
injustiça da multidão, almas que esqueceram a visão sagrada
(\emph{léthên ôn tóte eîdon hierôn}) que outrora tiveram. Poucas
(\emph{olígai}) são as [almas] deixadas com suficiente memória
(\emph{mnḗmês}~\emph{ikanôs}). Estas mesmas, quando têm visão de algo
semelhante (\emph{homoíoma ídôsin}), ficam fora de si
(\emph{ekplḗttontai}) e de nenhum modo voltam a si. Outras ignoram a
afecção (\emph{tò páthos}) por não a ter percebido com força suficiente
(\emph{mè ikanôs}~\emph{diaisthánesthai}).

[250b] Justiça, prudência (\emph{dikaiosýnês mèn oûn kaì
sôphrosýnês}) e tantas outras preciosidades da alma não resplandecem
(\emph{phéggous})~em nenhuma das semelhanças (\emph{homoiómasin}) daqui,
mas poucos, através de órgãos obscuros, com fadiga, contemplam o gênero
da similitude (\emph{theôntai to toû eikasthéntos génos}) partindo dos
ícones (\emph{epí tas eikónas ióntes}).

Era de se ver a luminosa beleza quando, outrora, juntamente com o feliz
cortejo, [as almas] visualizavam e contemplavam a bem"-aventurança
(\emph{makarían}~\emph{ópsin te kaì théan}). Nós somos seguidores de
Zeus, outros seguem outros deuses. Vimos e nos completávamos nas
iniciações que, com justiça, são ditas as mais felizes que celebramos,
íntegros e indiferentes aos males que nos surgiriam em tempos
posteriores. Nas completas, simples, [250c] calmas e felizes
aparições estávamos iniciados e havíamos chegado à essa revelação~pela
mais pura luz (\emph{epopteúontes en augêi katharâi}), estando puros
(\emph{katharoì óntes}), pois não havia a marca que nos é trazida pelo
que agora chamamos corpo, motivo pelo qual permanecemos atados [a
ele] como uma ostra [a sua concha]. Que estas coisas sejam
agraciadas (\emph{kecharísthô}) pela memória (\emph{mnḗmêi}),~a qual
pela ausência de outrora foi agora longamente enunciada.

Sobre a beleza, como dissemos, sendo em cada um de nós luminosa,
chegamos aqui [250d] tomando"-a com a máxima clareza de nossos
sentidos, com o mais radiante brilho. A~visão (\emph{ópsis}) é a mais
aguda das sensações que nos chega pelo corpo, mas por ela a prudência
(\emph{phrónêsis}) não é vista.~Cairíamos em terríveis amores, se algum
ídolo (\emph{eídôlon}) de tal classe, por sua própria evidência, fosse
enviado e desejado pela visão (\emph{ópsin}), assim como tantas outras
coisas amáveis. Só a beleza (\emph{kállos}~\emph{mónon}) teve este
destino (\emph{moîran}), ser a mais evidente (\emph{ekphanéstaton}) e a
mais amada (\emph{erasmiṓtaton}).

[250e] Um recém"-iniciado ou alguém que veio a corromper"-se não é
rapidamente trazido daqui para lá, para a beleza mesma (\emph{pròs autò
tò kállos}), contemplando o mesmo que aqui leva seu nome. Não venera o
olhar, mas, entregue ao prazer (\emph{hêdonêi}), põe"-se a andar na lei de
um quadrúpede, produz filhos e, familiarizado com a desmesura
(\emph{hýbrei}), não teme nem se envergonha, perseguindo um prazer
contrário à natureza (\emph{parà phýsin hêdonḕn}).

[251a] O recém"-iniciado (\emph{artitelḗs}) que contemplou muitas
coisas (\emph{polytheámon}), quando vê um rosto de forma divina
(\emph{theoeidés}), bem imitando o belo (\emph{kállos eû memimêménon})
ou alguma forma corpórea (\emph{sṓmatos idéan}), primeiro estremece
(\emph{éphrixe}), enquanto algo dos medos de outrora chega até ele,
depois de ter visto, venera"-o como a um deus e, se não temesse a fama de
uma excessiva loucura (\emph{sphódra manías dóxan}), sacrificaria
(\emph{thúoi}) ao predileto (\emph{paidikoîs}) como a uma imagem e como
a um deus (\emph{agálmati kai theôi}). Depois dessa visão, surge nele
uma mudança do tremor (\emph{tês phríkês}~\emph{metabolḗ}), pois um suor
e um calor atípico o toma e o aquece, tendo recebido [251b] o fluxo
(\emph{aporroḕn}) da beleza pelos olhos (\emph{ommátôn}), motivo pelo
qual a natureza alada é irrigada (\emph{pteroû phýsis ádretai}).
Aquecida a asa, dissolve"-se uma natureza que há muito tempo não
germinava, por endurecimento, aprisionamento e clausura conjunta.
Túrgido de alimento, o caule da asa incha e começa a brotar da raiz em
todas as formas de almas, pois toda [alma] era anteriormente alada
(\emph{pterôtḗ}). Nesse momento, ela ferve toda e irrita"-se
(\emph{anakêkíei}), [251c] como quando sofremos pelo surgimento dos
dentes, que recém"-saídos raspam e irritam por toda a passagem, e o mesmo
sofre a alma no começo do crescer das asas (\emph{pterophyeîn}), ferve e
irrita"-se com as cócegas provocadas pelo brotar das asas (\emph{phýousa
tà pterá}). Quando, ao olhar para a beleza do amado, e dele recebendo
parte do fluxo que sobrevém -- o qual precisamente é chamado de desejo
(\emph{hímeros})~\mbox{---,} é irrigada e aquecida, recompondo"-se da dor ela
fica alegre. [251d] Quando ficam separadas e áridas, as vias que
deságuam onde crescem (\emph{hormâi}) as asas ficam secas, fechando e
obstruindo o germinar das asas, as quais, em seu interior, após terem
sido fechadas ao fluxo do desejo (\emph{himérou}), ficam agitadas,
arranhando cada uma das vias de saída, justamente porque a alma enfurece
todas as feridas ao redor, causando dor.

Por outro lado, [a alma] alegra"-se tendo a memória (\emph{mnḗmen})
da beleza. Nessa mistura de ambos ela se atormenta pela estranheza da
afecção (\emph{atopíai toû páthous}), não conseguindo saída pela fúria,
e, enlouquecida (\emph{emmanès}), nem a noite pode dormir, nem durante o
dia permanece num só lugar. [251e] Corre (\emph{theî}) ansiosa para
onde considera ver (\emph{ópsesthai}) o possuidor da beleza
(\emph{kállos}). Tendo visto e canalizado o desejo (\emph{hímeron}),
libera o que até então estava conjuntamente obstruído, tomando fôlego,
tendo apaziguado as picadas e dores. Este é o dulcíssimo prazer
(\emph{hedonḕn}) de que, no momento, ela desfruta. Por isso não é
voluntariamente afastada e a ninguém mais atende se não ao belo,
[252a] esquece de todos: mãe, irmãos, companheiros. Sendo arruinada
pela negligência, não realiza nada e, quanto aos hábitos e conveniências
com as quais antes se embelezava, a todos passa a desprezar, pronta a
escravizar"-se e deleitar"-se onde lhe permitam, o mais próximo possível
de seu desejo (\emph{póthou}). Além de adorar aquele que porta a beleza,
nele encontra único médico para os seus maiores sofrimentos
(\emph{iatrón hêurêke mónon tôn megístôn pónôn}).

[252b] Essa afecção (\emph{tò páthos}), ó bela criança a quem se
dirige meu discurso (\emph{lógos}), os homens denominam Amor
(\emph{Erôta}), mas se ouvires como os deuses o designam, tu
possivelmente rirás, por conta de sua juventude. Alguns homéridas,
segundo penso, em dois de seus versos secretos (\emph{apothéton epôn})
falam o seguinte sobre o Amor (\emph{Érôta}) -- o segundo verso é
excessivo (\emph{hybristikòn}) e não precisamente na métrica \mbox{---,} eles
cantam assim:

\begin{quote}
 


\begin{verse}[\versewidth]
os mortais o designam por Eros (\emph{Érôta}) alado (\emph{potênón}),\\!

e os imortais por\emph{~Ptérôta}~(Alado), pela força do brotar das asas
(\emph{pterophýtor}).\\!
\end{verse} 
\end{quote}

 

É possível ser persuadido por estes homens, assim como é possível não
ser. Apesar disso, a causa (\emph{aitía}) e a afecção (\emph{to páthos})
para aqueles que amam (\emph{tôn eróntôn}) são estas mesmas. [252c]
Dentre os acompanhantes, o que foi tomado com mais força pode carregar o
fardo de Zeus, aquele que é denominado como alado (\emph{pterônúmou}).
Quanto aos que foram servidores de Ares e com este circularam, quando
estão tomados por Eros e consideram que foram injustiçados pelo amado,
prontificam"-se ao homicídio, querem sacrificar a si mesmos e aos seus
prediletos (\emph{tà paidiká}). E~assim, cada qual sendo coreuta para
cada deus, honra"-o e imita"-o (\emph{mimoúmenos}) na vida o quanto podem.
[252d] Durante algum tempo, por não se corromperem, vivem aqui nesta
primeira geração, sendo levados a reunirem"-se aos amantes e a outros por
esse modo. Então, cada um elege (\emph{eklégetai}) a sua maneira no que
concerne ao Amor dos belos, e, sendo aquele amado como um deus
(\emph{theòn}), [os amantes] a si mesmos fabricam
(\emph{tektaínetaí}) e adornam (\emph{katakosmeî}) uma imagem
(\emph{ágalma}), para lhes honrar e celebrar (\emph{timḕsôn te kaì
orgiásôn}). [252e] Os que acompanham alguém tal qual o divino Zeus,
buscam que a alma do seu amado seja similar a dele, daí observam se sua
natureza (\emph{phýsin}) é de filósofo (\emph{philósophós}) ou de
comandante (\emph{hêgemonikòs}), tornam"-se amantes dele quando o
encontram (\emph{heuróntes}) e tudo fazem para que permaneça assim.~Caso
anteriormente eles não tenham se empenhado nestas ocupações, logo
atentamente aprendem, a partir de onde for possível, se acercam dos seus
amados e como rastreadores eles mesmos descobrem (\emph{aneurískein}) a
natureza do deus que lhes é próprio, prosperam assim através do severo
esforço em olhar para o deus (\emph{pròs tòn theòn blépein}).~[253a]
Sendo eles apoderados pela memória (\emph{têi mnémei}), tomam, em
entusiasmo (\emph{enthousiôntes}), os hábitos (\emph{tà éthê}) e
ocupações a partir daquele [deus], tanto quanto é possível um homem
partilhar (\emph{matascheîn}) da divindade.

E essa é a causa de tanto amarem (\emph{agapôsin}) seus amantes, tanto
os que pertencem a Zeus, como as bacantes que atingem a alma do amado e
adotam a máxima semelhança (\emph{homoiótaton}) com relação ao seu deus
[Dioniso], [253b] como os que são seguidores de Hera e buscam
[um amado de] natureza real (\emph{basilikòn}), tendo"-o encontrado
(\emph{heuróntes}) fazem com ele tudo do mesmo modo. [253b] Os que
são de Apolo, bem como de cada um dos deuses, avançando com o deus,
buscam que seu amado seja de tal natureza, e, depois de o enredarem,
persuadem (\emph{peíthontes}) e disciplinam (\emph{rhythmídzontes}) o
predileto (\emph{tà paidiká}) a imitá"-lo (\emph{mimoúmenoi}),
conduzem"-no, assim, à ocupação e ao aspecto (\emph{idéian}) daquele
[deus] tanto quanto é possível a cada um, não por inveja ou
mesquinha hostilidade para com o predileto (\emph{tà paidiká}), mas
tentando, em tudo, levá"-lo a maior semelhança (\emph{homoióteta})
possível consigo mesmo e, portanto, com o deus honrado, assim eles
fazem. [253c] A boa vontade (\emph{prothymía}) e a iniciação
(\emph{teletḗ}) dos verdadeiros amorosos (\emph{alêthôs erṓntôn}), caso
realizem essa benevolência que digo, é bela e feliz na loucura amorosa
do amante para com o amado, se ele foi mesmo arrebatado pelo amor. O~eleito é dessa maneira tomado.

Desde o começo dessa narrativa (\emph{mýthou}) dividimos a alma em três
partes, duas delas na forma de cavalos e a terceira na forma do auriga.
Então, agora, vamos manter isso. [253d] Entre os cavalos, dissemos
que um é bom e o outro não, mas não explicamos ainda qual é a virtude do
bom e o vício do mau. Façamos isso agora. Um deles tem uma bela postura,
uma forma correta e articulada, altivo, nariz adunco, branco, olhos
negros, amante da honra (\emph{timês erastḕs}) de acordo com a
temperança e o pudor (\emph{sôphosýnês te kaì aidoûs}).~Ele é
companheiro da opinião verdadeira (\emph{alêthinês dóxês hetaîros}), ele
não insulta, obedece a um só e é conduzido pela palavra (\emph{lógoi}).
[253e] Já o outro é oblíquo, vulgar, levado ao acaso, tem pescoço
forte e curto, nariz achatado, é negro, tem olhos acinzentados, é
sanguíneo, companheiro da desmesura (\emph{hýbreôs}) e da jactância
(\emph{aladzoneías}), orelhas peludas, surdo (\emph{kôphós}), e só
obedece com dificuldade ao açoite e ao aguilhão. Então, quando o auriga
vê o olhar do amante (\emph{tò erôtikòn ómma}), sente toda alma
aquecer"-se, enchendo"-se de prurido e dos aguilhões do desejo
(\emph{póthou}). [254a] Então, o cavalo que é bem persuadido pelo
auriga e sempre constrangido pela força do pudor (\emph{aidoî}),
permanece sob seu próprio domínio e não é levado para a direção do
amado. Já o outro, nem pelo aguilhão do auriga nem pelo açoite recua,
saltando e sendo conduzido pela força. Esse oferece todo tipo de apuros
ao companheiro de jugo e ao auriga, forçando"-os na direção do predileto,
fazendo com que ele rememore (\emph{mneían}) gracejos afrodisíacos
(\emph{aphrodisíôn cháritos}).

Ambos, desde o começo, opõem"-se de modo irritadiço, [254b] uma vez
que são forçados a coisas terríveis e violentas. Mas ao final, quando
nem mesmo conseguem evitar a maldade, atravessam e seguem, agindo como
se concordassem (\emph{homologḗsante}) em fazer o que lhes foi ordenado.
Ao chegarem diante do amado e observarem a face (\emph{ópsin}) luminosa
do predileto, a memória (\emph{hê mnḗmê}) do auriga, pela visão, é
levada à natureza do belo, momento em que novamente vê (\emph{pálin
eîden}) aquela beleza, de acordo com a prudência (\emph{sôphrosýnês}),
estabelecida num sagrado pedestal.

Ao vê"-lo, ele o teme e sente vergonha, a ponto de cair de costas, e ao
mesmo tempo o constrange ao puxar as rédeas com tanta força que ambos os
cavalos se assentam sobre os próprios quadris [254c], um por vontade
própria, sem oferecer oposição, mas o outro, rebelde (\emph{hubrístḕn}),
faz isso muito a contragosto. Chegando a um lugar mais afastado, um por
estar com vergonha e estupefato, banha toda a alma com suor, o outro,
estando apaziguado da dor, causada pelo freio e pela queda, toma fôlego
e, com ímpeto, vitupera os muitos abusos do auriga e do companheiro de
jugo, como se por timidez ou covardia eles houvessem abandonado a ordem
(\emph{táxin}) e o acordo (\emph{homologían}).~[254d] E novamente
(\emph{pálin}), não desejando ser conduzido à força, às duras penas ele,
[o cavalo negro], aceita o que lhe foi demandado, adiando sua nova
investida (\emph{hyperbalésthai}). Chegado o tempo determinado, como se
estivesse esquecido (\emph{amnêmoneîn}), ele, [o cavalo negro], é
levado à rememoração (\emph{anamimnḗiskôn}), e usando toda sua energia,
relinchando, puxa fortemente para o lado contrário, arrastando
(\emph{hélkôn}) para onde está o favorito e oferecendo"-lhe os mesmos
discursos (\emph{lógous}). Logo em seguida, quando ele se aproxima,
agacha e estica a calda, morde o freio e arrasta (\emph{hélkei}) sem
nenhum pudor (\emph{anaideías}).

[254e] O auriga sofre a maior dessas afecções (\emph{páthos
pathón}), como se estivesse impedido por uma corda, uma vez que há a
desmesura (\emph{hybristoû}) do cavalo que é arrastado com força pelo
freio dos dentes, tendo a língua maledicente e a mandíbula
ensanguentadas, aí então suas patas e sua anca são lançadas à terra e
expostas ao sofrimento. Quando esse malvado sofre todas essas coisas,
cessa sua desmesura (\emph{hýbreôs}),~submetendo"-se à condução e à
intenção (\emph{pronoíai}) do auriga, e quando vê o belo [novamente]
ele é aniquilado pelo medo (\emph{phóbôi}). A~partir daí, ocorre que a
alma do amante (\emph{erastoû}) passa a seguir o predileto com pudor e
respeito (\emph{aidouménên te kaì dediuîan}). [255a] Por conta de
todos os cuidados dispensados pelo amante, que eram similares aos
dispensados a um deus, não por que o amante fingia, mas por sentir"-se
assim de verdade, fizeram com que o próprio amado, naturalmente, se
tornasse amigo do seu servidor, mesmo que anteriormente ele tivesse sido
reprovado pelos companheiros ou por quaisquer outros que diziam ser
vergonhoso associar"-se a um amante, e que por isso tivesse repelido o
amante, mas com o passar do tempo, [255b] a idade e a necessidade
fazem com que aquele seja aceito em sua companhia. Não quis o destino
nem que o malvado fosse amigo do malvado, nem que o bondoso não fosse
amigo do bondoso. Tendo oferecido o discurso e o recebido em sua
companhia, a proximidade e a benevolência do amante provocam no amado a
sensação de perturbação, uma vez que nem outros amigos, nem familiares,
ninguém frente ao amigo entusiasmado (\emph{éntheon})~oferece parcela
alguma de amizade. E~quando por muito tempo age assim, aproximando"-se
dele para tocá"-lo nos ginásios e em outras ocasiões, [255c] aí então
[surge] a fonte (\emph{pêgḗ}) daquele fluxo (\emph{rheúmatos}), o
qual Zeus, amante de Ganimedes, denominou de desejo (\emph{hímeron}),
que chega em abundância no amante, preenchendo"-lhe e, uma vez
preenchido, transborda para o exterior. Tal qual um sopro
(\emph{pneûma}) ou algum eco (\emph{êchô}) que numa superfície lisa ou
sólida é levado novamente (\emph{pálin}) ao ponto de partida, assim o
fluxo da beleza é novamente (\emph{pálin}) direcionado ao belo, através
dos olhos (\emph{ommátôn}), por onde a alma é acessada e tem as asas
acrescidas (\emph{anapterôsan}). [255d] Então, as vias das asas são
irrigadas (\emph{pterôn árdei}), iniciando o seu brotar
(\emph{pterophyeîn}), enquanto o amor preenche (\emph{érôtos enéplêsen})
a alma do amado (\emph{erôménou}). Ele ama, mas não sabe o quê. Não sabe
o que sofre e não tem como expressar isso. Tal qual uma oftalmia
(\emph{ophthalmías})~adquirida de outrem, ele não tem como expressar a
causa, uma vez que lhe escapa (\emph{lélêthen}) que vê a si mesmo no seu
amante, como se fosse em um espelho (\emph{katóptrôi}). E~quando está
junto dele, cessa o seu sofrimento, tal qual no amante, mas quando está
separado, ele deseja e é também desejado (\emph{potheî kaì potheîtai}),
pois adquire um ídolo do amor, um Ânteros (\emph{eídôlon érôtos antérôta
échôn}).~[255e] A este nomeia e considera não como amor
(\emph{érota}), mas como amizade (\emph{philían}). O~seu desejo
(\emph{epithymeî}) é quase o mesmo daquele, só que menos intenso, o de
ver, tocar, beijar, deitar"-se ao seu lado, ações que, como é verossímil
(\emph{eikós}), não tardará a realizar. Então, quando partilha o mesmo
leito do amante, o cavalo indisciplinado (\emph{akólastos}) tem algo a
dizer ao auriga, esperando depois de todos os sofrimentos tirar"-lhe um
pequeno benefício. [256a] O predileto não tem nada a dizer e, pleno
de desejo, perplexo, acerca"-se do amante e o adora, amigavelmente
saudando a quem bem lhe quer, como quem não pode recusar os gracejos
(\emph{charísasthai}) do amante quando está a seu lado, se por acaso
(\emph{tycheîn}) ele demandar. O~companheiro de jugo se opõe a isso,
juntamente com o auriga, seguindo seu pudor e sua razão (\emph{aidoûs
kaì lógou}). E~se, por acaso, predominarem as melhores partes do
pensamento (\emph{dianoías}), as que conduzem a um regime de vida
ordenado e amante da sabedoria (\emph{philosophían}), [256b] são
felizes e conduzem uma vida de concórdia, estando eles senhores de si
(\emph{egkrateîs}) e disciplinados (\emph{kósmioi}), subjugam aquilo que
faz nascer a maldade na alma e libertam aquilo que nela gera a virtude
(\emph{areté}). Então, no fim da vida, ganham asas e leveza
(\emph{hypópteroi kaì elaphroì}), pois venceram um dos três combates
verdadeiramente olímpicos, o qual é o maior bem (\emph{agathòn}), não
podendo ser alcançado pelo homem, nem pela prudência
(\emph{sôphrosýnê}), nem pela loucura divina (\emph{theía manía}). Mas
se, pelo contrário, levarem um regime de vida mais vulgar e sem amor
pela sabedoria (\emph{aphilosóphôi}), [256c] valendo"-se do amor pela
honra (\emph{philotímôi}), então, rapidamente, nas bebedeiras
(\emph{méthais}) ou em outras ocasiões de despreocupação
(\emph{ameleíai}), os dois [cavalos] libertinos (\emph{akolástô}),
sob o mesmo jugo, tomam as almas desprevenidas (\emph{aphroúrous}),
unindo"-se ambos para o mesmo fim, escolhendo (\emph{airesin}) o que a
maioria (\emph{tôn pollôn}) toma por excelente e assim praticam.~Tendo
realizado isso, valem"-se desse comportamento em ocasiões futuras, embora
raramente isso ocorra, visto que a praticam sem a aprovação plena da
reflexão (\emph{dianoíai}). Estes são obviamente amigos, mas em menor
grau que os anteriores, [256d] e, o amor lhes é recíproco e até
mesmo depois dele acabar, (\emph{písteis}) acreditam terem oferecido e
recebido mutuamente as melhores coisas, o que torna ilícito que fiquem
apartados a ponto de serem hostis entre si. No final da vida, sem asas
(\emph{ápteroi}), mas desejosos de as terem adquirido, eles saem do
corpo, e não é pequena a sua recompensa (\emph{âthlon}) advinda dessa
loucura amorosa (\emph{erôtikês manías}). Não há uma lei que designe que
aqueles que iniciam seu trajeto sob o céu devam passar pela travessia
escura e pelo subterrâneo, mas sim que atravessem celebrando entre si
uma vida luminosa (\emph{phanòn bíon}) e feliz (\emph{eudaimoneîn}),
[256e] e que sejam agraciados pelo amor com asas semelhantes
(\emph{homópteros érôtos chárin}), quando chegar o momento de seu
surgimento. São essas então, ó jovem, as coisas divinas (\emph{theîa})
que lhes são entregues pela amizade para com o amante (\emph{par'
erastoû philía}).

A familiaridade com o não afetado pelo amor (\emph{mè erôntos}),
mesclada com a prudência (\emph{sôphrosýnêi}) mortal na administração de
bens mortais e miseráveis (\emph{thnêtá te kaì pheidôlà}), é uma
servilidade (\emph{aneleutherían}) elogiada por muitos (\emph{plḗthous
epainouménen}) como uma virtude (\emph{aretḕn}) gerada pela amizade na
alma, [257a] mas que faz com que ela gire por nove mil anos, ao
redor e debaixo da terra, num percurso sem intelecto (\emph{ánoun}).

Esta é, ó querido Eros, dentro das nossas possibilidades, a mais bela e
melhor palinódia que eu poderia oferecer"-te como pagamento, entre tantas
outras razões, mas especialmente no vocabulário~poético a que fui
forçado por Fedro. Desculpe"-me (\emph{suggnṓmên}) pelos primeiros
[discursos] e que este último o tenha agradado (\emph{chárin}), seja
para mim benévolo e propício na arte de amar (\emph{tḕn erôtikḗn
téchnên}) que me destes, que eu não seja dela subtraído nem incapacitado
pelo impulso (\emph{orgén}), e que me seja concedido ser ainda mais
honrado (\emph{tímion}) junto aos belos. [257b] E, se com os dois
primeiros discursos eu e Fedro fomos dissonantes a ti, o causador foi
Lísias, o pai do discurso (\emph{tòn toû lógoû patéra}). Então,
interrompe (\emph{paûe}) nele tais discursos (\emph{lógôn}) e o conduz
(\emph{trépson}) para a filosofia, como foi conduzido (\emph{tétraptai})
o seu próprio irmão Polemarco, para que este seu amante (\emph{erastès})
aqui não fique mais entre dois caminhos (\emph{epamphoterídzêi}),
justamente como agora, mas tenha a vida devotada somente para o Amor, de
acordo com discursos filosóficos''.

 

F: Junto minhas preces às tuas, ó Sócrates, e, se isso for o melhor para
nós, que assim seja. [257c] O teu discurso há tempo que admiro
(\emph{thaumásas}), tanto mais belo que o anterior o fizeste. Assim,
receio que Lísias me apareça inferior mesmo, especialmente se queres
(\emph{ethelêsêi}) contra ele competir (\emph{antiparateînai}).~Pois é
algo assim, ó admirável (\emph{thaumásie}), agora mesmo um dos políticos
insultava e censurava Lísias, e entre os insultos proferidos o designava
por logógrafo (\emph{logográphon}). Talvez, então, tenha sido o amor
pela honra (\emph{philotimías}) o motivo pelo qual ele se absteve de nos
escrever (\emph{gráphein}).

 

S: Engraçado, ó jovem, o parecer (\emph{dógma}) que proferes, pois sobre
o teu companheiro (\emph{hetaírou}) estás completamente enganado
(\emph{diamartáneis}), se o consideras como alguém tímido. [257d]
Talvez aquele que o insultava considerasse censurável dizer o que disse
(\emph{légein hà élegen}).

 

F: É o que parece, ó Sócrates. Tu sabes como os poderosos e
reverenciados nas cidades envergonham"-se de escrever discursos
(\emph{lógous te gráphein}) e de deixar composições suas
(\emph{kataleípein suggrámmata heautôn}), temerosos da reputação
(\emph{dóxan}) que, com o tempo, pode atingi"-los, sendo designados por
sofistas (\emph{mè sophistaì kalôntai}).

 

S: Doce rodeio, ó Fedro, mas esqueces ainda do grande rodeio mencionado
pelos que descem o Nilo. [257e] E além desse rodeio, esqueceste que
os maiores amantes da logografia (\emph{erôsi logographías}), bem como
do legado de composições escritas (\emph{kataleípseôs suggrammátôn}),
são os grandes e os mais notáveis políticos (\emph{oi mégiston
phronoûntes tôn politikôn}).~Em seguida, em discurso escrito
(\emph{gráphôsi lógon}), eles agradam aos seus panegiristas
(\emph{epainétas}), uma vez que estes são os primeiros a elogiá"-los
(\emph{epainôsin}), previamente e em qualquer situação.

 

F: Como dizes isso? Não compreendo.

 

[258a] S: Não compreendes porque os políticos, no início das suas
composições escritas, inscrevem (\emph{gégraptai}) primeiramente o nome
dos seus panegiristas (\emph{epainétês}).

 

F: Como?

 

S: ``Foi resolvido'', como ele diz, ``pelo conselho'' ou ``pelo povo'',
ou por ambos, e ao dizer ``aquele que'', refere"-se ao seu próprio
discurso, no que há de mais sagrado e elogiável (\emph{egkômiádzôn}) no
escritor (\emph{suggrapheús}). Depois de tudo isso, mostra aos seus
panegiristas (\emph{epainétais}) a sua própria sabedoria
(\emph{sophía}), por vezes redigindo composições escritas
(\emph{sýggramma}) bastante longas. Que outra coisa te parece isso,
se não uma composição de discurso escrito (\emph{lógos syggegramménos})?

 

F: Não me parece outra coisa.

 

S: Então, se for bem recebido, o poeta deixa o teatro com júbilo, se for
rejeitado, é privado da logografia e da dignidade (\emph{áxios}) de
escrever (\emph{suggráphein}), lamentando"-se ele e os seus companheiros.

 

F: E muito.

 

S: Parece que não desprezam essa ocupação, mas a admiram.

 

F: Perfeitamente.

 

S: O quê? Quando alguém vem a ser um rétor ou um rei, tal qual Licurgo,
Sólon ou Dario, não é possível que ele venha a se tornar um logógrafo
imortal da cidade? [258c] Enquanto está vivo, ele é visto como um
deus~e, depois, os subsequentes cidadãos não o considerariam da mesma
maneira, ao contemplarem suas composições escritas (\emph{suggrámmata})?

 

F: E como.

 

S: Consideras que um desses, qualquer um, com qualquer tipo de desavença
contra Lísias, poderia censurá"-lo (\emph{oneidízein}) porque escreveu
(\emph{suggráphei})?

 

F: Não é verossímil (\emph{eikós}) pelo que dizes. Pois seria, como
parece, uma censura (\emph{oneidízoi}) contra o próprio desejo
(\emph{epithymíai}).

 

[258d] S: E isso é claro para todos, que não é vergonhoso por si só
escrever discursos (\emph{grapheîn lógous}).

 

F: Como?

 

S: Considero vergonhoso falar (\emph{légein}) e escrever
(\emph{gráphein}) sem nenhuma beleza, além de uma vergonha é algo
malvado.

 

F: É claro.

 

S: Qual é, então, a maneira (\emph{trópos}) de escrever com beleza ou sem?
Precisamos, ó Fedro, examinar (\emph{exetásai}) esse assunto junto a
Lísias ou a qualquer outro que tenha escrito (\emph{gégraphen}) ou que
ainda vá escrever (\emph{grápsei}), seja sobre um escrito político
(\emph{politikòn súggramma}) ou um assunto particular, seja na métrica
como poeta ou sem, como um prosador?

 

[258e] F: Perguntas (\emph{erôtais}) se precisamos? Que motivo teria
alguém para viver se não em vista, por assim dizer, desses mesmos
prazeres (\emph{hêdonôn})?~Pois não são daqueles que necessitam de
sofrimento prévio (\emph{prolupêthênai}), sem o que nem mesmo o prazer
(\emph{hêsthênai}) haveria, mas estão entre os poucos (\emph{olígou})
que fornecem todos os prazeres corpóreos (\emph{sôma hêdonaì}), motivo
pelo qual, justamente, são designados por servis (\emph{andrapodṓdeis}).

 

S: Temos tempo livre (\emph{scholḕ}), como parece. Enquanto isso, as
cigarras cantoras~conversam entre si nesse calor e nos observam
(\emph{kathorân}) lá de cima. [259a] Se elas nos vissem, como a
maioria, ao meio"-dia e sem dialogarmos (\emph{mḕ~dialegoménous}), quase
dormindo, encantados pela preguiça da reflexão (\emph{dianoías}), elas
justamente nos desprezariam, considerando"-nos como criaturas cativas que
chegaram a um recanto, como ovelhas, ao meio"-dia, a dormir junto à fonte
(\emph{tḕn krḗnen eúdein}). Mas se elas nos vissem a dialogar
(\emph{dialegoménous}) e a evitá"-las, como [quem evita] as Sirenas,
sem nos deixarmos encantar (\emph{akêlḗtous}), então rapidamente nos
admirariam e conceder"-nos"-iam as dádivas divinas atribuídas aos homens.

 

[259b] F: Quais são essas dádivas? Não ouvi, como parece, acerca de
nenhuma delas?

 

S: Não é adequado (\emph{prépei}) a um homem amigo das Musas
(\emph{philómouson}) não ter ouvido falar nisso. Dizem que, antes do
tempo das Musas, as cigarras eram homens e que, quando estas [Musas]
surgiram e lhes mostraram os cantos (\emph{phaineísês}~\emph{oidês}),
alguns deles foram tomados por esse prazer (\emph{hêdonês}). Envolvidos
com o canto (\emph{áidontes}), eles, sem perceber, acabaram descuidando
da comida e da bebida, sendo levados à morte. Deles é que a família das
cigarras descende, pois, junto às Musas, tendo recebido essa dádiva
(\emph{géras}), elas não têm necessidade de alimentos, mas vivem a
cantar (\emph{aidein}) ininterruptamente, sem comer e sem beber, até a
morte e, depois disso, para as Musas relatam (\emph{apaggéllein}) quais
foram aqueles que as honraram (\emph{timâi}) aqui. Terpsicore
(Alegra"-coro) é venerada (\emph{tetimêkótas}) nas danças
(\emph{choroîs}), relato que proporciona maior benevolência aos seus
realizadores.~[259d] Érato (Amorosa) com a [poesia] erótica
(\emph{erôtikoîs}) é venerada, assim também em outras ocasiões, segundo
cada forma de honra (\emph{timês}). As mais velhas delas são Calíope
(Belavoz) e em seguida Urânia (Celeste), para aqueles que se dedicam à
filosofia e que estimam (\emph{timôntas}) a música, pois especialmente
as Musas enviam bela"-voz acerca do céu, dos discursos dos deuses e dos
homens.~Muitas são as razões para que falemos ao meio"-dia e não
cochilemos.

 

F: Falemos então.

 

[259e] S: Vamos agora estabelecer uma verificação
(\emph{sképsasthai}) sobre o discurso. Verifiquemos (\emph{skeptéon}) em
que medida é possível falar e escrever (\emph{légein te kaì gráphein})
de modo belo (\emph{kalôs}) ou não.

 

F: Claro.

 

S: Não é necessário àqueles que desejam falar bem e de modo belo, que o
pensamento (\emph{diánoian}) de quem fala conheça a verdade
(\emph{eiduîan tò alêthès}) acerca do que será tratado?

 

[260a] F: Acerca disso ouvi o seguinte, ó querido Sócrates: aquele
que deseja tornar"-se rétor não necessita compreender (\emph{manthánein})
o que é verdadeiramente justo (\emph{tôi ónti díkaia}), mas o que parece
ser para aqueles muitos (\emph{tà dóxant' an plḗthei}), nem o verdadeiro
bom e belo, mas o que lhes parecer assim (\emph{tà óntos agathà ê kalà
all' hósa dóxei}). Disso deriva a persuasão (\emph{peíthei}), e não da
verdade (\emph{alêtheías}).

 

S: ``Palavras nada desprezíveis'', ó Fedro, essas que os sábios
(\emph{sophoí}) proferem, mas vamos examinar (\emph{skopeîn}) se elas
nos dizem algo. Certamente o que foi dito não deve ser abandonado.

 

F: Dizes bem.

 

S: Examinemos.

 

F: Como?

 

[260b] S: Se eu quisesse convencer"-te e ajudá"-lo na aquisição
(\emph{ktêsámenon}) de um cavalo de combate, ambos desconhecendo
(\emph{agnooîmen}) o que é um cavalo, mas, se alguma coisa, entretanto,
eu soubesse sobre você, que Fedro considera que ele é o animal doméstico
que tem a maior orelha.

 

F: Seria engraçado, ó Sócrates.

 

S: Nem tanto. Mas, na ocasião de ocupar"-me da tua persuasão
(\emph{peíthoimi}), colocando o discurso elogioso (\emph{épainon}) no
asno, designando"-o por cavalo, falando acerca de todas as qualidades da
criatura no uso doméstico, na aquisição, na guerra, defendendo sua
utilidade (\emph{ôphélimon}) com bagagens e outras tantas tarefas.

 

[260c] F: Isso seria realmente engraçado.

 

S: Mas então não seria melhor o engraçado (\emph{geloîon}) e o amistoso
(\emph{phílon}) do que o terrível (\emph{deinón}), ou o hostil
(\emph{echthròn})?

 

F: Parece.

 

S: Mas, quando o rétor desconhece (\emph{agnoôn}) o bom (\emph{agathòn})
e o mau (\emph{kakón}), tomando uma cidade pela a persuasão
(\emph{peíthêi}), não faria um elogio (\emph{épainon}) da sombra
(\emph{skiâs}) de um asno como se fosse de um cavalo, mas elogiaria o
mau como sendo o bom, e, exercitado na opinião da maioria (\emph{dóxas
dè plḗthous memeletêkòs}), ele poderia persuadi"-los (\emph{peísêi}) a
fazer o mau e não o bom. Considerando isso tudo, que tipo de fruto
(\emph{karpòn}) a retórica poderia colher dessa semeadura?

 

F: Um fruto não muito agradável.

 

S: Então, ó bondoso, fomos mais grosseiros que o necessário ao
detratarmos a arte dos discursos (\emph{lógôn téchnên})? E ela talvez
nos dissesse: ``Ó admiráveis, porque dizeis tais bobagens? Eu não obrigo
ninguém que desconheça a verdade a aprender a falar (\emph{agnooûnta
talêthès anagkádzô manthánein légein}), mas, se em algo vale o meu
conselho (\emph{sumboulḗ}), que adquiram (\emph{ktêsámenon}) aquela
[verdade] antes de me tomar (\emph{lambánein}). Eis, então, o que digo
veementemente: que, sem mim, aquele que conhece a verdade nunca
alcançará a arte de persuadir (\emph{peíthein téchnei}).''

 

[260e] F: E ela não diria coisas justas ao proferir isso?

 

S: É o que digo, se os discursos apresentados testemunham que ela é uma
arte. É~como se eu ouvisse a aproximação de alguns contestadores da arte
do discurso a dizer que ela é falsa (\emph{pseúdetai}), que ela não é
uma arte, mas uma ocupação isenta de arte (\emph{ouk esti téchne alla
atechnos tribé}). Os Lacônios afirmam que ``não existe uma arte
verdadeira (\emph{étymos téchnê}) sem estar atada à verdade
(\emph{alêtheías}), nem mesmo poderá existir no futuro.''

 

 

[261a] Precisamos desses discursos, ó Sócrates, traze"-os agora para
junto de nós para examinarmos (\emph{exétaze}) o quê e como eles falam.

 

S: Vinde, nobres criaturas, persuadi (\emph{peíthete}) Fedro de belos
filhos de que, quem não filosofar suficientemente (\emph{ikanôs
philosophḗsêi}), não será nunca capaz de falar sobre coisa alguma.
Responde agora, Fedro.

 

F: Pergunta.

 

S: Então, o todo da retórica não seria a arte da condução das almas
(\emph{téchnê}~\emph{psychagôgía}) por meio das palavras (\emph{dià
lógôn}), não só nos tribunais (\emph{dikastêríois}) e em outras
assembleias públicas (\emph{dêmósioi súllogoi}), mas também nas questões
particulares (\emph{idíois}), naquelas insignificantes e nas grandiosas,
e que não há nada de mais honrado (\emph{entimóteron}) que o seu
emprego, quando correto, seja nos assuntos sérios ou nos banais?
[261b] Ou como ouviste falar disso tudo?

 

F: Não, por Zeus, não foi assim absolutamente, mas que especialmente nos
tribunais (\emph{dikás}) fala"-se e escreve"-se com arte, bem como nas
assembleias públicas (\emph{dêmêgorías}). Não ouvi mais do que isso.

 

S: Mas, então, apenas ouviste sobre as artes discursivas de Nestor e de
Odisseu, as quais foram escritas (\emph{sunegrapsátên}) por eles em
Troia, nas horas vagas (\emph{scholázontes}), e nem mesmo chegaste a
ouvir aquela composta por Palamedes?

 

[261c] F: Por Zeus, nem mesmo ouvi a de Nestor, a não ser que
consideres Górgias uma espécie de Nestor, ou Trasímaco e Teodoro
distintos tais quais Odisseu.

 

S: Talvez, mas deixemos estes aí por hora. E~tu dize"-me o que fazem
(\emph{tí drôsin}) os que disputam nos tribunais (\emph{dikastêríois oi
antídikoi}), eles não entram em litígio (\emph{antilégousin})? Ou o que
diremos?

 

F: Isso mesmo.

 

S: Acerca do justo e do injusto (\emph{dikaíou te kaì adíkou})?

 

F: Sim.

 

S: Então, quem lançar mão dessa atividade com arte fará as mesmas coisas
parecerem justas às mesmas pessoas, e, por outro lado, quando quiser
(\emph{boúlêtai}), parecerem injustas?

 

[261d] F: O que tem isso?

 

S: E, também, nas assembleias públicas (\emph{dêmêgoríai}) da cidade,
fará parecer as mesmas coisas ora boas ora o seu contrário
(\emph{tanantía})?

 

F: É assim.

 

S: Então não conhecemos os dizeres com arte do eleático Palamedes, por
meio do qual mostrava aos ouvintes as mesmas coisas como semelhantes e
dessemelhantes (\emph{hómoia kaì anómoia}), unas e múltiplas (\emph{hén
kaì pollá}), em repouso e em movimento (\emph{ménontá te aû kaì
pherómena})?

 

F: E como!

 

[261e] S: Então, não só no tribunal e nas assembleias públicas
existe a antilogia (\emph{antilogikḕ}), mas, como parece, em todas as
coisas que são ditas há uma só arte (\emph{mía tis téchnê}), se é que
existe, aquela que é capaz de assemelhar tudo a todas as coisas
possíveis (\emph{pân pantì homoíoûn tôn dynatôn}), na medida do
possível, e também de trazer à luz (\emph{eis phôs agein}) o que outros,
operando essas semelhanças (\emph{homoioûntos}), tentam dissimular
(\emph{apokruptôménou}).

 

F: Como é que dizes isso?

 

S: Procuro mostrar"-te isso que buscamos. O~engano nasce
predominantemente naquilo que difere muito ou pouco?

 

[262a] F: No que difere pouco.

 

S: Mas, então, seria melhor para transportar às ocultas (\emph{metabaínôn
mâllon lḗseis}), conduzindo os discursos ao seu contrário, guiar"-se pelo
que difere pouco (\emph{smikròn}) ou, ao contrário, pelo que difere
muito.

 

F: Como não seria assim?

 

S: É necessário àquele que se prontifica a enganar (\emph{apatḗsein})
outros, sem enganar a si mesmo (\emph{autòn dè mḕ~apatḗsesthai}), que
conheça exatamente as semelhanças e as dessemelhanças entre os seres
(\emph{tèn homoiótêta tôn óntôn kaì anomoiótêta akribôs dieidénai}).

 

F: É necessário.

 

S: E será possível a este mesmo homem, desconhecendo a verdade,
reconhecer as semelhanças (\emph{homoiótêta}) menores e maiores em
outros seres?

 

[262b] F: Impossível.

 

S: Dessa forma, para os que opinam contra [a existência] dos seres e
são enganados, é evidente como sua afecção (\emph{páthos}) foi arrastada
por aquelas semelhanças (\emph{di'homoiotḗtôn}).

 

F: É assim mesmo que acontece.

 

S: Como o artífice (\emph{technikòs}) irá transladar
(\emph{metabibázein}), seguindo as menores semelhanças
(\emph{homoiotḗtôn}) entre os seres, levando cada um deles ao seu
contrário (\emph{tounantíon})? E como ele poderia esquivar"-se
(\emph{diapheúgein}) desse mesmo efeito sem reconhecer o que são cada um
dos seres (\emph{hékaston}~\emph{tôn óntôn})?

 

F: Não poderia.

 

[262c] S: Então, a arte do discurso (\emph{lógôn téchnên}), ó
companheiro, sem o conhecimento do que é verdadeiro (\emph{tèn alétheian
mè eidôs}), é como uma caça das opiniões (\emph{dóxas dè tethêreukós}),
ocupação risível e, como bem parece, desprovida de arte (\emph{átechnon
paréxetai}).

 

F: É bem possível.

 

S: Sobre esse discurso de Lísias que trazes consigo ou esses que
pronunciamos em seguida, pretendes observar o que neles há desprovido de
arte (\emph{atéchnôn}), bem como o que está de acordo com a arte
(\emph{entéchnôn})?

 

F: É o melhor a fazer, especialmente porque, até agora, só falamos no
vazio, sem paradigmas suficientes.

 

S: E foi por sorte (\emph{tychên}), como bem parece, que nós temos dois
discursos como paradigmas, o que mostra que aquele que conhece a verdade
(\emph{eidôs tò alêthès}), brincando com as palavras (\emph{prospaídzôn
en lógois}), pode demover os ouvintes. [262d] E eu, ó Fedro, atribuo
isso aos deuses desse lugar (\emph{entopíous theoús}). Talvez tenham
sido as cigarras, intérpretes das Musas, que, sobre nossas cabeças,
cantam e inspiram"-nos essa honra, pois eu não partilho (\emph{métochos})
de nenhuma arte no meu discurso (\emph{téchnês toû légein}).

 

F: Que assim seja, mas apenas mostra o que dizes.

 

S: Lá vai então. Lê"-me o início do discurso de Lísias.

 

[262e] F: ``Já estás ciente acerca dos meus assuntos e creio que
ouviste acerca do que pode acontecer conosco. Espero que não me advenha
nenhum infortúnio (\emph{atychêsai}) só porque me ocorreu
(\emph{tygcháno}) de não estar te amando (\emph{ouk}~\emph{erastès}),
como aqueles que \redondo{[…]} arrependem"-se (\emph{metamélei})
\redondo{[…]}''.

 

S: Para. Tratemos agora do que ele errou (\emph{hamartánei}) e no que
procedeu sem arte (\emph{átechnon}), não é mesmo?

 

[263a] F: Sim.

 

S: Mas isso não é evidente para todos, que acerca de algumas coisas nós
concordamos (\emph{homonoêtikôs}) e de outras discordamos
(\emph{stasiôtikôs})?

 

F: Perece que entendo o que dizes, mas explica ainda de modo mais claro.

 

S: Quando dizes um nome como ferro (\emph{sidḗrou}) ou prata
(\emph{argúrou}), não entendemos (\emph{dienoḗthemen}) todos nós a mesma
coisa?

 

F: E como.

 

S: E quanto ao justo ou ao bom? Não ocorre que nos dirijamos uns para um
lado e outros para outro, fazendo como que entremos em controvérsia
mútua e até conosco mesmo (\emph{amphisbêtoûmen allélois te kaì hêmîn
autoîs})?

 

F: É assim mesmo.

 

[263b] S: Então, existem coisas a respeito das quais nós chegamos a
um acordo (\emph{sumphônoûmen}) e outras não?

 

F: De fato.

 

S: Em quais delas é mais fácil nos enganarmos e em qual delas a retórica
(\emph{rhetorikḕ}) tem maior poder (\emph{meîdzon dýnatai})?

 

F: É evidente que naquelas em que nós somos errantes
(\emph{planṓmetha}).

 

S: Então, para aquele que deseja seguir a arte retórica (\emph{téchnên
rhetorikḕn}), primeiramente, é preciso que diferencie esses dois
caminhos e que detecte os caracteres de cada um deles, onde
necessariamente a multidão erra
(\emph{to}~\emph{plêthos}~\emph{planâsthai}) e onde não.

 

[261c] F: Belo seria, ó Sócrates, deter essa forma de apreensão
(\emph{katanenoêkṑs}).

 

S: Em seguida, creio que não devemos deixar de observar cada um dos
assuntos que surgem, mas percebê"-los com agudeza (\emph{oxéôs
aisthánesthai}), bem como o gênero daquilo que falaremos.

 

F: Sem dúvida.

 

S: E o que diremos do amor? É algo controverso (\emph{amphisbêtêsímôn})
ou não?

 

F: Presumo que seja algo controverso (\emph{amphisbêtêsímôn}), ou
consideras ser possível admitir o que há pouco disseste acerca dele, que
é danoso (\emph{blábê}) ao amante e ao amado (\emph{tôi erôménôi kaì
erônti}) e, logo depois, que é o maior dos bens ocorridos (\emph{agathôn
tugchánei})?

 

[263d] S: Dizes muito bem, mas também diz"-me isso -- pois, pelo meu
entusiasmo (\emph{enthousiastikòn}), não me lembro bem (\emph{ou pánu
mémnêmai}) \mbox{---,} se defini o amor no início do discurso.

 

F: Sim, por Zeus, e com extraordinária precisão.

 

S: Ah! Proclamas superiores na arte (\emph{technikôtéras}) as Ninfas,
filhas de Aqueloo, e Pã, filho de Hermes, comparados a Lísias, filho de
Céfalo, em seu discurso. Talvez eu esteja errado, mas não é verdade que
Lísias, no início do seu discurso erótico (\emph{toû erôtikoû}),
forçou"-nos a entender o amor tal qual ele desejou (\emph{eboulḗthê}), e
a partir disso compôs (\emph{suntaxámenos}) tudo o que veio depois,
levando o discurso a seu termo? [263e] Queres novamente que leiamos
o seu começo (\emph{boúlei pálin anagnômen tèn archèn autoû})?

 

F: Se te parece conveniente. Mas o que procuras (\emph{dzêteîs}) não
está aí.

 

S: Diz, para que eu possa ouvi"-lo dele mesmo.

 

F: ``Já estás ciente acerca dos meus assuntos e creio que ouviste acerca
do que pode acontecer conosco. Espero que não me advenha nenhum
infortúnio (\emph{atychêsai}) só porque me ocorreu (\emph{tygcháno}) de
não estar te amando (\emph{ouk}~\emph{erastḕs}), [264a]como aqueles
que tão logo tenha cessado o seu desejo (\emph{epithymías paúsôntai}),
arrependem"-se (\emph{metamélei}) do que bem fizeram.''

 

S: Falta muito ainda, ao que parece, para que ele realize isso que
procuramos. Ele nem começa pelo começo, mas pelo final, empreendendo seu
discurso como alguém que nada de costas e para trás, iniciando pelas
coisas que o amante diria ao seu predileto somente no final. Ou não é
como digo, Fedro, querida cabeça.

 

[264b] F: É assim mesmo, ó Sócrates, uma peroração (\emph{teleutḗ})
em torno da qual é realizado o discurso.

 

S: E quanto ao resto? Não perece que foi lançado indiscriminadamente no
discurso? Ou o que veio depois do discurso deveria ser de fato colocado
depois por alguma necessidade, ou alguma outra coisa entre as que foram
ditas? Pois a mim me parece, como não sei de nada, que não é vil o que
foi proferido pelo escritor (\emph{tôi gráphonti}). E~tu conheces alguma
necessidade logográfica pela qual ele dispôs assim o discurso de modo
sucessivo, uma coisa ao lado da outra?

 

F: És muito gentil, uma vez que me considera suficientemente capaz de
assim discerni"-lo com precisão (\emph{akribôs diideîn}).

 

[264c] S: Mas te considero capaz de mostrar isso, que é necessário
que todo discurso esteja combinado como um ser vivo (\emph{hṓsper
dzôion}), tendo corpo próprio, não sendo acéfalo nem ápodo, e que tenha
tronco e membros convenientes entre si e com relação ao todo do escrito
(\emph{tôi hólôi gegramména}).

 

F: Como não?

 

S: Verifica (\emph{sképsai}) esse discurso do teu companheiro, seja ele
assim ou de outra maneira. Não encontras\emph{~}(\emph{heurḗseis}) no
escrito nada de diferente do epigrama (\emph{epigrámmatos}) da tumba de
Midas da Frígia, segundo alguns relatos escritos
(\emph{epigegráphthai}).

 

F: Como é e do que ele trata?

 

S: É assim:

 

\begin{quote}
Eu sou a virgem de bronze que jaz sobre a tumba de Midas,

enquanto a água fluir e grandes árvores florescerem,

eu permaneço sobre este túmulo tão chorado,

e anuncio aos que passam que Midas está aqui sepulto.
\end{quote}

 

Suponho que percebestes como não há diferença entre o que vem dito antes
ou depois.

 

[264e] F: Tu zombas do nosso discurso, ó Sócrates!

 

S: Deixemo"-lo então para não te irritar. Ainda que ele me pareça um
exemplo (\emph{paradeígmata}) àqueles que podem observá"-lo com algum
proveito, sem, contudo, imitá"-lo na performance, mas vamos para outros
discursos, pois neles há algo, como me parece, que diz respeito aos que
querem conhecer (\emph{ideîn}) e examinar discursos (\emph{perì lógôn
skopeîn}).

 

[265a] F: A que tipo de coisas te referes?

 

S: Meus dois discursos eram como que opostos (\emph{enantíô}), pois um
dizia que é necessário agraciar (\emph{charídzesthai}) ao amoroso
(\emph{tôi erônti}) e o outro ao que não é amoroso.

 

F: E com que virilidade.

 

S: Considerei que tu dirias ``com que loucura'' (\emph{manikôs}), que
seria o termo verdadeiro (\emph{talêthès}). Era ele que de fato eu
procurava. Pois dizemos ser o amor (\emph{erôta}) uma loucura
(\emph{manían}) ou não?

 

F: Sim.

 

S: Mas há duas (\emph{eîde dúo}) espécies de loucura (\emph{manías}): a
que afeta os homens como uma enfermidade (\emph{nosêmátôn}) e a que os
transporta (\emph{exallagês}) das normas habituais sob a influência da
divindade (\emph{hypò theías}).

 

[265b] F: Exato.

 

S: No que diz respeito aos deuses, são quatro as divindades e quatro as
partes pelas quais foram divididas. A~Apolo atribui"-se a inspiração da
adivinhação (\emph{mantikḕn}), a Dioniso as iniciações
(\emph{telestikḗn}), às Musas a poética (\emph{poêtikḗn}), e a loucura
amorosa (\emph{erôtikḕn manían}), a quarta, que dizemos ser a melhor
(\emph{arístên}), atribui"-se à Afrodite e ao Amor. E~não sei como, mas
ao apresentar (\emph{apeikádzontes}) a afecção amorosa (\emph{erôtikòn
páthos}), talvez por atingirmos algo verdadeiro (\emph{alethoûs tinos}),
talvez por chegarmos a outros lugares, forjamos um discurso não
totalmente isento de força persuasiva (\emph{apíthanon}) e celebramos
com uma espécie de hino, em algo mítico, [265c] bem medido e
respeitoso (\emph{metríôs te kaì euphḗmôs}) em nome desse meu e teu
senhor, o Amor, ó Fedro, o guardião dos belos jovens.

 

F: E a audição não me desagradou.

 

S: Captemos a partir disso como o discurso muda (\emph{metabênai}) do
vitupério (\emph{pségein}) ao elogio (\emph{epaineîn}).

 

F: Como dizes?

 

S: Parece"-me tudo isso ser uma brincadeira de criança, mas nessas
afirmações proferidas ao acaso (\emph{ek týchês}) há dois aspectos, e
não será desagradável se deles pudermos captar algo ligado à arte
(\emph{téchnêi}).

 

[265d] F: Quais deles?

 

S: Levar a uma só ideia, a uma visão de conjunto (\emph{sunorônta}), as
muitas coisas que estão dispersas (\emph{diesparména}), para que se
possa tornar evidente (\emph{dêlon}), pela definição
(\emph{horizómenos}), cada tema, sempre que pretendemos ensinar
(\emph{didáskein}), como agora mesmo foi feito com o Amor -- que foi
definido \mbox{---,} quer tenha sido bem ou mal definido, e que proporcionou, ao
mencionarmos o discurso, certa clareza (\emph{saphès}) e concordância
(\emph{homologoúmenon}) consigo mesmo.

 

F: E o outro aspecto (\emph{eîdos}) de que falas, ó Sócrates?

 

[265e] S: Poder novamente (\emph{pálin}) discernir
(\emph{diatémnein}) em espécies (\emph{eídê})\emph{,}~segundo as
articulações naturais, procurando não causar roturas em nenhuma parte,
ao modo do cozinheiro inexperiente. Mas que sirvam de exemplo os dois
discursos anteriores, que reuniram a insanidade do pensamento
(\emph{áphron tês dianoías}) a uma ideia comum
(\emph{koinêi}~\emph{eidos}). Tal como de um só corpo nascem membros
duplos e homônimos (\emph{diplâ kaì homṓnuma}), chamados sinistros e
destros, [266a] assim também o discurso nos apresentou uma ideia do
desvio do intelecto (\emph{paranóias}). Um deles, discernindo
(\emph{temnómenos}) a sua parte esquerda, não cessou de novamente
(\emph{pálin}) discerní"-la (\emph{témnôn}) enquanto não descobriu
(\emph{epheurṑn}) aí uma espécie de amor denominado sinistro, a quem com
toda a razão encheu de censuras (\emph{eloidórêsen}), e o outro nos
levou para a parte destra da loucura (\emph{manías})\emph{,}~homônima
àquela (\emph{homônumon mèn ekeínôi}), mas divina (\emph{theîon}) foi
essa parte descoberta (\emph{epheurṑn}) do amor, apresentando"-a diante
dos nossos olhos e cantando"-lhe elogios (\emph{epḗinesen}), como sendo a
causa dos nossos maiores bens (\emph{agathôn}).

 

[266b] F: É verdade o que dizes.

 

S: Eu mesmo sou um amante (\emph{erastḗs}), ó Fedro, dessas divisões e
sínteses (\emph{diairéseôn kaì sunagôgôn}), meio pelo qual é possível
falar e pensar (\emph{légein te kaì phroneîn}). Se considero qualquer
outra pessoa capaz de observar a natureza do uno e do múltiplo, este eu
persigo, ``seguindo seus passos como os de um deus''. Os que são capazes
disso, quer tenha eu os designado bem ou não, deus o sabe, até agora os
referi como dialéticos (\emph{dialektikoús}). [266c] Mas aos que
aprendem (\emph{mathóntas}) junto a ti e a Lísias, como é necessário que
os designemos? Ou essa não é aquela arte discursiva (\emph{lógôn
téchnê}), segundo a qual Trasímaco e outros sábios manejavam o falar,
proporcionando que outros assim também o fizessem, aqueles que queriam
presenteá"-los como se fossem reis (\emph{basileûsin})?

 

F: Há de fato uma realeza (\emph{basilikoì}) nesses homens, embora não
conheçam isso que perguntas. Parece"-me correta essa forma de dizer, que
esses são chamados de dialéticos (\emph{dialektikòn}), mas parece"-me,
todavia, que a retórica (\emph{rhêtorikòn}) ainda nos escapa
(\emph{diapheúgein}).

 

[266d] S: Como dizes? Onde poderia existir algo belo que, mesmo
afastado dessas mesmas características, fosse adquirido como uma arte
(\emph{téchnêi})? Em todo caso é preciso que não a desprezemos, tu e eu,
mas que falemos o que ficou de lado sobre a retórica
(\emph{rhêtorikês}).

 

F: E é bastante vasto, ó Sócrates, aquilo que foi escrito nos livros
acerca da arte discursiva (\emph{en toîs biblíois toîs perì lógôn
téchnês gegramménois}).

 

S: Bem me recordaste (\emph{hupémnêsas}) disto. Segundo creio,
primeiramente é necessário proferir no início dos discursos o
``proêmio''. É~a isso que te referes ou não? A esses refinamentos da
arte?

 

[266e] F: Sim.

 

S: Em segundo lugar vem a ``narração'' (\emph{diḗgesín}) e alguns
``testemunhos'' (\emph{marturías})\emph{~}que lhes dizem respeito, em
terceiro lugar a ``prova'' (\emph{tekmḗria}) e em quarto as
``verossimilhanças'' (\emph{eikóta})\emph{.~}Também existe, segundo
creio, a ``confirmação'' (\emph{pístôsin}) e a ``confirmação
suplementar'' (\emph{epipístôsin}), nos dizeres do excelente burilador
de discursos (\emph{logodaídalon}), o homem de Bizâncio.

 

F: Mencionas o auspicioso Teodoro?

 

[267a] S: Quem mais se não ele? O qual disse haver nas composições
uma ``refutação'' (\emph{elegchón}) e uma ``refutação suplementar''
(\emph{epexélegchon}), tanto na acusação como na defesa
(\emph{katêgoríai te kaì apologíai}). E~o belíssimo Eveno de Paros, não
o traremos para o debate? Ele foi quem primeiro inventou (\emph{hêuren})
as ``insinuações'' (\emph{hypodḗlosin})\emph{~}e os ``para"-elogios''
(\emph{parepaínous})\emph{.~}Dizem que ele compôs ``para"-vitupérios''
(\emph{parapsógous}) em versos, para auxiliar a memória (\emph{mnḗmes
chárin}). Foi, portanto, um homem sábio (\emph{sophòs gàr anḗr}).~E
Tísias~e Górgias, vamos deixá"-los dormindo, eles que souberam honrar
mais a verossimilhança do que a verdade, que pela força discursiva
fizeram o grande parecer pequeno e o pequeno parecer grande, [267b]
o novo parecer arcaico, bem como o seu contrário, o arcaico parecer
novo, e que acerca de todos os assuntos encontraram (\emph{anêuron}) a
concisão discursiva e seu prolongamento
indefinido?\textsuperscript{~}Ouvindo isso de mim, outrora, Pródico
sorriu e disse que somente ele havia descoberto (\emph{heurêkénai}) o
que é preciso na arte discursiva, discursos que não sejam nem longos nem
curtos, mas na medida (\emph{metríôn}).

 

F: Ó Pródico, sapientíssimo!

 

S: Não falamos ainda de Hípias?~Creio eu que o estrangeiro de Élis
votaria também com Pródico.

 

F: E por que não?

 

S: E o que diremos do~\emph{Museu Discursivo}~de Polo? [267c] Com
sua duplicação discursiva (\emph{diplasiologían})\emph{,~}coleção de
máximas (\emph{gnômologían}) e estilo imagético (\emph{eikonologian}).~E
do~\emph{Vocabulário}~que Licímnio~havia presenteado Polo, em vista da
composição do seu belo falar?

 

F: E de Protágoras, ó Sócrates, não há nada desse tipo?

 

S: A~\emph{Dicção Correta~}(\emph{Orthoépeia}) é uma delas, ó jovem,
entre muitas outras e belas composições. E~dos discursos piedosos
escritos sobre a velhice e a pobreza, o que me parece dominar pela arte
é o do grande Calcedônio, homem terrível que a muitos enfurecia e, em
seguida, novamente, conduzia pelos encantos da palavra (\emph{epáidôn
kêleîn}), a dissipar essa fúria, como ele mesmo dizia. [267d]
Fortíssimo ele era em gerar e destruir qualquer tipo de calúnia. Quanto
à conclusão dos discursos parecem estar todos em comum acordo, embora
alguns a chamem de peroração (\emph{epánodon}) e outros estabeleçam
outro nome.

 

F: Aludes à recordação (\emph{hypomnêsai}) de cada um dos pontos
capitais, no final do que foi dito aos ouvintes?

 

S: Falo disso mesmo, e se tens algo mais a dizer acerca da arte
discursiva…

 

F: Insignificâncias indignas de menção.

 

[268a] S: Deixemos de lado as insignificâncias e, à luz do sol,
vejamos melhor que potencialidade (\emph{dúnamin}) tem quem detém essa
arte.

 

F: Muita é sua força, ó Sócrates, sobretudo nas reuniões populares
(\emph{plḗthous}~\emph{sunódois}).

 

S: Tem mesmo, mas, ó divino, vê também se essa urdidura não te parece
frouxa, como a mim me parece.

 

F: Mostra"-me.

 

S: Diz"-me, se alguém chegasse a teu amigo Erixímaco ou a seu pai Acúmeno
afirmando o seguinte: ``Eu conheço aplicações para aquecer o corpo ou,
se desejar, resfriá"-lo, e se me parecer adequado fazê"-lo vomitar ou, ao
contrário, evacuar, além de muitos outros tantos efeitos semelhantes.
[268b] Tendo conhecimento disso, considero"-me um médico capaz de
fazer com que outros assim procedam, transmitindo tais saberes''. O~que
pensas que os seus ouvintes, nesse caso, diriam?

 

F: Que outra coisa, se não perguntar se ele sabe em quem e quando é
preciso aplicar cada um deles, e também em que quantidade?

 

S: Se então dissesse: ``De modo algum, mas considero que aquele que
junto a mim aprender essas coisas, poderá fazer o que perguntas''.

 

[268c] F: Poderiam dizer, creio eu, que esse homem estivesse louco,
pois só por ter colhido de algum livro~ou por calhar de conhecer alguns
fármacos, considera"-se um médico, sem nenhum conhecimento da arte.

 

S: E o que diria, se alguém chegasse junto a Sófocles e Eurípides,
dizendo saber compor grandes falas sobre temas insignificantes e sobre
temas grandiosos falas curtas, e quando quisesse falas piedosas ou, ao
contrário, terríveis e ameaçadoras, e tantas outras desse tipo? E~ainda
que, com tais ensinamentos, considerava"-se capaz de transmitir a própria
arte da composição de tragédias?

 

[268d] F: E esses também, ó Sócrates, creio que ririam, se alguém
considerasse ser a tragédia outra coisa se não a adequação
(\emph{prépousan}) dos elementos entre si e com o todo (\emph{tôi
holôi}) da composição.

 

S: Penso eu que tais censuras não seriam grosseiras, tal qual o músico
que encontra uma pessoa que se considera um harmonista, só porque lhe
ocorreu aprender a fazer soar uma corda aguda e grave. O~músico não
seria grosseiro dizendo: ``Ó desgraçado, melancólico'', mas, por ser
músico, gentilmente diria: ``Ó meu querido, para quem deseja vir a ser
um harmonista, esses conhecimentos são necessários, mas ninguém adentra
nem aprende o mínimo da harmonia, só por possuir essa tua disposição.
Conheces então os saberes prévios necessários à harmonia
(\emph{prò}~\emph{harmonías}), mas não a harmonia propriamente dita
(\emph{tà harmoniká})''.

 

F: Corretíssimo.

 

[269a] S: Então, Sófocles e Eurípides mostrariam que aquilo era
apenas um rudimento para a composição da tragédia (\emph{prò
tragoidías),~}mas não a arte trágica propriamente dita (\emph{tà
tragiká),~}bem como Acúmeno diria que apresentavam os saberes prévios
necessários à medicina (\emph{prò}~\emph{iatrikês}), mas não detinham a
medicina mesma (\emph{tà iatriká}).

 

F: Sem dúvida.

 

S: E o que pensaremos de ``Adrasto voz de mel'' ou de Péricles, se
ouvissem o que agora mesmo percorremos sobre todas as artes, acerca dos
discursos breves (\emph{brachylogiôn})\emph{,~}dos estilos imagéticos
(\emph{eikonologiôn})\emph{~}e tantos outros a que chegamos, os quais já
mencionamos e foram verificados à luz do dia? [269b] Qual deles
seria o mais cruel, assim como eu e tu, ao falar mal, pela própria
rusticidade (\emph{hup' agroikías}), desses escritores e professores da
arte retórica? Ou, sendo mais sábios que nós, eles nos reprovariam
dizendo: ``Ó Fedro e Sócrates, não é preciso ser odioso para com eles,
mas desculpá"-los, se alguns que não conheceram a dialética se tornaram
incapazes de definir o que é a retórica, e por essa mesma condição
consideraram ter descoberto a arte retórica, quando apenas eram
detentores de conhecimentos prévios necessários à arte. [269c] Por
ensinarem tais coisas a outros, eles se consideram perfeitos professores
de retórica, embora dizer em cada ocasião o persuasivo e arranjar o todo
no discurso não lhes pareça um trabalho especial, de modo que seus
aprendizes precisam por si próprios adquirir essas habilidades nos
discursos''.

 

F: Ó Sócrates, isso parece ser aquela arte retórica que os homens
ensinaram e sobre a qual escreveram, e me parece ser proclamada
verdadeira. Mas a arte da retórica propriamente dita e sua
credibilidade, [269d] como e a partir de onde seria possível
alcançá"-la?

 

S: Para ser capaz, ó Fedro, de se tornar um competidor
(\emph{agonistḕn}) perfeito é verossímil (\emph{eikós}) -- talvez
necessário \mbox{---,} que detenhas outras coisas. Se está na tua natureza
(\emph{phýsei}) ser um retórico, serás um rétor consumado, acrescendo a
isso conhecimento (\emph{epistḗmen})\emph{~}e exercício
(\emph{melétên}). Se deixares de lado qualquer um desses elementos,
serás imperfeito.~Tal é a arte, a qual não me parece evidente atingi"-la
através do método de Lísias e de Trasímaco.

 

F: E por que modo então?

 

S: É possível, ó querido, que Péricles tenha se tornado de modo
verossímil (\emph{eikótôs}) o maior perito na arte retórica.

 

F: E por quê?

 

S: Todas as grandes artes requerem a tagarelice e a meteorologia
(\emph{adoleschías}~\emph{kai}~\emph{meteôrologías phýseos péri}) acerca
da natureza, pois parece que é justamente delas que se pode adquirir
sublime pensamento (\emph{hypselónoun}) e perfeição. Péricles foi capaz
de adquiri"-las, além da sua inclinação natural (\emph{euphyḕs}), e me
parece que envolto com Anaxágoras, pleno daquela meteorologia, chegou à
natureza do intelecto e da sua ausência (\emph{noû te kaì anoías}),~tal
qual Anaxágoras apresentava em muitos dos seus discursos, e a partir
disso Péricles forjou utilidades para sua arte discursiva
(\emph{enteûthen~eílkysen~epì tèn tôn lógôn téchnên tò prósphoron
autêi}).

 

F: Como dizes?

 

S: [270b] É que o recurso técnico (\emph{trópos téchnês}) da
medicina é similar ao da retórica.

 

F: Como assim?

 

S: Em ambas é preciso dividir a natureza (\emph{phýsin}), numa delas a
do corpo, noutra, a da alma. Se pretendes, por um lado, fornecer
fármacos\emph{~}e alimento para a saúde e para a força, e, por outro,
discursos e estudos prescritos como úteis à persuasão desejada e à
virtude transmitida, isso não só aconteceria por treino e experiência,
mas por arte (\emph{mḕ~tribêi mónon kai empeiríai allà téchnêi}).

 

F: É verossímil (\emph{eikós}) que seja assim, ó Sócrates.

 

[270c] S: E tu consideras ser possível compreender o valor da
natureza da alma sem, contudo, compreender a natureza do todo (\emph{tês
toû hólou phýseôs})?

 

F: Se devemos acreditar em Hipócrates, que pertence aos Asclepíades,
quando diz que sem esse método nem mesmo o corpo seria possível
conhecer.

 

S: Belo dizer, ó companheiro, mas é necessário, todavia, examinarmos
isso em vista do discurso (\emph{tòn lógon}) de Hipócrates e
verificarmos se estamos de acordo.

 

F: É o que digo.

 

S: Verifica então agora o que diz Hipócrates e a verdadeira razão
(\emph{alêthḕs lógos}) acerca da natureza (\emph{perì phýseos}). Não é
preciso, para compreender a natureza de qualquer coisa, primeiramente
verificar se ela é simples ou de múltiplas formas, sobretudo se
desejaremos nós mesmos sermos os artífices (\emph{technikoì}) e capazes
de transmitir isso a outros, em seguida, se forem simples, verificar sua
potencialidade, saber em que tipo de relação sua natureza produz uma
determinada ação ou pelo que é afetada por algo externo. Se são
múltiplas as suas formas, estas devem ser também enumeradas e, tal qual
a unidade, devem ser observadas, cada uma delas, em que sua natureza
pode produzir ou ser afetada por algo externo?

 

F: É possível, ó Sócrates.

 

[270e] S: Sem esse método pareceríamos fazer uma travessia de cego
(\emph{typhloû poreíai}), mas não devemos comparar um cego ou um surdo
àquele que persegue com arte alguma coisa, como é evidente a quem
ofereça a arte discursiva e mostre, de forma acurada, a essência da
natureza para quem os discursos serão oferecidos. E~essa será, sem
dúvida, a alma.

 

F: Seguramente.

 

[271a] S: Então, o combate se estende por toda alma, pois a
persuasão (\emph{peithṑ}) é produzida nela, ou não?

 

F: Sim.

 

S: É evidente que Trasímaco e tantos outros dos que zelosamente nos
legaram tratados da arte retórica, inicialmente e com muito cuidado
inscreverão e produzirão uma visão\emph{~}da alma, em seguida observarão
se sua natureza é una e semelhante, ou se, como o corpo, tem múltiplas
formas. Dizemos que mostrar a natureza de algo é isso.

 

F: É assim mesmo.

 

S: Em segundo lugar, é preciso entender o que sua natureza pode produzir
e pelo que é afetada.

 

 

F: Seguramente.

 

[271b] S: Em terceiro lugar, é preciso realizar uma disposição entre
gêneros de discursos e almas, bem como todas as causas que as afetam,
harmonizar cada qual ao seu correspondente e ensinar por quais causas,
necessariamente, alguns são persuadidos por determinados discursos e
outros não são persuadidos.

 

F: Parece que essa é a mais bela maneira de agir.

 

S: Não haverá, então, ó querido, outra forma de demonstrar ou proferir
com arte, pela qual se dirá ou se escreverá algo, seja acerca desse ou
de outro tema. [271c] Os que hoje escrevem, como tu ouviste, são
hábeis em todas essas artes discursivas e as dissimulam, tendo muito
conhecimento acerca da alma. Antes de falarem e escreverem com tais
recursos, não nos deixemos persuadir de que eles escrevem com arte.

 

F: Quais são esses recursos?

 

S: Dizer com todas as letras não é fácil. Mas o modo pelo qual é preciso
escrever, se queres mesmo dominar a arte, tanto quanto se possa admitir,
acerca disso eu quero falar.

 

F: Diz então.

 

S: Uma vez que a potência do discurso está na condução das almas
(\emph{psychagôgía}),\emph{~}[271d] aquele que pretende ser rétor
deve necessariamente conhecer as formas (\emph{eíde}) que a alma tem. Há
tantas e tantos tipos de almas, motivo pelo qual existem umas de um modo
e outras de outro. Tendo assim realizado tal distinção, existem
discursos correspondentes para cada uma delas e cada uma tem o seu
específico. Por causa disso, umas [almas] são levadas à plena
persuasão (\emph{eupeitheîs}) pelo efeito do discurso, e outras, por
outro lado, pelo mesmo discurso, são levadas à desconfiança
(\emph{duspeitheîs}). É~preciso ter pensado (\emph{noḗsanta})
suficientemente nessas coisas, ter contemplado os próprios seres em ação
e ter praticado [271e], bem como ser capaz de acompanhar com agudeza
pela sensação tudo isso, ou não haverá plenitude dos saberes que outrora
ouviu nos discursos que trazia consigo. Quando for capaz de dizer como e
pelo que há persuasão, quando for capaz de, junto a alguém, perceber e
mostrar a si mesmo qual é a natureza [272a] acerca da qual versavam
os discursos de outrora, depois disso, junto a eles tendo trabalhado, é
preciso administrar os discursos pelos quais serão persuadidos. É~necessário ser detentor disso tudo, escolhendo também o momento oportuno
(\emph{kairós}) de falar e o de calar, os discursos breves
(\emph{brachylogias}),\emph{~}os discursos piedosos
(\emph{eleinologias}) e cada uma das formas dos discursos veementes
(\emph{deinṓseos})\emph{~}aprendidos, reconhecendo neles o momento
oportuno (\emph{eukairían}) e a falta de oportunidade (\emph{akairían}).
Bela e acabada estará, nesse ponto, a arte adquirida, antes disso, não.
[272b] Mas se alguém deixar de lado qualquer um desses elementos ao
falar, ensinar ou escrever, dizendo que o faz com arte, não terá força
persuasiva. Daí talvez dissesse o escritor: ``O quê? Ó Fedro e Sócrates,
parece"-vos mesmo assim? Não há então outro modo de conceber a arte dos
discursos?''

 

F: Impossível, ó Sócrates, que seja de outro modo, e não me parece uma
tarefa de pouca monta.

 

S: Dizes a verdade. E, em consequência disso, é necessário percorrer de
cima abaixo todos os discursos para verificar se há, em alguma parte, um
caminho mais fácil e curto, [272c] a fim de que este não seja vão e
muito árduo, mas, se possível, curto e suave. Se em alguma ocasião
tiveste o auxílio da audição de Lísias ou de algum outro, procura
lembrar"-te e diz.

 

F: Eu poderia até tentar, mas assim, agora, não posso.

 

S: Queres que eu te exponha um discurso que ouvi de alguns, acerca desse
tema?

 

F: Como não?

 

S: Dizem, ó Fedro, que é justo mencionar também a razão do lobo.

 

[272d] F: Então faça isso.

 

S: Dizem alguns que não é preciso honrar tanto essas coisas, nem se
elevar tanto a altos rodeios. Foi perfeito o que dissemos no início
dessa discussão: que aquele que pretende ser um rétor pleno, não precisa
participar da verdade (\emph{alêtheías}), nem da justiça
(\emph{dikaíôn}), nem da bondade (\emph{agathôn}) acerca daquilo que
trata, quer tenham os homens tais naturezas, quer as tenham adquirido
pela educação. Ocorre que nos tribunais ninguém se preocupa com a
verdade (\emph{alêtheías}), mas com o persuasivo (\emph{pithanoû}), isto
é, com o verossímil (\emph{eikós}), questão na qual é preciso aplicar"-se
quem pretenda falar com arte. [272e] Algumas vezes, nem mesmo os
fatos ocorridos devem ser mencionados, caso não sejam verossímeis
(\emph{eikótôs}), mas somente as coisas verossímeis (\emph{eikóta}),
seja na acusação, seja na defesa, bem como tudo aquilo que se diz de
forma verossímil (\emph{eikòs}) é necessário perseguir, motivo pelo qual
muitas vezes é preciso renunciar à verdade (\emph{alêtheî}). [273a]
O verossímil surge ao longo dos discursos e proporciona toda a arte
(\emph{téchnên}).

 

F: Detalhaste com propriedade, ó Sócrates, os dizeres daqueles que
professam serem detentores dessa arte dos discursos. Recordo"-me que, no
começo dessa discussão, tocamos rapidamente nesse tema, o qual parece
ser muito importante para aqueles que se ocupam disso.

 

S: Mas tu certamente tens degustado bem do seu Tísias. [273b] Pois
então ele que nos diga, também, se o verossímil (\emph{tò eikòs}) é
outra coisa se não a opinião da maioria (\emph{tôi plḗthei dokoûn}).

 

F: E o que mais seria?

 

S: Como parece, esse sábio artista aqui encontrado (\emph{heurṑn})
escreveu que se por acaso um homem fraco e corajoso assaltasse um
homem forte e covarde, levando sua toga e outros pertences, ao serem
ambos levados ao tribunal, seria necessário que nenhum deles dissesse a
verdade, uma vez que o covarde diria que não havia sido assaltado
unicamente pelo corajoso, o qual, por sua vez, diria que estava sozinho,
[273c] usando aquele argumento: ``Como eu, sendo assim fraco,
poderia ter executado o assalto contra ele que é forte?'' E o outro não
expressaria sua própria maldade, mas usaria alguma outra mentira para,
rapidamente, oferecer refutação ao oponente. E~acerca de outras tantas
coisas dessas, também ligadas à arte, não dizemos que é assim, ó Fedro?

 

F: Como não?

 

S: Ah! Parece"-me terrivelmente escondida essa arte encontrada
(\emph{aneureîn}) por Tísias ou por outro qualquer, bem como o nome pelo
qual ela pode ser designada. Mas, ó companheiro, qual deles, dizemos ou
não --.

 

[273d] F: O qual?

 

S: Ó Tísias, antes de ti, os mais antigos já haviam dito que o
verossímil surge para a multidão pela semelhança que ele tem com o
verdadeiro (\emph{to eikòs toîs polloîs dia homoiótêta toû alethoûs
tygchánei eggignómenon}). E~as semelhanças (\emph{homoiótêta}), como
dissemos há pouco, são encontradas (\emph{heurískein}) em toda parte e
da melhor maneira possível por aquele que conhece a verdade. Então, se
tens algo mais acerca da arte dos discursos, diz, pois te escutaremos,
caso contrário chegaremos ao mesmo ponto onde estávamos persuadidos
[273e]. Se alguém não enumera a natureza dos seus ouvintes, discerne
os seres segundo sua forma, não é capaz de levá"-los a uma só ideia,
abarcando cada uma delas, e não será um artista do discurso, tanto
quanto isso é possível ao homem. Essas coisas não se podem adquirir sem
muito empenho (\emph{pragmateías}). E~não é por causa do falar e do agir
com outros que o homem moderado precisa cultivar sua prudência
(\emph{sṓphrona}), mas para poder agradar aos deuses, ao falar, ao agir
e em tudo que seja possível. Não é assim, ó Tísias? Dizem"-nos os mais
sábios que é necessário seguir e agradar o que possui intelecto
(\emph{tòn noûn échonta}), [274a] porque esse não é simples
acessório, mas senhor bondoso em tudo aquilo que é bom. Se essa estrada
é longa, não te espantes, pois é pelas grandes coisas que se faz esse
percurso, ao contrário do que tu supões. Como o discurso já afirmou, se
for algo que desejas, o que surge desse trajeto será belíssimo.

 

F: Perece"-me que afirmas algo magnífico, ó Sócrates, se for assim mesmo.

 

S: É belo ocupar"-se das coisas belas, bem como suportar aquilo que lhes
advém.

 

F: É certo.

 

[274b] S: Então, acerca da arte e da sua ausência (\emph{téchnês te
kai atechnías}) nos discursos, dissemos o suficiente.

 

F: E o que mais haveria?

 

S: E acerca da conveniência ou inconveniência da escrita, como ela pode
ser bela ou inconveniente, omitimos isso?

 

F: Sim.

 

S: Conheces, pois, o meio pelo qual em matéria de discursos, devemos
agradar aos deuses, agindo ou falando?

 

F: Não conheço, e tu?

 

[274c] S: Escuta o que posso te contar dos antigos, pois eles
conheciam a verdade. Se nós a descobríssemos (\emph{heuroimen}), não nos
preocuparíamos com a opinião dos homens, não é mesmo?

 

F: Engraçada tua pergunta, mas conta o que ouviste.

 

S: Escutei que perto de Náucratis, no Egito, existia um dos deuses
antigos, cujo pássaro sagrado era chamado Íbis, o nome
desse~\emph{daímon}~era Theuth. Ele foi o primeiro a inventar
(\emph{heureîn}) os números, o cálculo, a geometria, a astronomia, o
jogo de tabuleiro, o jogo de dados e especialmente a escrita
(\emph{grámmata}).~[274d] O rei de todo o Egito, nessa época, era
Thamous, que vivia na grande cidade alta, a qual os gregos chamavam de
Tebas egípcia e onde o mesmo Thamous Amon era o deus.~Theuth veio junto
ao rei para mostrar"-lhe suas artes, que, segundo ele, deveriam ser
presenteadas a todos os egípcios. Indagado, então, acerca das utilidades
(\emph{ôphelían}) de cada uma delas, ele as expunha, de modo que o rei
dizia o que parecia, aos seus olhos, ser belo ou feio em cada uma, ora
elogiando, ora vituperando.~[274e] Muitas foram as artes para as
quais Thamous apresentou seu comentário a Theuth, e todo o seu discurso
seria muito longo para referi"-lo aqui, mas acerca da escrita, foi assim:
``Ó rei, disse Theuth, esse conhecimento tornará os egípcios mais sábios
e com maior disposição para a memória. Foi inventado (\emph{heuréthê})
então o fármaco da memória e da sabedoria``. Ao que o rei replicou: ''Ó
artificiosíssimo Theuth, enquanto uns são capazes de criar uma arte,
outros são capazes de julgá"-la, especialmente em que aspectos elas serão
nocivas ou úteis para quem poderá usá"-las. [275a] Agora, aqui, tu,
como pai da escrita que és, por tua benevolência para com ela, dizes o
contrário do que ela é capaz. Ela produz esquecimento nas almas daqueles
que aprendem pela falta de cuidado com a memória, sendo então por
escritos externos e alheios que adquirem a crença
(\emph{pístis})\emph{,~}não adquirindo mais a reminiscência internamente
e por si mesmos. Portanto, não inventaste (\emph{hêures}) o fármaco da
memória (\emph{mnḗmês})\emph{,~}mas o da recordação
(\emph{hypomnḗseos}). Ela oferece uma aparente sabedoria aos discípulos,
que não alcançam a verdade propriamente dita. [275b] Muitos dos teus
ouvintes, sem aprendizado, parecem conhecedores de muitas coisas, quando
na verdade são geralmente ignorantes e difíceis no trato, tornando"-se
sábios apenas aparentemente(\emph{doxósophoi}), em vez de sábios
(\emph{sophôn}) de fato''.

 

F: Ó Sócrates, que facilidade tens para apresentar histórias egípcias e
de qualquer lugar que queiras.

 

S: Ó amigo, dizem que os antigos discursos divinatórios provinham de um
carvalho situado no templo de Zeus em Dodona. Naquele tempo os homens
não eram tão sábios quanto vós, os jovens, motivo pelo qual lhes
bastava, devido à sua simplicidade, ouvir um carvalho ou uma pedra,
desde que estes lhes dissessem somente a verdade. [275c] Tu talvez
possas discernir qual é o discurso e de onde ele provém. E~não observes
somente se é assim ou não.

F: Correta é a tua repreensão e me parece que, acerca dos escritos,
ocorre o que tebano já havia afirmado.

 

S: Tanto aquele que supõe deixar alguma arte por meio da escrita
(\emph{en grámmasi}), quanto aquele que espera recebê"-la por esse meio,
ambos consideram que a escrita (\emph{grammátôn}) porta algo de claro e
seguro, o que é muita ingenuidade e prova de desconhecimento
(\emph{agnooî}) do oráculo de Amon, [275d] segundo o qual os
discursos escritos (\emph{lógos gegramménous}) nada mais são do que um
meio de recordar (\emph{hypomnêsai}) aquele que já conhece
(\emph{eidóta}) os assuntos tratados nos escritos (\emph{gegramména}).

 

F: Corretíssimo.

 

S: É terrível mesmo, ó Fedro, essa escrita (\emph{graphḕ}) e como tem
verdadeira semelhança (\emph{hómoion}) com a pintura
(\emph{zôgraphíai}). Os frutos desta são estabelecidos como vivos, mas
se lhe perguntas algo, ela permanece sempre num silêncio sagrado
(\emph{semnôs pánu sigâi}), e assim também acontece com os discursos
(\emph{oi lógoi}). Eles parecem dizer algo de sensato, mas, se alguém
que deseja aprender lhes pergunta algo sobre o que foi dito, eles só
significam a mesma coisa sempre. [275e] E a grafia (\emph{graphêi})
roda por todo lado conservando o mesmo discurso, seja para os que a
elogiam, seja para os que nela não têm nenhum interesse, pois ela não
conhece o momento de falar ou de calar. E~se ela for atacada num
tribunal, sempre haverá a necessidade de que o seu pai a socorra
(\emph{boêthoû}) das injúrias, pois ela não é capaz de defender ou
socorrer (\emph{amúnasthai oute boêthêsai}) a si mesma.

 

F: Também isso que dizes é corretíssimo.

 

[276a] S: O quê? Dizemos que há outro discurso, irmão legítimo
deste, mas surgido por outro modo, melhor quanto à natureza e mais
poderoso?

 

F: Sobre qual discurso te referes e como ele surge?

 

S: Sobre aquele que é inscrito na alma (\emph{gráphetai en têi psychêi})
daquele que aprende, segundo o conhecimento (\emph{met'epistḗme}), ele é
capaz de defender (\emph{amûnai}) a si mesmo, conhecedor da ocasião
frente a qual é preciso falar (\emph{légein}) ou calar (\emph{sigân}).

 

F: O discurso de quem efetivamente sabe, ao qual te referes, é vivo e
animado (\emph{zônta kaì empsuchon}), de modo que o discurso escrito
(\emph{gegramménos}), poderíamos dizer com justiça, é um ídolo
(\emph{eídôlon}) seu.

 

[276b] S: É assim mesmo, agora me diz, quanto ao agricultor
(\emph{geôrgós}) que tem inteligência (\emph{noûn}) e deseja cuidar das
suas sementes para que frutifiquem, o que ele faria? Haveria de
lançá"-las, durante o verão, no jardim de Adônis (\emph{Adónidos
kêpous}), para homenagear a sua festa, para que floresçam em oito dias?
Ou isso ele poderia fazer só por brincadeira (\emph{paidiâs}) e
exclusivamente de bom grado para o festival, quando muito. Ou, quanto às
sementes que ele realmente despende atenção, valendo"-se da arte da
agricultura (\emph{georgikêi}), ele semearia em local adequado,
felicitando"-se em oito meses, quando as sementes atingem sua maturidade?

 

[276c] F: Ó Sócrates, num caso ele faria com atenção, no outro, não,
como tu dizes.

 

S: O que dizemos daquele que tem conhecimento do justo, do belo e do
bom? Que ele tem menor inteligência (\emph{noûn}) que a do agricultor,
com relação às suas sementes?

 

F: De modo algum.

 

S: Então não vai cuidadosamente escrevê"-las na água escura com uma pena,
compondo discursos incapazes de socorrerem"-se (\emph{boêtheîn}) a si
mesmos, insuficientes para ensinar a verdade (\emph{adunátôn dè ikanôs
talêthḕs didáxai}).

 

F: Não é mesmo verossímil (\emph{eikós}).

 

[276d] S: Não? Mas nos jardins da escritura (\emph{en grámmasi
kḗpous}), como parece (\emph{eoike}), semeiam e escrevem pelo deleite da
brincadeira (\emph{paidiâs chárin spereî te kaì grápsei}), e quando
escrevem entesouram recordações (\emph{hypomnḗmata}) de si mesmos, para
o ``oblívio da velhice'' (\emph{lḗthes}~\emph{gêras}), se ela
``chegar''. E~todos que buscam seguir seus passos~serão agraciados pela
contemplação dessas delicadas plantas. Por outro lado, quando outros se
valem de outras diversões, bebendo nos simpósios, entregues a prazeres
similares a este, e, como parece, divertir"-se-ão exatamente com as
coisas referidas.

 

[276e] F: Boa diversão frente àquela frívola, ó Sócrates, essa de
poder brincar (\emph{paídzein}) com os discursos, sejam eles judiciais
ou outros que dizes nos quais possamos também narrar
(\emph{mythologoûnta})

 

S: É assim, ó querido Fedro, considero muito mais belo o empenho daquele
que, pela arte da dialética, toma uma alma para plantar e nela semear
discursos com conhecimento (\emph{met'epistḗmês}), [277a] aqueles
que são capazes de socorrer (\emph{boêtheîn}) quem os plantou. Então, os
discursos não são infrutíferos, mas têm sementes, pelas quais outros em
outros lugares se habituarão a crescer, tornando"-as sempre imortais o
bastante, tornando felizes os homens, tanto quanto possível.

 

F: Muito mais belo é o que dizes agora.

 

S: Agora que chegamos a esse acordo, ó Fedro, somos então capazes de
julgar (\emph{krínein}).

 

F: Julgar o quê?

 

S: Aquilo que queríamos saber e que nos trouxe até aqui, justamente para
que pudéssemos examinar a censura endereçada a Lísias pelos seus
discursos escritos (\emph{tês tôn lógôn graphês péri}), [277b] e
para examinarmos os próprios discursos, se haviam sido escritos com arte
(\emph{téchnêi}) ou sem arte (\emph{aneu~techné}). Os que estão de
acordo com a arte (\emph{éntechnon}) parecem"-me terem sido expostos de
modo bem medido (\emph{metríôs})\emph{.}

 

F: Parece mesmo. Mas recorda"-me (\emph{hypómnêson}) novamente
(\emph{palin}) como.

 

S: Antes, devemos saber a verdade acerca de cada coisa sobre o que se
fala e escreve, tudo deve poder ser definido por si mesmo, e uma vez
definido, devemos conhecer como dividi"-lo (\emph{témnein}) novamente
(\emph{pálin}) até a forma indivisível (\emph{toû atmétou}). E~a
respeito da natureza da alma, que se distinga tudo da mesma forma,
[277c] descobrindo (\emph{aneurískôn}) a forma natural que se
harmoniza com cada uma delas, para então estabelecer e ordenar o
discurso. Um discurso variegado é oferecido para uma alma variegada, um
simples para uma alma simples, antes disso não é possível haver um
gênero discursivo que faça uso natural da arte, nem para ensinar nem
para persuadir, como nos foi revelado pelo discurso anterior.

 

F: É tudo mesmo dessa forma, tal qual nos pareceu.

 

S: E a respeito do falar e do escrever discursos ser algo belo ou
vergonhoso, e de quando é possível dizer, com justiça, o que é
vergonhoso ou não. O~que há pouco foi dito não ficou bem claro?

 

F: O quê?

 

S: Que Lísias ou qualquer outro que tenha escrito ou venha a escrever
leis particulares ou públicas, quando consideram o tratado escrito sobre
política algo grandioso, estável e claro, é nesse momento que eles podem
se envergonhar dos discursos, quer isso seja mencionado ou não. O~fato
de alguém ignorar, sob o efeito do sono, [277e] o justo e o injusto,
o mau e o bom, não pode livrá"-lo da verdade de ser censurado, ainda que
toda a turba o elogie.

 

F: Não mesmo.

 

S: É necessário que haja muito divertimento (\emph{paidián}) em cada um
desses discursos escritos, e que nenhum deles, em metro ou sem, mereça
grande esforço para ser escrito, ou mesmo lido, como fazem os rapsodos,
sem preparo ou didática naquilo que é dito para persuadir. [278a] Os
melhores entre eles são os que, pela recordação (\emph{hypómnêsin}),
levam ao saber. Por outro lado, os que são feitos para ensinar,
discursos que agradam ao aluno e inscrevem na alma (\emph{graphoménois
en psychêi}) algo acerca do justo, do belo e do bom, somente estes são
visíveis, acabados e dignos de esforço. É~preciso que tais discursos
sejam enunciados como filhos legítimos, [278b] primeiro por eles
mesmos, se eles os descobrirem (\emph{heuretheìs}) em si, e, em seguida,
se alguns desses seus descendentes e irmãos plantam concomitantemente em
outras almas, em outros lugares, de acordo com a dignidade. Quanto a
outros discursos, é melhor afastar"-nos deles, ó Fedro, pois essa é a
atitude do homem que ambos, eu e tu, gostaríamos de ser.

 

F: Quanto a mim, desejo e faço votos para que seja assim, tudo da
maneira que dizes.

 

S: Então nós já nos divertimos (\emph{pepaísthô}) o bastante
(\emph{metríôs}) acerca dos discursos, e tu vai até Lísias e diz a ele
que nós dois descemos até a fonte das ninfas e ao santuário das Musas e
que escutamos um discurso [278c] para ser enviado a Lísias e para
qualquer outro que componha discursos, a Homero e a qualquer outro que
tenha composto poesia com ou sem acompanhamento musical (\emph{ôidêi}),
e em terceiro lugar a Sólon e aos que escreveram discursos políticos,
tratados que foram chamados de leis escritas: ``Se conheces a verdade
daquilo que está composto nesse escrito e és capaz de socorrê"-lo
(\emph{boêtheîn}), nas refutações que lhes são endereçadas, e ainda és
capaz de mostrar o que é ineficiente no teu próprio escrito, então, na
verdade, pelo qual epônimo deverá ser designado, por esta atividade de
escrever ou por aquela atividade a qual se dedicou?''

 

F: Qual dos epônimos tu atribuis a ele?

 

S: O de sábio, ó Fedro, acredito parecer demasiado, conveniente somente
a um deus. O~de filósofo ou outro desse tipo poderia ser mais ajustado e
adequado.

 

F: E de nenhum modo inapropriado.

 

S: Aquele que não tem, por outro lado, nada de mais honrado
(\emph{timiṓtera}) do que aquilo que outrora escreveu e passa o tempo a
percorrer (\emph{stréphôn}) seus escritos de cima abaixo, separando
trechos e trocando"-os de lugar, [278e] é com justiça que o
designarás por poeta, compositor de discursos ou escritor de leis?

 

F: É certo.

 

S: E é isso mesmo que deves dizer ao teu companheiro.

 

F: E tu? Como farás? Não deves pôr de lado o teu companheiro.

 

S: Qual deles?

 

F: O belo Isócrates.~O que dirás a ele, ó Sócrates, e nós diremos o quê?

 

S: Isócrates é jovem ainda, ó Fedro, [279a] entretanto adivinho algo
sobre ele e quero dizer.

 

F: O quê?

 

S: Parece"-me que ele é superior a Lísias quanto à natureza de seus
discursos, e ainda temperado por um caráter mais nobre (\emph{ḗthei
gennikôtéroi}), de modo que não seria espantoso se, com a idade, ele
superasse nessa prática os que hoje em dia se ocupam disso, tornando
infantis os que sempre se ocuparam de discursos. Se isso ainda não for
suficiente, ele será guiado por um impulso maior e mais divino. Pois há,
ó querido, certa filosofia no intelecto desse homem. [279b] É isso,
então, que eu vou, junto aos deuses, anunciar a Isócrates, o meu
favorito, e tu, por sua parte, faça o mesmo ao teu Lísias.

 

F: Assim será, partamos agora que o calor se tornou ameno.

 

S: Não é adequado fazermos uma prece antes de partir?

 

F: Sim, é.

 

S: Ó querido Pã e outros deuses, concedam"-me uma beleza interior. Que
tudo que há fora de mim possa ser amigo do que está no meu interior
[279c]~e que eu considere rico o sábio. Quanto à quantidade de ouro,
que eu possua tanto quanto o homem prudente seja capaz de levar e
trazer. Precisamos de algo mais, ó Fedro? Pois me parece bem medida
(\emph{metríôs}) a prece.

 

F: E eu partilho dessa súplica, pois tudo é comum entre amigos.

 

S: Partamos.


%\theendnotes	
