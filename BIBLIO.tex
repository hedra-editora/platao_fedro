\chapter{Bibliografia}

\section{Edições, traduções e comentários específicos dedicados
ao~\emph{Fedro}}

 

\begin{bibliohedra}
\tit{Brisson, L.}~\emph{Phèdre}. Paris: Flammarion, 1989.
\tit{Burnet, J.}~\emph{Platonis Opera}. Oxford: Oxford Classical
  Texts, 1901, tomo \versal{II}.
\tit{Chambry, E.}~\emph{Phèdre}. Paris: Flammarion, 1992.
\tit{De Vries, G.J.}~\emph{A~Commentary on the Phaedrus of Plato}.
  Amsterdam: Adolf \versal{M.}~Hakkert"-Publisher, 1969.
\tit{Diano, C.}~\emph{Dialoghi}. Bari: Laterza, 1934 (vol. \versal{III}).
\tit{Fernandez, L.G.}~\emph{Fedro}. Madrid: Alianza, 2004.
\tit{Ferreira, J.R.}~\emph{Fedro}. Lisboa: Verbo, 1973.
\tit{Fowler, H.N.}~\emph{Plato, Euthyphro, Apology, Crito, Phaedo, Phaedrus}. Cambridge (\versal{EUA}): Loeb
  Classical Library, 1914.
\tit{Gil, L.F.}~\emph{Fedro. Traducción, notas y studio preliminar}.
  Madrid: Instituto de Estudios Políticos, 1957.
\tit{Hackforth, R.}~\emph{Plato's Phaedrus. Translated with introduction
  and commentary}. Cambridge: Cambridge University Press, 1952.
\tit{Hermias}~Alexandrinus.~\emph{In Platonis Phaedrum Scholia, ad fidem
  codicis parisini 1810 denuo collati edidit et aparatu critic ornavit}.
  Ed. Paul Couvreur. Paris: Émile Bouillon, 1901.
\tit{Hermias}~Alexandrinus.~\emph{In Platonis Phaedrum Scholia}. Eds.~Carlo
  M.~Lucarini \& Claudio Moreschini. Berlin: De Gruyer,
  2012.
\tit{Lledó, E.}~\emph{Fedro}. Madrid: Gredos, 2008.
\tit{Nunes, C.A.}~\emph{Fedro. Cartas. Primeiro Alcebíades}.
  Belém: \versal{UFPA}, 1975.
\tit{Poratti, A.}~\emph{Fedro}. España: Akal,
  2010.
\tit{Pucci, P.}~\emph{Fedro}. Bari: Laterza, 1998.
\tit{Robin, L.}~\emph{Platon, Oeuvres completes. Phèdre}.
  Paris: Les Belles Letres, 1933, tomo \versal{IV}.
\tit{Rowe, C.J.}~\emph{Plato: Phaedrus}. Warminster: Aris \& Phillips Ltd., 1986.
\tit{Ryan, P.}~\emph{Plato's Phaedrus, a commentary for greek readers}.
  Norman: University of Oklahoma Press, 2012.
\tit{Santa"-Cruz M.I.}~\& \versal{Crespo, M.I.}~\emph{Fedro}. España: Losada, 2007.
\tit{Souza, J.C.}~de. O~amor alado, Platão, Fedro, 243e-257b.
  \emph{Letras Clássicas}, ano 2, n. 2, São Paulo: Humanitas, 1998.
\tit{Thompson, W.H.}~\emph{The Phaedrus of Plato}. London: Wittaker and Co., 1868 [1973].
\tit{Vicaire, P.}~\& Moreschini, \versal{C.}~\emph{Phèdre}. Paris: Les Belles Letres,
  1998.
\tit{Vicaire, P.}~\emph{Phèdre}. Paris: Les Belles Letres, 1991.
\tit{Yunis, H.}~\emph{Phaedrus}. Cambridge:
  Cambridge University Press, 2011.
\end{bibliohedra}

 

\section{Autores antigos}

 

\begin{bibliohedra}
\tit{Anacreon}. Trad. Bruno Gentili. Roma: Athenaei, 1958.
\tit{Antiphon}. \emph{Minor Attic Orators}.
  Andocides. Trad. K.J.~Maidment. Harvard University Press, 1941 (vol. \versal{I}: Antiphon).
\tit{Apulée}.~\emph{Les Métamorphoses}. Trad. P. Vallete. Paris: Les Belles
  Lettres, 1972.
\tit{Apuleio}, Lúcio.~\emph{O~asno de Ouro}. Trad. Francisco Antônio de
  Campos. Mem-de-Sá: Europa"-América, 1990.
\tit{Aristophanes}.~\emph{Clouds. Wasps. Peace}. Trad. J. Henderson. Cambridge: Harvard
  University Press, 1998.
\tit{Aristotele}.~\emph{Etica Nicomachea}.~Trad. C. Natali. Bari: Laterza,
  1999.
\tit{Aristotele}.~\emph{Politica}. Trad. C.A.~Viano. Milano: Bur, 2008.
\tit{Aristóteles}.~\emph{Física~\versal{I"-II}}. Trad. L. Angioni. Campinas: Unicamp,
  2002
\tit{Aristotelis}.~\emph{Physica}. Ed. W.D.~Ross.~Oxford: Clarendon Press,
  1950, Repr. 1966.
\tit{Aristotle}.~\emph{On Rhetoric}. Trad. G.A.~Kennedy. Oxford: Oxford
  University Press, 1991.
\tit{Aristotle}.~\emph{Posterior Analytics. Topica}. Trad. Hugh Tredennick
  \& E.~S.~Forster. Cambridge: Harvard University Press, 1960.
\tit{Aristotle}.~\emph{The Art of rhetoric}. Trad.~J.H.~Freese. Michigan:
  2006.
\tit{\emph{Comicorum Atticorum Fragmenta}}. Ed.~T.~Kock.~Leipzig: Teubner, 1880 (vol. \versal{I}).
\tit{Élio~Aristides}.~\emph{Pròs Platona perì rethorikês~}. Ed. W.~Dindorf.~Leipzig: Reimer, 1829 [1964].
\tit{Eurípides}.~\emph{Bacantes}. Trad. J.A.A.~Torrano. São Paulo: Hucitec,
  São Paulo: Iluminuras, 1995.
\tit{Hipócrates}.~\emph{Aforismos}. Trad. J.M.~de Rezende. São Paulo:
  Unifesp, 2010.
\tit{Hipócrates}.~Da natureza do homem. Trad. Henrique Cairus.
  ~\emph{História, Ciências, Saúde}, vol.~9, n. 2, 1999.
\tit{Hipócrates}.~\emph{Sobre o riso e a Loucura}. Trad. R.G.~de~Campos.
  São Paulo: Hedra, 2011.
\tit{Hipócrates}.~\emph{Tratados hipocráticos}. Trad. Nava, Gual, Férez y
  Alvarez. Barcelona: Gredos, 2000.
\tit{Isocrates}.~\emph{Opera Omnia}. Ed.~B.G.~Mandilaras. München:
  Monachii et Lipsiae, 2003 (vols.~\versal{I \& II}).
\tit{Isocrates}.~\emph{Oratory of Classical Greece}. Trad. David C.~Mirhady \&
  Yun Lee Too. Texas: University of Texas Press, 2013 (vol. \versal{I}).
\tit{Isocrates}.~\emph{Isocrates}. Trad. G.~Norlin. Harvard University Press, 1928 (vol. \versal{I}).
\tit{Isocrates \& Too}, Yun Lee.~\emph{A commentary on Isocrates' Antidosis}. New York:
  Oxford University Press, 2008.
\tit{Laertius, D.}~\emph{Lives of Eminent Philosophers}. Trad. R.D.~Hicks. Cambridge: Harvard University Press, 1925 (2 vols.).
\tit{\emph{Luciani}~Samosatensis~Opera}. Ed. Jacobitz \& Karl Gottfried.
  Leipzig: Teubneri, 1913 (vol.~\versal{II}).
\tit{Lysias}. Trad. W.R.M.~Lamb. Cambridge: Harvard University Press, 2006.
\tit{Menander}~(Rhetor).~\emph{Division of Epideictic Speeches}. Ed. D.A.~Russel \& N.G. Wilson.
  Oxford: Oxford University Press, 1981.
\tit{Omero}.~\emph{Iliade}. Trad. R.C.~Onesti. Torino: Einaudi, 1990.
\tit{Omero}.~\emph{Odissea}. Trad. R.C.~Onesti. Torino: Einaudi, 1989.
\tit{Pausanias}.~\emph{Description of Greece}. Leipzig: Teubner, 1967.
\tit{Photius}.~\emph{Bibliothèque}. Ed. R.~Henry. Paris: Les Belles
  Lettres, 1977.
\tit{Pindar}.~\emph{Nemean Odes. Isthmian Odes. Fragments}. Trad. William H.~Race. Cambridge: Harvard University Press, 1997.
\tit{Platão}.~\emph{Apologia de Sócrates. Eutifron.
  Críton}. Trad. André Malta. Porto Alegre: \versal{L\&PM}, 2014.
\tit{Platão}.~\emph{Filebo}. Trad. F.~Muniz. Rio de Janeiro/São Paulo: \versal{PUC} Rio \& Loyola,
  2015.
\tit{Platão}.~\emph{Górgias}. Trad. Daniel R.N.~Lopes. São Paulo:
  Perspectiva, 2011.
\tit{Platão}.\emph{~Íon. Hípias menor}. Trad. Andŕe Malta. Porto
  Alegre: \versal{L\&PM}, 2007.
\tit{Platão}.~\emph{O~Banquete. Apologia de Sócrates}. Trad. C.A.~Nunes.
  Pará: \versal{UFPA}, 2001.
\tit{Platão}.~\emph{Parmênides}. Trad. M.~Iglésias~\& F.~Rodrigues. Rio de Janeiro/São
  Paulo: \versal{PUC} Rio \& Loyola, 2003.
\tit{Platão}.~\emph{Protágoras. Górgias. Fedão}. Trad. C.A.~Nunes. Pará:
  \versal{UFPA}, 2002.
\tit{Platão}.~\emph{República}. Trad. Maria H.~da~R.~Pereira. Lisboa:
  \versal{F.C.}~Gulbenkian, 1993.
\tit{Platão}.~\emph{Teeteto. Crátilo}. Trad. C.A.~Nunes. Pará: \versal{UFPA}, 2001.
\tit{Platão}.~\emph{Timeu. Crítias. Segundo Alcebíades. Hípias Menor}.
  Trad. C.A.A.~Nunes. Pará: \versal{UFPA}, 2001.
\tit{Platon}.~\emph{Lysis}. Trad. A.~Croiset. Paris: Les Belles Lettres,
  1999.
\tit{Platone}.~\emph{La Repubblica}. Trad. G.~Lozza. Milano: Mondadori,
  1990.
\tit{Platone}.~\emph{Politico}. Trad. G.~Giorgini. Milano: \versal{BUR}, 2005.
\tit{Platone}.~\emph{Teeteto}. Trad. F.~Ferrari. Milano: \versal{BUR}, 2011.
\tit{Plotino}.~\emph{Tratados das Enéadas}. Trad. A.~Sommerman. São Paulo: Polar, 2000.
\tit{Safo~de~Lesbos}.~\emph{Hino a Afrodite e outros poemas}. Trad. G.~Ragusa. São Paulo: Hedra, 2011.
\tit{\emph{Scholia~graeca~in~Aristophanem}}. Trad. F.~Dübner. Paris:
  Didot, 1969.
\tit{Strabo}.~\emph{Geography}. Trad. H.~L.~Jones.
  Harvard University Press, 1928 (vol. \versal{V}, livros 10"-12).
\end{bibliohedra}

 

 

\section{Bibliografia Geral}


\begin{bibliohedra}
\tit{Albert, K.}~\emph{Platonismo, caminho e essência do filosofar
  occidental}. São Paulo: Loyola, 2011.
\tit{Arnaoutoglou, I.}~\emph{Leis da Grécia Antiga}. São Paulo: Odysseus,
  2003.
\tit{Ast}, F. \emph{Lexicon Platonicum}. Lipsiae: Weidmann,
  1835 (3 vols).
\tit{Austin, N.}~\emph{Helen of Troy and her shameless phantom}. New York:
  Cornell University Press, 1994.
\tit{Barnabé, A.}~\emph{Platón y el orfismo}. Madrid: Agapea, 1990.
\tit{Benson, H.H.}~[et. al.]~\emph{Platão}. Trad. M.~A. Zingano. Porto
  Alegre: Artmed, 2011.
\tit{Berg, R.M.}~van den.~\emph{Proclus' hymns: essays, translations and
  commentary}. Köln: Brill, 2001.
\tit{Bowra, C.M.}~Stesichorus in the Peloponnese. \emph{The Classical
  Quarterly}, v. 28, n.2, 1934.
\tit{Bowra C.M.}~The Two Palinodes of Stesichorus. \emph{The Classical
  Review}, vol. 13, n. 3, 1963.
\tit{Brisson, L.}~\emph{Orphée --- poèmes magiques et cosmologiques}, Paris:
  Les Belles Lettres, 1993.
\tit{Brisson, L.}~Proclus et l'orphisme. In: \emph{Proclus, lecteur et
  interprète des anciens}. Paris: Les Belles Letres, 1987.
\tit{Bruce, I.A.F.}~Athenian Embassies in the Early Fourth
  Century~B.C.~\emph{Zeitschrift für Alte Geschichte}, v. 15, n.3, 1966.
\tit{Cairus}, Henrique F.~\& Ribeiro Jr., W.~A.~\emph{Textos
  Hipocráticos}:~\emph{o doente, o médico e a doença}. Rio de Janeiro:
  Fiocruz, 2013.
\tit{Cambiano, G.}~Dialettica, medicina, retorica nel Fedro platonico.
  \emph{Rivista di Filosofia}, vol.~57, 1966.
\tit{Campbell, D.}~\emph{Greek Liric~\versal{III}}. Cambridge: Harvard University Press,
  1991.
\tit{Campos, R.G.}~de.~\emph{O}~Fedro \emph{de Platão à luz da tríade de
  Estesícoro}. Tese de Doutorado. São Paulo: \versal{FFLCH}"-\versal{USP}, 2012.
\tit{Cardoso, D.}~\emph{A~alma como centro do filosofar de Platão}. São
  Paulo: Loyola, 2006.
\tit{Casadio, G.}~La metempsicosi tra Orfeo e Pitagora. In: \versal{Borgeaud}, Philippe (org.).
  \emph{Orphisme et Orphée}. Genève: Droz, 1991.
\tit{Cassin, B.}~\emph{O~efeito sofístico}. São Paulo: Ed. 34, 2005.
\tit{Cerri, G.}~Il ruolo positivo della scrittura secondo Il Fedro di
  Platone.~In: \versal{Rosseti, L.} (org.). \emph{Understanding the Phaedrus}. Sankt
  Augustin: Verlag, 1992.
\tit{Chantraine, P.}~\emph{Dictionnaire Etymologique Langue Grecque: Histoire
  des mots}. Paris: Klincksieck 1968.
\tit{Cherniss, H.}~\emph{Selected Papers}. Leiden: Brill, 1977.
\tit{Colli, G.}~\emph{La Nascita della filosofia}. Milano: Adelphi, 1975.
\tit{Córdova, P.V.; Aquino, S.; Juárez, M.G.~\& Vidal, G.R.}~\emph{Oratória griega y oradores áticos del primer periodo}. México: Unam, 2004.
\tit{Cornford, F.M.}~\emph{Principium Sapientae, as origens do pensamento
  filosófico grego}. Lisboa: Calouste Gulbenkian, 1989.
\tit{Corrêa, P.C.}~\emph{Harmonia -- mito e música na Grécia antiga}. São
  Paulo: Humanitas, 2003.
\tit{D'Alfonso, F.}~La ``choreía'' astrale in un passo
  del~Fedro~platônico. \emph{Helikon}, vol.~33-34, 1993-1994.
\tit{D'Alfonso, F.}~\emph{Stesicoro et la performance}. Roma: \versal{GEI}, 1994.
\tit{Davies, M.}~\emph{Poetarum Melicorum Graecorum Fragmenta}. Oxford: Oxford University Press,
  1991.
\tit{Defradas, J.}~\emph{Les thèmes de la propagande delphique}. Paris:
  Librairie G.~Klincksieck, 1954.
\tit{Denniston, J.D.}~\emph{Greek prose style}. London: Bristol, 2002.
\tit{Derrida, J.}~\emph{A~Farmácia de Platão}. São Paulo: Iluminuras, 1997.
\tit{Diels, H.~\&~Kranz, W.}~\emph{Die fragmente der
  vorsokratiker}.~Berlin: Weidmann, 1989 (3 vols.).
\tit{Dixsaut, M.}~\emph{Métamorphoses de la dialectique de Platon}. Paris:
  Vrin, 2001.
\tit{Dodds, E.R.}~\emph{The Greeks and the Irrational}. London: University
  of California Press, 1997.
\tit{Dover, K.J.}~\emph{A~Homossexualidade na Grecia Antiga}. São Paulo:
  Nova Alexandria, 2007.
\tit{Dover, K.J.}~\emph{Greek word order}. London: Bristol, 2001.
\tit{Entralgo, P.L.}~\emph{La curación por la palabra en la antigüidad
  clásica}. Barcelona: Anthropos, 1987.
\tit{Fattal, M.}~L'alethès lógos du Phèdre
  en~270c10.~In:~\versal{Rosseti, L.}~(Org.).~\emph{Understanding the
  Phaedrus}. Sankt Augustin: Academia Verlag, 1992.
\tit{Ferrari, G.R.F.}~\emph{Listening to the Cicadas: A Study of Plato's
  Phaedrus}. Cambridge: Cambridge Classical Studies, 1990.
\tit{Festugière, A.J.}~Platon et L'orient. In: \emph{Études de philosophie
  grecque}. Paris: J.~Vrin, 1971.
\tit{Fontes, J.B.}~\emph{Eros, tecelão de mitos -- a poesia de Safo de
  Lesbos}. São Paulo: Estação Liberdade, 1991.
\tit{Friedländer, P.}~\emph{Platone}. Milano: Bompiani, 2004.
\tit{Furley W.D.~\& Bremer, J.M.}~\emph{Greek Hymns}. Tübingen: Mohr
  Siebeck, 2001 (vol. \versal{I}).
\tit{Gadamer, H"-G.}~\emph{A~ideia do Bem entre Platão e Aristóteles}. São
  Paulo: Martins Fontes, 2009.
\tit{Gaiser, K.}~\emph{La metafísica della storia in Platone}. Milano: Vita
  e Pensiero, 1991.
\tit{González, J.}~Psique y Eros en el~Fedro. In:~\versal{Conrado, E.L.}~(org.).~\emph{Platón: Los
  diálogos tardíos. Actas del Symposium
  Platonicum}. Sankt Augustin: Academia Verlag, 1986.
\tit{Graves, R.}~\emph{The Greek Myths}. Middlesex: Penguin Books, 1960.
\tit{Grimal, P.}~\emph{Dicionário da Mitologia grega e romana}. Rio de
  Janeiro: Bertrand, 1997.
\tit{Guthrie, W.K.C.}~\emph{The Greek philosophers}. London: Harper
  Colophon, 1975.
\tit{Guthrie, W.K.C.}~\emph{Os sofistas}. Brasil: Paulus, 1991.
\tit{Heath, M.}~The unity of Plato's Phaedrus. \emph{Oxford Studies in
  Ancient Philosophy}, vol.~7, 1989. 
\tit{Hösle, V.}~\emph{Interpretar Platão}. São Paulo: Loyola, 2004.
\tit{Howland, R.L.}~The attack on Isocrates in the Phaedrus. \emph{The
  Classical Quarterly}, vol. 31, n. 3-4, 1937.
\tit{Irwin, T.}~\emph{Plato's ethics}. New York: Oxford University Press, 1995.
\tit{Jaeger, W.}~\emph{Paideia}. México: Fondo de Cultura Economica, 2004.
\tit{Joly, R.}~La question hippocratique et le témoignage du Phèdre. \emph{Revue des Études Grecques}, tomo 74, fasc. 349"-350, p.~69"-92, jan.-jun.~1961.
\tit{Jouanna, J.}~\emph{Hippocrate}. Paris: Fayard,1992.
\tit{Käpel, L.}~\emph{Paian, Studien zur Geschichte einer Gattung}.
  Berlin: Walter de Gruyter, 1992.
\tit{Kastely, J.L.}~Respecting the rupture: not solving the problem
  of Unity in Plato's Phaedrus. \emph{Philosophy and Rhetoric}, vol.~35, n.~2, 2002.
\tit{Kelly, A.}~Stesikhoros and Helen. \emph{Museum Helveticum}, v.
  64, n. 1, p.~1"-21, mar. 2007.
\tit{Kennedy, G.A.}~\emph{Invention and Method: two rethorical treatises
  from the Hermogenic Corpus}.~Atlanta: Society of Biblical Literature,
  2005.
\tit{Kerényi, K.}~\emph{Arquétipos da religião grega}. Rio de Janeiro:
  Vozes, 2015.
\tit{Kern, O.}~\emph{Orphicorum Fragmenta}. Zurich: Weidmann, 1972.
\tit{Kirk, G.S., Raven, J.E.~\& Schofield, M.}~\emph{Os filósofos
  pré"-socráticos}. Lisboa: Calouste Gulbenkian, 2010.
\tit{Krämer, H.}~\emph{Platone e i fondamenti della metafísica}. 3. ed. Milano:
  Vita e Pensiero, 1989.
\tit{Krämer, H.}~\emph{La nuova imagine di Platone}. Napoli: Bibliopolis,
  1986.
\tit{Kraut, R.}~(ed.)~\emph{The Cambridge Companion to Plato}. Cambridge:
  Cambridge University Press, 1992.
\tit{Kucharski, P.}~La rhétorique dans le Gorgias et le Phèdre. \emph{Revue des Études Grecques}, tomo 74, fasc. 351"-353, p. 371"-406, jul.-dez. 1961.
 \tit{Lidell, H.G., Scott, R.~\& Jones, H.S.}~\emph{A~Greek"-English
  Lexicon}. Oxford: Oxford University Press, 1996.
\tit{Marques, M.P.}~\emph{Platão, pensador da diferença -- uma leitura do} Sofista. Belo Horizonte: \versal{UFMG}, 2006.
\tit{Marrou, H."-I.}~\emph{Histoire de l'Education dans L'Antiquité}. Paris:
  Seuil, 1965.
\tit{Mattéi, J."-F.}~\emph{Platão}. São Paulo: Unesp, 2010.
\tit{Mazzara, G.}~\emph{Gorgia, la retorica del verosimile}. Sankt Augustin: Academia Verlag, 1999.
\tit{Moravcsik, J.}~\emph{Platão e platonismo}. São Paulo: Loyola, 2006.
\tit{Moss, J.}~Soul"-Leading: The unity of the Phaedrus, Again. \emph{Oxford Studies in Ancient Philosophy}, vol. 43, 2012.
\tit{Mossé, C.}~\emph{Histoire d'une démocratie: Athènes}. Paris: Editions
  du Seuil, 1971.
\tit{Mossé, C.}~\emph{Péricles, o inventor da democracia}. São Paulo:
  Estação Liberdade, 2008.
\tit{Nightingale, A.W.}~\emph{Genres in dialogue: Plato and the
  construct of philosophy}. Cambridge: Cambridge University Press, 1995.
\tit{Paleologos, K.}~As modalidades esportivas. In: \versal{Cabral, L.A.} et al.~\emph{Os Jogos
  Olímpicos na Grécia Antiga}. São Paulo: Odysseus, 2004.
\tit{Pradeau, J."-F.~\& Brisson, L.}~\emph{Dictionnaire Platon}. Paris:
  Elipses, 1998.
\tit{Pulquério, M.}~de~\versal{O.}~O~problema das duas Palinódias de
  Estesícoro. \emph{Humanitas}, vol.~25"-26, 1973.
\tit{Ragon, E.}~\emph{Grammaire grecque}. Paris: Gigord, 1961.
\tit{Reale, G.}~\emph{Por uma nova interpretação de Platão}. São Paulo:
  Loyola, 1997.
\tit{Rijksbaron, A.}~\emph{The syntax and Semantics of the verb in classical
  Greek}. Chicago: University of Chicago Press, 2002.
\tit{Robin, L.}~\emph{Théorie platonicienne de l'amour}. Paris: Felix Alcan,
  1908.
\tit{Robinson, T.M.}~\emph{A~psicologia de Platão}. Trad. Marcelo P. Marques. São Paulo: Loyola, 2007.
\tit{Robinson, T.M.}~The argument for immortality in
  Plato's~\emph{Phaedrus}. In: \versal{Anton, J. \& Kustas, G.} (eds.). \emph{Essays in ancient Greek Philosophy}.
  New York: \versal{NY} Press, 1971.
\tit{Rosseti, L.}~(org.).~\emph{Understanding the Phaedrus}. Sankt Augustin:
  Academia Verlag, 1992.
\tit{Rougier, L.}~\emph{La religion astrale des pythagoriciens}. Paris:
  Presses Universitaires de France, 1959.
\tit{Rowe, C.}~The unity of the \emph{Phaedrus}: a replay to Heath. \emph{Oxford Studies in Ancient Philosophy}, vol.~8, 1989.
\tit{Schäfer, C.}~(org.).~\emph{Léxico de Platão}. São Paulo: Loyola,
  2012.
\tit{Schäfer, C.}~Rhetoric as part of an initiation into the
  mysteries: a new interpretation of the platonic \emph{Phaedrus}. In: \versal{Michelini, A.N.} (ed.). \emph{Plato as
  author: the rhetoric of philosophy}. Boston: Brill, 2003.
\tit{Slezák, T.}~\emph{Ler Platão}. São Paulo: Loyola, 2005.
\tit{Smith, H.W.}~\emph{Greek Grammar}. 2\textsuperscript{a}~ed. 1920
  [revised by Messing, 1956].
\tit{Sorel, R.}~\emph{Les cosmogonies grecques}. Paris: \versal{PUF}, 1994.
\tit{Trabattoni, F.}~\emph{Oralidade e escrita em Platão}. Trad. R. Bolzani
  Filho \& F. R. Puente. São Paulo: Discurso Editorial, 2003.
\tit{Trabattoni, F.}~\emph{Scrivere nell'anima. Verità, dialettica e
  persuasione in Platone}. Firenze: La Nuova Italia, 1994.
\tit{Trivigno, F.}~Putting Unity in its Place: Organic Unity in
  Plato's Phaedrus. \emph{Literature and Aesthetics}, vol. 19, 2009.
\tit{Untersteiner. M.}~\emph{I sofisti}. Milano: Mondadori, 1996.
\tit{Urbina, J.M.P.}~\emph{Diccionario bilingüe manual Griego clássico"-Español}. Madrid: Vox, 2012.
\tit{Vernant, J."-P.}~\emph{Mito e religião na Grécia antiga}. Campinas:
  Papirus, 1992.
\tit{Vernant, J."-P.}~\emph{Mito e sociedade na Grécia antiga}. Brasil: \versal{UNB},
  1992.
\tit{Vidal, G.R.}~La dimensión política de la retórica griega.
  \emph{Rétor}, vol.~1, n.~1, 2011.
\tit{Vidal, G.R.}~Notas sobre la retórica de Isócrates. \emph{Noua Tellus},
  vol. 24, n. 1, 2006.
\tit{Vürtheim, J.}~\emph{Stesichoros' Fragmente und Biographie}. Lieden:
  \versal{A.W.}~Sijthoff's, 1919.
\tit{Watanabe, L.A.}~\emph{ Platão -- por mitos e hipóteses}. São Paulo:
  Moderna, 1995.
\tit{Watanabe, L.A.}~\emph{Vérité du mythe, vérité de l'histoire chez
  Platon}. Paris: \versal{EHESS}, 1983.
\tit{West, M.L.}~\emph{Greek Metre}. London: Oxford, 1982.
\end{bibliohedra}
